\documentclass[a4paper,10pt]{article}

% This bunch of macros grew to that size in a couple of years,
% some parts are even decades old, so be aware of the risks of diving in!

\usepackage{palatino}
%\usepackage{lmodern}

% Narrow the margins a bit to fit in the code
\usepackage[margin=1.2in]{geometry}
\usepackage{amsmath, amsthm,amsfonts,amssymb}
\usepackage{graphicx}
\usepackage{fixltx2e}
\usepackage{listings}
\usepackage{algorithm}

\usepackage{hyperref}
\usepackage{longtable}
\usepackage{siunitx}
\usepackage{graphicx}

\usepackage{float}
% pseudocode
\usepackage[end]{algpseudocode}
\newcommand*\Let[2]{\State #1 $\gets$ #2}
\algrenewcommand\alglinenumber[1]{{\sf\scriptsize#1}}
\algrenewcommand\algorithmicrequire{\textbf{Input:}}
\algrenewcommand\algorithmicensure{\textbf{Output:}}

\makeatletter
\let\OldStatex\Statex
\renewcommand{\Statex}[1][3]{%
  \setlength\@tempdima{\algorithmicindent}%
  \OldStatex\hskip\dimexpr#1\@tempdima\relax}
\makeatother

\algnewcommand\algorithmicblackcomment[1]{\hfill\(\blacktriangleright\) #1}
\algnewcommand{\CommentMulti}[1]{ \(\blacktriangleright\) #1}
\makeatletter
\algnewcommand{\CommentInlineMulti}[1]{\Statex[\theALG@nested] \(\triangleright\) #1}

% "algorithmic" without "algorithm" lacks all of the horizontal lines
% The top lines are in the  "captionof" redefinition below and the trailing
% line is here
\algrenewcommand\ALG@endalgorithmic{\Statex[-1]\hrulefill}

\usepackage{xpatch}
% too many fractions in the algorithms
% TODO: do it individually per case
\xpatchcmd{\algorithmic}{\itemsep\z@}{\itemsep=.25ex}{}{}
\makeatother


% shamelessly stolen from http://tex.stackexchange.com/questions/53357/switch-cases-in-algorithmic
% New definitions
\algnewcommand\algorithmicswitch{\textbf{switch}}
\algnewcommand\algorithmiccase{\textbf{case}}
\algnewcommand\algorithmicdefault{\textbf{default}}

\newcommand{\longsub}[1]{\text{\textit{\scriptsize{#1}}}}
\newcommand{\RETURN}{\State \textbf{return} }
\newcommand{\Break}{\State \textbf{break} }
\newcommand{\Continue}{\State \textbf{break} }

% New "environments"
\algdef{SE}[SWITCH]{Switch}{EndSwitch}[1]{\algorithmicswitch\ #1\ \algorithmicdo}{\algorithmicend\ \algorithmicswitch}%
\algdef{SE}[CASE]{Case}{EndCase}[1]{\algorithmiccase\ #1}{\algorithmicend\ \algorithmiccase}%
\algdef{SE}[DEFAULT]{Default}{EndDefault}[1]{\algorithmicdefault\ #1}{\algorithmicend\ \algorithmicdefault}%
\algtext*{EndSwitch}%
\algtext*{EndCase}%

\algdef{SE}[DOWHILE]{Do}{doWhile}{\algorithmicdo}[1]{\algorithmicwhile\ #1}


\makeatletter
\def\hrulefillthick{\leavevmode\leaders\hrule height .85pt\hfill\kern\z@}
\makeatother
% shamelessly stolen from http://tex.stackexchange.com/questions/33866/algorithm-tag-and-page-break
\usepackage[font=small,labelfont=bf,width=.8\linewidth]{caption}
\DeclareCaptionFormat{algor}{%
  \hrulefillthick\par\offinterlineskip\vskip2pt%
     \textbf{#1#2}#3\offinterlineskip\hrulefill}
\DeclareCaptionStyle{algori}{singlelinecheck=off,format=algor,labelsep=space}
\captionsetup[algorithm]{style=algori}
%Still necessary?
\MakeRobust{\Call}

% http://tex.stackexchange.com/questions/77996/how-to-show-a-hint-when-lstlisting-is-breaking-page
\usepackage[framemethod=tikz]{mdframed}
% define the frame style for the listing:
\mdfdefinestyle{note}
  {
    hidealllines = true,
    skipabove    = .5\baselineskip,
    skipbelow    = .5\baselineskip,
    singleextra  = {},
    firstextra   = {
      \node[below right,overlay,align=center,font=\continuingfont]
        at (O) {\continuingtext};
    },
    secondextra  = {
      \node[above right,overlay,align=center,font=\continuingfont]
        at (O |- P) {\continuedtext};
    },
    middleextra  = {
      \node[below right,overlay,align=center,font=\continuingfont]
        at (O) {\continuingtext};
      \node[above right,overlay,align=center,font=\continuingfont]
        at (O |- P) {\continuedtext};
    }
  }
\newcommand*\continuingfont{\footnotesize\itshape}
\newcommand*\continuingtext{\hspace{2em}Listing continues on next page}
\newcommand*\continuedtext{\hspace{2em}Continuing from last page}
% the trick with mdframed writes over the footnote-line,
% this lowers the footnote
\setlength{\skip\footins}{5ex}

\ifpdf
\pdfcompresslevel=9
\pdfinfo{
   /Title      (LibTomMath's Algorithms: Fast Number Conversion)
   /Author     (Editor: Christoph Zurnieden)
   /Keywords   (arbitrary-precision algorithm libtommath libtom number-conversion)
}
\fi


\providecommand{\abs}[1]{\left\lvert#1\right\rvert}
\providecommand{\floor}[1]{\left\lfloor#1\right\rfloor}
\providecommand{\ceil}[1]{\left\lceil#1\right\rceil}

\DeclareMathOperator{\AND}{\wedge}
\DeclareMathOperator{\OR}{\vee}
\DeclareMathOperator{\sgn}{sgn}
\DeclareMathOperator{\flog}{flog}

% don't forget to add an empty statement {} after that command to get a space
% I did not want to add another package to this already very heavy prelude
% but you can include the package "xspace" and change the macro to
% \newcommand{\nthroot}{$n^{\text{\tiny th}}$-root\xspace}
% some say that xspace produces more problems than it solves, vid.:
% https://tex.stackexchange.com/questions/86565/drawbacks-of-xspace
% where you'll find out that its own creator does not recommend it.
\newcommand{\nthroot}{$n^{\text{\tiny th}}$-root}

\newcommand{\RaiseNum}[2]{\raisebox{#1}{$\scriptstyle #2$}}

% wonder if I'll ever need it
% C&P from the amsthm documentation
\theoremstyle{plain} % is said to be the default
\newtheorem{thm}{Theorem}[section]
\newtheorem{lem}[thm]{Lemma}
\newtheorem{prop}[thm]{Proposition}

\theoremstyle{definition}
\newtheorem{defn}{Definition}[section]
\newtheorem{conj}{Conjecture}[section]
\newtheorem{exmp}{Example}[section]

\theoremstyle{remark}
\newtheorem*{rem}{Remark}
\newtheorem*{note}{Note}
\newtheorem{case}{Case}

% to get the notes a counter (lacks internationalization)
%\newcounter{notice}
%\newenvironment{notice}{
%\stepcounter{notice}
%\par\bigskip\noindent{\bfseries Note \arabic{notice}}
%\par\medskip}{\par\medskip}

% shamelessly stolen from https://tex.stackexchange.com/a/94466
% still lacks internationalization
\global\mdfdefinestyle{notice}{%
linecolor=gray,linewidth=1pt,%
leftmargin=1cm,rightmargin=1cm,
}
\newenvironment{fnotice}[1]{%
\mdfsetup{%
frametitle={\colorbox{white}{\,#1\,}},
%frametitle={\tikz\node[fill=white,rectangle,inner sep=0pt,outer sep=0pt]{#1};},
frametitleaboveskip=-\ht\strutbox,
%frametitleaboveskip=-0.5\ht\strutbox,
frametitlealignment=\raggedright
}%
\begin{mdframed}[style=notice]
}{\end{mdframed}}

\definecolor{lightgray}{rgb}{.9,.9,.9}
\definecolor{darkgray}{rgb}{.4,.4,.4}
\definecolor{purple}{rgb}{0.65, 0.12, 0.82}

\lstdefinelanguage{parigp}{
  alsoletter={\\},
  keywords={ for, forcomposite, fordiv, fordivfactored, forell, forfactored, forpart,%
             forperm, forprime, forsquarefree, forstep, forsubgroup, forsubset,%
             forvec, if, iferr, next, return, until, while, local, alias, allocatemem,%
             apply, call, default, extern, externstr, fold, getabstime, getenv,%
             getheap, getrand, getstack, gettime, getwalltime, global, inline, input,%
             localbitprec, localprec, print, print1, printf, printp, printsep,%
             printsep1, printtex, quit, read, readstr, readvec, self, setrand,%
             system, type, uninline, version, write, write1, writebin, writetex},
  keywordstyle=\bfseries,
  morekeywords={[2]{log, floor, ceil,gcd, lcm,sqrt, sqrtn,sqrtint,%
              sin,cos,tan,asin,acos,atan,sinh,cosh,tanh, asinh,acosh,atanh,%
              gamma,lngamma,psi,dilog,%
              % own functions
              flog2, newton, halley, bisection}},
  keywordstyle={[2]{\color{darkgray}\bfseries}},
  sensitive=false,
  comment=[l]{\\\\},
  commentstyle=\ttfamily,
  stringstyle=\ttfamily,
  morestring=[b]',
  morestring=[b]"
}

\lstdefinestyle{code}{
   extendedchars=true,
   basicstyle=\footnotesize\ttfamily,
   showstringspaces=false,
   showspaces=false,
   tabsize=2,
   breaklines=true,
   showtabs=false,
   xleftmargin=.2\textwidth, xrightmargin=.2\textwidth,
   frame=lines
%   texcl= true
}

\lstdefinestyle{widercode}{
   extendedchars=true,
   basicstyle=\footnotesize\ttfamily,
   showstringspaces=false,
   showspaces=false,
   tabsize=2,
   breaklines=true,
   showtabs=false,
   xleftmargin=.05\textwidth, xrightmargin=.05\textwidth,
   frame=lines
}

\lstnewenvironment{pblisting}[1]
  {%
    \lstset{style=widercode,#1}%
    \mdframed[style=note]%
  }
  {%
    \endmdframed
  }
% to get subsubsection numbered, too
\setcounter{secnumdepth}{3}
\setcounter{tocdepth}{3}

% Boxes with the gray background
% shamelessly stolen from https://tex.stackexchange.com/a/7545
\usepackage{calc}
\newlength{\DepthReference}
\settodepth{\DepthReference}{g}
\newlength{\HeightReference}
\settoheight{\HeightReference}{T}
\newlength{\Width}
\newcommand{\newcolorbox}[2][gray!30] {%
  \settowidth{\Width}{#2}%
  \setlength{\fboxsep}{.9pt}%
  \fcolorbox{black}{#1}{%
    \raisebox{-\DepthReference} {%
      \parbox[b][\HeightReference+\DepthReference][c]{\Width}{\centering#2}%
    }%
  }%
}
\setlength{\fboxsep}{1pt}
%no, a framed ox with grey background is just too much here
%\newcommand*{\ttchar}[1]{\newcolorbox{\texttt{\strut#1}}}
\newcommand*{\ttchar}[1]{\texttt{\textbf{\strut#1}}}
\newcommand*{\smalltt}[1]{\small\texttt{#1}}

\makeatletter
% Idea shamlessly stolen from Stephan Lehmke in
% https://tex.stackexchange.com/a/53716
\newcommand\circled
{%
  \mathpalette\@circled
}
\newcommand\@circled[2]
{%
  \mathbin%
  {%
    \ooalign{\hidewidth$#1#2$\hidewidth\crcr$#1\bigcirc$}%
  }%
}
\newcommand{\odiv}{\circled{\div}}
\makeatother

\sisetup{output-exponent-marker=\ensuremath{\mathrm{e}}}

\begin{document}
\title{LibTomMath's Algorithms: Fast Number Conversions}
\author{Current editor: Christoph Zurnieden\\
        \small{\texttt{$<$czurnieden@gmx.de$>$}}}
\date{Last change: \today}
\maketitle

%\tableofcontents
% Courier because Computer Modern has no boldface
\renewcommand{\ttdefault}{pcr}



\begin{abstract}
This artikel describes an algorithm for fast number conversion based on the function \smalltt{Barret\_todecimal} in Lars Helmstr\"om's post to the TCL list\footnote{Available online at: \url{https://sourceforge.net/p/tcl/mailman/message/31972670/}}.
\end{abstract}

\section{Introduction}

\section{Big--integer to String}

\subsection{Setup}

The setup does nothing more than to check the input and fills the cache with the reciprocals.

% TODO: no minipage enclosure to reduce width, it is put on the next page if it is in one.
\begin{center}
  \captionof{algorithm}{mp\_get\_str\label{alg:mpgetstr}}
  \begin{algorithmic}[1]
    \Require{$a > 0$, big--integer; $b$, a buffer of suitable size; $2 \le r \le 64$ the radix}
    \Ensure{$b$ filled with the value of $a$ converted to a string with radix $r$}
    \Function{mp\_get\_str}{$a$}
      \If{$a < C_{\textnormal{\textit{cut--off}}}$}
         \State\Call{mp\_to\_radix}{$a,b,r$}
      \EndIf
      \CommentInlineMulti{Here $t$ is a table with the start--values given in table \ref{tab:startvalues}}
      \Let{$n$}{$t_n$}
      \Let{$M$}{$t_m$}
      \CommentInlineMulti{If the values of the table \ref{tab:startvalues} is used, $s = 40$ for all entries}
      \Let{$s$}{$t_s$}
      \CommentInlineMulti{Introducing $\flog_2(x) = \floor{\log_2(x)}$ for simplicity}
      \Let{$c$}{$\flog_2\left(\floor{\frac{\flog_2\left(a\right)}{2\flog_2\left(n\right)}}\right)$}
      \CommentInlineMulti{Setup and zero out three buffers able to hold $c$ big--integers each}
      \Let{$n^\ell$}{$\{c,0\}$}
      \Let{$s^\ell$}{$\{c,0\}$}
      \Let{$m^\ell$}{$\{c,0\}$}

      \CommentInlineMulti{Set start--values: radix, shift and reciprocal}
      \Let{$n^\ell_0$}{$n$}
      \Let{$s^\ell_0$}{$s$}
      \Let{$m^\ell_0$}{$M$}

      \Let{$\rho$}{$1$}\Comment{Index into the buffers $n^\ell, s^\ell, m^\ell$}
      \While{$\rho < c$}
        \Let{$M_2$}{$2M^2$}
        \Let{$M_4$}{$\floor{\frac{nM^4}{2^{s + 6}}}$}
        \Let{$M$}{$\floor{\frac{\left(M_2 - M_4\right)}{2^6}} + 1$}
        \Let{$m^\ell_\rho$}{$M$}
        \Let{$\rho$}{$\rho + 1$}
        \Let{$M$}{$8M$}
      \EndWhile
      \CommentInlineMulti{Use the precomputed reciprocals recursively}
      \State \Call{mp\_get\_str\_recursion}{$a, r, n^\ell, s^\ell, m^\ell, \rho - 1, 1, b$}
    \EndFunction
  \end{algorithmic}
\end{center}




\begin{center}
\begin{longtable}{r r r r | r r r r}
\multicolumn{1}{r}{$\mathbf{b}$}&
\multicolumn{1}{r}{$\mathbf{k}$}&
\multicolumn{1}{r}{$\mathbf{n = b^k}$}&
\multicolumn{1}{r}{$\mathbf{M = \ceil{\ceil{2^{s+3}/n}/8}}$}&
\multicolumn{1}{r}{$\mathbf{b}$}&
\multicolumn{1}{r}{$\mathbf{k}$}&
\multicolumn{1}{r}{$\mathbf{n = b^k}$}&
\multicolumn{1}{r}{$\mathbf{M = \ceil{\ceil{2^{s+3}/n}/8}}$}\\
\endfirsthead
%
\multicolumn{1}{r}{$\mathbf{b}$}&
\multicolumn{1}{r}{$\mathbf{k}$}&
\multicolumn{1}{r}{$\mathbf{n = b^k}$}&
\multicolumn{1}{r}{$\mathbf{M = \ceil{\ceil{2^{s+3}/n}/8}}$}&
\multicolumn{1}{r}{$\mathbf{b}$}&
\multicolumn{1}{r}{$\mathbf{k}$}&
\multicolumn{1}{r}{$\mathbf{n = b^k}$}&
\multicolumn{1}{r}{$\mathbf{M = \ceil{\ceil{2^{s+3}/n}/8}}$}\\
\endhead
% To get a bit more "headroom"
\rule{0pt}{1ex}\\
\multicolumn{6}{c}{{\footnotesize{\textit{Continued on next page}}}} \\
\endfoot
\endlastfoot
2  & 20 & 1048576 & 1048576  &  33 & 3  & 35937   & 30595532 \\
3  & 12 & 531441  & 2068926  &  34 & 3  & 39304   & 27974548 \\
4  & 10 & 1048576 & 1048576  &  35 & 3  & 42875   & 25644587 \\
5  & 8  & 390625  & 2814750  &  36 & 3  & 46656   & 23566351 \\
6  & 7  & 279936  & 3927726  &  37 & 3  & 50653   & 21706743 \\
7  & 7  & 823543  & 1335100  &  38 & 3  & 54872   & 20037754 \\
8  & 6  & 262144  & 4194304  &  39 & 3  & 59319   & 18535573 \\
9  & 6  & 531441  & 2068926  &  40 & 3  & 64000   & 17179870 \\
10 & 6  & 1000000 & 1099512  &  41 & 3  & 68921   & 15953217 \\
11 & 5  & 161051  & 6827103  &  42 & 3  & 74088   & 14840617 \\
12 & 5  & 248832  & 4418691  &  43 & 3  & 79507   & 13829118 \\
13 & 5  & 371293  & 2961305  &  44 & 3  & 85184   & 12907490 \\
14 & 5  & 537824  & 2044371  &  45 & 3  & 91125   & 12065972 \\
15 & 5  & 759375  & 1447917  &  46 & 3  & 97336   & 11296043 \\
16 & 5  & 1048576 & 1048576  &  47 & 3  & 103823  & 10590251 \\
17 & 4  & 83521   & 13164494 &  48 & 3  & 110592  & 9942054  \\
18 & 4  & 104976  & 10473934 &  49 & 3  & 117649  & 9345695  \\
19 & 4  & 130321  & 8436949  &  50 & 3  & 125000  & 8796094  \\
20 & 4  & 160000  & 6871948  &  51 & 3  & 132651  & 8288755  \\
21 & 4  & 194481  & 5653569  &  52 & 3  & 140608  & 7819695  \\
22 & 4  & 234256  & 4693633  &  53 & 3  & 148877  & 7385370  \\
23 & 4  & 279841  & 3929059  &  54 & 3  & 157464  & 6982623  \\
24 & 4  & 331776  & 3314018  &  55 & 3  & 166375  & 6608635  \\
25 & 4  & 390625  & 2814750  &  56 & 3  & 175616  & 6260886  \\
26 & 4  & 456976  & 2406060  &  57 & 3  & 185193  & 5937113  \\
27 & 4  & 531441  & 2068926  &  58 & 3  & 195112  & 5635285  \\
28 & 4  & 614656  & 1788825  &  59 & 3  & 205379  & 5353574  \\
29 & 4  & 707281  & 1554562  &  60 & 3  & 216000  & 5090332  \\
30 & 4  & 810000  & 1357422  &  61 & 3  & 226981  & 4844070  \\
31 & 4  & 923521  & 1190565  &  62 & 3  & 238328  & 4613439  \\
32 & 4  & 1048576 & 1048576  &  63 & 3  & 250047  & 4397220  \\
   &    &         &          &  64 & 3  & 262144  & 4194304
\end{longtable}
\captionof{table}{Start--values for $n$ for assorted bases $b$ with $s = 40$}\label{tab:startvalues}
\end{center}

\subsection{Recursion}
The actual working horse is this recursively called function.
\begin{center}
\begin{minipage}{.9\linewidth}
  \captionof{algorithm}{mp\_get\_str\_recursion\label{alg:mpgetstrrecursion}}
  \begin{algorithmic}[1]
    \Require{$a > 0$, big-integer, $r, n^\ell, s^\ell, m^\ell, \rho, L, b$, from mp\_get\_str}
    \Ensure{$b$ filled with the value of $a$ converted to  a string with radix $r$}
    \Function{mp\_get\_str\_recursion}{$a,n^\ell, s^\ell, m^\ell, \rho, L, b$}
    \If{$\rho < 0$}
      \Let{$b$}{$b + a$}\Comment{Convert $a$ to a string with radix $r$ and add to $b$. See text for details}
      \RETURN
    \Else
      \Let{$q$}{$\floor{\left(a\cdot m^\ell_\rho\right) / 2^{s^\ell_\rho}}$}\Comment{Quotient}
      \Let{$r$}{$ a - n^\ell_\rho \cdot q$}\Comment{Approximate remainder}
      \If{$r < 0$}
        \Let{$q$}{$q - 1$}
        \Let{$r$}{$r + n^\ell_\rho$}\Comment{Exact remainder}
      \EndIf
      \Let{$\rho$}{$\rho + 1$}
      \If{$\left(L = 1\right) \AND \left(q = 0\right) $}
        \State \Call{mp\_get\_str\_recursion}{$r,n^\ell, s^\ell, m^\ell, \rho, 1, b$}
      \Else
        \State \Call{mp\_get\_str\_recursion}{$q,n^\ell, s^\ell, m^\ell, \rho, L, b$}
        \State \Call{mp\_get\_str\_recursion}{$r,n^\ell, s^\ell, m^\ell, \rho, 0, b$}
      \EndIf
    \EndIf
    \EndFunction
  \end{algorithmic}
 \end{minipage}
\end{center}


If the chunk is small enough we can build that part of the output. The number of digits is in column $k$ in table \ref{tab:startvalues}. This is also a place where a bit of optimization can take place because the number of digits in the leafs is very small.

\subsubsection{Larger Leafs}
The algorithm as it is now goes to the very end of the recursion until the actual number conversion is done. The 
amount of converted digits is very small down there, just the number of digits in $n^1-1$.
To get more digits we need to stop at an earlier stage $\rho > 0$ of the recursion to do the conversion instead of $
\rho = 0$. The maximum number of digits needed for radix $r$ is
\begin{equation}
\ceil{\log_r\left(n^{\left(2^k\right)}-1\right)}\quad\text{with}\quad n = r^k
\end{equation}

\begin{center}
\begin{minipage}{.9\linewidth}
  \captionof{algorithm}{to\_string\label{alg:tostring}}
  \begin{algorithmic}[1]
    \Require{$a$, big--integer, $b$, buffer, $r$, $k$, $\rho$, $L$, bool (is left?)}
    \Ensure{$b$ filled with the value of $a$ converted to base $r$}
    \Function{to\_string}{$a$, $b$, $r$, $k$, $\rho$, $L$}
    \Let{$K$}{$k + 2^\rho - 1$}\Comment{Cut--off at $\rho=5$ on the authors machine, around 600 bits for radix $10$.}
    \CommentInlineMulti{This can be an actual buffer or the corresponding range in $b$ directly}
    \Let{$S$}{A temporary buffer of size $K$}
    \Let{$S$}{Fill the whole buffer with the ASCII digit zero, \smalltt{0x30}}
    \Let{$i$}{$0$}
    \While{$a \not= 0$}
       \Let{$q$}{$\floor{\frac{a}{r}}$} \Comment{Big--integer single digit division}
       \Let{$d$}{$a - \left(q \cdot r\right)$}
       \Let{$c$}{$T_d$}\Comment{$T$ is a table mapping $d$ to a character}
       \Let{$S_{K-i}$}{$c$}\Comment{Fill buffer $S$ from the end}
       \Let{$i$}{$i + 1$}
       \Let{$a$}{$q$}
    \EndWhile
    \EndFunction
  \end{algorithmic}
\end{minipage}
\end{center}

\subsubsection{Converting Radices that are Powers of Two}
The radices $2$, $4$, $8$, $16$, $32$, and $64$ can be converted in the loop in in algorithm \ref{alg:tostring} without any big--integer division, by masking of a sliding window of the size of the radix. Two limbs are needed if \smalltt{MP\_DIGIT\_BIT} is not a power of two, just one otherwise.


% place adapted alg:tostring here

\section{String to Big--integer}


\subsection{Setup}
\subsection{Recursion}



\end{document}














