\documentclass[synpaper]{book}
\usepackage{hyperref}
\usepackage{makeidx}
\usepackage{amssymb}
\usepackage{color}
\usepackage{alltt}
\usepackage{graphicx}
\usepackage{layout}
\def\union{\cup}
\def\intersect{\cap}
\def\getsrandom{\stackrel{\rm R}{\gets}}
\def\cross{\times}
\def\cat{\hspace{0.5em} \| \hspace{0.5em}}
\def\catn{$\|$}
\def\divides{\hspace{0.3em} | \hspace{0.3em}}
\def\nequiv{\not\equiv}
\def\approx{\raisebox{0.2ex}{\mbox{\small $\sim$}}}
\def\lcm{{\rm lcm}}
\def\gcd{{\rm gcd}}
\def\log{{\rm log}}
\def\ord{{\rm ord}}
\def\abs{{\mathit abs}}
\def\rep{{\mathit rep}}
\def\mod{{\mathit\ mod\ }}
\renewcommand{\pmod}[1]{\ ({\rm mod\ }{#1})}
\newcommand{\floor}[1]{\left\lfloor{#1}\right\rfloor}
\newcommand{\ceil}[1]{\left\lceil{#1}\right\rceil}
\def\Or{{\rm\ or\ }}
\def\And{{\rm\ and\ }}
\def\iff{\hspace{1em}\Longleftrightarrow\hspace{1em}}
\def\implies{\Rightarrow}
\def\undefined{{\rm ``undefined"}}
\def\Proof{\vspace{1ex}\noindent {\bf Proof:}\hspace{1em}}
\let\oldphi\phi
\def\phi{\varphi}
\def\Pr{{\rm Pr}}
\newcommand{\str}[1]{{\mathbf{#1}}}
\def\F{{\mathbb F}}
\def\N{{\mathbb N}}
\def\Z{{\mathbb Z}}
\def\R{{\mathbb R}}
\def\C{{\mathbb C}}
\def\Q{{\mathbb Q}}
\definecolor{DGray}{gray}{0.5}
\newcommand{\emailaddr}[1]{\mbox{$<${#1}$>$}}
\def\twiddle{\raisebox{0.3ex}{\mbox{\tiny $\sim$}}}
\def\gap{\vspace{0.5ex}}
\makeindex
\begin{document}
\frontmatter
\pagestyle{empty}
\title{LibTomMath User Manual \\ v1.1.0}
\author{LibTom Projects \\ www.libtom.net}
\maketitle
This text, the library and the accompanying textbook are all hereby placed in the public domain.  This book has been
formatted for B5 [176x250] paper using the \LaTeX{} {\em book} macro package.

\vspace{10cm}

\begin{flushright}Open Source.  Open Academia.  Open Minds.

\mbox{ }
LibTom Projects

\& originally

Tom St Denis,

Ontario, Canada
\end{flushright}

\tableofcontents
\listoffigures
\mainmatter
\pagestyle{headings}
\chapter{Introduction}
\section{What is LibTomMath?}
LibTomMath is a library of source code which provides a series of efficient and carefully written functions for manipulating
large integer numbers.  It was written in portable ISO C source code so that it will build on any platform with a conforming
C compiler.

In a nutshell the library was written from scratch with verbose comments to help instruct computer science students how
to implement ``bignum'' math.  However, the resulting code has proven to be very useful.  It has been used by numerous
universities, commercial and open source software developers.  It has been used on a variety of platforms ranging from
Linux and Windows based x86 to ARM based Gameboys and PPC based MacOS machines.

\section{License}
As of the v0.25 the library source code has been placed in the public domain with every new release.  As of the v0.28
release the textbook ``Implementing Multiple Precision Arithmetic'' has been placed in the public domain with every new
release as well.  This textbook is meant to compliment the project by providing a more solid walkthrough of the development
algorithms used in the library.

Since both\footnote{Note that the MPI files under mtest/ are copyrighted by Michael Fromberger.  They are not required to use LibTomMath.} are in the
public domain everyone is entitled to do with them as they see fit.

\section{Building LibTomMath}

LibTomMath is meant to be very ``GCC friendly'' as it comes with a makefile well suited for GCC.  However, the library will
also build in MSVC, Borland C out of the box.  For any other ISO C compiler a makefile will have to be made by the end
developer. Please consider to commit such a makefile to the LibTomMath developers, currently residing at
\url{http://github.com/libtom/libtommath}, if successfully done so.

Intel's C-compiler (ICC) is sufficiently compatible with GCC, at least the newer versions, to replace GCC for building the static and the shared library. Editing the makfiles is not needed, just set the shell variable \texttt{CC} as shown below.
\begin{alltt}
CC=/home/czurnieden/intel/bin/icc make
\end{alltt}

ICC does not know all options available for GCC and LibTomMath uses two diagnostics \texttt{-Wbad-function-cast} and \texttt{-Wcast-align} that are not supported by ICC resulting in the warnings:
\begin{alltt}
icc: command line warning #10148: option '-Wbad-function-cast' not supported
icc: command line warning #10148: option '-Wcast-align' not supported
\end{alltt}
It is possible to mute this ICC warning with the compiler flag \texttt{-diag-disable=10006}\footnote{It is not recommended to suppress warnings without a very good reason but there is no harm in doing so in this very special case.}.

\subsection{Static Libraries}
To build as a static library for GCC issue the following
\begin{alltt}
make
\end{alltt}

command.  This will build the library and archive the object files in ``libtommath.a''.  Now you link against
that and include ``tommath.h'' within your programs.  Alternatively to build with MSVC issue the following
\begin{alltt}
nmake -f makefile.msvc
\end{alltt}

This will build the library and archive the object files in ``tommath.lib''.  This has been tested with MSVC
version 6.00 with service pack 5.

\subsection{Shared Libraries}
\subsubsection{GNU based Operating Systems}
To build as a shared library for GCC issue the following
\begin{alltt}
make -f makefile.shared
\end{alltt}
This requires the ``libtool'' package (common on most Linux/BSD systems).  It will build LibTomMath as both shared
and static then install (by default) into /usr/lib as well as install the header files in /usr/include.  The shared
library (resource) will be called ``libtommath.la'' while the static library called ``libtommath.a''.  Generally
you use libtool to link your application against the shared object.
\subsubsection{Microsoft Windows based Operating Systems}
There is limited support for making a ``DLL'' in windows via the ``makefile.cygwin\_dll'' makefile.  It requires
Cygwin to work with since it requires the auto-export/import functionality.  The resulting DLL and import library
``libtommath.dll.a'' can be used to link LibTomMath dynamically to any Windows program using Cygwin.
\subsubsection{OpenBSD}
OpenBSD replaced some of their GNU-tools, especially \texttt{libtool} with their own, slightly different versions. To ease the workload of LibTomMath's developer team, only a static library can be build with the included \texttt{makefile.unix}.

The wrong \texttt{make} will result in errors like:
\begin{alltt}
*** Parse error in /home/user/GITHUB/libtommath: Need an operator in 'LIBNAME' )
*** Parse error: Need an operator in 'endif' (makefile.shared:8)
*** Parse error: Need an operator in 'CROSS_COMPILE' (makefile_include.mk:16)
*** Parse error: Need an operator in 'endif' (makefile_include.mk:18)
*** Parse error: Missing dependency operator (makefile_include.mk:22)
*** Parse error: Missing dependency operator (makefile_include.mk:23)
...
\end{alltt}
The wrong \texttt{libtool} will build it all fine but when it comes to the final linking fails with
\begin{alltt}
...
cc -I./ -Wall -Wsign-compare -Wextra -Wshadow -Wsystem-headers -Wdeclaration-afo...
cc -I./ -Wall -Wsign-compare -Wextra -Wshadow -Wsystem-headers -Wdeclaration-afo...
cc -I./ -Wall -Wsign-compare -Wextra -Wshadow -Wsystem-headers -Wdeclaration-afo...
libtool --mode=link --tag=CC cc  bn_error.lo bn_fast_mp_invmod.lo bn_fast_mp_mo 
libtool: link: cc bn_error.lo bn_fast_mp_invmod.lo bn_fast_mp_montgomery_reduce0
bn_error.lo: file not recognized: File format not recognized
cc: error: linker command failed with exit code 1 (use -v to see invocation)
Error while executing cc bn_error.lo bn_fast_mp_invmod.lo bn_fast_mp_montgomery0
gmake: *** [makefile.shared:64: libtommath.la] Error 1
\end{alltt}

To build a shared library with OpenBSD\footnote{Tested with OpenBSD version 6.4} the GNU versions of \texttt{make} and \texttt{libtool} are needed.
\begin{alltt}
$ sudo pkg_add gmake libtool
\end{alltt}
At this time two versions of \texttt{libtool} are installed and both are named \texttt{libtool}, unfortunately but GNU \texttt{libtool} has been placed in \texttt{/usr/local/bin/} and the native version in \texttt{/usr/bin/}. The path might be different in other versions of OpenBSD but both programms differ in the output of \texttt{libtool --version}
\begin{alltt}
$ /usr/local/bin/libtool --version                              
libtool (GNU libtool) 2.4.2
Written by Gordon Matzigkeit <gord@gnu.ai.mit.edu>, 1996

Copyright (C) 2011 Free Software Foundation, Inc.
This is free software; see the source for copying conditions.  There is NO
warranty; not even for MERCHANTABILITY or FITNESS FOR A PARTICULAR PURPOSE.
$ libtool --version
libtool (not (GNU libtool)) 1.5.26
\end{alltt}

The shared library should build now with
\begin{alltt}
LIBTOOL="/usr/local/bin/libtool" gmake -f makefile.shared
\end{alltt}
You might need to run a \texttt{gmake -f makefile.shared clean} first.

\subsubsection{NetBSD}
NetBSD is not as strict as OpenBSD but still needs \texttt{gmake} to build the shared library. \texttt{libtool} may also not exist in a fresh install.
\begin{alltt}
pkg_add gmake libtool
\end{alltt}
Please check with \texttt{libtool --version} that installed libtool is indeed a GNU libtool.
Build the shared library by typing:
\begin{alltt}
gmake -f makefile.shared
\end{alltt}

\subsection{Testing}
To build the library and the test harness type

\begin{alltt}
make test
\end{alltt}

This will build the library, ``test'' and ``mtest/mtest''.  The ``test'' program will accept test vectors and verify the
results.  ``mtest/mtest'' will generate test vectors using the MPI library by Michael Fromberger\footnote{A copy of MPI
is included in the package}.  Simply pipe mtest into test using

\begin{alltt}
mtest/mtest | test
\end{alltt}

If you do not have a ``/dev/urandom'' style RNG source you will have to write your own PRNG and simply pipe that into
mtest.  For example, if your PRNG program is called ``myprng'' simply invoke

\begin{alltt}
myprng | mtest/mtest | test
\end{alltt}

This will output a row of numbers that are increasing.  Each column is a different test (such as addition, multiplication, etc)
that is being performed.  The numbers represent how many times the test was invoked.  If an error is detected the program
will exit with a dump of the relevant numbers it was working with.

\section{Build Configuration}
LibTomMath can configured at build time in three phases we shall call ``depends'', ``tweaks'' and ``trims''.
Each phase changes how the library is built and they are applied one after another respectively.

To make the system more powerful you can tweak the build process.  Classes are defined in the file
``tommath\_superclass.h''.  By default, the symbol ``LTM\_ALL'' shall be defined which simply
instructs the system to build all of the functions.  This is how LibTomMath used to be packaged.  This will give you
access to every function LibTomMath offers.

However, there are cases where such a build is not optional.  For instance, you want to perform RSA operations.  You
don't need the vast majority of the library to perform these operations.  Aside from LTM\_ALL there is
another pre--defined class ``SC\_RSA\_1'' which works in conjunction with the RSA from LibTomCrypt.  Additional
classes can be defined base on the need of the user.

\subsection{Build Depends}
In the file tommath\_class.h you will see a large list of C ``defines'' followed by a series of ``ifdefs''
which further define symbols.  All of the symbols (technically they're macros $\ldots$) represent a given C source
file.  For instance, BN\_MP\_ADD\_C represents the file ``bn\_mp\_add.c''.  When a define has been enabled the
function in the respective file will be compiled and linked into the library.  Accordingly when the define
is absent the file will not be compiled and not contribute any size to the library.

You will also note that the header tommath\_class.h is actually recursively included (it includes itself twice).
This is to help resolve as many dependencies as possible.  In the last pass the symbol LTM\_LAST will be defined.
This is useful for ``trims''.

\subsection{Build Tweaks}
A tweak is an algorithm ``alternative''.  For example, to provide tradeoffs (usually between size and space).
They can be enabled at any pass of the configuration phase.

\begin{small}
\begin{center}
\begin{tabular}{|l|l|}
\hline \textbf{Define} & \textbf{Purpose} \\
\hline BN\_MP\_DIV\_SMALL & Enables a slower, smaller and equally \\
                          & functional mp\_div() function \\
\hline
\end{tabular}
\end{center}
\end{small}

\subsection{Build Trims}
A trim is a manner of removing functionality from a function that is not required.  For instance, to perform
RSA cryptography you only require exponentiation with odd moduli so even moduli support can be safely removed.
Build trims are meant to be defined on the last pass of the configuration which means they are to be defined
only if LTM\_LAST has been defined.

\subsubsection{Moduli Related}
\begin{small}
\begin{center}
\begin{tabular}{|l|l|}
\hline \textbf{Restriction} & \textbf{Undefine} \\
\hline Exponentiation with odd moduli only & BN\_S\_MP\_EXPTMOD\_C \\
                                           & BN\_MP\_REDUCE\_C \\
                                           & BN\_MP\_REDUCE\_SETUP\_C \\
                                           & BN\_S\_MP\_MUL\_HIGH\_DIGS\_C \\
                                           & BN\_FAST\_S\_MP\_MUL\_HIGH\_DIGS\_C \\
\hline Exponentiation with random odd moduli & (The above plus the following) \\
                                           & BN\_MP\_REDUCE\_2K\_C \\
                                           & BN\_MP\_REDUCE\_2K\_SETUP\_C \\
                                           & BN\_MP\_REDUCE\_IS\_2K\_C \\
                                           & BN\_MP\_DR\_IS\_MODULUS\_C \\
                                           & BN\_MP\_DR\_REDUCE\_C \\
                                           & BN\_MP\_DR\_SETUP\_C \\
\hline Modular inverse odd moduli only     & BN\_MP\_INVMOD\_SLOW\_C \\
\hline Modular inverse (both, smaller/slower) & BN\_FAST\_MP\_INVMOD\_C \\
\hline
\end{tabular}
\end{center}
\end{small}

\subsubsection{Operand Size Related}
\begin{small}
\begin{center}
\begin{tabular}{|l|l|}
\hline \textbf{Restriction} & \textbf{Undefine} \\
\hline Moduli $\le 2560$ bits              & BN\_MP\_MONTGOMERY\_REDUCE\_C \\
                                           & BN\_S\_MP\_MUL\_DIGS\_C \\
                                           & BN\_S\_MP\_MUL\_HIGH\_DIGS\_C \\
                                           & BN\_S\_MP\_SQR\_C \\
\hline Polynomial Schmolynomial            & BN\_MP\_KARATSUBA\_MUL\_C \\
                                           & BN\_MP\_KARATSUBA\_SQR\_C \\
                                           & BN\_MP\_TOOM\_MUL\_C \\
                                           & BN\_MP\_TOOM\_SQR\_C \\

\hline
\end{tabular}
\end{center}
\end{small}


\section{Purpose of LibTomMath}
Unlike  GNU MP (GMP) Library, LIP, OpenSSL or various other commercial kits (Miracl), LibTomMath was not written with
bleeding edge performance in mind.  First and foremost LibTomMath was written to be entirely open.  Not only is the
source code public domain (unlike various other GPL/etc licensed code), not only is the code freely downloadable but the
source code is also accessible for computer science students attempting to learn ``BigNum'' or multiple precision
arithmetic techniques.

LibTomMath was written to be an instructive collection of source code.  This is why there are many comments, only one
function per source file and often I use a ``middle-road'' approach where I don't cut corners for an extra 2\% speed
increase.

Source code alone cannot really teach how the algorithms work which is why I also wrote a textbook that accompanies
the library (beat that!).

So you may be thinking ``should I use LibTomMath?'' and the answer is a definite maybe.  Let me tabulate what I think
are the pros and cons of LibTomMath by comparing it to the math routines from GnuPG\footnote{GnuPG v1.2.3 versus LibTomMath v0.28}.

\newpage\begin{figure}[h]
\begin{small}
\begin{center}
\begin{tabular}{|l|c|c|l|}
\hline \textbf{Criteria} & \textbf{Pro} & \textbf{Con} & \textbf{Notes} \\
\hline Few lines of code per file & X & & GnuPG $ = 300.9$, LibTomMath  $ = 71.97$ \\
\hline Commented function prototypes & X && GnuPG function names are cryptic. \\
\hline Speed && X & LibTomMath is slower.  \\
\hline Totally free & X & & GPL has unfavourable restrictions.\\
\hline Large function base & X & & GnuPG is barebones. \\
\hline Five modular reduction algorithms & X & & Faster modular exponentiation for a variety of moduli. \\
\hline Portable & X & & GnuPG requires configuration to build. \\
\hline
\end{tabular}
\end{center}
\end{small}
\caption{LibTomMath Valuation}
\end{figure}

It may seem odd to compare LibTomMath to GnuPG since the math in GnuPG is only a small portion of the entire application.
However, LibTomMath was written with cryptography in mind.  It provides essentially all of the functions a cryptosystem
would require when working with large integers.

So it may feel tempting to just rip the math code out of GnuPG (or GnuMP where it was taken from originally) in your
own application but I think there are reasons not to.  While LibTomMath is slower than libraries such as GnuMP it is
not normally significantly slower.  On x86 machines the difference is normally a factor of two when performing modular
exponentiations.  It depends largely on the processor, compiler and the moduli being used.

Essentially the only time you wouldn't use LibTomMath is when blazing speed is the primary concern.  However,
on the other side of the coin LibTomMath offers you a totally free (public domain) well structured math library
that is very flexible, complete and performs well in resource constrained environments.  Fast RSA for example can
be performed with as little as 8KB of ram for data (again depending on build options).

\chapter{Getting Started with LibTomMath}
\section{Building Programs}
In order to use LibTomMath you must include ``tommath.h'' and link against the appropriate library file (typically
libtommath.a).  There is no library initialization required and the entire library is thread safe.

\section{Return Codes}
There are three possible return codes a function may return.

\index{MP\_OKAY}\index{MP\_YES}\index{MP\_NO}\index{MP\_VAL}\index{MP\_MEM}
\begin{figure}[h!]
\begin{center}
\begin{small}
\begin{tabular}{|l|l|}
\hline \textbf{Code} & \textbf{Meaning} \\
\hline MP\_OKAY & The function succeeded. \\
\hline MP\_VAL  & The function input was invalid. \\
\hline MP\_MEM  & Heap memory exhausted. \\
\hline &\\
\hline MP\_YES  & Response is yes. \\
\hline MP\_NO   & Response is no. \\
\hline
\end{tabular}
\end{small}
\end{center}
\caption{Return Codes}
\end{figure}

The last two codes listed are not actually ``return'ed'' by a function.  They are placed in an integer (the caller must
provide the address of an integer it can store to) which the caller can access.  To convert one of the three return codes
to a string use the following function.

\index{mp\_error\_to\_string}
\begin{alltt}
char *mp_error_to_string(int code);
\end{alltt}

This will return a pointer to a string which describes the given error code.  It will not work for the return codes
MP\_YES and MP\_NO.

\section{Data Types}
The basic ``multiple precision integer'' type is known as the ``mp\_int'' within LibTomMath.  This data type is used to
organize all of the data required to manipulate the integer it represents.  Within LibTomMath it has been prototyped
as the following.

\index{mp\_int}
\begin{alltt}
typedef struct  \{
    int used, alloc, sign;
    mp_digit *dp;
\} mp_int;
\end{alltt}

Where ``mp\_digit'' is a data type that represents individual digits of the integer.  By default, an mp\_digit is the
ISO C ``unsigned long'' data type and each digit is $28-$bits long.  The mp\_digit type can be configured to suit other
platforms by defining the appropriate macros.

All LTM functions that use the mp\_int type will expect a pointer to mp\_int structure.  You must allocate memory to
hold the structure itself by yourself (whether off stack or heap it doesn't matter).  The very first thing that must be
done to use an mp\_int is that it must be initialized.

\section{Function Organization}

The arithmetic functions of the library are all organized to have the same style prototype.  That is source operands
are passed on the left and the destination is on the right.  For instance,

\begin{alltt}
mp_add(&a, &b, &c);       /* c = a + b */
mp_mul(&a, &a, &c);       /* c = a * a */
mp_div(&a, &b, &c, &d);   /* c = [a/b], d = a mod b */
\end{alltt}

Another feature of the way the functions have been implemented is that source operands can be destination operands as well.
For instance,

\begin{alltt}
mp_add(&a, &b, &b);       /* b = a + b */
mp_div(&a, &b, &a, &c);   /* a = [a/b], c = a mod b */
\end{alltt}

This allows operands to be re-used which can make programming simpler.

\section{Initialization}
\subsection{Single Initialization}
A single mp\_int can be initialized with the ``mp\_init'' function.

\index{mp\_init}
\begin{alltt}
int mp_init (mp_int * a);
\end{alltt}

This function expects a pointer to an mp\_int structure and will initialize the members of the structure so the mp\_int
represents the default integer which is zero.  If the functions returns MP\_OKAY then the mp\_int is ready to be used
by the other LibTomMath functions.

\begin{small} \begin{alltt}
int main(void)
\{
   mp_int number;
   int result;

   if ((result = mp_init(&number)) != MP_OKAY) \{
      printf("Error initializing the number.  \%s",
             mp_error_to_string(result));
      return EXIT_FAILURE;
   \}

   /* use the number */

   return EXIT_SUCCESS;
\}
\end{alltt} \end{small}

\subsection{Single Free}
When you are finished with an mp\_int it is ideal to return the heap it used back to the system.  The following function
provides this functionality.

\index{mp\_clear}
\begin{alltt}
void mp_clear (mp_int * a);
\end{alltt}

The function expects a pointer to a previously initialized mp\_int structure and frees the heap it uses.  It sets the
pointer\footnote{The ``dp'' member.} within the mp\_int to \textbf{NULL} which is used to prevent double free situations.
Is is legal to call mp\_clear() twice on the same mp\_int in a row.

\begin{small} \begin{alltt}
int main(void)
\{
   mp_int number;
   int result;

   if ((result = mp_init(&number)) != MP_OKAY) \{
      printf("Error initializing the number.  \%s",
             mp_error_to_string(result));
      return EXIT_FAILURE;
   \}

   /* use the number */

   /* We're done with it. */
   mp_clear(&number);

   return EXIT_SUCCESS;
\}
\end{alltt} \end{small}

\subsection{Multiple Initializations}
Certain algorithms require more than one large integer.  In these instances it is ideal to initialize all of the mp\_int
variables in an ``all or nothing'' fashion.  That is, they are either all initialized successfully or they are all
not initialized.

The  mp\_init\_multi() function provides this functionality.

\index{mp\_init\_multi} \index{mp\_clear\_multi}
\begin{alltt}
int mp_init_multi(mp_int *mp, ...);
\end{alltt}

It accepts a \textbf{NULL} terminated list of pointers to mp\_int structures.  It will attempt to initialize them all
at once.  If the function returns MP\_OKAY then all of the mp\_int variables are ready to use, otherwise none of them
are available for use.  A complementary mp\_clear\_multi() function allows multiple mp\_int variables to be free'd
from the heap at the same time.

\begin{small} \begin{alltt}
int main(void)
\{
   mp_int num1, num2, num3;
   int result;

   if ((result = mp_init_multi(&num1,
                               &num2,
                               &num3, NULL)) != MP\_OKAY) \{
      printf("Error initializing the numbers.  \%s",
             mp_error_to_string(result));
      return EXIT_FAILURE;
   \}

   /* use the numbers */

   /* We're done with them. */
   mp_clear_multi(&num1, &num2, &num3, NULL);

   return EXIT_SUCCESS;
\}
\end{alltt} \end{small}

\subsection{Other Initializers}
To initialized and make a copy of an mp\_int the mp\_init\_copy() function has been provided.

\index{mp\_init\_copy}
\begin{alltt}
int mp_init_copy (mp_int * a, mp_int * b);
\end{alltt}

This function will initialize $a$ and make it a copy of $b$ if all goes well.

\begin{small} \begin{alltt}
int main(void)
\{
   mp_int num1, num2;
   int result;

   /* initialize and do work on num1 ... */

   /* We want a copy of num1 in num2 now */
   if ((result = mp_init_copy(&num2, &num1)) != MP_OKAY) \{
     printf("Error initializing the copy.  \%s",
             mp_error_to_string(result));
      return EXIT_FAILURE;
   \}

   /* now num2 is ready and contains a copy of num1 */

   /* We're done with them. */
   mp_clear_multi(&num1, &num2, NULL);

   return EXIT_SUCCESS;
\}
\end{alltt} \end{small}

Another less common initializer is mp\_init\_size() which allows the user to initialize an mp\_int with a given
default number of digits.  By default, all initializers allocate \textbf{MP\_PREC} digits.  This function lets
you override this behaviour.

\index{mp\_init\_size}
\begin{alltt}
int mp_init_size (mp_int * a, int size);
\end{alltt}

The $size$ parameter must be greater than zero.  If the function succeeds the mp\_int $a$ will be initialized
to have $size$ digits (which are all initially zero).

\begin{small} \begin{alltt}
int main(void)
\{
   mp_int number;
   int result;

   /* we need a 60-digit number */
   if ((result = mp_init_size(&number, 60)) != MP_OKAY) \{
      printf("Error initializing the number.  \%s",
             mp_error_to_string(result));
      return EXIT_FAILURE;
   \}

   /* use the number */

   return EXIT_SUCCESS;
\}
\end{alltt} \end{small}

\section{Maintenance Functions}
\subsection{Clear Leading Zeros}

This is used to ensure that leading zero digits are trimed and the leading "used" digit will be non-zero.
It also fixes the sign if there are no more leading digits.

\index{mp\_clamp}
\begin{alltt}
void mp_clamp(mp_int *a);
\end{alltt}

\subsection{Zero Out}

This function will set the ``bigint'' to zeros without changing the amount of allocated memory.

\index{mp\_zero}
\begin{alltt}
void mp_zero(mp_int *a);
\end{alltt}


\subsection{Reducing Memory Usage}
When an mp\_int is in a state where it won't be changed again\footnote{A Diffie-Hellman modulus for instance.} excess
digits can be removed to return memory to the heap with the mp\_shrink() function.

\index{mp\_shrink}
\begin{alltt}
int mp_shrink (mp_int * a);
\end{alltt}

This will remove excess digits of the mp\_int $a$.  If the operation fails the mp\_int should be intact without the
excess digits being removed.  Note that you can use a shrunk mp\_int in further computations, however, such operations
will require heap operations which can be slow.  It is not ideal to shrink mp\_int variables that you will further
modify in the system (unless you are seriously low on memory).

\begin{small} \begin{alltt}
int main(void)
\{
   mp_int number;
   int result;

   if ((result = mp_init(&number)) != MP_OKAY) \{
      printf("Error initializing the number.  \%s",
             mp_error_to_string(result));
      return EXIT_FAILURE;
   \}

   /* use the number [e.g. pre-computation]  */

   /* We're done with it for now. */
   if ((result = mp_shrink(&number)) != MP_OKAY) \{
      printf("Error shrinking the number.  \%s",
             mp_error_to_string(result));
      return EXIT_FAILURE;
   \}

   /* use it .... */


   /* we're done with it. */
   mp_clear(&number);

   return EXIT_SUCCESS;
\}
\end{alltt} \end{small}

\subsection{Adding additional digits}

Within the mp\_int structure are two parameters which control the limitations of the array of digits that represent
the integer the mp\_int is meant to equal.   The \textit{used} parameter dictates how many digits are significant, that is,
contribute to the value of the mp\_int.  The \textit{alloc} parameter dictates how many digits are currently available in
the array.  If you need to perform an operation that requires more digits you will have to mp\_grow() the mp\_int to
your desired size.

\index{mp\_grow}
\begin{alltt}
int mp_grow (mp_int * a, int size);
\end{alltt}

This will grow the array of digits of $a$ to $size$.  If the \textit{alloc} parameter is already bigger than
$size$ the function will not do anything.

\begin{small} \begin{alltt}
int main(void)
\{
   mp_int number;
   int result;

   if ((result = mp_init(&number)) != MP_OKAY) \{
      printf("Error initializing the number.  \%s",
             mp_error_to_string(result));
      return EXIT_FAILURE;
   \}

   /* use the number */

   /* We need to add 20 digits to the number  */
   if ((result = mp_grow(&number, number.alloc + 20)) != MP_OKAY) \{
      printf("Error growing the number.  \%s",
             mp_error_to_string(result));
      return EXIT_FAILURE;
   \}


   /* use the number */

   /* we're done with it. */
   mp_clear(&number);

   return EXIT_SUCCESS;
\}
\end{alltt} \end{small}

\chapter{Basic Operations}
\section{Copying}

A so called ``deep copy'', where new memory is allocated and all contents of $a$ are copied verbatim into $b$ such that $b = a$ at the end.

\index{mp\_copy}
\begin{alltt}
int mp_copy (mp_int * a, mp_int *b);
\end{alltt}

You can also just swap $a$ and $b$. It does the normal pointer changing with a temporary pointer variable, just that you do not have to.

\index{mp\_exch}
\begin{alltt}
void mp_exch (mp_int * a, mp_int *b);
\end{alltt}

\section{Bit Counting}

To get the position of the lowest bit set (LSB, the Lowest Significant Bit; the number of bits which are zero before the first zero bit )

\index{mp\_cnt\_lsb}
\begin{alltt}
int mp_cnt_lsb(const mp_int *a);
\end{alltt}

To get the position of the highest bit set (MSB, the Most Significant Bit; the number of bits in teh ``bignum'')

\index{mp\_count\_bits}
\begin{alltt}
int mp_count_bits(const mp_int *a);
\end{alltt}


\section{Small Constants}
Setting mp\_ints to small constants is a relatively common operation.  To accommodate these instances there are two
small constant assignment functions.  The first function is used to set a single digit constant while the second sets
an ISO C style ``unsigned long'' constant.  The reason for both functions is efficiency.  Setting a single digit is quick but the
domain of a digit can change (it's always at least $0 \ldots 127$).

\subsection{Single Digit}

Setting a single digit can be accomplished with the following function.

\index{mp\_set}
\begin{alltt}
void mp_set (mp_int * a, mp_digit b);
\end{alltt}

This will zero the contents of $a$ and make it represent an integer equal to the value of $b$.  Note that this
function has a return type of \textbf{void}.  It cannot cause an error so it is safe to assume the function
succeeded.

\begin{small} \begin{alltt}
int main(void)
\{
   mp_int number;
   int result;

   if ((result = mp_init(&number)) != MP_OKAY) \{
      printf("Error initializing the number.  \%s",
             mp_error_to_string(result));
      return EXIT_FAILURE;
   \}

   /* set the number to 5 */
   mp_set(&number, 5);

   /* we're done with it. */
   mp_clear(&number);

   return EXIT_SUCCESS;
\}
\end{alltt} \end{small}

\subsection{Long Constants}

To set a constant that is the size of an ISO C ``unsigned long'' and larger than a single digit the following function
can be used.

\index{mp\_set\_int}
\begin{alltt}
int mp_set_int (mp_int * a, unsigned long b);
\end{alltt}

This will assign the value of the 32-bit variable $b$ to the mp\_int $a$.  Unlike mp\_set() this function will always
accept a 32-bit input regardless of the size of a single digit.  However, since the value may span several digits
this function can fail if it runs out of heap memory.

To get the ``unsigned long'' copy of an mp\_int the following function can be used.

\index{mp\_get\_int}
\begin{alltt}
unsigned long mp_get_int (mp_int * a);
\end{alltt}

This will return the 32 least significant bits of the mp\_int $a$.

\begin{small} \begin{alltt}
int main(void)
\{
   mp_int number;
   int result;

   if ((result = mp_init(&number)) != MP_OKAY) \{
      printf("Error initializing the number.  \%s",
             mp_error_to_string(result));
      return EXIT_FAILURE;
   \}

   /* set the number to 654321 (note this is bigger than 127) */
   if ((result = mp_set_int(&number, 654321)) != MP_OKAY) \{
      printf("Error setting the value of the number.  \%s",
             mp_error_to_string(result));
      return EXIT_FAILURE;
   \}

   printf("number == \%lu", mp_get_int(&number));

   /* we're done with it. */
   mp_clear(&number);

   return EXIT_SUCCESS;
\}
\end{alltt} \end{small}

This should output the following if the program succeeds.

\begin{alltt}
number == 654321
\end{alltt}

\subsection{Long Constants - platform dependant}

\index{mp\_set\_long}
\begin{alltt}
int mp_set_long (mp_int * a, unsigned long b);
\end{alltt}

This will assign the value of the platform-dependent sized variable $b$ to the mp\_int $a$.

To get the ``unsigned long'' copy of an mp\_int the following function can be used.

\index{mp\_get\_long}
\begin{alltt}
unsigned long mp_get_long (mp_int * a);
\end{alltt}

This will return the least significant bits of the mp\_int $a$ that fit into an ``unsigned long''.

\subsection{Long Long Constants}

\index{mp\_set\_long\_long}
\begin{alltt}
int mp_set_long_long (mp_int * a, unsigned long long b);
\end{alltt}

This will assign the value of the 64-bit variable $b$ to the mp\_int $a$.

To get the ``unsigned long long'' copy of an mp\_int the following function can be used.

\index{mp\_get\_long\_long}
\begin{alltt}
unsigned long long mp_get_long_long (mp_int * a);
\end{alltt}

This will return the 64 least significant bits of the mp\_int $a$.

\subsection{Initialize and Setting Constants}
To both initialize and set small constants the following two functions are available.
\index{mp\_init\_set} \index{mp\_init\_set\_int}
\begin{alltt}
int mp_init_set (mp_int * a, mp_digit b);
int mp_init_set_int (mp_int * a, unsigned long b);
\end{alltt}

Both functions work like the previous counterparts except they first mp\_init $a$ before setting the values.

\begin{alltt}
int main(void)
\{
   mp_int number1, number2;
   int    result;

   /* initialize and set a single digit */
   if ((result = mp_init_set(&number1, 100)) != MP_OKAY) \{
      printf("Error setting number1: \%s",
             mp_error_to_string(result));
      return EXIT_FAILURE;
   \}

   /* initialize and set a long */
   if ((result = mp_init_set_int(&number2, 1023)) != MP_OKAY) \{
      printf("Error setting number2: \%s",
             mp_error_to_string(result));
      return EXIT_FAILURE;
   \}

   /* display */
   printf("Number1, Number2 == \%lu, \%lu",
          mp_get_int(&number1), mp_get_int(&number2));

   /* clear */
   mp_clear_multi(&number1, &number2, NULL);

   return EXIT_SUCCESS;
\}
\end{alltt}

If this program succeeds it shall output.
\begin{alltt}
Number1, Number2 == 100, 1023
\end{alltt}

\section{Comparisons}

Comparisons in LibTomMath are always performed in a ``left to right'' fashion.  There are three possible return codes
for any comparison.

\index{MP\_GT} \index{MP\_EQ} \index{MP\_LT}
\begin{figure}[h]
\begin{center}
\begin{tabular}{|c|c|}
\hline \textbf{Result Code} & \textbf{Meaning} \\
\hline MP\_GT & $a > b$ \\
\hline MP\_EQ & $a = b$ \\
\hline MP\_LT & $a < b$ \\
\hline
\end{tabular}
\end{center}
\caption{Comparison Codes for $a, b$}
\label{fig:CMP}
\end{figure}

In figure \ref{fig:CMP} two integers $a$ and $b$ are being compared.  In this case $a$ is said to be ``to the left'' of
$b$.

\subsection{Unsigned comparison}

An unsigned comparison considers only the digits themselves and not the associated \textit{sign} flag of the
mp\_int structures.  This is analogous to an absolute comparison.  The function mp\_cmp\_mag() will compare two
mp\_int variables based on their digits only.

\index{mp\_cmp\_mag}
\begin{alltt}
int mp_cmp_mag(mp_int * a, mp_int * b);
\end{alltt}
This will compare $a$ to $b$ placing $a$ to the left of $b$.  This function cannot fail and will return one of the
three compare codes listed in figure \ref{fig:CMP}.

\begin{small} \begin{alltt}
int main(void)
\{
   mp_int number1, number2;
   int result;

   if ((result = mp_init_multi(&number1, &number2, NULL)) != MP_OKAY) \{
      printf("Error initializing the numbers.  \%s",
             mp_error_to_string(result));
      return EXIT_FAILURE;
   \}

   /* set the number1 to 5 */
   mp_set(&number1, 5);

   /* set the number2 to -6 */
   mp_set(&number2, 6);
   if ((result = mp_neg(&number2, &number2)) != MP_OKAY) \{
      printf("Error negating number2.  \%s",
             mp_error_to_string(result));
      return EXIT_FAILURE;
   \}

   switch(mp_cmp_mag(&number1, &number2)) \{
       case MP_GT:  printf("|number1| > |number2|"); break;
       case MP_EQ:  printf("|number1| = |number2|"); break;
       case MP_LT:  printf("|number1| < |number2|"); break;
   \}

   /* we're done with it. */
   mp_clear_multi(&number1, &number2, NULL);

   return EXIT_SUCCESS;
\}
\end{alltt} \end{small}

If this program\footnote{This function uses the mp\_neg() function which is discussed in section \ref{sec:NEG}.} completes
successfully it should print the following.

\begin{alltt}
|number1| < |number2|
\end{alltt}

This is because $\vert -6 \vert = 6$ and obviously $5 < 6$.

\subsection{Signed comparison}

To compare two mp\_int variables based on their signed value the mp\_cmp() function is provided.

\index{mp\_cmp}
\begin{alltt}
int mp_cmp(mp_int * a, mp_int * b);
\end{alltt}

This will compare $a$ to the left of $b$.  It will first compare the signs of the two mp\_int variables.  If they
differ it will return immediately based on their signs.  If the signs are equal then it will compare the digits
individually.  This function will return one of the compare conditions codes listed in figure \ref{fig:CMP}.

\begin{small} \begin{alltt}
int main(void)
\{
   mp_int number1, number2;
   int result;

   if ((result = mp_init_multi(&number1, &number2, NULL)) != MP_OKAY) \{
      printf("Error initializing the numbers.  \%s",
             mp_error_to_string(result));
      return EXIT_FAILURE;
   \}

   /* set the number1 to 5 */
   mp_set(&number1, 5);

   /* set the number2 to -6 */
   mp_set(&number2, 6);
   if ((result = mp_neg(&number2, &number2)) != MP_OKAY) \{
      printf("Error negating number2.  \%s",
             mp_error_to_string(result));
      return EXIT_FAILURE;
   \}

   switch(mp_cmp(&number1, &number2)) \{
       case MP_GT:  printf("number1 > number2"); break;
       case MP_EQ:  printf("number1 = number2"); break;
       case MP_LT:  printf("number1 < number2"); break;
   \}

   /* we're done with it. */
   mp_clear_multi(&number1, &number2, NULL);

   return EXIT_SUCCESS;
\}
\end{alltt} \end{small}

If this program\footnote{This function uses the mp\_neg() function which is discussed in section \ref{sec:NEG}.} completes
successfully it should print the following.

\begin{alltt}
number1 > number2
\end{alltt}

\subsection{Single Digit}

To compare a single digit against an mp\_int the following function has been provided.

\index{mp\_cmp\_d}
\begin{alltt}
int mp_cmp_d(mp_int * a, mp_digit b);
\end{alltt}

This will compare $a$ to the left of $b$ using a signed comparison.  Note that it will always treat $b$ as
positive.  This function is rather handy when you have to compare against small values such as $1$ (which often
comes up in cryptography).  The function cannot fail and will return one of the tree compare condition codes
listed in figure \ref{fig:CMP}.


\begin{small} \begin{alltt}
int main(void)
\{
   mp_int number;
   int result;

   if ((result = mp_init(&number)) != MP_OKAY) \{
      printf("Error initializing the number.  \%s",
             mp_error_to_string(result));
      return EXIT_FAILURE;
   \}

   /* set the number to 5 */
   mp_set(&number, 5);

   switch(mp_cmp_d(&number, 7)) \{
       case MP_GT:  printf("number > 7"); break;
       case MP_EQ:  printf("number = 7"); break;
       case MP_LT:  printf("number < 7"); break;
   \}

   /* we're done with it. */
   mp_clear(&number);

   return EXIT_SUCCESS;
\}
\end{alltt} \end{small}

If this program functions properly it will print out the following.

\begin{alltt}
number < 7
\end{alltt}

\section{Logical Operations}

Logical operations are operations that can be performed either with simple shifts or boolean operators such as
AND, XOR and OR directly.  These operations are very quick.

\subsection{Multiplication by two}

Multiplications and divisions by any power of two can be performed with quick logical shifts either left or
right depending on the operation.

When multiplying or dividing by two a special case routine can be used which are as follows.
\index{mp\_mul\_2} \index{mp\_div\_2}
\begin{alltt}
int mp_mul_2(mp_int * a, mp_int * b);
int mp_div_2(mp_int * a, mp_int * b);
\end{alltt}

The former will assign twice $a$ to $b$ while the latter will assign half $a$ to $b$.  These functions are fast
since the shift counts and maskes are hardcoded into the routines.

\begin{small} \begin{alltt}
int main(void)
\{
   mp_int number;
   int result;

   if ((result = mp_init(&number)) != MP_OKAY) \{
      printf("Error initializing the number.  \%s",
             mp_error_to_string(result));
      return EXIT_FAILURE;
   \}

   /* set the number to 5 */
   mp_set(&number, 5);

   /* multiply by two */
   if ((result = mp\_mul\_2(&number, &number)) != MP_OKAY) \{
      printf("Error multiplying the number.  \%s",
             mp_error_to_string(result));
      return EXIT_FAILURE;
   \}
   switch(mp_cmp_d(&number, 7)) \{
       case MP_GT:  printf("2*number > 7"); break;
       case MP_EQ:  printf("2*number = 7"); break;
       case MP_LT:  printf("2*number < 7"); break;
   \}

   /* now divide by two */
   if ((result = mp\_div\_2(&number, &number)) != MP_OKAY) \{
      printf("Error dividing the number.  \%s",
             mp_error_to_string(result));
      return EXIT_FAILURE;
   \}
   switch(mp_cmp_d(&number, 7)) \{
       case MP_GT:  printf("2*number/2 > 7"); break;
       case MP_EQ:  printf("2*number/2 = 7"); break;
       case MP_LT:  printf("2*number/2 < 7"); break;
   \}

   /* we're done with it. */
   mp_clear(&number);

   return EXIT_SUCCESS;
\}
\end{alltt} \end{small}

If this program is successful it will print out the following text.

\begin{alltt}
2*number > 7
2*number/2 < 7
\end{alltt}

Since $10 > 7$ and $5 < 7$.

To multiply by a power of two the following function can be used.

\index{mp\_mul\_2d}
\begin{alltt}
int mp_mul_2d(mp_int * a, int b, mp_int * c);
\end{alltt}

This will multiply $a$ by $2^b$ and store the result in ``c''.  If the value of $b$ is less than or equal to
zero the function will copy $a$ to ``c'' without performing any further actions.  The multiplication itself
is implemented as a right-shift operation of $a$ by $b$ bits.

To divide by a power of two use the following.

\index{mp\_div\_2d}
\begin{alltt}
int mp_div_2d (mp_int * a, int b, mp_int * c, mp_int * d);
\end{alltt}
Which will divide $a$ by $2^b$, store the quotient in ``c'' and the remainder in ``d'.  If $b \le 0$ then the
function simply copies $a$ over to ``c'' and zeroes $d$.  The variable $d$ may be passed as a \textbf{NULL}
value to signal that the remainder is not desired.  The division itself is implemented as a left-shift
operation of $a$ by $b$ bits.

\index{mp\_tc\_div\_2d}\label{arithrightshift}
\begin{alltt}
int mp_tc_div_2d (mp_int * a, int b, mp_int * c, mp_int * d);
\end{alltt}
The two-co,mplement version of the function above. This can be used to implement arbitrary-precision two-complement integers together with the two-complement bit-wise operations at page \ref{tcbitwiseops}.


It is also not very uncommon to need just the power of two $2^b$;  for example the startvalue for the Newton method.

\index{mp\_2expt}
\begin{alltt}
int mp_2expt(mp_int *a, int b);
\end{alltt}
It is faster than doing it by shifting $1$ with \texttt{mp\_mul\_2d}.

\subsection{Polynomial Basis Operations}

Strictly speaking the organization of the integers within the mp\_int structures is what is known as a
``polynomial basis''.  This simply means a field element is stored by divisions of a radix.  For example, if
$f(x) = \sum_{i=0}^{k} y_ix^k$ for any vector $\vec y$ then the array of digits in $\vec y$ are said to be
the polynomial basis representation of $z$ if $f(\beta) = z$ for a given radix $\beta$.

To multiply by the polynomial $g(x) = x$ all you have todo is shift the digits of the basis left one place.  The
following function provides this operation.

\index{mp\_lshd}
\begin{alltt}
int mp_lshd (mp_int * a, int b);
\end{alltt}

This will multiply $a$ in place by $x^b$ which is equivalent to shifting the digits left $b$ places and inserting zeroes
in the least significant digits.  Similarly to divide by a power of $x$ the following function is provided.

\index{mp\_rshd}
\begin{alltt}
void mp_rshd (mp_int * a, int b)
\end{alltt}
This will divide $a$ in place by $x^b$ and discard the remainder.  This function cannot fail as it performs the operations
in place and no new digits are required to complete it.

\subsection{AND, OR, XOR and COMPLEMENT Operations}

While AND, OR and XOR operations are not typical ``bignum functions'' they can be useful in several instances.  The
three functions are prototyped as follows.

\index{mp\_or} \index{mp\_and} \index{mp\_xor}
\begin{alltt}
int mp_or  (mp_int * a, mp_int * b, mp_int * c);
int mp_and (mp_int * a, mp_int * b, mp_int * c);
int mp_xor (mp_int * a, mp_int * b, mp_int * c);
\end{alltt}

Which compute $c = a \odot b$ where $\odot$ is one of OR, AND or XOR.

The following four functions allow implementing arbitrary-precision two-complement numbers.

\index{mp\_tc\_or} \index{mp\_tc\_and} \index{mp\_tc\_xor} \index{mp\_complement} \label{tcbitwiseops}
\begin{alltt}
int mp_tc_or  (mp_int * a, mp_int * b, mp_int * c);
int mp_tc_and (mp_int * a, mp_int * b, mp_int * c);
int mp_tc_xor (mp_int * a, mp_int * b, mp_int * c);
int mp_complement(const mp_int *a, mp_int *b);
\end{alltt}

They compute $c = a \odot b$ as above if both $a$ and $b$ are positive. Negative values are converted into their two-complement representations first. The function \texttt{mp\_complement} computes a two-complement $b = \sim a$.


\subsection{Bit Picking}
\index{mp\_get\_bit}
\begin{alltt}
int mp_get_bit(mp_int *a, int b)
\end{alltt}

Pick a bit: returns \texttt{MP\_YES} if the bit at position $b$ (0-index) is set, that is if it is 1 (one), \texttt{MP\_NO}
if the bit is 0 (zero) and \texttt{MP\_VAL} if $b < 0$.

\section{Addition and Subtraction}

To compute an addition or subtraction the following two functions can be used.

\index{mp\_add} \index{mp\_sub}
\begin{alltt}
int mp_add (mp_int * a, mp_int * b, mp_int * c);
int mp_sub (mp_int * a, mp_int * b, mp_int * c)
\end{alltt}

Which perform $c = a \odot b$ where $\odot$ is one of signed addition or subtraction.  The operations are fully sign
aware.

\section{Sign Manipulation}
\subsection{Negation}
\label{sec:NEG}
Simple integer negation can be performed with the following.

\index{mp\_neg}
\begin{alltt}
int mp_neg (mp_int * a, mp_int * b);
\end{alltt}

Which assigns $-a$ to $b$.

\subsection{Absolute}
Simple integer absolutes can be performed with the following.

\index{mp\_abs}
\begin{alltt}
int mp_abs (mp_int * a, mp_int * b);
\end{alltt}

Which assigns $\vert a \vert$ to $b$.

\section{Integer Division and Remainder}
To perform a complete and general integer division with remainder use the following function.

\index{mp\_div}
\begin{alltt}
int mp_div (mp_int * a, mp_int * b, mp_int * c, mp_int * d);
\end{alltt}

This divides $a$ by $b$ and stores the quotient in $c$ and $d$.  The signed quotient is computed such that
$bc + d = a$.  Note that either of $c$ or $d$ can be set to \textbf{NULL} if their value is not required.  If
$b$ is zero the function returns \textbf{MP\_VAL}.


\chapter{Multiplication and Squaring}
\section{Multiplication}
A full signed integer multiplication can be performed with the following.
\index{mp\_mul}
\begin{alltt}
int mp_mul (mp_int * a, mp_int * b, mp_int * c);
\end{alltt}
Which assigns the full signed product $ab$ to $c$.  This function actually breaks into one of four cases which are
specific multiplication routines optimized for given parameters.  First there are the Toom-Cook multiplications which
should only be used with very large inputs.  This is followed by the Karatsuba multiplications which are for moderate
sized inputs.  Then followed by the Comba and baseline multipliers.

Fortunately for the developer you don't really need to know this unless you really want to fine tune the system.  mp\_mul()
will determine on its own\footnote{Some tweaking may be required.} what routine to use automatically when it is called.

\begin{alltt}
int main(void)
\{
   mp_int number1, number2;
   int result;

   /* Initialize the numbers */
   if ((result = mp_init_multi(&number1,
                               &number2, NULL)) != MP_OKAY) \{
      printf("Error initializing the numbers.  \%s",
             mp_error_to_string(result));
      return EXIT_FAILURE;
   \}

   /* set the terms */
   if ((result = mp_set_int(&number, 257)) != MP_OKAY) \{
      printf("Error setting number1.  \%s",
             mp_error_to_string(result));
      return EXIT_FAILURE;
   \}

   if ((result = mp_set_int(&number2, 1023)) != MP_OKAY) \{
      printf("Error setting number2.  \%s",
             mp_error_to_string(result));
      return EXIT_FAILURE;
   \}

   /* multiply them */
   if ((result = mp_mul(&number1, &number2,
                        &number1)) != MP_OKAY) \{
      printf("Error multiplying terms.  \%s",
             mp_error_to_string(result));
      return EXIT_FAILURE;
   \}

   /* display */
   printf("number1 * number2 == \%lu", mp_get_int(&number1));

   /* free terms and return */
   mp_clear_multi(&number1, &number2, NULL);

   return EXIT_SUCCESS;
\}
\end{alltt}

If this program succeeds it shall output the following.

\begin{alltt}
number1 * number2 == 262911
\end{alltt}

\section{Squaring}
Since squaring can be performed faster than multiplication it is performed it's own function instead of just using
mp\_mul().

\index{mp\_sqr}
\begin{alltt}
int mp_sqr (mp_int * a, mp_int * b);
\end{alltt}

Will square $a$ and store it in $b$.  Like the case of multiplication there are four different squaring
algorithms all which can be called from mp\_sqr().  It is ideal to use mp\_sqr over mp\_mul when squaring terms because
of the speed difference.

\section{Tuning Polynomial Basis Routines}

Both of the Toom-Cook and Karatsuba multiplication algorithms are faster than the traditional $O(n^2)$ approach that
the Comba and baseline algorithms use.  At $O(n^{1.464973})$ and $O(n^{1.584962})$ running times respectively they require
considerably less work.  For example, a 10000-digit multiplication would take roughly 724,000 single precision
multiplications with Toom-Cook or 100,000,000 single precision multiplications with the standard Comba (a factor
of 138).

So why not always use Karatsuba or Toom-Cook?   The simple answer is that they have so much overhead that they're not
actually faster than Comba until you hit distinct  ``cutoff'' points.  For Karatsuba with the default configuration,
GCC 3.3.1 and an Athlon XP processor the cutoff point is roughly 110 digits (about 70 for the Intel P4).  That is, at
110 digits Karatsuba and Comba multiplications just about break even and for 110+ digits Karatsuba is faster.

Toom-Cook has incredible overhead and is probably only useful for very large inputs.  So far no known cutoff points
exist and for the most part I just set the cutoff points very high to make sure they're not called.

A demo program in the ``etc/'' directory of the project called ``tune.c'' can be used to find the cutoff points.  This
can be built with GCC as follows

\begin{alltt}
make XXX
\end{alltt}
Where ``XXX'' is one of the following entries from the table \ref{fig:tuning}.

\begin{figure}[h]
\begin{center}
\begin{small}
\begin{tabular}{|l|l|}
\hline \textbf{Value of XXX} & \textbf{Meaning} \\
\hline tune & Builds portable tuning application \\
\hline tune86 & Builds x86 (pentium and up) program for COFF \\
\hline tune86c & Builds x86 program for Cygwin \\
\hline tune86l & Builds x86 program for Linux (ELF format) \\
\hline
\end{tabular}
\end{small}
\end{center}
\caption{Build Names for Tuning Programs}
\label{fig:tuning}
\end{figure}

When the program is running it will output a series of measurements for different cutoff points.  It will first find
good Karatsuba squaring and multiplication points.  Then it proceeds to find Toom-Cook points.  Note that the Toom-Cook
tuning takes a very long time as the cutoff points are likely to be very high.

\chapter{Modular Reduction}

Modular reduction is process of taking the remainder of one quantity divided by another.  Expressed
as (\ref{eqn:mod}) the modular reduction is equivalent to the remainder of $b$ divided by $c$.

\begin{equation}
a \equiv b \mbox{ (mod }c\mbox{)}
\label{eqn:mod}
\end{equation}

Of particular interest to cryptography are reductions where $b$ is limited to the range $0 \le b < c^2$ since particularly
fast reduction algorithms can be written for the limited range.

Note that one of the four optimized reduction algorithms are automatically chosen in the modular exponentiation
algorithm mp\_exptmod when an appropriate modulus is detected.

\section{Straight Division}
In order to effect an arbitrary modular reduction the following algorithm is provided.

\index{mp\_mod}
\begin{alltt}
int mp_mod(mp_int *a, mp_int *b, mp_int *c);
\end{alltt}

This reduces $a$ modulo $b$ and stores the result in $c$.  The sign of $c$ shall agree with the sign
of $b$.  This algorithm accepts an input $a$ of any range and is not limited by $0 \le a < b^2$.

\section{Barrett Reduction}

Barrett reduction is a generic optimized reduction algorithm that requires pre--computation to achieve
a decent speedup over straight division.  First a $\mu$ value must be precomputed with the following function.

\index{mp\_reduce\_setup}
\begin{alltt}
int mp_reduce_setup(mp_int *a, mp_int *b);
\end{alltt}

Given a modulus in $b$ this produces the required $\mu$ value in $a$.  For any given modulus this only has to
be computed once.  Modular reduction can now be performed with the following.

\index{mp\_reduce}
\begin{alltt}
int mp_reduce(mp_int *a, mp_int *b, mp_int *c);
\end{alltt}

This will reduce $a$ in place modulo $b$ with the precomputed $\mu$ value in $c$.  $a$ must be in the range
$0 \le a < b^2$.

\begin{alltt}
int main(void)
\{
   mp_int   a, b, c, mu;
   int      result;

   /* initialize a,b to desired values, mp_init mu,
    * c and set c to 1...we want to compute a^3 mod b
    */

   /* get mu value */
   if ((result = mp_reduce_setup(&mu, b)) != MP_OKAY) \{
      printf("Error getting mu.  \%s",
             mp_error_to_string(result));
      return EXIT_FAILURE;
   \}

   /* square a to get c = a^2 */
   if ((result = mp_sqr(&a, &c)) != MP_OKAY) \{
      printf("Error squaring.  \%s",
             mp_error_to_string(result));
      return EXIT_FAILURE;
   \}

   /* now reduce `c' modulo b */
   if ((result = mp_reduce(&c, &b, &mu)) != MP_OKAY) \{
      printf("Error reducing.  \%s",
             mp_error_to_string(result));
      return EXIT_FAILURE;
   \}

   /* multiply a to get c = a^3 */
   if ((result = mp_mul(&a, &c, &c)) != MP_OKAY) \{
      printf("Error reducing.  \%s",
             mp_error_to_string(result));
      return EXIT_FAILURE;
   \}

   /* now reduce `c' modulo b  */
   if ((result = mp_reduce(&c, &b, &mu)) != MP_OKAY) \{
      printf("Error reducing.  \%s",
             mp_error_to_string(result));
      return EXIT_FAILURE;
   \}

   /* c now equals a^3 mod b */

   return EXIT_SUCCESS;
\}
\end{alltt}

This program will calculate $a^3 \mbox{ mod }b$ if all the functions succeed.

\section{Montgomery Reduction}

Montgomery is a specialized reduction algorithm for any odd moduli.  Like Barrett reduction a pre--computation
step is required.  This is accomplished with the following.

\index{mp\_montgomery\_setup}
\begin{alltt}
int mp_montgomery_setup(mp_int *a, mp_digit *mp);
\end{alltt}

For the given odd moduli $a$ the precomputation value is placed in $mp$.  The reduction is computed with the
following.

\index{mp\_montgomery\_reduce}
\begin{alltt}
int mp_montgomery_reduce(mp_int *a, mp_int *m, mp_digit mp);
\end{alltt}
This reduces $a$ in place modulo $m$ with the pre--computed value $mp$.   $a$ must be in the range
$0 \le a < b^2$.

Montgomery reduction is faster than Barrett reduction for moduli smaller than the ``comba'' limit.  With the default
setup for instance, the limit is $127$ digits ($3556$--bits).   Note that this function is not limited to
$127$ digits just that it falls back to a baseline algorithm after that point.

An important observation is that this reduction does not return $a \mbox{ mod }m$ but $aR^{-1} \mbox{ mod }m$
where $R = \beta^n$, $n$ is the n number of digits in $m$ and $\beta$ is radix used (default is $2^{28}$).

To quickly calculate $R$ the following function was provided.

\index{mp\_montgomery\_calc\_normalization}
\begin{alltt}
int mp_montgomery_calc_normalization(mp_int *a, mp_int *b);
\end{alltt}
Which calculates $a = R$ for the odd moduli $b$ without using multiplication or division.

The normal modus operandi for Montgomery reductions is to normalize the integers before entering the system.  For
example, to calculate $a^3 \mbox { mod }b$ using Montgomery reduction the value of $a$ can be normalized by
multiplying it by $R$.  Consider the following code snippet.

\begin{alltt}
int main(void)
\{
   mp_int   a, b, c, R;
   mp_digit mp;
   int      result;

   /* initialize a,b to desired values,
    * mp_init R, c and set c to 1....
    */

   /* get normalization */
   if ((result = mp_montgomery_calc_normalization(&R, b)) != MP_OKAY) \{
      printf("Error getting norm.  \%s",
             mp_error_to_string(result));
      return EXIT_FAILURE;
   \}

   /* get mp value */
   if ((result = mp_montgomery_setup(&c, &mp)) != MP_OKAY) \{
      printf("Error setting up montgomery.  \%s",
             mp_error_to_string(result));
      return EXIT_FAILURE;
   \}

   /* normalize `a' so now a is equal to aR */
   if ((result = mp_mulmod(&a, &R, &b, &a)) != MP_OKAY) \{
      printf("Error computing aR.  \%s",
             mp_error_to_string(result));
      return EXIT_FAILURE;
   \}

   /* square a to get c = a^2R^2 */
   if ((result = mp_sqr(&a, &c)) != MP_OKAY) \{
      printf("Error squaring.  \%s",
             mp_error_to_string(result));
      return EXIT_FAILURE;
   \}

   /* now reduce `c' back down to c = a^2R^2 * R^-1 == a^2R */
   if ((result = mp_montgomery_reduce(&c, &b, mp)) != MP_OKAY) \{
      printf("Error reducing.  \%s",
             mp_error_to_string(result));
      return EXIT_FAILURE;
   \}

   /* multiply a to get c = a^3R^2 */
   if ((result = mp_mul(&a, &c, &c)) != MP_OKAY) \{
      printf("Error reducing.  \%s",
             mp_error_to_string(result));
      return EXIT_FAILURE;
   \}

   /* now reduce `c' back down to c = a^3R^2 * R^-1 == a^3R */
   if ((result = mp_montgomery_reduce(&c, &b, mp)) != MP_OKAY) \{
      printf("Error reducing.  \%s",
             mp_error_to_string(result));
      return EXIT_FAILURE;
   \}

   /* now reduce (again) `c' back down to c = a^3R * R^-1 == a^3 */
   if ((result = mp_montgomery_reduce(&c, &b, mp)) != MP_OKAY) \{
      printf("Error reducing.  \%s",
             mp_error_to_string(result));
      return EXIT_FAILURE;
   \}

   /* c now equals a^3 mod b */

   return EXIT_SUCCESS;
\}
\end{alltt}

This particular example does not look too efficient but it demonstrates the point of the algorithm.  By
normalizing the inputs the reduced results are always of the form $aR$ for some variable $a$.  This allows
a single final reduction to correct for the normalization and the fast reduction used within the algorithm.

For more details consider examining the file \textit{bn\_mp\_exptmod\_fast.c}.

\section{Restricted Diminished Radix}

``Diminished Radix'' reduction refers to reduction with respect to moduli that are amenable to simple
digit shifting and small multiplications.  In this case the ``restricted'' variant refers to moduli of the
form $\beta^k - p$ for some $k \ge 0$ and $0 < p < \beta$ where $\beta$ is the radix (default to $2^{28}$).

As in the case of Montgomery reduction there is a pre--computation phase required for a given modulus.

\index{mp\_dr\_setup}
\begin{alltt}
void mp_dr_setup(mp_int *a, mp_digit *d);
\end{alltt}

This computes the value required for the modulus $a$ and stores it in $d$.  This function cannot fail
and does not return any error codes.  After the pre--computation a reduction can be performed with the
following.

\index{mp\_dr\_reduce}
\begin{alltt}
int mp_dr_reduce(mp_int *a, mp_int *b, mp_digit mp);
\end{alltt}

This reduces $a$ in place modulo $b$ with the pre--computed value $mp$.  $b$ must be of a restricted
diminished radix form and $a$ must be in the range $0 \le a < b^2$.  Diminished radix reductions are
much faster than both Barrett and Montgomery reductions as they have a much lower asymptotic running time.

Since the moduli are restricted this algorithm is not particularly useful for something like Rabin, RSA or
BBS cryptographic purposes.  This reduction algorithm is useful for Diffie-Hellman and ECC where fixed
primes are acceptable.

Note that unlike Montgomery reduction there is no normalization process.  The result of this function is
equal to the correct residue.

\section{Unrestricted Diminished Radix}

Unrestricted reductions work much like the restricted counterparts except in this case the moduli is of the
form $2^k - p$ for $0 < p < \beta$.  In this sense the unrestricted reductions are more flexible as they
can be applied to a wider range of numbers.

\index{mp\_reduce\_2k\_setup}
\begin{alltt}
int mp_reduce_2k_setup(mp_int *a, mp_digit *d);
\end{alltt}

This will compute the required $d$ value for the given moduli $a$.

\index{mp\_reduce\_2k}
\begin{alltt}
int mp_reduce_2k(mp_int *a, mp_int *n, mp_digit d);
\end{alltt}

This will reduce $a$ in place modulo $n$ with the pre--computed value $d$.  From my experience this routine is
slower than mp\_dr\_reduce but faster for most moduli sizes than the Montgomery reduction.

\section{Combined Modular Reduction}

Some of the combinations of an arithmetic operations followed by a modular reduction can be done in a faster way. The ones implemented are:

Addition $d = (a + b) \mod c$
\index{mp\_addmod}
\begin{alltt}
int mp_addmod(const mp_int *a, const mp_int *b, const mp_int *c, mp_int *d);
\end{alltt}

Subtraction  $d = (a - b) \mod c$
\begin{alltt}
int mp_submod(const mp_int *a, const mp_int *b, const mp_int *c, mp_int *d);
\end{alltt}

Multiplication $d = (ab) \mod c$
\begin{alltt}
int mp_mulmod(const mp_int *a, const mp_int *b, const mp_int *c, mp_int *d);
\end{alltt}

Squaring  $d = (a^2) \mod c$
\begin{alltt}
int mp_sqrmod(const mp_int *a, const mp_int *b, const mp_int *c, mp_int *d);
\end{alltt}



\chapter{Exponentiation}
\section{Single Digit Exponentiation}
\index{mp\_expt\_d\_ex}
\begin{alltt}
int mp_expt_d_ex (mp_int * a, mp_digit b, mp_int * c, int fast)
\end{alltt}
This function computes $c = a^b$.

With parameter \textit{fast} set to $0$ the old version of the algorithm is used,
when \textit{fast} is $1$, a faster but not statically timed version of the algorithm is used.

The old version uses a simple binary left-to-right algorithm.
It is faster than repeated multiplications by $a$ for all values of $b$ greater than three.

The new version uses a binary right-to-left algorithm.

The difference between the old and the new version is that the old version always
executes $DIGIT\_BIT$ iterations. The new algorithm executes only $n$ iterations
where $n$ is equal to the position of the highest bit that is set in $b$.

\index{mp\_expt\_d}
\begin{alltt}
int mp_expt_d (mp_int * a, mp_digit b, mp_int * c)
\end{alltt}
mp\_expt\_d(a, b, c) is a wrapper function to mp\_expt\_d\_ex(a, b, c, 0).

\section{Modular Exponentiation}
\index{mp\_exptmod}
\begin{alltt}
int mp_exptmod (mp_int * G, mp_int * X, mp_int * P, mp_int * Y)
\end{alltt}
This computes $Y \equiv G^X \mbox{ (mod }P\mbox{)}$ using a variable width sliding window algorithm.  This function
will automatically detect the fastest modular reduction technique to use during the operation.  For negative values of
$X$ the operation is performed as $Y \equiv (G^{-1} \mbox{ mod }P)^{\vert X \vert} \mbox{ (mod }P\mbox{)}$ provided that
$gcd(G, P) = 1$.

This function is actually a shell around the two internal exponentiation functions.  This routine will automatically
detect when Barrett, Montgomery, Restricted and Unrestricted Diminished Radix based exponentiation can be used.  Generally
moduli of the a ``restricted diminished radix'' form lead to the fastest modular exponentiations.  Followed by Montgomery
and the other two algorithms.

\section{Modulus a Power of Two}
\index{mp\_mod\_2d}
\begin{alltt}
int mp_mod_2d(const mp_int *a, int b, mp_int *c)
\end{alltt}
It calculates $c = a \mod 2^b$.

\section{Root Finding}
\index{mp\_n\_root}
\begin{alltt}
int mp_n_root (mp_int * a, mp_digit b, mp_int * c)
\end{alltt}
This computes $c = a^{1/b}$ such that $c^b \le a$ and $(c+1)^b > a$. Will return a positive root only for even roots and return
a root with the sign of the input for odd roots.  For example, performing $4^{1/2}$ will return $2$ whereas $(-8)^{1/3}$
will return $-2$.

This algorithm uses the ``Newton Approximation'' method and will converge on the correct root fairly quickly.

The square root  $c = a^{1/2}$ (with the same conditions $c^2 \le a$ and $(c+1)^2 > a$) is implemented with a faster algorithm.

\index{mp\_sqrt}
\begin{alltt}
int mp_sqrt (mp_int * a, mp_digit b, mp_int * c)
\end{alltt}


\chapter{Prime Numbers}
\section{Trial Division}
\index{mp\_prime\_is\_divisible}
\begin{alltt}
int mp_prime_is_divisible (mp_int * a, int *result)
\end{alltt}
This will attempt to evenly divide $a$ by a list of primes\footnote{Default is the first 256 primes.} and store the
outcome in ``result''.  That is if $result = 0$ then $a$ is not divisible by the primes, otherwise it is.  Note that
if the function does not return \textbf{MP\_OKAY} the value in ``result'' should be considered undefined\footnote{Currently
the default is to set it to zero first.}.

\section{Fermat Test}
\index{mp\_prime\_fermat}
\begin{alltt}
int mp_prime_fermat (mp_int * a, mp_int * b, int *result)
\end{alltt}
Performs a Fermat primality test to the base $b$.  That is it computes $b^a \mbox{ mod }a$ and tests whether the value is
equal to $b$ or not.  If the values are equal then $a$ is probably prime and $result$ is set to one.  Otherwise $result$
is set to zero.

\section{Miller-Rabin Test}
\index{mp\_prime\_miller\_rabin}
\begin{alltt}
int mp_prime_miller_rabin (mp_int * a, mp_int * b, int *result)
\end{alltt}
Performs a Miller-Rabin test to the base $b$ of $a$.  This test is much stronger than the Fermat test and is very hard to
fool (besides with Carmichael numbers).  If $a$ passes the test (therefore is probably prime) $result$ is set to one.
Otherwise $result$ is set to zero.

Note that is suggested that you use the Miller-Rabin test instead of the Fermat test since all of the failures of
Miller-Rabin are a subset of the failures of the Fermat test.

\subsection{Required Number of Tests}
Generally to ensure a number is very likely to be prime you have to perform the Miller-Rabin with at least a half-dozen
or so unique bases.  However, it has been proven that the probability of failure goes down as the size of the input goes up.
This is why a simple function has been provided to help out.

\index{mp\_prime\_rabin\_miller\_trials}
\begin{alltt}
int mp_prime_rabin_miller_trials(int size)
\end{alltt}
This returns the number of trials required for a $2^{-96}$ (or lower) probability of failure for a given ``size'' expressed
in bits.  This comes in handy specially since larger numbers are slower to test.  For example, a 512-bit number would
require ten tests whereas a 1024-bit number would only require four tests.

You should always still perform a trial division before a Miller-Rabin test though.

A small table, broke in two for typographical reasons, with the number of rounds of Miller-Rabin tests is shown below.
The first column is the number of bits $b$ in the prime $p = 2^b$, the numbers in the first row represent the
probability that the number that all of the Miller-Rabin tests deemed a pseudoprime is actually a composite. There is a deterministic test for numbers smaller than $2^{80}$.

\begin{table}[h]
\begin{center}
\begin{tabular}{c c c c c c c}
\textbf{bits} & $\mathbf{2^{-80}}$ & $\mathbf{2^{-96}}$ & $\mathbf{2^{-112}}$ & $\mathbf{2^{-128}}$ & $\mathbf{2^{-160}}$ & $\mathbf{2^{-192}}$ \\
80    & 31 & 39 & 47 & 55 & 71 & 87  \\
96    & 29 & 37 & 45 & 53 & 69 & 85  \\
128   & 24 & 32 & 40 & 48 & 64 & 80  \\
160   & 19 & 27 & 35 & 43 & 59 & 75  \\
192   & 15 & 21 & 29 & 37 & 53 & 69  \\
256   & 10 & 15 & 20 & 27 & 43 & 59  \\
384   & 7  & 9  & 12 & 16 & 25 & 38  \\
512   & 5  & 7  & 9  & 12 & 18 & 26  \\
768   & 4  & 5  & 6  & 8  & 11 & 16  \\
1024  & 3  & 4  & 5  & 6  & 9  & 12  \\
1536  & 2  & 3  & 3  & 4  & 6  & 8   \\
2048  & 2  & 2  & 3  & 3  & 4  & 6   \\
3072  & 1  & 2  & 2  & 2  & 3  & 4   \\
4096  & 1  & 1  & 2  & 2  & 2  & 3   \\
6144  & 1  & 1  & 1  & 1  & 2  & 2   \\
8192  & 1  & 1  & 1  & 1  & 2  & 2   \\
12288 & 1  & 1  & 1  & 1  & 1  & 1   \\
16384 & 1  & 1  & 1  & 1  & 1  & 1   \\
24576 & 1  & 1  & 1  & 1  & 1  & 1   \\
32768 & 1  & 1  & 1  & 1  & 1  & 1
\end{tabular}
\caption{ Number of Miller-Rabin rounds. Part I } \label{table:millerrabinrunsp1}
\end{center}
\end{table}
\newpage
\begin{table}[h]
\begin{center}
\begin{tabular}{c c c c c c c c}
\textbf{bits} &$\mathbf{2^{-224}}$ & $\mathbf{2^{-256}}$ & $\mathbf{2^{-288}}$ & $\mathbf{2^{-320}}$ & $\mathbf{2^{-352}}$ & $\mathbf{2^{-384}}$ & $\mathbf{2^{-416}}$\\
80    & 103 & 119 & 135 & 151 & 167 & 183 & 199 \\
96    & 101 & 117 & 133 & 149 & 165 & 181 & 197 \\
128   & 96  & 112 & 128 & 144 & 160 & 176 & 192 \\
160   & 91  & 107 & 123 & 139 & 155 & 171 & 187 \\
192   & 85  & 101 & 117 & 133 & 149 & 165 & 181 \\
256   & 75  & 91  & 107 & 123 & 139 & 155 & 171 \\
384   & 54  & 70  & 86  & 102 & 118 & 134 & 150 \\
512   & 36  & 49  & 65  & 81  & 97  & 113 & 129 \\
768   & 22  & 29  & 37  & 47  & 58  & 70  & 86  \\
1024  & 16  & 21  & 26  & 33  & 40  & 48  & 58  \\
1536  & 10  & 13  & 17  & 21  & 25  & 30  & 35  \\
2048  & 8   & 10  & 13  & 15  & 18  & 22  & 26  \\
3072  & 5   & 7   & 8	& 10  & 12  & 14  & 17  \\
4096  & 4   & 5   & 6	& 8   & 9   & 11  & 12  \\
6144  & 3   & 4   & 4	& 5   & 6   & 7   & 8	\\
8192  & 2   & 3   & 3	& 4   & 5   & 6   & 6	\\
12288 & 2   & 2   & 2	& 3   & 3   & 4   & 4	\\
16384 & 1   & 2   & 2	& 2   & 3   & 3   & 3	\\
24576 & 1   & 1   & 2	& 2   & 2   & 2   & 2	\\
32768 & 1   & 1   & 1	& 1   & 2   & 2   & 2
\end{tabular}
\caption{ Number of Miller-Rabin rounds. Part II } \label{table:millerrabinrunsp2}
\end{center}
\end{table}

Determining the probability needed to pick the right column is a bit harder. Fips 186.4, for example has $2^{-80}$ for $512$ bit large numbers, $2^{-112}$ for $1024$ bits, and $2^{128}$ for $1536$ bits. It can be seen in table \ref{table:millerrabinrunsp1} that those combinations follow the diagonal from $(512,2^{-80})$ downwards and to the right to gain a lower probabilty of getting a composite declared a pseudoprime for the same amount of work or less.

If this version of the library has the strong Lucas-Selfridge and/or the Frobenius-Underwood test implemented only one or two rounds of the Miller-Rabin test with a random base is necesssary for numbers larger than or equal to $1024$ bits.


\section{Strong Lucas-Selfridge Test}
\index{mp\_prime\_strong\_lucas\_selfridge}
\begin{alltt}
int mp_prime_strong_lucas_selfridge(const mp_int *a, int *result)
\end{alltt}
Performs a strong Lucas-Selfridge test. The strong Lucas-Selfridge test together with the Rabin-Miler test with bases $2$ and $3$ resemble the BPSW test. The single internal use is a compile-time option in \texttt{mp\_prime\_is\_prime} and can be excluded
from the Libtommath build if not needed.

\section{Frobenius (Underwood)  Test}
\index{mp\_prime\_frobenius\_underwood}
\begin{alltt}
int mp_prime_frobenius_underwood(const mp_int *N, int *result)
\end{alltt}
Performs the variant of the Frobenius test as described by Paul Underwood. The single internal use is in
\texttt{mp\_prime\_is\_prime} for \texttt{MP\_8BIT} only but can be included at build-time for all other sizes
if the preprocessor macro \texttt{LTM\_USE\_FROBENIUS\_TEST} is defined.

It returns \texttt{MP\_ITER} if the number of iterations is exhausted, assumes a composite as the input and sets \texttt{result} accordingly. This will reduce the set of available pseudoprimes by a very small amount: test with large datasets (more than $10^{10}$ numbers, both randomly chosen and sequences of odd numbers with a random start point) found only 31 (thirty-one) numbers with $a > 120$ and none at all with just an additional simple check for divisors $d < 2^8$.

\section{Primality Testing}
Testing if a number is a square can be done a bit faster than just by calculating the square root. It is used by the primality testing function described below.
\index{mp\_is\_square}
\begin{alltt}
int mp_is_square(const mp_int *arg, int *ret);
\end{alltt}


\index{mp\_prime\_is\_prime}
\begin{alltt}
int mp_prime_is_prime (mp_int * a, int t, int *result)
\end{alltt}
This will perform a trial division followed by two rounds of Miller-Rabin with bases 2 and 3 and a Lucas-Selfridge test. The Lucas-Selfridge test is replaced with a Frobenius-Underwood for \texttt{MP\_8BIT}. The Frobenius-Underwood test for all other sizes is available as a compile-time option with the preprocessor macro \texttt{LTM\_USE\_FROBENIUS\_TEST}. See file
\texttt{bn\_mp\_prime\_is\_prime.c} for the necessary details. It shall be noted that both functions are much slower than
the Miller-Rabin test and if speed is an essential issue, the macro \texttt{LTM\_USE\_FIPS\_ONLY} switches both functions, the Frobenius-Underwood test and the Lucas-Selfridge test off and their code will not even be compiled into the library.

If $t$ is set to a positive value $t$ additional rounds of the Miller-Rabin test with random bases will be performed to allow for Fips 186.4 (vid.~p.~126ff) compliance. The function \texttt{mp\_prime\_rabin\_miller\_trials} can be used to determine the number of rounds. It is vital that the function \texttt{mp\_rand()} has a cryptographically strong random number generator available.

One Miller-Rabin tests with a random base will be run automatically, so by setting $t$ to a positive value this function will run $t + 1$ Miller-Rabin tests with random bases.

If  $t$ is set to a negative value the test will run the deterministic Miller-Rabin test for the primes up to
$3317044064679887385961981$. That limit has to be checked by the caller. If $-t > 13$ than $-t - 13$ additional rounds of the
Miller-Rabin test will be performed but note that $-t$ is bounded by $1 \le -t < PRIME\_SIZE$ where $PRIME\_SIZE$ is the number
of primes in the prime number table (by default this is $256$) and the first 13 primes have already been used. It will return
\texttt{MP\_VAL} in case of$-t > PRIME\_SIZE$.

If $a$ passes all of the tests $result$ is set to one, otherwise it is set to zero.

\section{Next Prime}
\index{mp\_prime\_next\_prime}
\begin{alltt}
int mp_prime_next_prime(mp_int *a, int t, int bbs_style)
\end{alltt}
This finds the next prime after $a$ that passes mp\_prime\_is\_prime() with $t$ tests but see the documentation for
mp\_prime\_is\_prime for details regarding the use of the argument $t$.  Set $bbs\_style$ to one if you
want only the next prime congruent to $3 \mbox{ mod } 4$, otherwise set it to zero to find any next prime.

\section{Random Primes}
\index{mp\_prime\_random}
\begin{alltt}
int mp_prime_random(mp_int *a, int t, int size, int bbs,
                    ltm_prime_callback cb, void *dat)
\end{alltt}
This will find a prime greater than $256^{size}$ which can be ``bbs\_style'' or not depending on $bbs$ and must pass
$t$ rounds of tests but see the documentation for mp\_prime\_is\_prime for details regarding the use of the argument $t$.
The ``ltm\_prime\_callback'' is a typedef for

\begin{alltt}
typedef int ltm_prime_callback(unsigned char *dst, int len, void *dat);
\end{alltt}

Which is a function that must read $len$ bytes (and return the amount stored) into $dst$.  The $dat$ variable is simply
copied from the original input.  It can be used to pass RNG context data to the callback.  The function
mp\_prime\_random() is more suitable for generating primes which must be secret (as in the case of RSA) since there
is no skew on the least significant bits.

\textit{Note:}  As of v0.30 of the LibTomMath library this function has been deprecated.  It is still available
but users are encouraged to use the new mp\_prime\_random\_ex() function instead.

\subsection{Extended Generation}
\index{mp\_prime\_random\_ex}
\begin{alltt}
int mp_prime_random_ex(mp_int *a,    int t,
                       int     size, int flags,
                       ltm_prime_callback cb, void *dat);
\end{alltt}
This will generate a prime in $a$ using $t$ tests of the primality testing algorithms.  The variable $size$
specifies the bit length of the prime desired.  The variable $flags$ specifies one of several options available
(see fig. \ref{fig:primeopts}) which can be OR'ed together.  The callback parameters are used as in
mp\_prime\_random().

\begin{figure}[h]
\begin{center}
\begin{small}
\begin{tabular}{|r|l|}
\hline \textbf{Flag}         & \textbf{Meaning} \\
\hline LTM\_PRIME\_BBS       & Make the prime congruent to $3$ modulo $4$ \\
\hline LTM\_PRIME\_SAFE      & Make a prime $p$ such that $(p - 1)/2$ is also prime. \\
                             & This option implies LTM\_PRIME\_BBS as well. \\
\hline LTM\_PRIME\_2MSB\_OFF & Makes sure that the bit adjacent to the most significant bit \\
                             & Is forced to zero.  \\
\hline LTM\_PRIME\_2MSB\_ON  & Makes sure that the bit adjacent to the most significant bit \\
                             & Is forced to one. \\
\hline
\end{tabular}
\end{small}
\end{center}
\caption{Primality Generation Options}
\label{fig:primeopts}
\end{figure}

\chapter{Random Number Generation}
\section{PRNG}
\index{mp\_rand\_digit}
\begin{alltt}
int mp_rand_digit(mp_digit *r)
\end{alltt}
This function generates a random number in \texttt{r} of the size given in \texttt{r} (that is, the variable is used for in- and output) but not more than \texttt{MP\_MASK} bits.

\index{mp\_rand}
\begin{alltt}
int mp_rand(mp_int *a, int digits)
\end{alltt}
This function generates a random number of \texttt{digits} bits.

The random number generated with these two functions is cryptographically secure if the source of random numbers the operating systems offers is cryptographically secure. It will use \texttt{arc4random()} if the OS is a BSD flavor, Wincrypt on Windows, or \texttt{/dev/urandom} on all operating systems that have it.

\chapter{Factorizing}
\subsection{Pollard-Rho-Brent}
One of the methods to decompose a larger number into its prime factors is the Pollard-Rho algorithm. This implementation uses Richard P. Brent's method for loop detection.
\index{mp\_pollard\_rho}
\begin{alltt}
int mp_pollard_rho(const mp_int *n, mp_int *factor);
\end{alltt}
The function \texttt{mp\_pollard\_rho} is not meant to be used as a standalone function, please consult the source of the function and the test for this function \texttt{test\_mp\_pollard\_rho} in \texttt{demp/test.c} for more information.



\chapter{Input and Output}
\section{ASCII Conversions}
\subsection{To ASCII}
\index{mp\_toradix}
\begin{alltt}
int mp_toradix (mp_int * a, char *str, int radix);
\end{alltt}
This still store $a$ in ``str'' as a base-``radix'' string of ASCII chars.  This function appends a NUL character
to terminate the string.  Valid values of ``radix'' line in the range $[2, 64]$.  To determine the size (exact) required
by the conversion before storing any data use the following function.

\index{mp\_toradix\_n}
\begin{alltt}
int mp_toradix_n (mp_int * a, char *str, int radix, int maxlen);
\end{alltt}

Like \texttt{mp\_toradix} but stores upto maxlen-1 chars and always a NULL byte.

\index{mp\_radix\_size}
\begin{alltt}
int mp_radix_size (mp_int * a, int radix, int *size)
\end{alltt}
This stores in ``size'' the number of characters (including space for the NUL terminator) required.  Upon error this
function returns an error code and ``size'' will be zero.

If \texttt{LTM\_NO\_FILE} is not defined a function to write to a file is also available.
\index{mp\_fwrite}
\begin{alltt}
int mp_fwrite(const mp_int *a, int radix, FILE *stream);
\end{alltt}


\subsection{From ASCII}
\index{mp\_read\_radix}
\begin{alltt}
int mp_read_radix (mp_int * a, char *str, int radix);
\end{alltt}
This will read the base-``radix'' NUL terminated string from ``str'' into $a$.  It will stop reading when it reads a
character it does not recognize (which happens to include th NUL char... imagine that...).  A single leading $-$ sign
can be used to denote a negative number.

If \texttt{LTM\_NO\_FILE} is not defined a function to read from a file is also available.
\index{mp\_fread}
\begin{alltt}
int mp_fread(mp_int *a, int radix, FILE *stream);
\end{alltt}


\section{Binary Conversions}

Converting an mp\_int to and from binary is another keen idea.

\index{mp\_unsigned\_bin\_size}
\begin{alltt}
int mp_unsigned_bin_size(mp_int *a);
\end{alltt}

This will return the number of bytes (octets) required to store the unsigned copy of the integer $a$.

\index{mp\_to\_unsigned\_bin}
\begin{alltt}
int mp_to_unsigned_bin(mp_int *a, unsigned char *b);
\end{alltt}
This will store $a$ into the buffer $b$ in big--endian format.  Fortunately this is exactly what DER (or is it ASN?)
requires.  It does not store the sign of the integer.

\index{mp\_to\_unsigned\_bin\_n}
\begin{alltt}
int mp_to_unsigned_bin_n(const mp_int *a, unsigned char *b, unsigned long *outlen)
\end{alltt}
Like \texttt{mp\_to\_unsigned\_bin} but checks if the value at \texttt{*outlen} is larger than or equal to the output of \texttt{mp\_unsigned\_bin\_size(a)} and sets \texttt{*outlen} to the output of \texttt{mp\_unsigned\_bin\_size(a)} or returns \texttt{MP\_VAL} if the test failed.


\index{mp\_read\_unsigned\_bin}
\begin{alltt}
int mp_read_unsigned_bin(mp_int *a, unsigned char *b, int c);
\end{alltt}
This will read in an unsigned big--endian array of bytes (octets) from $b$ of length $c$ into $a$.  The resulting
integer $a$ will always be positive.

For those who acknowledge the existence of negative numbers (heretic!) there are ``signed'' versions of the
previous functions.
\index{mp\_signed\_bin\_size} \index{mp\_to\_signed\_bin} \index{mp\_read\_signed\_bin}
\begin{alltt}
int mp_signed_bin_size(mp_int *a);
int mp_read_signed_bin(mp_int *a, unsigned char *b, int c);
int mp_to_signed_bin(mp_int *a, unsigned char *b);
\end{alltt}
They operate essentially the same as the unsigned copies except they prefix the data with zero or non--zero
byte depending on the sign.  If the sign is zpos (e.g. not negative) the prefix is zero, otherwise the prefix
is non--zero.

The two functions \texttt{mp\_import} and \texttt{mp\_export} implement the corresponding GMP functions as described at \url{http://gmplib.org/manual/Integer-Import-and-Export.html}.
\index{mp\_import} \index{mp\_export}
\begin{alltt}
int mp_import(mp_int *rop, size_t count, int order, size_t size, int endian, size_t nails, const void *op);
int mp_export(void *rop, size_t *countp, int order, size_t size, int endian, size_t nails, const mp_int *op);
\end{alltt}

\chapter{Algebraic Functions}
\section{Extended Euclidean Algorithm}
\index{mp\_exteuclid}
\begin{alltt}
int mp_exteuclid(mp_int *a, mp_int *b,
                 mp_int *U1, mp_int *U2, mp_int *U3);
\end{alltt}

This finds the triple U1/U2/U3 using the Extended Euclidean algorithm such that the following equation holds.

\begin{equation}
a \cdot U1 + b \cdot U2 = U3
\end{equation}

Any of the U1/U2/U3 parameters can be set to \textbf{NULL} if they are not desired.

\section{Greatest Common Divisor}
\index{mp\_gcd}
\begin{alltt}
int mp_gcd (mp_int * a, mp_int * b, mp_int * c)
\end{alltt}
This will compute the greatest common divisor of $a$ and $b$ and store it in $c$.

\section{Least Common Multiple}
\index{mp\_lcm}
\begin{alltt}
int mp_lcm (mp_int * a, mp_int * b, mp_int * c)
\end{alltt}
This will compute the least common multiple of $a$ and $b$ and store it in $c$.

\section{Jacobi Symbol}
\index{mp\_jacobi}
\begin{alltt}
int mp_jacobi (mp_int * a, mp_int * p, int *c)
\end{alltt}
This will compute the Jacobi symbol for $a$ with respect to $p$.  If $p$ is prime this essentially computes the Legendre
symbol.  The result is stored in $c$ and can take on one of three values $\lbrace -1, 0, 1 \rbrace$.  If $p$ is prime
then the result will be $-1$ when $a$ is not a quadratic residue modulo $p$.  The result will be $0$ if $a$ divides $p$
and the result will be $1$ if $a$ is a quadratic residue modulo $p$.

\section{Kronecker Symbol}
\index{mp\_kronecker}
\begin{alltt}
int mp_kronecker (mp_int * a, mp_int * p, int *c)
\end{alltt}
Extension of the Jacoby symbol to all $\lbrace a, p \rbrace \in \mathbb{Z}$ .


\section{Modular square root}
\index{mp\_sqrtmod\_prime}
\begin{alltt}
int mp_sqrtmod_prime(mp_int *n, mp_int *p, mp_int *r)
\end{alltt}

This will solve the modular equatioon $r^2 = n \mod p$ where $p$ is a prime number greater than 2 (odd prime).
The result is returned in the third argument $r$, the function returns \textbf{MP\_OKAY} on success,
other return values indicate failure.

The implementation is split for two different cases:

1. if $p \mod 4 == 3$ we apply \href{http://cacr.uwaterloo.ca/hac/}{Handbook of Applied Cryptography algorithm 3.36} and compute $r$ directly as
$r = n^{(p+1)/4} \mod p$

2. otherwise we use \href{https://en.wikipedia.org/wiki/Tonelli-Shanks_algorithm}{Tonelli-Shanks algorithm}

The function does not check the primality of parameter $p$ thus it is up to the caller to assure that this parameter
is a prime number. When $p$ is a composite the function behaviour is undefined, it may even return a false-positive
\textbf{MP\_OKAY}.

\section{Modular Inverse}
\index{mp\_invmod}
\begin{alltt}
int mp_invmod (mp_int * a, mp_int * b, mp_int * c)
\end{alltt}
Computes the multiplicative inverse of $a$ modulo $b$ and stores the result in $c$ such that $ac \equiv 1 \mbox{ (mod }b\mbox{)}$.

\section{Single Digit Functions}

For those using small numbers (\textit{snicker snicker}) there are several ``helper'' functions

\index{mp\_add\_d} \index{mp\_sub\_d} \index{mp\_mul\_d} \index{mp\_div\_d} \index{mp\_mod\_d}
\begin{alltt}
int mp_add_d(mp_int *a, mp_digit b, mp_int *c);
int mp_sub_d(mp_int *a, mp_digit b, mp_int *c);
int mp_mul_d(mp_int *a, mp_digit b, mp_int *c);
int mp_div_d(mp_int *a, mp_digit b, mp_int *c, mp_digit *d);
int mp_mod_d(mp_int *a, mp_digit b, mp_digit *c);
\end{alltt}

These work like the full mp\_int capable variants except the second parameter $b$ is a mp\_digit.  These
functions fairly handy if you have to work with relatively small numbers since you will not have to allocate
an entire mp\_int to store a number like $1$ or $2$.

The functions \texttt{mp\_incr} and \texttt{mp\_decr} mimic the postfix operators \texttt{++} and \texttt{--} respectively, to increment the input by one. They call the full single-digit functions if the addition would carry. Both functions need to be included in a minimized library because they call each other in case of a negative input, These functions change the inputs!
\begin{alltt}
int mp_incr(mp_int *a);
int mp_decr(mp_int *a);
\end{alltt}


The division by three can be made faster by replacing the division with a multiplication by the multiplicative inverse of three.

\index{mp\_div\_3}
\begin{alltt}
int mp_div_3(const mp_int *a, mp_int *c, mp_digit *d);
\end{alltt}

\chapter{Little Helpers}
It is never wrong to have some useful little shortcuts at hand.
\section{Function Macros}
To make this overview simpler the macros are given as function prototypes. The return of logic macros is \texttt{MP\_NO} or \texttt{MP\_YES} respectively.

\index{mp\_iseven}
\begin{alltt}
int mp_iseven(mp_int *a)
\end{alltt}
Checks if $a = 0 mod 2$

\index{mp\_isodd}
\begin{alltt}
int mp_isodd(mp_int *a)
\end{alltt}
Checks if $a = 1 mod 2$

\index{mp\_isneg}
\begin{alltt}
int mp_isneg(mp_int *a)
\end{alltt}
Checks if $a < 0$


\index{mp\_iszero}
\begin{alltt}
int mp_iszero(mp_int *a)
\end{alltt}
Checks if $a = 0$. It does not check if the amount of memory allocated for $a$ is also minimal.


Other macros which are either shortcuts to normal functions or just other names for them do have their place in a programmer's life, too!

\subsection{Renamings}
\index{mp\_mag\_size}
\begin{alltt}
#define mp_mag_size(mp) mp_unsigned_bin_size(mp)
\end{alltt}


\index{mp\_raw\_size}
\begin{alltt}
#define mp_raw_size(mp) mp_signed_bin_size(mp)
\end{alltt}


\index{mp\_read\_mag}
\begin{alltt}
#define mp_read_mag(mp, str, len) mp_read_unsigned_bin((mp), (str), (len))
\end{alltt}


\index{mp\_read\_raw}
\begin{alltt}
 #define mp_read_raw(mp, str, len) mp_read_signed_bin((mp), (str), (len))
\end{alltt}


\index{mp\_tomag}
\begin{alltt}
#define mp_tomag(mp, str) mp_to_unsigned_bin((mp), (str))
\end{alltt}


\index{mp\_toraw}
\begin{alltt}
#define mp_toraw(mp, str)         mp_to_signed_bin((mp), (str))
\end{alltt}



\subsection{Shortcuts}

\index{mp\_tobinary}
\begin{alltt}
#define mp_tobinary(M, S) mp_toradix((M), (S), 2)
\end{alltt}


\index{mp\_tooctal}
\begin{alltt}
#define mp_tooctal(M, S) mp_toradix((M), (S), 8)
\end{alltt}


\index{mp\_todecimal}
\begin{alltt}
#define mp_todecimal(M, S) mp_toradix((M), (S), 10)
\end{alltt}


\index{mp\_tohex}
\begin{alltt}
#define mp_tohex(M, S)     mp_toradix((M), (S), 16)
\end{alltt}


\documentclass[synpaper]{book}
\usepackage{hyperref}
\usepackage{makeidx}
\usepackage{amssymb}
\usepackage{color}
\usepackage{alltt}
\usepackage{graphicx}
\usepackage{layout}
\def\union{\cup}
\def\intersect{\cap}
\def\getsrandom{\stackrel{\rm R}{\gets}}
\def\cross{\times}
\def\cat{\hspace{0.5em} \| \hspace{0.5em}}
\def\catn{$\|$}
\def\divides{\hspace{0.3em} | \hspace{0.3em}}
\def\nequiv{\not\equiv}
\def\approx{\raisebox{0.2ex}{\mbox{\small $\sim$}}}
\def\lcm{{\rm lcm}}
\def\gcd{{\rm gcd}}
\def\log{{\rm log}}
\def\ilog{{\rm ilog}}
\def\ord{{\rm ord}}
\def\abs{{\mathit abs}}
\def\rep{{\mathit rep}}
\def\mod{{\mathit\ mod\ }}
\renewcommand{\pmod}[1]{\ ({\rm mod\ }{#1})}
\newcommand{\floor}[1]{\left\lfloor{#1}\right\rfloor}
\newcommand{\ceil}[1]{\left\lceil{#1}\right\rceil}
\def\Or{{\rm\ or\ }}
\def\And{{\rm\ and\ }}
\def\iff{\hspace{1em}\Longleftrightarrow\hspace{1em}}
\def\implies{\Rightarrow}
\def\undefined{{\rm ``undefined"}}
\def\Proof{\vspace{1ex}\noindent {\bf Proof:}\hspace{1em}}
\let\oldphi\phi
\def\phi{\varphi}
\def\Pr{{\rm Pr}}
\newcommand{\str}[1]{{\mathbf{#1}}}
\def\F{{\mathbb F}}
\def\N{{\mathbb N}}
\def\Z{{\mathbb Z}}
\def\R{{\mathbb R}}
\def\C{{\mathbb C}}
\def\Q{{\mathbb Q}}
\definecolor{DGray}{gray}{0.5}
\newcommand{\emailaddr}[1]{\mbox{$<${#1}$>$}}
\def\twiddle{\raisebox{0.3ex}{\mbox{\tiny $\sim$}}}
\def\gap{\vspace{0.5ex}}
\makeindex
\begin{document}
\frontmatter
\pagestyle{empty}
\title{LibTomMath User Manual \\ v1.1.0}
\author{LibTom Projects \\ www.libtom.net}
\maketitle
This text, the library and the accompanying textbook are all hereby placed in the public domain.  This book has been
formatted for B5 [176x250] paper using the \LaTeX{} {\em book} macro package.

\vspace{10cm}

\begin{flushright}Open Source.  Open Academia.  Open Minds.

\mbox{ }
LibTom Projects

\& originally

Tom St Denis,

Ontario, Canada
\end{flushright}

\tableofcontents
\listoffigures
\mainmatter
\pagestyle{headings}
\chapter{Introduction}
\section{What is LibTomMath?}
LibTomMath is a library of source code which provides a series of efficient and carefully written functions for manipulating
large integer numbers.  It was written in portable ISO C source code so that it will build on any platform with a conforming
C compiler.

In a nutshell the library was written from scratch with verbose comments to help instruct computer science students how
to implement ``bignum'' math.  However, the resulting code has proven to be very useful.  It has been used by numerous
universities, commercial and open source software developers.  It has been used on a variety of platforms ranging from
Linux and Windows based x86 to ARM based Gameboys and PPC based MacOS machines.

\section{License}
As of the v0.25 the library source code has been placed in the public domain with every new release.  As of the v0.28
release the textbook ``Implementing Multiple Precision Arithmetic'' has been placed in the public domain with every new
release as well.  This textbook is meant to compliment the project by providing a more solid walkthrough of the development
algorithms used in the library.

Since both\footnote{Note that the MPI files under mtest/ are copyrighted by Michael Fromberger.  They are not required to use LibTomMath.} are in the
public domain everyone is entitled to do with them as they see fit.

\section{Building LibTomMath}

LibTomMath is meant to be very ``GCC friendly'' as it comes with a makefile well suited for GCC.  However, the library will
also build in MSVC, Borland C out of the box.  For any other ISO C compiler a makefile will have to be made by the end
developer. Please consider to commit such a makefile to the LibTomMath developers, currently residing at
\url{http://github.com/libtom/libtommath}, if successfully done so.

Intel's C-compiler (ICC) is sufficiently compatible with GCC, at least the newer versions, to replace GCC for building the static and the shared library. Editing the makefiles is not needed, just set the shell variable \texttt{CC} as shown below.
\begin{alltt}
CC=/home/czurnieden/intel/bin/icc make
\end{alltt}

ICC does not know all options available for GCC and LibTomMath uses two diagnostics \texttt{-Wbad-function-cast} and \texttt{-Wcast-align} that are not supported by ICC resulting in the warnings:
\begin{alltt}
icc: command line warning #10148: option '-Wbad-function-cast' not supported
icc: command line warning #10148: option '-Wcast-align' not supported
\end{alltt}
It is possible to mute this ICC warning with the compiler flag \texttt{-diag-disable=10006}\footnote{It is not recommended to suppress warnings without a very good reason but there is no harm in doing so in this very special case.}.

\subsection{Static Libraries}
To build as a static library for GCC issue the following
\begin{alltt}
make
\end{alltt}

command.  This will build the library and archive the object files in ``libtommath.a''.  Now you link against
that and include ``tommath.h'' within your programs.  Alternatively to build with MSVC issue the following
\begin{alltt}
nmake -f makefile.msvc
\end{alltt}

This will build the library and archive the object files in ``tommath.lib''.  This has been tested with MSVC
version 6.00 with service pack 5.

\subsection{Shared Libraries}
\subsubsection{GNU based Operating Systems}
To build as a shared library for GCC issue the following
\begin{alltt}
make -f makefile.shared
\end{alltt}
This requires the ``libtool'' package (common on most Linux/BSD systems).  It will build LibTomMath as both shared
and static then install (by default) into /usr/lib as well as install the header files in /usr/include.  The shared
library (resource) will be called ``libtommath.la'' while the static library called ``libtommath.a''.  Generally
you use libtool to link your application against the shared object.
\subsubsection{Microsoft Windows based Operating Systems}
There is limited support for making a ``DLL'' in windows via the ``makefile.cygwin\_dll'' makefile.  It requires
Cygwin to work with since it requires the auto-export/import functionality.  The resulting DLL and import library
``libtommath.dll.a'' can be used to link LibTomMath dynamically to any Windows program using Cygwin.
\subsubsection{OpenBSD}
OpenBSD replaced some of their GNU-tools, especially \texttt{libtool} with their own, slightly different versions. To ease the workload of LibTomMath's developer team, only a static library can be build with the included \texttt{makefile.unix}.

The wrong \texttt{make} will result in errors like:
\begin{alltt}
*** Parse error in /home/user/GITHUB/libtommath: Need an operator in 'LIBNAME' )
*** Parse error: Need an operator in 'endif' (makefile.shared:8)
*** Parse error: Need an operator in 'CROSS_COMPILE' (makefile_include.mk:16)
*** Parse error: Need an operator in 'endif' (makefile_include.mk:18)
*** Parse error: Missing dependency operator (makefile_include.mk:22)
*** Parse error: Missing dependency operator (makefile_include.mk:23)
...
\end{alltt}
The wrong \texttt{libtool} will build it all fine but when it comes to the final linking fails with
\begin{alltt}
...
cc -I./ -Wall -Wsign-compare -Wextra -Wshadow -Wsystem-headers -Wdeclaration-afo...
cc -I./ -Wall -Wsign-compare -Wextra -Wshadow -Wsystem-headers -Wdeclaration-afo...
cc -I./ -Wall -Wsign-compare -Wextra -Wshadow -Wsystem-headers -Wdeclaration-afo...
libtool --mode=link --tag=CC cc  bn_error.lo bn_fast_mp_invmod.lo bn_fast_mp_mo 
libtool: link: cc bn_error.lo bn_fast_mp_invmod.lo bn_fast_mp_montgomery_reduce0
bn_error.lo: file not recognized: File format not recognized
cc: error: linker command failed with exit code 1 (use -v to see invocation)
Error while executing cc bn_error.lo bn_fast_mp_invmod.lo bn_fast_mp_montgomery0
gmake: *** [makefile.shared:64: libtommath.la] Error 1
\end{alltt}

To build a shared library with OpenBSD\footnote{Tested with OpenBSD version 6.4} the GNU versions of \texttt{make} and \texttt{libtool} are needed.
\begin{alltt}
$ sudo pkg_add gmake libtool
\end{alltt}
At this time two versions of \texttt{libtool} are installed and both are named \texttt{libtool}, unfortunately but GNU \texttt{libtool} has been placed in \texttt{/usr/local/bin/} and the native version in \texttt{/usr/bin/}. The path might be different in other versions of OpenBSD but both programs differ in the output of \texttt{libtool --version}
\begin{alltt}
$ /usr/local/bin/libtool --version                              
libtool (GNU libtool) 2.4.2
Written by Gordon Matzigkeit <gord@gnu.ai.mit.edu>, 1996

Copyright (C) 2011 Free Software Foundation, Inc.
This is free software; see the source for copying conditions.  There is NO
warranty; not even for MERCHANTABILITY or FITNESS FOR A PARTICULAR PURPOSE.
$ libtool --version
libtool (not (GNU libtool)) 1.5.26
\end{alltt}

The shared library should build now with
\begin{alltt}
LIBTOOL="/usr/local/bin/libtool" gmake -f makefile.shared
\end{alltt}
You might need to run a \texttt{gmake -f makefile.shared clean} first.

\subsubsection{NetBSD}
NetBSD is not as strict as OpenBSD but still needs \texttt{gmake} to build the shared library. \texttt{libtool} may also not exist in a fresh install.
\begin{alltt}
pkg_add gmake libtool
\end{alltt}
Please check with \texttt{libtool --version} that installed libtool is indeed a GNU libtool.
Build the shared library by typing:
\begin{alltt}
gmake -f makefile.shared
\end{alltt}

\subsection{Testing}
To build the library and the test harness type

\begin{alltt}
make test
\end{alltt}

This will build the library, ``test'' and ``mtest/mtest''.  The ``test'' program will accept test vectors and verify the
results.  ``mtest/mtest'' will generate test vectors using the MPI library by Michael Fromberger\footnote{A copy of MPI
is included in the package}.  Simply pipe mtest into test using

\begin{alltt}
mtest/mtest | test
\end{alltt}

If you do not have a ``/dev/urandom'' style RNG source you will have to write your own PRNG and simply pipe that into
mtest.  For example, if your PRNG program is called ``myprng'' simply invoke

\begin{alltt}
myprng | mtest/mtest | test
\end{alltt}

This will output a row of numbers that are increasing.  Each column is a different test (such as addition, multiplication, etc)
that is being performed.  The numbers represent how many times the test was invoked.  If an error is detected the program
will exit with a dump of the relevant numbers it was working with.

\section{Build Configuration}
LibTomMath can configured at build time in three phases we shall call ``depends'', ``tweaks'' and ``trims''.
Each phase changes how the library is built and they are applied one after another respectively.

To make the system more powerful you can tweak the build process.  Classes are defined in the file
``tommath\_superclass.h''.  By default, the symbol ``LTM\_ALL'' shall be defined which simply
instructs the system to build all of the functions.  This is how LibTomMath used to be packaged.  This will give you
access to every function LibTomMath offers.

However, there are cases where such a build is not optional.  For instance, you want to perform RSA operations.  You
don't need the vast majority of the library to perform these operations.  Aside from LTM\_ALL there is
another pre--defined class ``SC\_RSA\_1'' which works in conjunction with the RSA from LibTomCrypt.  Additional
classes can be defined base on the need of the user.

\subsection{Build Depends}
In the file tommath\_class.h you will see a large list of C ``defines'' followed by a series of ``ifdefs''
which further define symbols.  All of the symbols (technically they're macros $\ldots$) represent a given C source
file.  For instance, BN\_MP\_ADD\_C represents the file ``bn\_mp\_add.c''.  When a define has been enabled the
function in the respective file will be compiled and linked into the library.  Accordingly when the define
is absent the file will not be compiled and not contribute any size to the library.

You will also note that the header tommath\_class.h is actually recursively included (it includes itself twice).
This is to help resolve as many dependencies as possible.  In the last pass the symbol LTM\_LAST will be defined.
This is useful for ``trims''.

\subsection{Build Tweaks}
A tweak is an algorithm ``alternative''.  For example, to provide trade-offs (usually between size and space).
They can be enabled at any pass of the configuration phase.

\begin{small}
\begin{center}
\begin{tabular}{|l|l|}
\hline \textbf{Define} & \textbf{Purpose} \\
\hline BN\_MP\_DIV\_SMALL & Enables a slower, smaller and equally \\
                          & functional mp\_div() function \\
\hline
\end{tabular}
\end{center}
\end{small}

\subsection{Build Trims}
A trim is a manner of removing functionality from a function that is not required.  For instance, to perform
RSA cryptography you only require exponentiation with odd moduli so even moduli support can be safely removed.
Build trims are meant to be defined on the last pass of the configuration which means they are to be defined
only if LTM\_LAST has been defined.

\subsubsection{Moduli Related}
\begin{small}
\begin{center}
\begin{tabular}{|l|l|}
\hline \textbf{Restriction} & \textbf{Undefine} \\
\hline Exponentiation with odd moduli only & BN\_S\_MP\_EXPTMOD\_C \\
                                           & BN\_MP\_REDUCE\_C \\
                                           & BN\_MP\_REDUCE\_SETUP\_C \\
                                           & BN\_S\_MP\_MUL\_HIGH\_DIGS\_C \\
                                           & BN\_FAST\_S\_MP\_MUL\_HIGH\_DIGS\_C \\
\hline Exponentiation with random odd moduli & (The above plus the following) \\
                                           & BN\_MP\_REDUCE\_2K\_C \\
                                           & BN\_MP\_REDUCE\_2K\_SETUP\_C \\
                                           & BN\_MP\_REDUCE\_IS\_2K\_C \\
                                           & BN\_MP\_DR\_IS\_MODULUS\_C \\
                                           & BN\_MP\_DR\_REDUCE\_C \\
                                           & BN\_MP\_DR\_SETUP\_C \\
\hline Modular inverse odd moduli only     & BN\_MP\_INVMOD\_SLOW\_C \\
\hline Modular inverse (both, smaller/slower) & BN\_FAST\_MP\_INVMOD\_C \\
\hline
\end{tabular}
\end{center}
\end{small}

\subsubsection{Operand Size Related}
\begin{small}
\begin{center}
\begin{tabular}{|l|l|}
\hline \textbf{Restriction} & \textbf{Undefine} \\
\hline Moduli $\le 2560$ bits              & BN\_MP\_MONTGOMERY\_REDUCE\_C \\
                                           & BN\_S\_MP\_MUL\_DIGS\_C \\
                                           & BN\_S\_MP\_MUL\_HIGH\_DIGS\_C \\
                                           & BN\_S\_MP\_SQR\_C \\
\hline Polynomial Schmolynomial            & BN\_MP\_KARATSUBA\_MUL\_C \\
                                           & BN\_MP\_KARATSUBA\_SQR\_C \\
                                           & BN\_MP\_TOOM\_MUL\_C \\
                                           & BN\_MP\_TOOM\_SQR\_C \\

\hline
\end{tabular}
\end{center}
\end{small}


\section{Purpose of LibTomMath}
Unlike  GNU MP (GMP) Library, LIP, OpenSSL or various other commercial kits (Miracl), LibTomMath was not written with
bleeding edge performance in mind.  First and foremost LibTomMath was written to be entirely open.  Not only is the
source code public domain (unlike various other GPL/etc licensed code), not only is the code freely downloadable but the
source code is also accessible for computer science students attempting to learn ``BigNum'' or multiple precision
arithmetic techniques.

LibTomMath was written to be an instructive collection of source code.  This is why there are many comments, only one
function per source file and often I use a ``middle-road'' approach where I don't cut corners for an extra 2\% speed
increase.

Source code alone cannot really teach how the algorithms work which is why I also wrote a textbook that accompanies
the library (beat that!).

So you may be thinking ``should I use LibTomMath?'' and the answer is a definite maybe.  Let me tabulate what I think
are the pros and cons of LibTomMath by comparing it to the math routines from GnuPG\footnote{GnuPG v1.2.3 versus LibTomMath v0.28}.

\newpage\begin{figure}[h]
\begin{small}
\begin{center}
\begin{tabular}{|l|c|c|l|}
\hline \textbf{Criteria} & \textbf{Pro} & \textbf{Con} & \textbf{Notes} \\
\hline Few lines of code per file & X & & GnuPG $ = 300.9$, LibTomMath  $ = 71.97$ \\
\hline Commented function prototypes & X && GnuPG function names are cryptic. \\
\hline Speed && X & LibTomMath is slower.  \\
\hline Totally free & X & & GPL has unfavourable restrictions.\\
\hline Large function base & X & & GnuPG is barebones. \\
\hline Five modular reduction algorithms & X & & Faster modular exponentiation for a variety of moduli. \\
\hline Portable & X & & GnuPG requires configuration to build. \\
\hline
\end{tabular}
\end{center}
\end{small}
\caption{LibTomMath Valuation}
\end{figure}

It may seem odd to compare LibTomMath to GnuPG since the math in GnuPG is only a small portion of the entire application.
However, LibTomMath was written with cryptography in mind.  It provides essentially all of the functions a cryptosystem
would require when working with large integers.

So it may feel tempting to just rip the math code out of GnuPG (or GnuMP where it was taken from originally) in your
own application but I think there are reasons not to.  While LibTomMath is slower than libraries such as GnuMP it is
not normally significantly slower.  On x86 machines the difference is normally a factor of two when performing modular
exponentiations.  It depends largely on the processor, compiler and the moduli being used.

Essentially the only time you would not use LibTomMath is when blazing speed is the primary concern.  However,
on the other side of the coin LibTomMath offers you a totally free (public domain) well structured math library
that is very flexible, complete and performs well in resource constrained environments.  Fast RSA for example can
be performed with as little as 8KB of ram for data (again depending on build options).

\chapter{Getting Started with LibTomMath}
\section{Building Programs}
In order to use LibTomMath you must include ``tommath.h'' and link against the appropriate library file (typically
libtommath.a).  There is no library initialization required and the entire library is thread safe.

\section{Return Codes}
There are three possible return codes a function may return.

\index{MP\_OKAY}\index{MP\_YES}\index{MP\_NO}\index{MP\_VAL}\index{MP\_MEM}
\begin{figure}[h!]
\begin{center}
\begin{small}
\begin{tabular}{|l|l|}
\hline \textbf{Code} & \textbf{Meaning} \\
\hline MP\_OKAY & The function succeeded. \\
\hline MP\_VAL  & The function input was invalid. \\
\hline MP\_MEM  & Heap memory exhausted. \\
\hline &\\
\hline MP\_YES  & Response is yes. \\
\hline MP\_NO   & Response is no. \\
\hline
\end{tabular}
\end{small}
\end{center}
\caption{Return Codes}
\end{figure}

The last two codes listed are not actually ``return'ed'' by a function.  They are placed in an integer (the caller must
provide the address of an integer it can store to) which the caller can access.  To convert one of the three return codes
to a string use the following function.

\index{mp\_error\_to\_string}
\begin{alltt}
char *mp_error_to_string(int code);
\end{alltt}

This will return a pointer to a string which describes the given error code.  It will not work for the return codes
MP\_YES and MP\_NO.

\section{Data Types}
The basic ``multiple precision integer'' type is known as the ``mp\_int'' within LibTomMath.  This data type is used to
organize all of the data required to manipulate the integer it represents.  Within LibTomMath it has been prototyped
as the following.

\index{mp\_int}
\begin{alltt}
typedef struct  \{
    int used, alloc, sign;
    mp_digit *dp;
\} mp_int;
\end{alltt}

Where ``mp\_digit'' is a data type that represents individual digits of the integer.  By default, an mp\_digit is the
ISO C ``unsigned long'' data type and each digit is $28-$bits long.  The mp\_digit type can be configured to suit other
platforms by defining the appropriate macros.

All LTM functions that use the mp\_int type will expect a pointer to mp\_int structure.  You must allocate memory to
hold the structure itself by yourself (whether off stack or heap it does not matter).  The very first thing that must be
done to use an mp\_int is that it must be initialized.

\section{Function Organization}

The arithmetic functions of the library are all organized to have the same style prototype.  That is source operands
are passed on the left and the destination is on the right.  For instance,

\begin{alltt}
mp_add(&a, &b, &c);       /* c = a + b */
mp_mul(&a, &a, &c);       /* c = a * a */
mp_div(&a, &b, &c, &d);   /* c = [a/b], d = a mod b */
\end{alltt}

Another feature of the way the functions have been implemented is that source operands can be destination operands as well.
For instance,

\begin{alltt}
mp_add(&a, &b, &b);       /* b = a + b */
mp_div(&a, &b, &a, &c);   /* a = [a/b], c = a mod b */
\end{alltt}

This allows operands to be re-used which can make programming simpler.

\section{Initialization}
\subsection{Single Initialization}
A single mp\_int can be initialized with the ``mp\_init'' function.

\index{mp\_init}
\begin{alltt}
int mp_init (mp_int * a);
\end{alltt}

This function expects a pointer to an mp\_int structure and will initialize the members of the structure so the mp\_int
represents the default integer which is zero.  If the functions returns MP\_OKAY then the mp\_int is ready to be used
by the other LibTomMath functions.

\begin{small} \begin{alltt}
int main(void)
\{
   mp_int number;
   int result;

   if ((result = mp_init(&number)) != MP_OKAY) \{
      printf("Error initializing the number.  \%s",
             mp_error_to_string(result));
      return EXIT_FAILURE;
   \}

   /* use the number */

   return EXIT_SUCCESS;
\}
\end{alltt} \end{small}

\subsection{Single Free}
When you are finished with an mp\_int it is ideal to return the heap it used back to the system.  The following function
provides this functionality.

\index{mp\_clear}
\begin{alltt}
void mp_clear (mp_int * a);
\end{alltt}

The function expects a pointer to a previously initialized mp\_int structure and frees the heap it uses.  It sets the
pointer\footnote{The ``dp'' member.} within the mp\_int to \textbf{NULL} which is used to prevent double free situations.
Is is legal to call mp\_clear() twice on the same mp\_int in a row.

\begin{small} \begin{alltt}
int main(void)
\{
   mp_int number;
   int result;

   if ((result = mp_init(&number)) != MP_OKAY) \{
      printf("Error initializing the number.  \%s",
             mp_error_to_string(result));
      return EXIT_FAILURE;
   \}

   /* use the number */

   /* We're done with it. */
   mp_clear(&number);

   return EXIT_SUCCESS;
\}
\end{alltt} \end{small}

\subsection{Multiple Initializations}
Certain algorithms require more than one large integer.  In these instances it is ideal to initialize all of the mp\_int
variables in an ``all or nothing'' fashion.  That is, they are either all initialized successfully or they are all
not initialized.

The  mp\_init\_multi() function provides this functionality.

\index{mp\_init\_multi} \index{mp\_clear\_multi}
\begin{alltt}
int mp_init_multi(mp_int *mp, ...);
\end{alltt}

It accepts a \textbf{NULL} terminated list of pointers to mp\_int structures.  It will attempt to initialize them all
at once.  If the function returns MP\_OKAY then all of the mp\_int variables are ready to use, otherwise none of them
are available for use.  A complementary mp\_clear\_multi() function allows multiple mp\_int variables to be free'd
from the heap at the same time.

\begin{small} \begin{alltt}
int main(void)
\{
   mp_int num1, num2, num3;
   int result;

   if ((result = mp_init_multi(&num1,
                               &num2,
                               &num3, NULL)) != MP\_OKAY) \{
      printf("Error initializing the numbers.  \%s",
             mp_error_to_string(result));
      return EXIT_FAILURE;
   \}

   /* use the numbers */

   /* We're done with them. */
   mp_clear_multi(&num1, &num2, &num3, NULL);

   return EXIT_SUCCESS;
\}
\end{alltt} \end{small}

\subsection{Other Initializers}
To initialized and make a copy of an mp\_int the mp\_init\_copy() function has been provided.

\index{mp\_init\_copy}
\begin{alltt}
int mp_init_copy (mp_int * a, mp_int * b);
\end{alltt}

This function will initialize $a$ and make it a copy of $b$ if all goes well.

\begin{small} \begin{alltt}
int main(void)
\{
   mp_int num1, num2;
   int result;

   /* initialize and do work on num1 ... */

   /* We want a copy of num1 in num2 now */
   if ((result = mp_init_copy(&num2, &num1)) != MP_OKAY) \{
     printf("Error initializing the copy.  \%s",
             mp_error_to_string(result));
      return EXIT_FAILURE;
   \}

   /* now num2 is ready and contains a copy of num1 */

   /* We're done with them. */
   mp_clear_multi(&num1, &num2, NULL);

   return EXIT_SUCCESS;
\}
\end{alltt} \end{small}

Another less common initializer is mp\_init\_size() which allows the user to initialize an mp\_int with a given
default number of digits.  By default, all initializers allocate \textbf{MP\_PREC} digits.  This function lets
you override this behaviour.

\index{mp\_init\_size}
\begin{alltt}
int mp_init_size (mp_int * a, int size);
\end{alltt}

The $size$ parameter must be greater than zero.  If the function succeeds the mp\_int $a$ will be initialized
to have $size$ digits (which are all initially zero).

\begin{small} \begin{alltt}
int main(void)
\{
   mp_int number;
   int result;

   /* we need a 60-digit number */
   if ((result = mp_init_size(&number, 60)) != MP_OKAY) \{
      printf("Error initializing the number.  \%s",
             mp_error_to_string(result));
      return EXIT_FAILURE;
   \}

   /* use the number */

   return EXIT_SUCCESS;
\}
\end{alltt} \end{small}

\section{Maintenance Functions}
\subsection{Clear Leading Zeros}

This is used to ensure that leading zero digits are trimmed and the leading "used" digit will be non-zero.
It also fixes the sign if there are no more leading digits.

\index{mp\_clamp}
\begin{alltt}
void mp_clamp(mp_int *a);
\end{alltt}

\subsection{Zero Out}

This function will set the ``bigint'' to zeros without changing the amount of allocated memory.

\index{mp\_zero}
\begin{alltt}
void mp_zero(mp_int *a);
\end{alltt}


\subsection{Reducing Memory Usage}
When an mp\_int is in a state where it won't be changed again\footnote{A Diffie-Hellman modulus for instance.} excess
digits can be removed to return memory to the heap with the mp\_shrink() function.

\index{mp\_shrink}
\begin{alltt}
int mp_shrink (mp_int * a);
\end{alltt}

This will remove excess digits of the mp\_int $a$.  If the operation fails the mp\_int should be intact without the
excess digits being removed.  Note that you can use a shrunk mp\_int in further computations, however, such operations
will require heap operations which can be slow.  It is not ideal to shrink mp\_int variables that you will further
modify in the system (unless you are seriously low on memory).

\begin{small} \begin{alltt}
int main(void)
\{
   mp_int number;
   int result;

   if ((result = mp_init(&number)) != MP_OKAY) \{
      printf("Error initializing the number.  \%s",
             mp_error_to_string(result));
      return EXIT_FAILURE;
   \}

   /* use the number [e.g. pre-computation]  */

   /* We're done with it for now. */
   if ((result = mp_shrink(&number)) != MP_OKAY) \{
      printf("Error shrinking the number.  \%s",
             mp_error_to_string(result));
      return EXIT_FAILURE;
   \}

   /* use it .... */


   /* we're done with it. */
   mp_clear(&number);

   return EXIT_SUCCESS;
\}
\end{alltt} \end{small}

\subsection{Adding additional digits}

Within the mp\_int structure are two parameters which control the limitations of the array of digits that represent
the integer the mp\_int is meant to equal.   The \textit{used} parameter dictates how many digits are significant, that is,
contribute to the value of the mp\_int.  The \textit{alloc} parameter dictates how many digits are currently available in
the array.  If you need to perform an operation that requires more digits you will have to mp\_grow() the mp\_int to
your desired size.

\index{mp\_grow}
\begin{alltt}
int mp_grow (mp_int * a, int size);
\end{alltt}

This will grow the array of digits of $a$ to $size$.  If the \textit{alloc} parameter is already bigger than
$size$ the function will not do anything.

\begin{small} \begin{alltt}
int main(void)
\{
   mp_int number;
   int result;

   if ((result = mp_init(&number)) != MP_OKAY) \{
      printf("Error initializing the number.  \%s",
             mp_error_to_string(result));
      return EXIT_FAILURE;
   \}

   /* use the number */

   /* We need to add 20 digits to the number  */
   if ((result = mp_grow(&number, number.alloc + 20)) != MP_OKAY) \{
      printf("Error growing the number.  \%s",
             mp_error_to_string(result));
      return EXIT_FAILURE;
   \}


   /* use the number */

   /* we're done with it. */
   mp_clear(&number);

   return EXIT_SUCCESS;
\}
\end{alltt} \end{small}

\chapter{Basic Operations}
\section{Copying}

A so called ``deep copy'', where new memory is allocated and all contents of $a$ are copied verbatim into $b$ such that $b = a$ at the end.

\index{mp\_copy}
\begin{alltt}
int mp_copy (mp_int * a, mp_int *b);
\end{alltt}

You can also just swap $a$ and $b$. It does the normal pointer changing with a temporary pointer variable, just that you do not have to.

\index{mp\_exch}
\begin{alltt}
void mp_exch (mp_int * a, mp_int *b);
\end{alltt}

\section{Bit Counting}

To get the position of the lowest bit set (LSB, the Lowest Significant Bit; the number of bits which are zero before the first zero bit )

\index{mp\_cnt\_lsb}
\begin{alltt}
int mp_cnt_lsb(const mp_int *a);
\end{alltt}

To get the position of the highest bit set (MSB, the Most Significant Bit; the number of bits in teh ``bignum'')

\index{mp\_count\_bits}
\begin{alltt}
int mp_count_bits(const mp_int *a);
\end{alltt}


\section{Small Constants}
Setting mp\_ints to small constants is a relatively common operation.  To accommodate these instances there are two
small constant assignment functions.  The first function is used to set a single digit constant while the second sets
an ISO C style ``unsigned long'' constant.  The reason for both functions is efficiency.  Setting a single digit is quick but the
domain of a digit can change (it's always at least $0 \ldots 127$).

\subsection{Single Digit}

Setting a single digit can be accomplished with the following function.

\index{mp\_set}
\begin{alltt}
void mp_set (mp_int * a, mp_digit b);
\end{alltt}

This will zero the contents of $a$ and make it represent an integer equal to the value of $b$.  Note that this
function has a return type of \textbf{void}.  It cannot cause an error so it is safe to assume the function
succeeded.

\begin{small} \begin{alltt}
int main(void)
\{
   mp_int number;
   int result;

   if ((result = mp_init(&number)) != MP_OKAY) \{
      printf("Error initializing the number.  \%s",
             mp_error_to_string(result));
      return EXIT_FAILURE;
   \}

   /* set the number to 5 */
   mp_set(&number, 5);

   /* we're done with it. */
   mp_clear(&number);

   return EXIT_SUCCESS;
\}
\end{alltt} \end{small}

\subsection{Long Constants}

To set a constant that is the size of an ISO C ``unsigned long'' and larger than a single digit the following function
can be used.

\index{mp\_set\_int}
\begin{alltt}
int mp_set_int (mp_int * a, unsigned long b);
\end{alltt}

This will assign the value of the 32-bit variable $b$ to the mp\_int $a$.  Unlike mp\_set() this function will always
accept a 32-bit input regardless of the size of a single digit.  However, since the value may span several digits
this function can fail if it runs out of heap memory.

To get the ``unsigned long'' copy of an mp\_int the following function can be used.

\index{mp\_get\_int}
\begin{alltt}
unsigned long mp_get_int (mp_int * a);
\end{alltt}

This will return the 32 least significant bits of the mp\_int $a$.

\begin{small} \begin{alltt}
int main(void)
\{
   mp_int number;
   int result;

   if ((result = mp_init(&number)) != MP_OKAY) \{
      printf("Error initializing the number.  \%s",
             mp_error_to_string(result));
      return EXIT_FAILURE;
   \}

   /* set the number to 654321 (note this is bigger than 127) */
   if ((result = mp_set_int(&number, 654321)) != MP_OKAY) \{
      printf("Error setting the value of the number.  \%s",
             mp_error_to_string(result));
      return EXIT_FAILURE;
   \}

   printf("number == \%lu", mp_get_int(&number));

   /* we're done with it. */
   mp_clear(&number);

   return EXIT_SUCCESS;
\}
\end{alltt} \end{small}

This should output the following if the program succeeds.

\begin{alltt}
number == 654321
\end{alltt}

\subsection{Long Constants - platform dependent}

\index{mp\_set\_long}
\begin{alltt}
int mp_set_long (mp_int * a, unsigned long b);
\end{alltt}

This will assign the value of the platform-dependent sized variable $b$ to the mp\_int $a$.

To get the ``unsigned long'' copy of an mp\_int the following function can be used.

\index{mp\_get\_long}
\begin{alltt}
unsigned long mp_get_long (mp_int * a);
\end{alltt}

This will return the least significant bits of the mp\_int $a$ that fit into an ``unsigned long''.

\subsection{Long Long Constants}

\index{mp\_set\_long\_long}
\begin{alltt}
int mp_set_long_long (mp_int * a, unsigned long long b);
\end{alltt}

This will assign the value of the 64-bit variable $b$ to the mp\_int $a$.

To get the ``unsigned long long'' copy of an mp\_int the following function can be used.

\index{mp\_get\_long\_long}
\begin{alltt}
unsigned long long mp_get_long_long (mp_int * a);
\end{alltt}

This will return the 64 least significant bits of the mp\_int $a$.

\subsection{Initialize and Setting Constants}
To both initialize and set small constants the following two functions are available.
\index{mp\_init\_set} \index{mp\_init\_set\_int}
\begin{alltt}
int mp_init_set (mp_int * a, mp_digit b);
int mp_init_set_int (mp_int * a, unsigned long b);
\end{alltt}

Both functions work like the previous counterparts except they first mp\_init $a$ before setting the values.

\begin{alltt}
int main(void)
\{
   mp_int number1, number2;
   int    result;

   /* initialize and set a single digit */
   if ((result = mp_init_set(&number1, 100)) != MP_OKAY) \{
      printf("Error setting number1: \%s",
             mp_error_to_string(result));
      return EXIT_FAILURE;
   \}

   /* initialize and set a long */
   if ((result = mp_init_set_int(&number2, 1023)) != MP_OKAY) \{
      printf("Error setting number2: \%s",
             mp_error_to_string(result));
      return EXIT_FAILURE;
   \}

   /* display */
   printf("Number1, Number2 == \%lu, \%lu",
          mp_get_int(&number1), mp_get_int(&number2));

   /* clear */
   mp_clear_multi(&number1, &number2, NULL);

   return EXIT_SUCCESS;
\}
\end{alltt}

If this program succeeds it shall output.
\begin{alltt}
Number1, Number2 == 100, 1023
\end{alltt}

\section{Comparisons}

Comparisons in LibTomMath are always performed in a ``left to right'' fashion.  There are three possible return codes
for any comparison.

\index{MP\_GT} \index{MP\_EQ} \index{MP\_LT}
\begin{figure}[h]
\begin{center}
\begin{tabular}{|c|c|}
\hline \textbf{Result Code} & \textbf{Meaning} \\
\hline MP\_GT & $a > b$ \\
\hline MP\_EQ & $a = b$ \\
\hline MP\_LT & $a < b$ \\
\hline
\end{tabular}
\end{center}
\caption{Comparison Codes for $a, b$}
\label{fig:CMP}
\end{figure}

In figure \ref{fig:CMP} two integers $a$ and $b$ are being compared.  In this case $a$ is said to be ``to the left'' of
$b$.

\subsection{Unsigned comparison}

An unsigned comparison considers only the digits themselves and not the associated \textit{sign} flag of the
mp\_int structures.  This is analogous to an absolute comparison.  The function mp\_cmp\_mag() will compare two
mp\_int variables based on their digits only.

\index{mp\_cmp\_mag}
\begin{alltt}
int mp_cmp_mag(mp_int * a, mp_int * b);
\end{alltt}
This will compare $a$ to $b$ placing $a$ to the left of $b$.  This function cannot fail and will return one of the
three compare codes listed in figure \ref{fig:CMP}.

\begin{small} \begin{alltt}
int main(void)
\{
   mp_int number1, number2;
   int result;

   if ((result = mp_init_multi(&number1, &number2, NULL)) != MP_OKAY) \{
      printf("Error initializing the numbers.  \%s",
             mp_error_to_string(result));
      return EXIT_FAILURE;
   \}

   /* set the number1 to 5 */
   mp_set(&number1, 5);

   /* set the number2 to -6 */
   mp_set(&number2, 6);
   if ((result = mp_neg(&number2, &number2)) != MP_OKAY) \{
      printf("Error negating number2.  \%s",
             mp_error_to_string(result));
      return EXIT_FAILURE;
   \}

   switch(mp_cmp_mag(&number1, &number2)) \{
       case MP_GT:  printf("|number1| > |number2|"); break;
       case MP_EQ:  printf("|number1| = |number2|"); break;
       case MP_LT:  printf("|number1| < |number2|"); break;
   \}

   /* we're done with it. */
   mp_clear_multi(&number1, &number2, NULL);

   return EXIT_SUCCESS;
\}
\end{alltt} \end{small}

If this program\footnote{This function uses the mp\_neg() function which is discussed in section \ref{sec:NEG}.} completes
successfully it should print the following.

\begin{alltt}
|number1| < |number2|
\end{alltt}

This is because $\vert -6 \vert = 6$ and obviously $5 < 6$.

\subsection{Signed comparison}

To compare two mp\_int variables based on their signed value the mp\_cmp() function is provided.

\index{mp\_cmp}
\begin{alltt}
int mp_cmp(mp_int * a, mp_int * b);
\end{alltt}

This will compare $a$ to the left of $b$.  It will first compare the signs of the two mp\_int variables.  If they
differ it will return immediately based on their signs.  If the signs are equal then it will compare the digits
individually.  This function will return one of the compare conditions codes listed in figure \ref{fig:CMP}.

\begin{small} \begin{alltt}
int main(void)
\{
   mp_int number1, number2;
   int result;

   if ((result = mp_init_multi(&number1, &number2, NULL)) != MP_OKAY) \{
      printf("Error initializing the numbers.  \%s",
             mp_error_to_string(result));
      return EXIT_FAILURE;
   \}

   /* set the number1 to 5 */
   mp_set(&number1, 5);

   /* set the number2 to -6 */
   mp_set(&number2, 6);
   if ((result = mp_neg(&number2, &number2)) != MP_OKAY) \{
      printf("Error negating number2.  \%s",
             mp_error_to_string(result));
      return EXIT_FAILURE;
   \}

   switch(mp_cmp(&number1, &number2)) \{
       case MP_GT:  printf("number1 > number2"); break;
       case MP_EQ:  printf("number1 = number2"); break;
       case MP_LT:  printf("number1 < number2"); break;
   \}

   /* we're done with it. */
   mp_clear_multi(&number1, &number2, NULL);

   return EXIT_SUCCESS;
\}
\end{alltt} \end{small}

If this program\footnote{This function uses the mp\_neg() function which is discussed in section \ref{sec:NEG}.} completes
successfully it should print the following.

\begin{alltt}
number1 > number2
\end{alltt}

\subsection{Single Digit}

To compare a single digit against an mp\_int the following function has been provided.

\index{mp\_cmp\_d}
\begin{alltt}
int mp_cmp_d(mp_int * a, mp_digit b);
\end{alltt}

This will compare $a$ to the left of $b$ using a signed comparison.  Note that it will always treat $b$ as
positive.  This function is rather handy when you have to compare against small values such as $1$ (which often
comes up in cryptography).  The function cannot fail and will return one of the tree compare condition codes
listed in figure \ref{fig:CMP}.


\begin{small} \begin{alltt}
int main(void)
\{
   mp_int number;
   int result;

   if ((result = mp_init(&number)) != MP_OKAY) \{
      printf("Error initializing the number.  \%s",
             mp_error_to_string(result));
      return EXIT_FAILURE;
   \}

   /* set the number to 5 */
   mp_set(&number, 5);

   switch(mp_cmp_d(&number, 7)) \{
       case MP_GT:  printf("number > 7"); break;
       case MP_EQ:  printf("number = 7"); break;
       case MP_LT:  printf("number < 7"); break;
   \}

   /* we're done with it. */
   mp_clear(&number);

   return EXIT_SUCCESS;
\}
\end{alltt} \end{small}

If this program functions properly it will print out the following.

\begin{alltt}
number < 7
\end{alltt}

\section{Logical Operations}

Logical operations are operations that can be performed either with simple shifts or boolean operators such as
AND, XOR and OR directly.  These operations are very quick.

\subsection{Multiplication by two}

Multiplications and divisions by any power of two can be performed with quick logical shifts either left or
right depending on the operation.

When multiplying or dividing by two a special case routine can be used which are as follows.
\index{mp\_mul\_2} \index{mp\_div\_2}
\begin{alltt}
int mp_mul_2(mp_int * a, mp_int * b);
int mp_div_2(mp_int * a, mp_int * b);
\end{alltt}

The former will assign twice $a$ to $b$ while the latter will assign half $a$ to $b$.  These functions are fast
since the shift counts and masks are hardcoded into the routines.

\begin{small} \begin{alltt}
int main(void)
\{
   mp_int number;
   int result;

   if ((result = mp_init(&number)) != MP_OKAY) \{
      printf("Error initializing the number.  \%s",
             mp_error_to_string(result));
      return EXIT_FAILURE;
   \}

   /* set the number to 5 */
   mp_set(&number, 5);

   /* multiply by two */
   if ((result = mp\_mul\_2(&number, &number)) != MP_OKAY) \{
      printf("Error multiplying the number.  \%s",
             mp_error_to_string(result));
      return EXIT_FAILURE;
   \}
   switch(mp_cmp_d(&number, 7)) \{
       case MP_GT:  printf("2*number > 7"); break;
       case MP_EQ:  printf("2*number = 7"); break;
       case MP_LT:  printf("2*number < 7"); break;
   \}

   /* now divide by two */
   if ((result = mp\_div\_2(&number, &number)) != MP_OKAY) \{
      printf("Error dividing the number.  \%s",
             mp_error_to_string(result));
      return EXIT_FAILURE;
   \}
   switch(mp_cmp_d(&number, 7)) \{
       case MP_GT:  printf("2*number/2 > 7"); break;
       case MP_EQ:  printf("2*number/2 = 7"); break;
       case MP_LT:  printf("2*number/2 < 7"); break;
   \}

   /* we're done with it. */
   mp_clear(&number);

   return EXIT_SUCCESS;
\}
\end{alltt} \end{small}

If this program is successful it will print out the following text.

\begin{alltt}
2*number > 7
2*number/2 < 7
\end{alltt}

Since $10 > 7$ and $5 < 7$.

To multiply by a power of two the following function can be used.

\index{mp\_mul\_2d}
\begin{alltt}
int mp_mul_2d(mp_int * a, int b, mp_int * c);
\end{alltt}

This will multiply $a$ by $2^b$ and store the result in ``c''.  If the value of $b$ is less than or equal to
zero the function will copy $a$ to ``c'' without performing any further actions.  The multiplication itself
is implemented as a right-shift operation of $a$ by $b$ bits.

To divide by a power of two use the following.

\index{mp\_div\_2d}
\begin{alltt}
int mp_div_2d (mp_int * a, int b, mp_int * c, mp_int * d);
\end{alltt}
Which will divide $a$ by $2^b$, store the quotient in ``c'' and the remainder in ``d'.  If $b \le 0$ then the
function simply copies $a$ over to ``c'' and zeros $d$.  The variable $d$ may be passed as a \textbf{NULL}
value to signal that the remainder is not desired.  The division itself is implemented as a left-shift
operation of $a$ by $b$ bits.

\index{mp\_tc\_div\_2d}\label{arithrightshift}
\begin{alltt}
int mp_tc_div_2d (mp_int * a, int b, mp_int * c, mp_int * d);
\end{alltt}
The two-complement version of the function above. This can be used to implement arbitrary-precision two-complement integers together with the two-complement bit-wise operations at page \ref{tcbitwiseops}.


It is also not very uncommon to need just the power of two $2^b$;  for example the startvalue for the Newton method.

\index{mp\_2expt}
\begin{alltt}
int mp_2expt(mp_int *a, int b);
\end{alltt}
It is faster than doing it by shifting $1$ with \texttt{mp\_mul\_2d}.

\subsection{Polynomial Basis Operations}

Strictly speaking the organization of the integers within the mp\_int structures is what is known as a
``polynomial basis''.  This simply means a field element is stored by divisions of a radix.  For example, if
$f(x) = \sum_{i=0}^{k} y_ix^k$ for any vector $\vec y$ then the array of digits in $\vec y$ are said to be
the polynomial basis representation of $z$ if $f(\beta) = z$ for a given radix $\beta$.

To multiply by the polynomial $g(x) = x$ all you have todo is shift the digits of the basis left one place.  The
following function provides this operation.

\index{mp\_lshd}
\begin{alltt}
int mp_lshd (mp_int * a, int b);
\end{alltt}

This will multiply $a$ in place by $x^b$ which is equivalent to shifting the digits left $b$ places and inserting zeros
in the least significant digits.  Similarly to divide by a power of $x$ the following function is provided.

\index{mp\_rshd}
\begin{alltt}
void mp_rshd (mp_int * a, int b)
\end{alltt}
This will divide $a$ in place by $x^b$ and discard the remainder.  This function cannot fail as it performs the operations
in place and no new digits are required to complete it.

\subsection{AND, OR, XOR and COMPLEMENT Operations}

While AND, OR and XOR operations are not typical ``bignum functions'' they can be useful in several instances.  The
three functions are prototyped as follows.

\index{mp\_or} \index{mp\_and} \index{mp\_xor}
\begin{alltt}
int mp_or  (mp_int * a, mp_int * b, mp_int * c);
int mp_and (mp_int * a, mp_int * b, mp_int * c);
int mp_xor (mp_int * a, mp_int * b, mp_int * c);
\end{alltt}

Which compute $c = a \odot b$ where $\odot$ is one of OR, AND or XOR.

The following four functions allow implementing arbitrary-precision two-complement numbers.

\index{mp\_tc\_or} \index{mp\_tc\_and} \index{mp\_tc\_xor} \index{mp\_complement} \label{tcbitwiseops}
\begin{alltt}
int mp_tc_or  (mp_int * a, mp_int * b, mp_int * c);
int mp_tc_and (mp_int * a, mp_int * b, mp_int * c);
int mp_tc_xor (mp_int * a, mp_int * b, mp_int * c);
int mp_complement(const mp_int *a, mp_int *b);
\end{alltt}

They compute $c = a \odot b$ as above if both $a$ and $b$ are positive. Negative values are converted into their two-complement representations first. The function \texttt{mp\_complement} computes a two-complement $b = \sim a$.


\subsection{Bit Picking}
\index{mp\_get\_bit}
\begin{alltt}
int mp_get_bit(mp_int *a, int b)
\end{alltt}

Pick a bit: returns \texttt{MP\_YES} if the bit at position $b$ (0-index) is set, that is if it is 1 (one), \texttt{MP\_NO}
if the bit is 0 (zero) and \texttt{MP\_VAL} if $b < 0$.

\section{Addition and Subtraction}

To compute an addition or subtraction the following two functions can be used.

\index{mp\_add} \index{mp\_sub}
\begin{alltt}
int mp_add (mp_int * a, mp_int * b, mp_int * c);
int mp_sub (mp_int * a, mp_int * b, mp_int * c)
\end{alltt}

Which perform $c = a \odot b$ where $\odot$ is one of signed addition or subtraction.  The operations are fully sign
aware.

\section{Sign Manipulation}
\subsection{Negation}
\label{sec:NEG}
Simple integer negation can be performed with the following.

\index{mp\_neg}
\begin{alltt}
int mp_neg (mp_int * a, mp_int * b);
\end{alltt}

Which assigns $-a$ to $b$.

\subsection{Absolute}
Simple integer absolutes can be performed with the following.

\index{mp\_abs}
\begin{alltt}
int mp_abs (mp_int * a, mp_int * b);
\end{alltt}

Which assigns $\vert a \vert$ to $b$.

\section{Integer Division and Remainder}
To perform a complete and general integer division with remainder use the following function.

\index{mp\_div}
\begin{alltt}
int mp_div (mp_int * a, mp_int * b, mp_int * c, mp_int * d);
\end{alltt}

This divides $a$ by $b$ and stores the quotient in $c$ and $d$.  The signed quotient is computed such that
$bc + d = a$.  Note that either of $c$ or $d$ can be set to \textbf{NULL} if their value is not required.  If
$b$ is zero the function returns \textbf{MP\_VAL}.


\chapter{Multiplication and Squaring}
\section{Multiplication}
A full signed integer multiplication can be performed with the following.
\index{mp\_mul}
\begin{alltt}
int mp_mul (mp_int * a, mp_int * b, mp_int * c);
\end{alltt}
Which assigns the full signed product $ab$ to $c$.  This function actually breaks into one of four cases which are
specific multiplication routines optimized for given parameters.  First there are the Toom-Cook multiplications which
should only be used with very large inputs.  This is followed by the Karatsuba multiplications which are for moderate
sized inputs.  Then followed by the Comba and baseline multipliers.

Fortunately for the developer you don't really need to know this unless you really want to fine tune the system.  mp\_mul()
will determine on its own\footnote{Some tweaking may be required.} what routine to use automatically when it is called.

\begin{alltt}
int main(void)
\{
   mp_int number1, number2;
   int result;

   /* Initialize the numbers */
   if ((result = mp_init_multi(&number1,
                               &number2, NULL)) != MP_OKAY) \{
      printf("Error initializing the numbers.  \%s",
             mp_error_to_string(result));
      return EXIT_FAILURE;
   \}

   /* set the terms */
   if ((result = mp_set_int(&number, 257)) != MP_OKAY) \{
      printf("Error setting number1.  \%s",
             mp_error_to_string(result));
      return EXIT_FAILURE;
   \}

   if ((result = mp_set_int(&number2, 1023)) != MP_OKAY) \{
      printf("Error setting number2.  \%s",
             mp_error_to_string(result));
      return EXIT_FAILURE;
   \}

   /* multiply them */
   if ((result = mp_mul(&number1, &number2,
                        &number1)) != MP_OKAY) \{
      printf("Error multiplying terms.  \%s",
             mp_error_to_string(result));
      return EXIT_FAILURE;
   \}

   /* display */
   printf("number1 * number2 == \%lu", mp_get_int(&number1));

   /* free terms and return */
   mp_clear_multi(&number1, &number2, NULL);

   return EXIT_SUCCESS;
\}
\end{alltt}

If this program succeeds it shall output the following.

\begin{alltt}
number1 * number2 == 262911
\end{alltt}

\section{Squaring}
Since squaring can be performed faster than multiplication it is performed it's own function instead of just using
mp\_mul().

\index{mp\_sqr}
\begin{alltt}
int mp_sqr (mp_int * a, mp_int * b);
\end{alltt}

Will square $a$ and store it in $b$.  Like the case of multiplication there are four different squaring
algorithms all which can be called from mp\_sqr().  It is ideal to use mp\_sqr over mp\_mul when squaring terms because
of the speed difference.

\section{Tuning Polynomial Basis Routines}

Both of the Toom-Cook and Karatsuba multiplication algorithms are faster than the traditional $O(n^2)$ approach that
the Comba and baseline algorithms use.  At $O(n^{1.464973})$ and $O(n^{1.584962})$ running times respectively they require
considerably less work.  For example, a 10000-digit multiplication would take roughly 724,000 single precision
multiplications with Toom-Cook or 100,000,000 single precision multiplications with the standard Comba (a factor
of 138).

So why not always use Karatsuba or Toom-Cook?   The simple answer is that they have so much overhead that they're not
actually faster than Comba until you hit distinct  ``cutoff'' points.  For Karatsuba with the default configuration,
GCC 3.3.1 and an Athlon XP processor the cutoff point is roughly 110 digits (about 70 for the Intel P4).  That is, at
110 digits Karatsuba and Comba multiplications just about break even and for 110+ digits Karatsuba is faster.

Toom-Cook has incredible overhead and is probably only useful for very large inputs.  So far no known cutoff points
exist and for the most part I just set the cutoff points very high to make sure they're not called.

A demo program in the ``etc/'' directory of the project called ``tune.c'' can be used to find the cutoff points.  This
can be built with GCC as follows

\begin{alltt}
make XXX
\end{alltt}
Where ``XXX'' is one of the following entries from the table \ref{fig:tuning}.

\begin{figure}[h]
\begin{center}
\begin{small}
\begin{tabular}{|l|l|}
\hline \textbf{Value of XXX} & \textbf{Meaning} \\
\hline tune & Builds portable tuning application \\
\hline tune86 & Builds x86 (pentium and up) program for COFF \\
\hline tune86c & Builds x86 program for Cygwin \\
\hline tune86l & Builds x86 program for Linux (ELF format) \\
\hline
\end{tabular}
\end{small}
\end{center}
\caption{Build Names for Tuning Programs}
\label{fig:tuning}
\end{figure}

When the program is running it will output a series of measurements for different cutoff points.  It will first find
good Karatsuba squaring and multiplication points.  Then it proceeds to find Toom-Cook points.  Note that the Toom-Cook
tuning takes a very long time as the cutoff points are likely to be very high.

\chapter{Modular Reduction}

Modular reduction is process of taking the remainder of one quantity divided by another.  Expressed
as (\ref{eqn:mod}) the modular reduction is equivalent to the remainder of $b$ divided by $c$.

\begin{equation}
a \equiv b \mbox{ (mod }c\mbox{)}
\label{eqn:mod}
\end{equation}

Of particular interest to cryptography are reductions where $b$ is limited to the range $0 \le b < c^2$ since particularly
fast reduction algorithms can be written for the limited range.

Note that one of the four optimized reduction algorithms are automatically chosen in the modular exponentiation
algorithm mp\_exptmod when an appropriate modulus is detected.

\section{Straight Division}
In order to effect an arbitrary modular reduction the following algorithm is provided.

\index{mp\_mod}
\begin{alltt}
int mp_mod(mp_int *a, mp_int *b, mp_int *c);
\end{alltt}

This reduces $a$ modulo $b$ and stores the result in $c$.  The sign of $c$ shall agree with the sign
of $b$.  This algorithm accepts an input $a$ of any range and is not limited by $0 \le a < b^2$.

\section{Barrett Reduction}

Barrett reduction is a generic optimized reduction algorithm that requires pre--computation to achieve
a decent speedup over straight division.  First a $\mu$ value must be precomputed with the following function.

\index{mp\_reduce\_setup}
\begin{alltt}
int mp_reduce_setup(mp_int *a, mp_int *b);
\end{alltt}

Given a modulus in $b$ this produces the required $\mu$ value in $a$.  For any given modulus this only has to
be computed once.  Modular reduction can now be performed with the following.

\index{mp\_reduce}
\begin{alltt}
int mp_reduce(mp_int *a, mp_int *b, mp_int *c);
\end{alltt}

This will reduce $a$ in place modulo $b$ with the precomputed $\mu$ value in $c$.  $a$ must be in the range
$0 \le a < b^2$.

\begin{alltt}
int main(void)
\{
   mp_int   a, b, c, mu;
   int      result;

   /* initialize a,b to desired values, mp_init mu,
    * c and set c to 1...we want to compute a^3 mod b
    */

   /* get mu value */
   if ((result = mp_reduce_setup(&mu, b)) != MP_OKAY) \{
      printf("Error getting mu.  \%s",
             mp_error_to_string(result));
      return EXIT_FAILURE;
   \}

   /* square a to get c = a^2 */
   if ((result = mp_sqr(&a, &c)) != MP_OKAY) \{
      printf("Error squaring.  \%s",
             mp_error_to_string(result));
      return EXIT_FAILURE;
   \}

   /* now reduce `c' modulo b */
   if ((result = mp_reduce(&c, &b, &mu)) != MP_OKAY) \{
      printf("Error reducing.  \%s",
             mp_error_to_string(result));
      return EXIT_FAILURE;
   \}

   /* multiply a to get c = a^3 */
   if ((result = mp_mul(&a, &c, &c)) != MP_OKAY) \{
      printf("Error reducing.  \%s",
             mp_error_to_string(result));
      return EXIT_FAILURE;
   \}

   /* now reduce `c' modulo b  */
   if ((result = mp_reduce(&c, &b, &mu)) != MP_OKAY) \{
      printf("Error reducing.  \%s",
             mp_error_to_string(result));
      return EXIT_FAILURE;
   \}

   /* c now equals a^3 mod b */

   return EXIT_SUCCESS;
\}
\end{alltt}

This program will calculate $a^3 \mbox{ mod }b$ if all the functions succeed.

\section{Montgomery Reduction}

Montgomery is a specialized reduction algorithm for any odd moduli.  Like Barrett reduction a pre--computation
step is required.  This is accomplished with the following.

\index{mp\_montgomery\_setup}
\begin{alltt}
int mp_montgomery_setup(mp_int *a, mp_digit *mp);
\end{alltt}

For the given odd moduli $a$ the pre--computation value is placed in $mp$.  The reduction is computed with the
following.

\index{mp\_montgomery\_reduce}
\begin{alltt}
int mp_montgomery_reduce(mp_int *a, mp_int *m, mp_digit mp);
\end{alltt}
This reduces $a$ in place modulo $m$ with the pre--computed value $mp$.   $a$ must be in the range
$0 \le a < b^2$.

Montgomery reduction is faster than Barrett reduction for moduli smaller than the ``comba'' limit.  With the default
setup for instance, the limit is $127$ digits ($3556$--bits).   Note that this function is not limited to
$127$ digits just that it falls back to a baseline algorithm after that point.

An important observation is that this reduction does not return $a \mbox{ mod }m$ but $aR^{-1} \mbox{ mod }m$
where $R = \beta^n$, $n$ is the n number of digits in $m$ and $\beta$ is radix used (default is $2^{28}$).

To quickly calculate $R$ the following function was provided.

\index{mp\_montgomery\_calc\_normalization}
\begin{alltt}
int mp_montgomery_calc_normalization(mp_int *a, mp_int *b);
\end{alltt}
Which calculates $a = R$ for the odd moduli $b$ without using multiplication or division.

The normal modus operandi for Montgomery reductions is to normalize the integers before entering the system.  For
example, to calculate $a^3 \mbox { mod }b$ using Montgomery reduction the value of $a$ can be normalized by
multiplying it by $R$.  Consider the following code snippet.

\begin{alltt}
int main(void)
\{
   mp_int   a, b, c, R;
   mp_digit mp;
   int      result;

   /* initialize a,b to desired values,
    * mp_init R, c and set c to 1....
    */

   /* get normalization */
   if ((result = mp_montgomery_calc_normalization(&R, b)) != MP_OKAY) \{
      printf("Error getting norm.  \%s",
             mp_error_to_string(result));
      return EXIT_FAILURE;
   \}

   /* get mp value */
   if ((result = mp_montgomery_setup(&c, &mp)) != MP_OKAY) \{
      printf("Error setting up montgomery.  \%s",
             mp_error_to_string(result));
      return EXIT_FAILURE;
   \}

   /* normalize `a' so now a is equal to aR */
   if ((result = mp_mulmod(&a, &R, &b, &a)) != MP_OKAY) \{
      printf("Error computing aR.  \%s",
             mp_error_to_string(result));
      return EXIT_FAILURE;
   \}

   /* square a to get c = a^2R^2 */
   if ((result = mp_sqr(&a, &c)) != MP_OKAY) \{
      printf("Error squaring.  \%s",
             mp_error_to_string(result));
      return EXIT_FAILURE;
   \}

   /* now reduce `c' back down to c = a^2R^2 * R^-1 == a^2R */
   if ((result = mp_montgomery_reduce(&c, &b, mp)) != MP_OKAY) \{
      printf("Error reducing.  \%s",
             mp_error_to_string(result));
      return EXIT_FAILURE;
   \}

   /* multiply a to get c = a^3R^2 */
   if ((result = mp_mul(&a, &c, &c)) != MP_OKAY) \{
      printf("Error reducing.  \%s",
             mp_error_to_string(result));
      return EXIT_FAILURE;
   \}

   /* now reduce `c' back down to c = a^3R^2 * R^-1 == a^3R */
   if ((result = mp_montgomery_reduce(&c, &b, mp)) != MP_OKAY) \{
      printf("Error reducing.  \%s",
             mp_error_to_string(result));
      return EXIT_FAILURE;
   \}

   /* now reduce (again) `c' back down to c = a^3R * R^-1 == a^3 */
   if ((result = mp_montgomery_reduce(&c, &b, mp)) != MP_OKAY) \{
      printf("Error reducing.  \%s",
             mp_error_to_string(result));
      return EXIT_FAILURE;
   \}

   /* c now equals a^3 mod b */

   return EXIT_SUCCESS;
\}
\end{alltt}

This particular example does not look too efficient but it demonstrates the point of the algorithm.  By
normalizing the inputs the reduced results are always of the form $aR$ for some variable $a$.  This allows
a single final reduction to correct for the normalization and the fast reduction used within the algorithm.

For more details consider examining the file \textit{bn\_mp\_exptmod\_fast.c}.

\section{Restricted Diminished Radix}

``Diminished Radix'' reduction refers to reduction with respect to moduli that are amenable to simple
digit shifting and small multiplications.  In this case the ``restricted'' variant refers to moduli of the
form $\beta^k - p$ for some $k \ge 0$ and $0 < p < \beta$ where $\beta$ is the radix (default to $2^{28}$).

As in the case of Montgomery reduction there is a pre--computation phase required for a given modulus.

\index{mp\_dr\_setup}
\begin{alltt}
void mp_dr_setup(mp_int *a, mp_digit *d);
\end{alltt}

This computes the value required for the modulus $a$ and stores it in $d$.  This function cannot fail
and does not return any error codes.  After the pre--computation a reduction can be performed with the
following.

\index{mp\_dr\_reduce}
\begin{alltt}
int mp_dr_reduce(mp_int *a, mp_int *b, mp_digit mp);
\end{alltt}

This reduces $a$ in place modulo $b$ with the pre--computed value $mp$.  $b$ must be of a restricted
diminished radix form and $a$ must be in the range $0 \le a < b^2$.  Diminished radix reductions are
much faster than both Barrett and Montgomery reductions as they have a much lower asymptotic running time.

Since the moduli are restricted this algorithm is not particularly useful for something like Rabin, RSA or
BBS cryptographic purposes.  This reduction algorithm is useful for Diffie-Hellman and ECC where fixed
primes are acceptable.

Note that unlike Montgomery reduction there is no normalization process.  The result of this function is
equal to the correct residue.

\section{Unrestricted Diminished Radix}

Unrestricted reductions work much like the restricted counterparts except in this case the moduli is of the
form $2^k - p$ for $0 < p < \beta$.  In this sense the unrestricted reductions are more flexible as they
can be applied to a wider range of numbers.

\index{mp\_reduce\_2k\_setup}
\begin{alltt}
int mp_reduce_2k_setup(mp_int *a, mp_digit *d);
\end{alltt}

This will compute the required $d$ value for the given moduli $a$.

\index{mp\_reduce\_2k}
\begin{alltt}
int mp_reduce_2k(mp_int *a, mp_int *n, mp_digit d);
\end{alltt}

This will reduce $a$ in place modulo $n$ with the pre--computed value $d$.  From my experience this routine is
slower than mp\_dr\_reduce but faster for most moduli sizes than the Montgomery reduction.

\section{Combined Modular Reduction}

Some of the combinations of an arithmetic operations followed by a modular reduction can be done in a faster way. The ones implemented are:

Addition $d = (a + b) \mod c$
\index{mp\_addmod}
\begin{alltt}
int mp_addmod(const mp_int *a, const mp_int *b, const mp_int *c, mp_int *d);
\end{alltt}

Subtraction  $d = (a - b) \mod c$
\begin{alltt}
int mp_submod(const mp_int *a, const mp_int *b, const mp_int *c, mp_int *d);
\end{alltt}

Multiplication $d = (ab) \mod c$
\begin{alltt}
int mp_mulmod(const mp_int *a, const mp_int *b, const mp_int *c, mp_int *d);
\end{alltt}

Squaring  $d = (a^2) \mod c$
\begin{alltt}
int mp_sqrmod(const mp_int *a, const mp_int *b, const mp_int *c, mp_int *d);
\end{alltt}



\chapter{Exponentiation}
\section{Single Digit Exponentiation}
\index{mp\_expt\_d\_ex}
\begin{alltt}
int mp_expt_d_ex (mp_int * a, mp_digit b, mp_int * c, int fast)
\end{alltt}
This function computes $c = a^b$.

With parameter \textit{fast} set to $0$ the old version of the algorithm is used,
when \textit{fast} is $1$, a faster but not statically timed version of the algorithm is used.

The old version uses a simple binary left-to-right algorithm.
It is faster than repeated multiplications by $a$ for all values of $b$ greater than three.

The new version uses a binary right-to-left algorithm.

The difference between the old and the new version is that the old version always
executes $DIGIT\_BIT$ iterations. The new algorithm executes only $n$ iterations
where $n$ is equal to the position of the highest bit that is set in $b$.

\index{mp\_expt\_d}
\begin{alltt}
int mp_expt_d (mp_int * a, mp_digit b, mp_int * c)
\end{alltt}
mp\_expt\_d(a, b, c) is a wrapper function to mp\_expt\_d\_ex(a, b, c, 0).

\index{mp\_expt}
\begin{alltt}
int mp_expt (mp_int * a, const mp_int *b, mp_int * c)
\end{alltt}
Same as \texttt{mp\_expt\_d(a, b, c)} except that \texttt{b} is a \texttt{mp\_int}, Useful for small \texttt{MP\_xBIT}.

\section{Modular Exponentiation}
\index{mp\_exptmod}
\begin{alltt}
int mp_exptmod (mp_int * G, mp_int * X, mp_int * P, mp_int * Y)
\end{alltt}
This computes $Y \equiv G^X \mbox{ (mod }P\mbox{)}$ using a variable width sliding window algorithm.  This function
will automatically detect the fastest modular reduction technique to use during the operation.  For negative values of
$X$ the operation is performed as $Y \equiv (G^{-1} \mbox{ mod }P)^{\vert X \vert} \mbox{ (mod }P\mbox{)}$ provided that
$gcd(G, P) = 1$.

This function is actually a shell around the two internal exponentiation functions.  This routine will automatically
detect when Barrett, Montgomery, Restricted and Unrestricted Diminished Radix based exponentiation can be used.  Generally
moduli of the a ``restricted diminished radix'' form lead to the fastest modular exponentiations.  Followed by Montgomery
and the other two algorithms.

\section{Modulus a Power of Two}
\index{mp\_mod\_2d}
\begin{alltt}
int mp_mod_2d(const mp_int *a, int b, mp_int *c)
\end{alltt}
It calculates $c = a \mod 2^b$.

\section{Root Finding}
\index{mp\_n\_root}
\begin{alltt}
int mp_n_root (mp_int * a, mp_digit b, mp_int * c)
\end{alltt}
This computes $c = a^{1/b}$ such that $c^b \le a$ and $(c+1)^b > a$.  The implementation of this function is not
ideal for values of $b$ greater than three.  It will work but become very slow.  So unless you are working with very small
numbers (less than 1000 bits) I'd avoid $b > 3$ situations.  Will return a positive root only for even roots and return
a root with the sign of the input for odd roots.  For example, performing $4^{1/2}$ will return $2$ whereas $(-8)^{1/3}$
will return $-2$.

This algorithm uses the ``Newton Approximation'' method and will converge on the correct root fairly quickly.  Since
the algorithm requires raising $a$ to the power of $b$ it is not ideal to attempt to find roots for large
values of $b$.  If particularly large roots are required then a factor method could be used instead.  For example,
$a^{1/16}$ is equivalent to $\left (a^{1/4} \right)^{1/4}$ or simply
$\left ( \left ( \left ( a^{1/2} \right )^{1/2} \right )^{1/2} \right )^{1/2}$


The square root  $c = a^{1/2}$ (with the same conditions $c^2 \le a$ and $(c+1)^2 > a$) is implemented with a faster algorithm.

\index{mp\_sqrt}
\begin{alltt}
int mp_sqrt (mp_int * a, mp_digit b, mp_int * c)
\end{alltt}


\chapter{Logarithm}
\section{Integer Logarithm}
A logarithm function for positive integer input \texttt{a, base} computing  $\floor{\log_bx}$ such that $(\ilog_bx)^b \le x$.
\index{mp\_ilogb}
\begin{alltt}
int mp_ilogb(mp_int *a, mp_digit base, mp_int *c)
\end{alltt}
\subsection{Example}
\begin{alltt}
#include <stdlib.h>
#include <stdio.h>
#include <errno.h>
/* Must be defined before including tommath.h */
#define LTM_USE_EXTRA_FUNCTIONS
#include <tommath.h>

int main(int argc, char **argv)
{
   mp_int x, output;
   mp_digit base;
   int e;

   if (argc != 3) {
      fprintf(stderr,"Usage %s base x\textbackslash{}n", argv[0]);
      exit(EXIT_FAILURE);
   }
   if ((e = mp_init_multi(&x, &output, NULL)) != MP_OKAY) {
      fprintf(stderr,"mp_init failed: \textbackslash{}"%s\textbackslash{}"\textbackslash{}n",
                     mp_error_to_string(e));
              exit(EXIT_FAILURE);
   }
   errno = 0;
#ifdef MP_64BIT
   base = (mp_digit)strtoull(argv[1], NULL, 10);
#else
   base = (mp_digit)strtoul(argv[1], NULL, 10);
#endif
   if ((errno == ERANGE) || (base > (base & MP_MASK))) {
      fprintf(stderr,"strtoul(l) failed: input out of range\textbackslash{}n");
      exit(EXIT_FAILURE);
   }
   if ((e = mp_read_radix(&x, argv[2], 10)) != MP_OKAY) {
      fprintf(stderr,"mp_read_radix failed: \textbackslash{}"%s\textbackslash{}"\textbackslash{}n",
                      mp_error_to_string(e));
      exit(EXIT_FAILURE);
   }

   if ((e = mp_ilogb(&x, base, &output)) != MP_OKAY) {
      fprintf(stderr,"mp_ilogb failed: \textbackslash{}"%s\textbackslash{}"\textbackslash{}n",
                      mp_error_to_string(e));
      exit(EXIT_FAILURE);
   }

   if ((e = mp_fwrite(&output, 10, stdout)) != MP_OKAY) {
      fprintf(stderr,"mp_fwrite failed: \textbackslash{}"%s\textbackslash{}"\textbackslash{}n",
                      mp_error_to_string(e));
      exit(EXIT_FAILURE);
   }
   putchar('\textbackslash{}n');

   mp_clear_multi(&x, &output, NULL);
   exit(EXIT_SUCCESS);
}
\end{alltt}

\chapter{Large Prime Numbers}
\section{Trial Division}
\index{mp\_prime\_is\_divisible}
\begin{alltt}
int mp_prime_is_divisible (mp_int * a, int *result)
\end{alltt}
This will attempt to evenly divide $a$ by a list of primes\footnote{Default is the first 256 primes.} and store the
outcome in ``result''.  That is if $result = 0$ then $a$ is not divisible by the primes, otherwise it is.  Note that
if the function does not return \textbf{MP\_OKAY} the value in ``result'' should be considered undefined\footnote{Currently
the default is to set it to zero first.}.

\section{Fermat Test}
\index{mp\_prime\_fermat}
\begin{alltt}
int mp_prime_fermat (mp_int * a, mp_int * b, int *result)
\end{alltt}
Performs a Fermat primality test to the base $b$.  That is it computes $b^a \mbox{ mod }a$ and tests whether the value is
equal to $b$ or not.  If the values are equal then $a$ is probably prime and $result$ is set to one.  Otherwise $result$
is set to zero.

\section{Miller-Rabin Test}
\index{mp\_prime\_miller\_rabin}
\begin{alltt}
int mp_prime_miller_rabin (mp_int * a, mp_int * b, int *result)
\end{alltt}
Performs a Miller-Rabin test to the base $b$ of $a$.  This test is much stronger than the Fermat test and is very hard to
fool (besides with Carmichael numbers).  If $a$ passes the test (therefore is probably prime) $result$ is set to one.
Otherwise $result$ is set to zero.

Note that is suggested that you use the Miller-Rabin test instead of the Fermat test since all of the failures of
Miller-Rabin are a subset of the failures of the Fermat test.

\subsection{Required Number of Tests}
Generally to ensure a number is very likely to be prime you have to perform the Miller-Rabin with at least a half-dozen
or so unique bases.  However, it has been proven that the probability of failure goes down as the size of the input goes up.
This is why a simple function has been provided to help out.

\index{mp\_prime\_rabin\_miller\_trials}
\begin{alltt}
int mp_prime_rabin_miller_trials(int size)
\end{alltt}
This returns the number of trials required for a $2^{-96}$ (or lower) probability of failure for a given ``size'' expressed
in bits.  This comes in handy specially since larger numbers are slower to test.  For example, a 512-bit number would
require ten tests whereas a 1024-bit number would only require four tests.

You should always still perform a trial division before a Miller-Rabin test though.

A small table, broke in two for typographical reasons, with the number of rounds of Miller-Rabin tests is shown below.
The first column is the number of bits $b$ in the prime $p = 2^b$, the numbers in the first row represent the
probability that the number that all of the Miller-Rabin tests deemed a pseudoprime is actually a composite. There is a deterministic test for numbers smaller than $2^{80}$.

\begin{table}[h]
\begin{center}
\begin{tabular}{c c c c c c c}
\textbf{bits} & $\mathbf{2^{-80}}$ & $\mathbf{2^{-96}}$ & $\mathbf{2^{-112}}$ & $\mathbf{2^{-128}}$ & $\mathbf{2^{-160}}$ & $\mathbf{2^{-192}}$ \\
80    & 31 & 39 & 47 & 55 & 71 & 87  \\
96    & 29 & 37 & 45 & 53 & 69 & 85  \\
128   & 24 & 32 & 40 & 48 & 64 & 80  \\
160   & 19 & 27 & 35 & 43 & 59 & 75  \\
192   & 15 & 21 & 29 & 37 & 53 & 69  \\
256   & 10 & 15 & 20 & 27 & 43 & 59  \\
384   & 7  & 9  & 12 & 16 & 25 & 38  \\
512   & 5  & 7  & 9  & 12 & 18 & 26  \\
768   & 4  & 5  & 6  & 8  & 11 & 16  \\
1024  & 3  & 4  & 5  & 6  & 9  & 12  \\
1536  & 2  & 3  & 3  & 4  & 6  & 8   \\
2048  & 2  & 2  & 3  & 3  & 4  & 6   \\
3072  & 1  & 2  & 2  & 2  & 3  & 4   \\
4096  & 1  & 1  & 2  & 2  & 2  & 3   \\
6144  & 1  & 1  & 1  & 1  & 2  & 2   \\
8192  & 1  & 1  & 1  & 1  & 2  & 2   \\
12288 & 1  & 1  & 1  & 1  & 1  & 1   \\
16384 & 1  & 1  & 1  & 1  & 1  & 1   \\
24576 & 1  & 1  & 1  & 1  & 1  & 1   \\
32768 & 1  & 1  & 1  & 1  & 1  & 1
\end{tabular}
\caption{ Number of Miller-Rabin rounds. Part I } \label{table:millerrabinrunsp1}
\end{center}
\end{table}
\newpage
\begin{table}[h]
\begin{center}
\begin{tabular}{c c c c c c c c}
\textbf{bits} &$\mathbf{2^{-224}}$ & $\mathbf{2^{-256}}$ & $\mathbf{2^{-288}}$ & $\mathbf{2^{-320}}$ & $\mathbf{2^{-352}}$ & $\mathbf{2^{-384}}$ & $\mathbf{2^{-416}}$\\
80    & 103 & 119 & 135 & 151 & 167 & 183 & 199 \\
96    & 101 & 117 & 133 & 149 & 165 & 181 & 197 \\
128   & 96  & 112 & 128 & 144 & 160 & 176 & 192 \\
160   & 91  & 107 & 123 & 139 & 155 & 171 & 187 \\
192   & 85  & 101 & 117 & 133 & 149 & 165 & 181 \\
256   & 75  & 91  & 107 & 123 & 139 & 155 & 171 \\
384   & 54  & 70  & 86  & 102 & 118 & 134 & 150 \\
512   & 36  & 49  & 65  & 81  & 97  & 113 & 129 \\
768   & 22  & 29  & 37  & 47  & 58  & 70  & 86  \\
1024  & 16  & 21  & 26  & 33  & 40  & 48  & 58  \\
1536  & 10  & 13  & 17  & 21  & 25  & 30  & 35  \\
2048  & 8   & 10  & 13  & 15  & 18  & 22  & 26  \\
3072  & 5   & 7   & 8	& 10  & 12  & 14  & 17  \\
4096  & 4   & 5   & 6	& 8   & 9   & 11  & 12  \\
6144  & 3   & 4   & 4	& 5   & 6   & 7   & 8	\\
8192  & 2   & 3   & 3	& 4   & 5   & 6   & 6	\\
12288 & 2   & 2   & 2	& 3   & 3   & 4   & 4	\\
16384 & 1   & 2   & 2	& 2   & 3   & 3   & 3	\\
24576 & 1   & 1   & 2	& 2   & 2   & 2   & 2	\\
32768 & 1   & 1   & 1	& 1   & 2   & 2   & 2
\end{tabular}
\caption{ Number of Miller-Rabin rounds. Part II } \label{table:millerrabinrunsp2}
\end{center}
\end{table}

Determining the probability needed to pick the right column is a bit harder. Fips 186.4, for example has $2^{-80}$ for $512$ bit large numbers, $2^{-112}$ for $1024$ bits, and $2^{128}$ for $1536$ bits. It can be seen in table \ref{table:millerrabinrunsp1} that those combinations follow the diagonal from $(512,2^{-80})$ downwards and to the right to gain a lower probability of getting a composite declared a pseudoprime for the same amount of work or less.

If this version of the library has the strong Lucas-Selfridge and/or the Frobenius-Underwood test implemented only one or two rounds of the Miller-Rabin test with a random base is necessary for numbers larger than or equal to $1024$ bits.


\section{Strong Lucas-Selfridge Test}
\index{mp\_prime\_strong\_lucas\_selfridge}
\begin{alltt}
int mp_prime_strong_lucas_selfridge(const mp_int *a, int *result)
\end{alltt}
Performs a strong Lucas-Selfridge test. The strong Lucas-Selfridge test together with the Rabin-Miler test with bases $2$ and $3$ resemble the BPSW test. The single internal use is a compile-time option in \texttt{mp\_prime\_is\_prime} and can be excluded
from the Libtommath build if not needed.

\section{Frobenius (Underwood)  Test}
\index{mp\_prime\_frobenius\_underwood}
\begin{alltt}
int mp_prime_frobenius_underwood(const mp_int *N, int *result)
\end{alltt}
Performs the variant of the Frobenius test as described by Paul Underwood. The single internal use is in
\texttt{mp\_prime\_is\_prime} for \texttt{MP\_8BIT} only but can be included at build-time for all other sizes
if the preprocessor macro \texttt{LTM\_USE\_FROBENIUS\_TEST} is defined.

It returns \texttt{MP\_ITER} if the number of iterations is exhausted, assumes a composite as the input and sets \texttt{result} accordingly. This will reduce the set of available pseudoprimes by a very small amount: test with large datasets (more than $10^{10}$ numbers, both randomly chosen and sequences of odd numbers with a random start point) found only 31 (thirty-one) numbers with $a > 120$ and none at all with just an additional simple check for divisors $d < 2^8$.

\section{Primality Testing}
Testing if a number is a square can be done a bit faster than just by calculating the square root. It is used by the primality testing function described below.
\index{mp\_is\_square}
\begin{alltt}
int mp_is_square(const mp_int *arg, int *ret);
\end{alltt}


\index{mp\_prime\_is\_prime}
\begin{alltt}
int mp_prime_is_prime (mp_int * a, int t, int *result)
\end{alltt}
This will perform a trial division followed by two rounds of Miller-Rabin with bases 2 and 3 and a Lucas-Selfridge test. The Lucas-Selfridge test is replaced with a Frobenius-Underwood for \texttt{MP\_8BIT}. The Frobenius-Underwood test for all other sizes is available as a compile-time option with the preprocessor macro \texttt{LTM\_USE\_FROBENIUS\_TEST}. See file
\texttt{bn\_mp\_prime\_is\_prime.c} for the necessary details. It shall be noted that both functions are much slower than
the Miller-Rabin test and if speed is an essential issue, the macro \texttt{LTM\_USE\_FIPS\_ONLY} switches both functions, the Frobenius-Underwood test and the Lucas-Selfridge test off and their code will not even be compiled into the library.

If $t$ is set to a positive value $t$ additional rounds of the Miller-Rabin test with random bases will be performed to allow for Fips 186.4 (vid.~p.~126ff) compliance. The function \texttt{mp\_prime\_rabin\_miller\_trials} can be used to determine the number of rounds. It is vital that the function \texttt{mp\_rand()} has a cryptographically strong random number generator available.

One Miller-Rabin tests with a random base will be run automatically, so by setting $t$ to a positive value this function will run $t + 1$ Miller-Rabin tests with random bases.

If  $t$ is set to a negative value the test will run the deterministic Miller-Rabin test for the primes up to
$3317044064679887385961981$. That limit has to be checked by the caller. If $-t > 13$ than $-t - 13$ additional rounds of the
Miller-Rabin test will be performed but note that $-t$ is bounded by $1 \le -t < PRIME\_SIZE$ where $PRIME\_SIZE$ is the number
of primes in the prime number table (by default this is $256$) and the first 13 primes have already been used. It will return
\texttt{MP\_VAL} in case of$-t > PRIME\_SIZE$.

If $a$ passes all of the tests $result$ is set to one, otherwise it is set to zero.

\subsection{Deterministic Prime Test}
The test above is a probabilistic test and can fail by calling a composite a prime. It is extremely rare, so rare that it is of almost no practical relevance. For those users who get nervous about the word ``almost'' a deterministic primes test has been put in the packet ``extra''. Add it to the library by typing \texttt{make extra}.
\index{mp\_prime\_is\_prime\_deterministic}
\begin{alltt}
int mp_prime_is_prime_deterministic(const mp_int *z, int *result);
\end{alltt}
It has two disadvantages: the theory behind assumes the general Riemann hypothesis to be true and, much more significant, is very slow for large primes.

\begin{table}[h]
\begin{center}
\begin{tabular}{c c c}
 $\mathbf{2^n}$ & \textbf{Testing Time} &  \textbf{Generating Time}\\
    128 &   0m00.061s  &  0m00.176s \\
    256 &   0m00.490s  &  0m00.788s \\
    512 &   0m06.233s  &  0m05.257s \\
    768 &   0m34.974s  &  0m02.916s \\
   1024 &   2m12.393s  &  0m45.018s \\
   1536 &  11m39.058s  &  2m25.890s \\
   2048 &  45m13.408s  &  2m57.433s
\end{tabular}
\caption{Benchmarking \texttt{mp\_prime\_is\_prime\_deterministic}} \label{table:benchmarkprimetestdet}
\end{center}
\end{table}

The machine for that benchmark was an AMD A8-6600K and the prime generator was LibTomMaths own
 \texttt{mp\_prime\_random\_ex(\&z, 8, size, LTM\_PRIME\_SAFE, myrng, NULL)}.

Despite its large runtime it was the only way to include a deterministic primality test with a small memory footprint, no need for floating point functions and one that works with low \texttt{MP\_xBIT}, too.

\section{Next Prime}
\index{mp\_prime\_next\_prime}
\begin{alltt}
int mp_prime_next_prime(mp_int *a, int t, int bbs_style)
\end{alltt}
This finds the next prime after $a$ that passes mp\_prime\_is\_prime() with $t$ tests but see the documentation for
mp\_prime\_is\_prime for details regarding the use of the argument $t$.  Set $bbs\_style$ to one if you
want only the next prime congruent to $3 \mbox{ mod } 4$, otherwise set it to zero to find any next prime.

\section{Random Primes}
\index{mp\_prime\_random}
\begin{alltt}
int mp_prime_random(mp_int *a, int t, int size, int bbs,
                    ltm_prime_callback cb, void *dat)
\end{alltt}
This will find a prime greater than $256^{size}$ which can be ``bbs\_style'' or not depending on $bbs$ and must pass
$t$ rounds of tests but see the documentation for mp\_prime\_is\_prime for details regarding the use of the argument $t$.
The ``ltm\_prime\_callback'' is a typedef for

\begin{alltt}
typedef int ltm_prime_callback(unsigned char *dst, int len, void *dat);
\end{alltt}

Which is a function that must read $len$ bytes (and return the amount stored) into $dst$.  The $dat$ variable is simply
copied from the original input.  It can be used to pass RNG context data to the callback.  The function
mp\_prime\_random() is more suitable for generating primes which must be secret (as in the case of RSA) since there
is no skew on the least significant bits.

\textit{Note:}  As of v0.30 of the LibTomMath library this function has been deprecated.  It is still available
but users are encouraged to use the new mp\_prime\_random\_ex() function instead.

\subsection{Extended Generation}
\index{mp\_prime\_random\_ex}
\begin{alltt}
int mp_prime_random_ex(mp_int *a,    int t,
                       int     size, int flags,
                       ltm_prime_callback cb, void *dat);
\end{alltt}
This will generate a prime in $a$ using $t$ tests of the primality testing algorithms.  The variable $size$
specifies the bit length of the prime desired.  The variable $flags$ specifies one of several options available
(see fig. \ref{fig:primeopts}) which can be OR'ed together.  The callback parameters are used as in
mp\_prime\_random().

\begin{figure}[h]
\begin{center}
\begin{small}
\begin{tabular}{|r|l|}
\hline \textbf{Flag}         & \textbf{Meaning} \\
\hline LTM\_PRIME\_BBS       & Make the prime congruent to $3$ modulo $4$ \\
\hline LTM\_PRIME\_SAFE      & Make a prime $p$ such that $(p - 1)/2$ is also prime. \\
                             & This option implies LTM\_PRIME\_BBS as well. \\
\hline LTM\_PRIME\_2MSB\_OFF & Makes sure that the bit adjacent to the most significant bit \\
                             & Is forced to zero.  \\
\hline LTM\_PRIME\_2MSB\_ON  & Makes sure that the bit adjacent to the most significant bit \\
                             & Is forced to one. \\
\hline
\end{tabular}
\end{small}
\end{center}
\caption{Primality Generation Options}
\label{fig:primeopts}
\end{figure}

\chapter{Random Number Generation}
\section{PRNG}
\index{mp\_rand\_digit}
\begin{alltt}
int mp_rand_digit(mp_digit *r)
\end{alltt}
This function generates a random number in \texttt{r} of the size given in \texttt{r} (that is, the variable is used for in- and output) but not more than \texttt{MP\_MASK} bits.

\index{mp\_rand}
\begin{alltt}
int mp_rand(mp_int *a, int digits)
\end{alltt}
This function generates a random number of \texttt{digits} bits.

The random number generated with these two functions is cryptographically secure if the source of random numbers the operating systems offers is cryptographically secure. It will use \texttt{arc4random()} if the OS is a BSD flavor, Wincrypt on Windows, or \texttt{/dev/urandom} on all operating systems that have it.

\chapter{Small Prime Numbers}

Small prime numbers are those primes that fit in a word of LibTomMath, that is they are smaller than or equal to the size of the type behind \texttt{mp\_word}. Examples at the end of this chapter at page \ref{sec:spnexamples},
\section{Prime Sieve}
A prime sieve is implemented as a simple segmented Sieve of Eratosthenes. It is only moderately optimized for space and runtime but should be small enough and also fast enough for almost all use-cases; quite quick for sequential access but relatively slow for random access.

 The prime sieve and its functions are part of the ``extra'' package and can be compiled in with \texttt{make extra}. The macro \texttt{LTM\_USE\_EXTRA\_FUNCTIONS} has to be set before \texttt{tommath.h} is included. Printing the small primes fneeds \texttt{inttypes.h} which must be included before \texttt{tommath.h}.
\subsection{Initialization and Clearing}
Initializing. It cannot fail because it only sets some default values. Memory is allocated later according to needs.
\index{mp\_sieve\_init}
\begin{alltt}
void mp_sieve_init(mp_sieve *sieve);
\end{alltt}
The function \texttt{mp\_sieve\_init} is equivalent to
\begin{alltt}
mp_sieve sieve = {NULL, NULL, 0};
\end{alltt}

Free the memory used by the sieve with
\index{mp\_sieve\_clear}
\begin{alltt}
void mp_sieve_clear(mp_sieve *sieve);
\end{alltt}
\subsection{Primality Test of Small Numbers}
Individual small numbers can be tested for primality with
\index{mp\_is\_small\_prime}
\begin{alltt}
int mp_is_small_prime(LTM_SIEVE_UINT n, LTM_SIEVE_UINT *result,
                      mp_sieve *sieve);
\end{alltt}
The implementation of this function also does all of the heavy lifting, the building of the base sieve and the segment if one is necessary. The base sieve stays, so this function can be used to ``warm up'' the sieve and, if \texttt{n} is slightly larger than the upper limit of the base sieve, ``warm up'' the first segment, too. It will return \texttt{LTM\_SIEVE\_MAX\_REACHED} to flag the content of \texttt{result} as the last valid one.
\subsection{Find Adjacent Primes}
To find the prime bigger than a number \texttt{n} use
\index{mp\_next\_small\_prime}
\begin{alltt}
int mp_next_small_prime(LTM_SIEVE_UINT n, LTM_SIEVE_UINT *result,
                        mp_sieve *sieve);
\end{alltt}
and to find the one smaller than \texttt{n}
\begin{alltt}
int mp_prec_small_prime(LTM_SIEVE_UINT n, LTM_SIEVE_UINT *result,
                        mp_sieve *sieve);
\end{alltt}
\subsection{Prime Sequence}
The most common use of the small primes is in the form of a continuous sequence. To produce this sequence utilize
\index{mp\_small\_prime\_array}. The array \texttt{prime\_array} is allocated with \texttt{malloc} internally and needs to be free'd after use.
\begin{alltt}
int mp_small_prime_array(LTM_SIEVE_UINT start, LTM_SIEVE_UINT end,
                         mp_factors *factors,
                         mp_sieve *sieve);
\end{alltt}

\subsection{Useful Constants}
\begin{description}
\item[\texttt{LTM\_SIEVE\_BIGGEST\_PRIME}] \texttt{read-only} The biggest prime the sieve can offer. It is be $65\,521$ for \texttt{MP\_8BIT},
 $4\,294\,967\,291$ for \texttt{MP\_16BIT}, \texttt{MP\_32BIT} and \texttt{MP\_64BIT}; and
 $18\,446\,744\,073\,709\,551\,557$ for \texttt{MP\_64BIT} if the macro\\
 \texttt{LTM\_SIEVE\_USE\_LARGE\_SIEVE} is defined.

\item[\texttt{LTM\_SIEVE\_UINT}] \texttt{read-only}  The basic type for the numbers in the sieve. It is be \texttt{uint16\_t} for \texttt{MP\_8BIT}, \texttt{uint32\_t} for \texttt{MP\_16BIT}, \texttt{MP\_32BIT} and \texttt{MP\_64BIT}; and \texttt{uint64\_t} for \texttt{MP\_64BIT} if the macro \texttt{LTM\_SIEVE\_USE\_LARGE\_SIEVE} is defined.

\item[\texttt{LTM\_SIEVE\_UINT\_MAX}] \texttt{read-only} The maximum value of the type for the numbers in the sieve. It is \texttt{UINT16\_MAX} for \texttt{MP\_8BIT}, \texttt{UINT32\_MAX} for \texttt{MP\_16BIT}, \texttt{MP\_32BIT} and \texttt{MP\_64BIT}; and \texttt{UINT\_64MAX} for \texttt{MP\_64BIT} if the macro\\
\texttt{LTM\_SIEVE\_USE\_LARGE\_SIEVE} is defined.

\item[\texttt{LTM\_SIEVE\_UINT\_MAX\_SQRT}] \texttt{read-only} The square root of the maximum value of the type for the numbers in the sieve. It is \texttt{UINT8\_MAX} for \texttt{MP\_8BIT}, \texttt{UINT16\_MAX} for \texttt{MP\_16BIT}, \texttt{MP\_32BIT} and \texttt{MP\_64BIT}; and\texttt{UINT32\_MAX} for \texttt{MP\_64BIT} if the macro \texttt{LTM\_SIEVE\_USE\_LARGE\_SIEVE} is defined.

\item[\texttt{LTM\_SIEVE\_USE\_LARGE\_SIEVE}] \texttt{read-only} A flag to make a large sieve.  No advantage has been seen in using 64-bit integers if available except the ability to get a sieve up to $2^64$. But in this case the base sieve gets 0.25 Gibibytes large and the segments 0.5 Gibibytes (although you can change \texttt{LTM\_SIEVE\_RANGE\_A\_B} to get smaller segments) and need a long time to fill.

\item[\texttt{LTM\_SIEVE\_RANGE\_A\_B}] \texttt{read-write} The size of the sieve for the segment. It is set to \texttt{LTM\_SIEVE\_UINT\_MAX\_SQRT} per default but it can be changed. The default size is already small enough to fit into most CPU's L-2 caches but if \texttt{LTM\_SIEVE\_USE\_LARGE\_SIEVE} is defined the segment sieve grows quite large and setting \texttt{LTM\_SIEVE\_RANGE\_A\_B} to the size of the CPU's L-2 caches will show a significant advantage regarding the runtime, it more than doubles it. Because of that large penalty the default value is set to \texttt{0x400000uL} if both \texttt{MP\_64BIT} and \texttt{LTM\_SIEVE\_USE\_LARGE\_SIEVE} are defined. Needs to be set at the compile time of LibTomMath.

\item[\texttt{LTM\_SIEVE\_UINT\_NUM\_BITS}] \texttt{read-only} The number of bits in the type for the numbers in the sieve.\\
It is a shortcut for \verb!CHAR\_BIT * sizeof(LTM\_SIEVE\_UINT)!.

\item[\texttt{LTM\_SIEVE\_SIZE(bst)}] \texttt{function macro, read-only} Returns the entry \texttt{size} of \texttt{struct mp\_sieve}. It is a shortcut for \verb!sieve->size!.

\item[\texttt{LTM\_SIEVE\_PR\_UINT}] Choses the correct macro from \texttt{inttypes.h} to print a\\
 \texttt{LTM\_SIEVE\_UINT}. The header \texttt{inttypes.h} must be included before\\
 \texttt{tommath.h} for it to work.
\end{description}


\subsection{Examples}\label{sec:spnexamples}
\subsubsection{Initialization and Clearing}
Using a sieve follows the same procedure as using a bigint:
\begin{alltt}
/* Declaration */
mp_sieve sieve;

/*
   Initialization.
   Cannot fail, only sets a handful of default values.
   Memory allocation is done in the library itself
   according to needs.
 */
mp_sieve_init(&sieve);

/* use the sieve */

/* Clean up */
mp_sieve_clear(&sieve);
\end{alltt}
\subsubsection{Primality Test}
A small program to test the input of a small number for primality.
\begin{alltt}
#include <stdlib.h>
#include <stdio.h>
#include <errno.h>
/*inttypes.h must be included before tommath.h*/
#include <inttypes.h>
/* Must be defined before tommath.h is included */
#define LTM_USE_EXTRA_FUNCTIONS
#include "tommath.h"
int main(int argc, char **argv)
{
   LTM_SIEVE_UINT number;
   mp_sieve *base = NULL;
   mp_sieve *segment = NULL;
   LTM_SIEVE_UINT single_segment_a = 0;
   int e;

   /* variable holding the result of mp_is_small_prime */
   LTM_SIEVE_UINT result;

   if (argc != 2) {
      fprintf(stderr,"Usage %s number\textbackslash{}n", argv[0]);
      exit(EXIT_FAILURE);
   }

   number = (LTM_SIEVE_UINT) strtoul(argv[1],NULL, 10);
   if (errno == ERANGE) {
      fprintf(stderr,"strtoul(l) failed: input out of range\textbackslash{}n");
      goto LTM_ERR;
   }

   mp_sieve_init(&sieve);

   if ((e = mp_is_small_prime(number, &result, &sieve)) != MP_OKAY) {
      fprintf(stderr,"mp_is_small_number failed: \textbackslash{}"%s\textbackslash{}"\textbackslash{}n",
              mp_error_to_string(e));
      goto LTM_ERR;
   }

   printf("The number %" LTM_SIEVE_PR_UINT " is %s prime\textbackslash{}n",
           number,(result)?"":"not");


   mp_sieve_clear(&sieve);
   exit(EXIT_SUCCESS);
LTM_ERR:
   mp_sieve_clear(&sieve);
   exit(EXIT_FAILURE);
}
\end{alltt}
\subsubsection{Find Adjacent Primes}
To sum up all primes up to and including \texttt{LTM\_SIEVE\_BIGGEST\_PRIME} you might do something like:
\begin{alltt}
#include <stdlib.h>
#include <stdio.h>
#include <errno.h>
/* Must be defined before tommath.h is included */
#define LTM_USE_EXTRA_FUNCTIONS
#include <tommath.h>
int main(int argc, char **argv)
{
   LTM_SIEVE_UINT number;
   mp_sieve sieve;
   LTM_SIEVE_UINT k, ret;
   mp_int total, t;
   int e;

   if (argc != 2) {
      fprintf(stderr,"Usage %s integer\textbackslash{}n", argv[0]);
      exit(EXIT_FAILURE);
   }

   if ((e = mp_init_multi(&total, &t, NULL)) != MP_OKAY) {
      fprintf(stderr,"mp_init_multi(segment): \textbackslash{}"%s\textbackslash{}"\textbackslash{}n",
              mp_error_to_string(e));
      goto LTM_ERR_1;
   }
   errno = 0;
#if ( (defined MP_64BIT) && (defined LTM_SIEVE_USE_LARGE_SIEVE) )
   number = (LTM_SIEVE_UINT) strtoull(argv[1],NULL, 10);
#else
   number = (LTM_SIEVE_UINT) strtoul(argv[1],NULL, 10);
#endif
   if (errno == ERANGE) {
      fprintf(stderr,"strtoul(l) failed: input out of range\textbackslash{}n");
      return EXIT_FAILURE
   }

   mp_sieve_init(&sieve);

   for (k = 0, ret = 0; ret < number; k = ret) {
      if ((e = mp_next_small_prime(k + 1, &ret, &sieve)) != MP_OKAY) {
         if (e == LTM_SIEVE_MAX_REACHED) {
#ifdef MP_64BIT
            if ((e = mp_add_d(&total, (mp_digit) k, &total)) != MP_OKAY) {
               fprintf(stderr,"mp_add_d (1) failed: \textbackslash{}"%s\textbackslash{}"\textbackslash{}n",
                       mp_error_to_string(e));
               goto LTM_ERR;
            }
#else
            if ((e = mp_set_long(&t, k)) != MP_OKAY) {
               fprintf(stderr,"mp_set_long (1) failed: \textbackslash{}"%s\textbackslash{}"\textbackslash{}n",
                       mp_error_to_string(e));
               goto LTM_ERR;
            }
            if ((e = mp_add(&total, &t, &total)) != MP_OKAY) {
               fprintf(stderr,"mp_add (1) failed: \textbackslash{}"%s\textbackslash{}"\textbackslash{}n",
                       mp_error_to_string(e));
               goto LTM_ERR;
            }
#endif
            break;
         }
         fprintf(stderr,"mp_next_small_prime failed: \textbackslash{}"%s\textbackslash{}"\textbackslash{}n",
                 mp_error_to_string(e));
         goto LTM_ERR;
      }
      /* The check if the prime is below the given limit
       * cannot be done in the for-loop conditions because if
       * it could we wouldn't need the sieve in the first place.
       */
      if (ret <= number) {
#ifdef MP_64BIT
         if ((e = mp_add_d(&total, (mp_digit) k, &total)) != MP_OKAY) {
            fprintf(stderr,"mp_add_d failed: \textbackslash{}"%s\textbackslash{}"\textbackslash{}n",
                    mp_error_to_string(e));
            goto LTM_ERR;
         }
#else
         if ((e = mp_set_long(&t, k)) != MP_OKAY) {
            fprintf(stderr,"mp_set_long failed: \textbackslash{}"%s\textbackslash{}"\textbackslash{}n",
                    mp_error_to_string(e));
            goto LTM_ERR;
         }
         if ((e = mp_add(&total, &t, &total)) != MP_OKAY) {
            fprintf(stderr,"mp_add failed: \textbackslash{}"%s\textbackslash{}"\textbackslash{}n",
            mp_error_to_string(e));
            goto LTM_ERR;
         }
#endif
      }
   }
   printf("total: ");
   mp_fwrite(&total,10,stdout);
   putchar('\textbackslash{}n');

   mp_clear_multi(&total, &t, NULL);
   mp_sieve_clear(&sieve);
   exit(EXIT_SUCCESS);
LTM_ERR:
   mp_clear_multi(&total, &t, NULL);
   mp_sieve_clear(&sieve);
   exit(EXIT_FAILURE);
}
\end{alltt}
It took about a minute on the authors machine from 2015 to get the expected $425\,649\,736\,193\,687\,430$ for the sum of all primes up to $2^{32}$, about the same runtime as Pari/GP version 2.9.4 (with a GMP-5.1.3 kernel).

\subsubsection{Prime Sequence}
A short sequence of primes can be produced with:
\begin{alltt}
#include <stdlib.h>
#include <stdio.h>
#include <errno.h>
/* Must be defined before tommath.h is included */
#define LTM_USE_EXTRA_FUNCTIONS
#include <tommath.h>
int main(int argc, char **argv)
{
   LTM_SIEVE_UINT a, b;
   mp_factors factors;
   int e;

   if (argc != 3) {
      fprintf(stderr,"Usage %s start stop\textbackslash{}n", argv[0]);
      exit(EXIT_FAILURE);
   }

   errno = 0;
#if ( (defined MP_64BIT) && (defined LTM_SIEVE_USE_LARGE_SIEVE) )
   a = (LTM_SIEVE_UINT) strtoull(argv[1], NULL, 10);
   if (errno == ERANGE) {
      fprintf(stderr,"strtoull(start) failed: input out of range\textbackslash{}n");
      goto LTM_ERR;
   }
   errno = 0;
   b = (LTM_SIEVE_UINT) strtoull(argv[2], NULL, 10);
   if (errno == ERANGE) {
      fprintf(stderr,"strtoull(end) failed: input out of range\textbackslash{}n");
      goto LTM_ERR;
   }
#else
   a = (LTM_SIEVE_UINT) strtoul(argv[1], NULL, 10);
   if (errno == ERANGE) {
      fprintf(stderr,"strtoul(start) failed: input out of range\textbackslash{}n");
      goto LTM_ERR;
   }
   errno = 0;
   b = (LTM_SIEVE_UINT) strtoul(argv[2], NULL, 10);
   if (errno == ERANGE) {
      fprintf(stderr,"strtoul(end) failed: input out of range\textbackslash{}n");
      goto LTM_ERR;
   }
#endif

   if ((e = mp_small_prime_array(a, b, &factors)) != MP_OKAY) {
      fprintf(stderr,"mp_small_prime_array failed: \textbackslash{}"%s\textbackslash{}"\textbackslash{}n",
              mp_error_to_string(e));
      goto LTM_ERR;
   }
   
   mp_factors_print(&factors, 10, 0, stdout);

   mp_factors_clear(&factors);
   exit(EXIT_SUCCESS);
LTM_ERR:
   mp_factors_clear(&factors);
   exit(EXIT_FAILURE);
}
\end{alltt}
The array \texttt{prime\_array} will be of size $\pi(b) - \pi(a)$ times \verb!sizeof(LTM_SIEVE_UINT)! which can get quite large quite quickly\footnote{There are $203\,280\,221$ primes smaller than $2^{32}$.}. You might find the method involving the function \texttt{mp\_next\_small\_prime} more applicable for larger sequences.

%\subsubsection{Using the Useful Constants}

\section{Factorizing}
All of the functions described in this section are in the packet ``extra''. Add it to the library by typing \texttt{make extra}.

The decomposition of numbers into their prime-factors is covered by the function
\index{mp\_factor}
\begin{alltt}
int mp_factor(const mp_int *z, mp_factors *factors);
\end{alltt}
It will decompose the factors of the integer $z > 0$ into its prime factors\footnote{The methods used in this algorithm are reasonably fast but definitely not the fastest. A practical limit is at about 30-35 bit large factors, 40 bit with some patience.}, checks the result by multiplying the found factors and comparing them with the input, and sample them as numbers of the type \texttt{mp\_int} in the list \texttt{factors}. The structure of this list is described by
\index{mp\_factors}
\begin{alltt}
typedef struct {
   int length, alloc;
   mp_int *factors;
} mp_factors;
\end{alltt}
There are a handful of functions to help with the management of that list.

\begin{description}
\item
\index{mp\_factors\_init}
\verb!int mp_factors_init(mp_factors *f);!\\
Initialize the factor list by allocating a certain amount of memory and setting \texttt{length = 0} and \texttt{alloc} to the amount of memory pre-allocated. The exact amount is defined at compile time by the macro \texttt{LTM\_TRIAL\_GROWTH} in \texttt{tommath.h}.
\item
\index{mp\_factors\_clear}
\verb!void mp_factors_clear(mp_factors *f);!\\
This function free's all memory used.
\item
\index{mp\_factors\_zero}
\verb!int mp_factors_zero(mp_factors *f);!\\
Remove the elements of the factor list and allocate (fresh) memory of default size in that order.
\item
\index{mp\_factors\_add}
\verb!int mp_factors_add(const mp_int *a, mp_factors *f);!\\
Add a factor of type \texttt{mp\_int} to the list.
\item
\index{mp\_factors\_sort}
\verb!int mp_factors_sort(mp_factors *f);!\\
The factors in the list are not necessarily in increasing order. This functions changes that. It does it with the insert-sort algorithm, a good choice for the task it has been written for (the list is most likely already ordered) but not for many other tasks involving large lists in random order.
\item
\index{mp\_factors\_print}
\texttt{int mp\_factors\_print(mp\_factors *f, int base, char delimiter,\\
\hphantom{int mp\_factors\_print(} FILE *stream);}\\
Prints the element of the list in base \texttt{base} to \texttt{stream} with the delimiter \texttt{delimiter}. The default delimiter is a comma (ASCII \texttt{0x2c}).
\item
\index{mp\_factors\_product}
\verb!int mp_factors_product(mp_factors *factors, mp_int *p);!\\
Multiplies all elements of the list. It does not recognize sparse lists, every zero in the list gets multiplied, too. It does multiply the list with a binary-splitting algorithm which assumes a highly or better fully sorted list to work optimally.
\end{description}

The function \texttt{mp\_factor} uses two different factorization algorithms. The first one does just trial division with the small primes generated by a sieve and is
\index{mp\_trial}
\begin{alltt}
int mp_trial(const mp_int *a, int limit, 
             mp_factors *factors, mp_int *r);
\end{alltt}
It tries all small primes up to the limit \texttt{limit}, puts all factors it finds in the list \texttt{factors} and the remainder in \texttt{r},

The other function is the Pollard-Rho algorithm.
\index{mp\_pollard\_rho}
\begin{alltt}
int mp_pollard_rho(const mp_int *n, mp_int *factor);
\end{alltt}
It is used to compute all factors left over by the function \texttt{mp\_factor}. Both functions \texttt{mp\_trial} and \texttt{mp\_pollard\_rho} are not meant to be used as a standalone function, please consult the source of the respective functions for the necessary information.


The function \texttt{mp\_factors\_product} does already most of the work so it was not much left to do to implement a function to compute a primorial.
\index{mp\_primorial}
\begin{alltt}
int mp_primorial(const LTM_SIEVE_UINT n, mp_int *p);
\end{alltt}

\subsection{Test for (perfect) Powers}
\index{mp\_ispower}
\begin{alltt}
int mp_ispower(const mp_int *z, int *result, mp_int *rootout,
                 mp_int *exponent);
\end{alltt}
Tests if $z$ is a power, that is $z = a^b$ with $b$ prime but $a$ might be composite. Computes the actual roots to do so and in case of success sets \texttt{result} to \texttt{MP\_YES}, puts $a$ in \texttt{rootout} and the exponent $b$ in \texttt{exponent} or $0$ (zero) in both and \texttt{result} to \texttt{MP\_NO} if case of a failure to find one.

\index{mp\_isperfpower}
\begin{alltt}
int mp_isperfpower(const mp_int *z, int *result, mp_int *rootout,
                   mp_int *exponent)
\end{alltt}
Same as \texttt{mp\_ispower} but searches for perfect or prime powers, that is $z = a^b$ such that $a, b$ are prime. Uses \texttt{mp\_prime\_is\_prime} which is a probabilistic prime tester and only roots below $2^{64}$ are save.
\chapter{Input and Output}
\section{ASCII Conversions}
\subsection{To ASCII}
\index{mp\_toradix}
\begin{alltt}
int mp_toradix (mp_int * a, char *str, int radix);
\end{alltt}
This still store $a$ in ``str'' as a base-``radix'' string of ASCII chars.  This function appends a NUL character
to terminate the string.  Valid values of ``radix'' line in the range $[2, 64]$.  To determine the size (exact) required
by the conversion before storing any data use the following function.

\index{mp\_toradix\_n}
\begin{alltt}
int mp_toradix_n (mp_int * a, char *str, int radix, int maxlen);
\end{alltt}

Like \texttt{mp\_toradix} but stores up to maxlen-1 chars and always a NULL byte.

\index{mp\_radix\_size}
\begin{alltt}
int mp_radix_size (mp_int * a, int radix, int *size)
\end{alltt}
This stores in ``size'' the number of characters (including space for the NUL terminator) required.  Upon error this
function returns an error code and ``size'' will be zero.

If \texttt{LTM\_NO\_FILE} is not defined a function to write to a file is also available.
\index{mp\_fwrite}
\begin{alltt}
int mp_fwrite(const mp_int *a, int radix, FILE *stream);
\end{alltt}


\subsection{From ASCII}
\index{mp\_read\_radix}
\begin{alltt}
int mp_read_radix (mp_int * a, char *str, int radix);
\end{alltt}
This will read the base-``radix'' NUL terminated string from ``str'' into $a$.  It will stop reading when it reads a
character it does not recognize (which happens to include th NUL char... imagine that...).  A single leading $-$ sign
can be used to denote a negative number.

If \texttt{LTM\_NO\_FILE} is not defined a function to read from a file is also available.
\index{mp\_fread}
\begin{alltt}
int mp_fread(mp_int *a, int radix, FILE *stream);
\end{alltt}


\section{Binary Conversions}

Converting an mp\_int to and from binary is another keen idea.

\index{mp\_unsigned\_bin\_size}
\begin{alltt}
int mp_unsigned_bin_size(mp_int *a);
\end{alltt}

This will return the number of bytes (octets) required to store the unsigned copy of the integer $a$.

\index{mp\_to\_unsigned\_bin}
\begin{alltt}
int mp_to_unsigned_bin(mp_int *a, unsigned char *b);
\end{alltt}
This will store $a$ into the buffer $b$ in big--endian format.  Fortunately this is exactly what DER (or is it ASN?)
requires.  It does not store the sign of the integer.

\index{mp\_to\_unsigned\_bin\_n}
\begin{alltt}
int mp_to_unsigned_bin_n(const mp_int *a, unsigned char *b, unsigned long *outlen)
\end{alltt}
Like \texttt{mp\_to\_unsigned\_bin} but checks if the value at \texttt{*outlen} is larger than or equal to the output of \texttt{mp\_unsigned\_bin\_size(a)} and sets \texttt{*outlen} to the output of \texttt{mp\_unsigned\_bin\_size(a)} or returns \texttt{MP\_VAL} if the test failed.


\index{mp\_read\_unsigned\_bin}
\begin{alltt}
int mp_read_unsigned_bin(mp_int *a, unsigned char *b, int c);
\end{alltt}
This will read in an unsigned big--endian array of bytes (octets) from $b$ of length $c$ into $a$.  The resulting
integer $a$ will always be positive.

For those who acknowledge the existence of negative numbers (heretic!) there are ``signed'' versions of the
previous functions.
\index{mp\_signed\_bin\_size} \index{mp\_to\_signed\_bin} \index{mp\_read\_signed\_bin}
\begin{alltt}
int mp_signed_bin_size(mp_int *a);
int mp_read_signed_bin(mp_int *a, unsigned char *b, int c);
int mp_to_signed_bin(mp_int *a, unsigned char *b);
\end{alltt}
They operate essentially the same as the unsigned copies except they prefix the data with zero or non--zero
byte depending on the sign.  If the sign is zpos (e.g. not negative) the prefix is zero, otherwise the prefix
is non--zero.

The two functions \texttt{mp\_import} and \texttt{mp\_export} implement the corresponding GMP functions as described at \url{http://gmplib.org/manual/Integer-Import-and-Export.html}.
\index{mp\_import} \index{mp\_export}
\begin{alltt}
int mp_import(mp_int *rop, size_t count, int order, size_t size, int endian, size_t nails, const void *op);
int mp_export(void *rop, size_t *countp, int order, size_t size, int endian, size_t nails, const mp_int *op);
\end{alltt}

\chapter{Algebraic Functions}
\section{Extended Euclidean Algorithm}
\index{mp\_exteuclid}
\begin{alltt}
int mp_exteuclid(mp_int *a, mp_int *b,
                 mp_int *U1, mp_int *U2, mp_int *U3);
\end{alltt}

This finds the triple U1/U2/U3 using the Extended Euclidean algorithm such that the following equation holds.

\begin{equation}
a \cdot U1 + b \cdot U2 = U3
\end{equation}

Any of the U1/U2/U3 parameters can be set to \textbf{NULL} if they are not desired.

\section{Greatest Common Divisor}
\index{mp\_gcd}
\begin{alltt}
int mp_gcd (mp_int * a, mp_int * b, mp_int * c)
\end{alltt}
This will compute the greatest common divisor of $a$ and $b$ and store it in $c$.

\section{Least Common Multiple}
\index{mp\_lcm}
\begin{alltt}
int mp_lcm (mp_int * a, mp_int * b, mp_int * c)
\end{alltt}
This will compute the least common multiple of $a$ and $b$ and store it in $c$.

\section{Jacobi Symbol}
\index{mp\_jacobi}
\begin{alltt}
int mp_jacobi (mp_int * a, mp_int * p, int *c)
\end{alltt}
This will compute the Jacobi symbol for $a$ with respect to $p$.  If $p$ is prime this essentially computes the Legendre
symbol.  The result is stored in $c$ and can take on one of three values $\lbrace -1, 0, 1 \rbrace$.  If $p$ is prime
then the result will be $-1$ when $a$ is not a quadratic residue modulo $p$.  The result will be $0$ if $a$ divides $p$
and the result will be $1$ if $a$ is a quadratic residue modulo $p$.

\section{Kronecker Symbol}
\index{mp\_kronecker}
\begin{alltt}
int mp_kronecker (mp_int * a, mp_int * p, int *c)
\end{alltt}
Extension of the Jacoby symbol to all $\lbrace a, p \rbrace \in \mathbb{Z}$ .


\section{Modular square root}
\index{mp\_sqrtmod\_prime}
\begin{alltt}
int mp_sqrtmod_prime(mp_int *n, mp_int *p, mp_int *r)
\end{alltt}

This will solve the modular equation $r^2 = n \mod p$ where $p$ is a prime number greater than 2 (odd prime).
The result is returned in the third argument $r$, the function returns \textbf{MP\_OKAY} on success,
other return values indicate failure.

The implementation is split for two different cases:

1. if $p \mod 4 == 3$ we apply \href{http://cacr.uwaterloo.ca/hac/}{Handbook of Applied Cryptography algorithm 3.36} and compute $r$ directly as
$r = n^{(p+1)/4} \mod p$

2. otherwise we use \href{https://en.wikipedia.org/wiki/Tonelli-Shanks_algorithm}{Tonelli-Shanks algorithm}

The function does not check the primality of parameter $p$ thus it is up to the caller to assure that this parameter
is a prime number. When $p$ is a composite the function behaviour is undefined, it may even return a false-positive
\textbf{MP\_OKAY}.

\section{Modular Inverse}
\index{mp\_invmod}
\begin{alltt}
int mp_invmod (mp_int * a, mp_int * b, mp_int * c)
\end{alltt}
Computes the multiplicative inverse of $a$ modulo $b$ and stores the result in $c$ such that $ac \equiv 1 \mbox{ (mod }b\mbox{)}$.

\section{Single Digit Functions}

For those using small numbers (\textit{snicker snicker}) there are several ``helper'' functions

\index{mp\_add\_d} \index{mp\_sub\_d} \index{mp\_mul\_d} \index{mp\_div\_d} \index{mp\_mod\_d}
\begin{alltt}
int mp_add_d(mp_int *a, mp_digit b, mp_int *c);
int mp_sub_d(mp_int *a, mp_digit b, mp_int *c);
int mp_mul_d(mp_int *a, mp_digit b, mp_int *c);
int mp_div_d(mp_int *a, mp_digit b, mp_int *c, mp_digit *d);
int mp_mod_d(mp_int *a, mp_digit b, mp_digit *c);
\end{alltt}

These work like the full mp\_int capable variants except the second parameter $b$ is a mp\_digit.  These
functions fairly handy if you have to work with relatively small numbers since you will not have to allocate
an entire mp\_int to store a number like $1$ or $2$.


The division by three can be made faster by replacing the division with a multiplication by the multiplicative inverse of three.

\index{mp\_div\_3}
\begin{alltt}
int mp_div_3(const mp_int *a, mp_int *c, mp_digit *d);
\end{alltt}

\chapter{Little Helpers}
It is never wrong to have some useful little shortcuts at hand.
\section{Function Macros}
To make this overview simpler the macros are given as function prototypes. The return of logic macros is \texttt{MP\_NO} or \texttt{MP\_YES} respectively.

\index{mp\_iseven}
\begin{alltt}
int mp_iseven(mp_int *a)
\end{alltt}
Checks if $a = 0 mod 2$

\index{mp\_isodd}
\begin{alltt}
int mp_isodd(mp_int *a)
\end{alltt}
Checks if $a = 1 mod 2$

\index{mp\_isneg}
\begin{alltt}
int mp_isneg(mp_int *a)
\end{alltt}
Checks if $a < 0$


\index{mp\_iszero}
\begin{alltt}
int mp_iszero(mp_int *a)
\end{alltt}
Checks if $a = 0$. It does not check if the amount of memory allocated for $a$ is also minimal.


Other macros which are either shortcuts to normal functions or just other names for them do have their place in a programmer's life, too!

\subsection{Renamings}
\index{mp\_mag\_size}
\begin{alltt}
#define mp_mag_size(mp) mp_unsigned_bin_size(mp)
\end{alltt}


\index{mp\_raw\_size}
\begin{alltt}
#define mp_raw_size(mp) mp_signed_bin_size(mp)
\end{alltt}


\index{mp\_read\_mag}
\begin{alltt}
#define mp_read_mag(mp, str, len) mp_read_unsigned_bin((mp), (str), (len))
\end{alltt}


\index{mp\_read\_raw}
\begin{alltt}
 #define mp_read_raw(mp, str, len) mp_read_signed_bin((mp), (str), (len))
\end{alltt}


\index{mp\_tomag}
\begin{alltt}
#define mp_tomag(mp, str) mp_to_unsigned_bin((mp), (str))
\end{alltt}


\index{mp\_toraw}
\begin{alltt}
#define mp_toraw(mp, str)         mp_to_signed_bin((mp), (str))
\end{alltt}



\subsection{Shortcuts}

\index{mp\_tobinary}
\begin{alltt}
#define mp_tobinary(M, S) mp_toradix((M), (S), 2)
\end{alltt}


\index{mp\_tooctal}
\begin{alltt}
#define mp_tooctal(M, S) mp_toradix((M), (S), 8)
\end{alltt}


\index{mp\_todecimal}
\begin{alltt}
#define mp_todecimal(M, S) mp_toradix((M), (S), 10)
\end{alltt}


\index{mp\_tohex}
\begin{alltt}
#define mp_tohex(M, S)     mp_toradix((M), (S), 16)
\end{alltt}


\documentclass[synpaper]{book}
\usepackage{hyperref}
\usepackage{makeidx}
\usepackage{amssymb}
\usepackage{color}
\usepackage{alltt}
\usepackage{graphicx}
\usepackage{layout}
\def\union{\cup}
\def\intersect{\cap}
\def\getsrandom{\stackrel{\rm R}{\gets}}
\def\cross{\times}
\def\cat{\hspace{0.5em} \| \hspace{0.5em}}
\def\catn{$\|$}
\def\divides{\hspace{0.3em} | \hspace{0.3em}}
\def\nequiv{\not\equiv}
\def\approx{\raisebox{0.2ex}{\mbox{\small $\sim$}}}
\def\lcm{{\rm lcm}}
\def\gcd{{\rm gcd}}
\def\log{{\rm log}}
\def\ilog{{\rm ilog}}
\def\ord{{\rm ord}}
\def\abs{{\mathit abs}}
\def\rep{{\mathit rep}}
\def\mod{{\mathit\ mod\ }}
\renewcommand{\pmod}[1]{\ ({\rm mod\ }{#1})}
\newcommand{\floor}[1]{\left\lfloor{#1}\right\rfloor}
\newcommand{\ceil}[1]{\left\lceil{#1}\right\rceil}
\def\Or{{\rm\ or\ }}
\def\And{{\rm\ and\ }}
\def\iff{\hspace{1em}\Longleftrightarrow\hspace{1em}}
\def\implies{\Rightarrow}
\def\undefined{{\rm ``undefined"}}
\def\Proof{\vspace{1ex}\noindent {\bf Proof:}\hspace{1em}}
\let\oldphi\phi
\def\phi{\varphi}
\def\Pr{{\rm Pr}}
\newcommand{\str}[1]{{\mathbf{#1}}}
\def\F{{\mathbb F}}
\def\N{{\mathbb N}}
\def\Z{{\mathbb Z}}
\def\R{{\mathbb R}}
\def\C{{\mathbb C}}
\def\Q{{\mathbb Q}}
\definecolor{DGray}{gray}{0.5}
\newcommand{\emailaddr}[1]{\mbox{$<${#1}$>$}}
\def\twiddle{\raisebox{0.3ex}{\mbox{\tiny $\sim$}}}
\def\gap{\vspace{0.5ex}}
\makeindex
\begin{document}
\frontmatter
\pagestyle{empty}
\title{LibTomMath User Manual \\ v1.1.0}
\author{LibTom Projects \\ www.libtom.net}
\maketitle
This text, the library and the accompanying textbook are all hereby placed in the public domain.  This book has been
formatted for B5 [176x250] paper using the \LaTeX{} {\em book} macro package.

\vspace{10cm}

\begin{flushright}Open Source.  Open Academia.  Open Minds.

\mbox{ }
LibTom Projects

\& originally

Tom St Denis,

Ontario, Canada
\end{flushright}

\tableofcontents
\listoffigures
\mainmatter
\pagestyle{headings}
\chapter{Introduction}
\section{What is LibTomMath?}
LibTomMath is a library of source code which provides a series of efficient and carefully written functions for manipulating
large integer numbers.  It was written in portable ISO C source code so that it will build on any platform with a conforming
C compiler.

In a nutshell the library was written from scratch with verbose comments to help instruct computer science students how
to implement ``bignum'' math.  However, the resulting code has proven to be very useful.  It has been used by numerous
universities, commercial and open source software developers.  It has been used on a variety of platforms ranging from
Linux and Windows based x86 to ARM based Gameboys and PPC based MacOS machines.

\section{License}
As of the v0.25 the library source code has been placed in the public domain with every new release.  As of the v0.28
release the textbook ``Implementing Multiple Precision Arithmetic'' has been placed in the public domain with every new
release as well.  This textbook is meant to compliment the project by providing a more solid walkthrough of the development
algorithms used in the library.

Since both\footnote{Note that the MPI files under mtest/ are copyrighted by Michael Fromberger.  They are not required to use LibTomMath.} are in the
public domain everyone is entitled to do with them as they see fit.

\section{Building LibTomMath}

LibTomMath is meant to be very ``GCC friendly'' as it comes with a makefile well suited for GCC.  However, the library will
also build in MSVC, Borland C out of the box.  For any other ISO C compiler a makefile will have to be made by the end
developer. Please consider to commit such a makefile to the LibTomMath developers, currently residing at
\url{http://github.com/libtom/libtommath}, if successfully done so.

Intel's C-compiler (ICC) is sufficiently compatible with GCC, at least the newer versions, to replace GCC for building the static and the shared library. Editing the makefiles is not needed, just set the shell variable \texttt{CC} as shown below.
\begin{alltt}
CC=/home/czurnieden/intel/bin/icc make
\end{alltt}

ICC does not know all options available for GCC and LibTomMath uses two diagnostics \texttt{-Wbad-function-cast} and \texttt{-Wcast-align} that are not supported by ICC resulting in the warnings:
\begin{alltt}
icc: command line warning #10148: option '-Wbad-function-cast' not supported
icc: command line warning #10148: option '-Wcast-align' not supported
\end{alltt}
It is possible to mute this ICC warning with the compiler flag \texttt{-diag-disable=10006}\footnote{It is not recommended to suppress warnings without a very good reason but there is no harm in doing so in this very special case.}.

\subsection{Static Libraries}
To build as a static library for GCC issue the following
\begin{alltt}
make
\end{alltt}

command.  This will build the library and archive the object files in ``libtommath.a''.  Now you link against
that and include ``tommath.h'' within your programs.  Alternatively to build with MSVC issue the following
\begin{alltt}
nmake -f makefile.msvc
\end{alltt}

This will build the library and archive the object files in ``tommath.lib''.  This has been tested with MSVC
version 6.00 with service pack 5.

\subsection{Shared Libraries}
\subsubsection{GNU based Operating Systems}
To build as a shared library for GCC issue the following
\begin{alltt}
make -f makefile.shared
\end{alltt}
This requires the ``libtool'' package (common on most Linux/BSD systems).  It will build LibTomMath as both shared
and static then install (by default) into /usr/lib as well as install the header files in /usr/include.  The shared
library (resource) will be called ``libtommath.la'' while the static library called ``libtommath.a''.  Generally
you use libtool to link your application against the shared object.
\subsubsection{Microsoft Windows based Operating Systems}
There is limited support for making a ``DLL'' in windows via the ``makefile.cygwin\_dll'' makefile.  It requires
Cygwin to work with since it requires the auto-export/import functionality.  The resulting DLL and import library
``libtommath.dll.a'' can be used to link LibTomMath dynamically to any Windows program using Cygwin.
\subsubsection{OpenBSD}
OpenBSD replaced some of their GNU-tools, especially \texttt{libtool} with their own, slightly different versions. To ease the workload of LibTomMath's developer team, only a static library can be build with the included \texttt{makefile.unix}.

The wrong \texttt{make} will result in errors like:
\begin{alltt}
*** Parse error in /home/user/GITHUB/libtommath: Need an operator in 'LIBNAME' )
*** Parse error: Need an operator in 'endif' (makefile.shared:8)
*** Parse error: Need an operator in 'CROSS_COMPILE' (makefile_include.mk:16)
*** Parse error: Need an operator in 'endif' (makefile_include.mk:18)
*** Parse error: Missing dependency operator (makefile_include.mk:22)
*** Parse error: Missing dependency operator (makefile_include.mk:23)
...
\end{alltt}
The wrong \texttt{libtool} will build it all fine but when it comes to the final linking fails with
\begin{alltt}
...
cc -I./ -Wall -Wsign-compare -Wextra -Wshadow -Wsystem-headers -Wdeclaration-afo...
cc -I./ -Wall -Wsign-compare -Wextra -Wshadow -Wsystem-headers -Wdeclaration-afo...
cc -I./ -Wall -Wsign-compare -Wextra -Wshadow -Wsystem-headers -Wdeclaration-afo...
libtool --mode=link --tag=CC cc  bn_error.lo bn_fast_mp_invmod.lo bn_fast_mp_mo 
libtool: link: cc bn_error.lo bn_fast_mp_invmod.lo bn_fast_mp_montgomery_reduce0
bn_error.lo: file not recognized: File format not recognized
cc: error: linker command failed with exit code 1 (use -v to see invocation)
Error while executing cc bn_error.lo bn_fast_mp_invmod.lo bn_fast_mp_montgomery0
gmake: *** [makefile.shared:64: libtommath.la] Error 1
\end{alltt}

To build a shared library with OpenBSD\footnote{Tested with OpenBSD version 6.4} the GNU versions of \texttt{make} and \texttt{libtool} are needed.
\begin{alltt}
$ sudo pkg_add gmake libtool
\end{alltt}
At this time two versions of \texttt{libtool} are installed and both are named \texttt{libtool}, unfortunately but GNU \texttt{libtool} has been placed in \texttt{/usr/local/bin/} and the native version in \texttt{/usr/bin/}. The path might be different in other versions of OpenBSD but both programs differ in the output of \texttt{libtool --version}
\begin{alltt}
$ /usr/local/bin/libtool --version                              
libtool (GNU libtool) 2.4.2
Written by Gordon Matzigkeit <gord@gnu.ai.mit.edu>, 1996

Copyright (C) 2011 Free Software Foundation, Inc.
This is free software; see the source for copying conditions.  There is NO
warranty; not even for MERCHANTABILITY or FITNESS FOR A PARTICULAR PURPOSE.
$ libtool --version
libtool (not (GNU libtool)) 1.5.26
\end{alltt}

The shared library should build now with
\begin{alltt}
LIBTOOL="/usr/local/bin/libtool" gmake -f makefile.shared
\end{alltt}
You might need to run a \texttt{gmake -f makefile.shared clean} first.

\subsubsection{NetBSD}
NetBSD is not as strict as OpenBSD but still needs \texttt{gmake} to build the shared library. \texttt{libtool} may also not exist in a fresh install.
\begin{alltt}
pkg_add gmake libtool
\end{alltt}
Please check with \texttt{libtool --version} that installed libtool is indeed a GNU libtool.
Build the shared library by typing:
\begin{alltt}
gmake -f makefile.shared
\end{alltt}

\subsection{Testing}
To build the library and the test harness type

\begin{alltt}
make test
\end{alltt}

This will build the library, ``test'' and ``mtest/mtest''.  The ``test'' program will accept test vectors and verify the
results.  ``mtest/mtest'' will generate test vectors using the MPI library by Michael Fromberger\footnote{A copy of MPI
is included in the package}.  Simply pipe mtest into test using

\begin{alltt}
mtest/mtest | test
\end{alltt}

If you do not have a ``/dev/urandom'' style RNG source you will have to write your own PRNG and simply pipe that into
mtest.  For example, if your PRNG program is called ``myprng'' simply invoke

\begin{alltt}
myprng | mtest/mtest | test
\end{alltt}

This will output a row of numbers that are increasing.  Each column is a different test (such as addition, multiplication, etc)
that is being performed.  The numbers represent how many times the test was invoked.  If an error is detected the program
will exit with a dump of the relevant numbers it was working with.

\section{Build Configuration}
LibTomMath can configured at build time in three phases we shall call ``depends'', ``tweaks'' and ``trims''.
Each phase changes how the library is built and they are applied one after another respectively.

To make the system more powerful you can tweak the build process.  Classes are defined in the file
``tommath\_superclass.h''.  By default, the symbol ``LTM\_ALL'' shall be defined which simply
instructs the system to build all of the functions.  This is how LibTomMath used to be packaged.  This will give you
access to every function LibTomMath offers.

However, there are cases where such a build is not optional.  For instance, you want to perform RSA operations.  You
don't need the vast majority of the library to perform these operations.  Aside from LTM\_ALL there is
another pre--defined class ``SC\_RSA\_1'' which works in conjunction with the RSA from LibTomCrypt.  Additional
classes can be defined base on the need of the user.

\subsection{Build Depends}
In the file tommath\_class.h you will see a large list of C ``defines'' followed by a series of ``ifdefs''
which further define symbols.  All of the symbols (technically they're macros $\ldots$) represent a given C source
file.  For instance, BN\_MP\_ADD\_C represents the file ``bn\_mp\_add.c''.  When a define has been enabled the
function in the respective file will be compiled and linked into the library.  Accordingly when the define
is absent the file will not be compiled and not contribute any size to the library.

You will also note that the header tommath\_class.h is actually recursively included (it includes itself twice).
This is to help resolve as many dependencies as possible.  In the last pass the symbol LTM\_LAST will be defined.
This is useful for ``trims''.

\subsection{Build Tweaks}
A tweak is an algorithm ``alternative''.  For example, to provide trade-offs (usually between size and space).
They can be enabled at any pass of the configuration phase.

\begin{small}
\begin{center}
\begin{tabular}{|l|l|}
\hline \textbf{Define} & \textbf{Purpose} \\
\hline BN\_MP\_DIV\_SMALL & Enables a slower, smaller and equally \\
                          & functional mp\_div() function \\
\hline
\end{tabular}
\end{center}
\end{small}

\subsection{Build Trims}
A trim is a manner of removing functionality from a function that is not required.  For instance, to perform
RSA cryptography you only require exponentiation with odd moduli so even moduli support can be safely removed.
Build trims are meant to be defined on the last pass of the configuration which means they are to be defined
only if LTM\_LAST has been defined.

\subsubsection{Moduli Related}
\begin{small}
\begin{center}
\begin{tabular}{|l|l|}
\hline \textbf{Restriction} & \textbf{Undefine} \\
\hline Exponentiation with odd moduli only & BN\_S\_MP\_EXPTMOD\_C \\
                                           & BN\_MP\_REDUCE\_C \\
                                           & BN\_MP\_REDUCE\_SETUP\_C \\
                                           & BN\_S\_MP\_MUL\_HIGH\_DIGS\_C \\
                                           & BN\_FAST\_S\_MP\_MUL\_HIGH\_DIGS\_C \\
\hline Exponentiation with random odd moduli & (The above plus the following) \\
                                           & BN\_MP\_REDUCE\_2K\_C \\
                                           & BN\_MP\_REDUCE\_2K\_SETUP\_C \\
                                           & BN\_MP\_REDUCE\_IS\_2K\_C \\
                                           & BN\_MP\_DR\_IS\_MODULUS\_C \\
                                           & BN\_MP\_DR\_REDUCE\_C \\
                                           & BN\_MP\_DR\_SETUP\_C \\
\hline Modular inverse odd moduli only     & BN\_MP\_INVMOD\_SLOW\_C \\
\hline Modular inverse (both, smaller/slower) & BN\_FAST\_MP\_INVMOD\_C \\
\hline
\end{tabular}
\end{center}
\end{small}

\subsubsection{Operand Size Related}
\begin{small}
\begin{center}
\begin{tabular}{|l|l|}
\hline \textbf{Restriction} & \textbf{Undefine} \\
\hline Moduli $\le 2560$ bits              & BN\_MP\_MONTGOMERY\_REDUCE\_C \\
                                           & BN\_S\_MP\_MUL\_DIGS\_C \\
                                           & BN\_S\_MP\_MUL\_HIGH\_DIGS\_C \\
                                           & BN\_S\_MP\_SQR\_C \\
\hline Polynomial Schmolynomial            & BN\_MP\_KARATSUBA\_MUL\_C \\
                                           & BN\_MP\_KARATSUBA\_SQR\_C \\
                                           & BN\_MP\_TOOM\_MUL\_C \\
                                           & BN\_MP\_TOOM\_SQR\_C \\

\hline
\end{tabular}
\end{center}
\end{small}


\section{Purpose of LibTomMath}
Unlike  GNU MP (GMP) Library, LIP, OpenSSL or various other commercial kits (Miracl), LibTomMath was not written with
bleeding edge performance in mind.  First and foremost LibTomMath was written to be entirely open.  Not only is the
source code public domain (unlike various other GPL/etc licensed code), not only is the code freely downloadable but the
source code is also accessible for computer science students attempting to learn ``BigNum'' or multiple precision
arithmetic techniques.

LibTomMath was written to be an instructive collection of source code.  This is why there are many comments, only one
function per source file and often I use a ``middle-road'' approach where I don't cut corners for an extra 2\% speed
increase.

Source code alone cannot really teach how the algorithms work which is why I also wrote a textbook that accompanies
the library (beat that!).

So you may be thinking ``should I use LibTomMath?'' and the answer is a definite maybe.  Let me tabulate what I think
are the pros and cons of LibTomMath by comparing it to the math routines from GnuPG\footnote{GnuPG v1.2.3 versus LibTomMath v0.28}.

\newpage\begin{figure}[h]
\begin{small}
\begin{center}
\begin{tabular}{|l|c|c|l|}
\hline \textbf{Criteria} & \textbf{Pro} & \textbf{Con} & \textbf{Notes} \\
\hline Few lines of code per file & X & & GnuPG $ = 300.9$, LibTomMath  $ = 71.97$ \\
\hline Commented function prototypes & X && GnuPG function names are cryptic. \\
\hline Speed && X & LibTomMath is slower.  \\
\hline Totally free & X & & GPL has unfavourable restrictions.\\
\hline Large function base & X & & GnuPG is barebones. \\
\hline Five modular reduction algorithms & X & & Faster modular exponentiation for a variety of moduli. \\
\hline Portable & X & & GnuPG requires configuration to build. \\
\hline
\end{tabular}
\end{center}
\end{small}
\caption{LibTomMath Valuation}
\end{figure}

It may seem odd to compare LibTomMath to GnuPG since the math in GnuPG is only a small portion of the entire application.
However, LibTomMath was written with cryptography in mind.  It provides essentially all of the functions a cryptosystem
would require when working with large integers.

So it may feel tempting to just rip the math code out of GnuPG (or GnuMP where it was taken from originally) in your
own application but I think there are reasons not to.  While LibTomMath is slower than libraries such as GnuMP it is
not normally significantly slower.  On x86 machines the difference is normally a factor of two when performing modular
exponentiations.  It depends largely on the processor, compiler and the moduli being used.

Essentially the only time you would not use LibTomMath is when blazing speed is the primary concern.  However,
on the other side of the coin LibTomMath offers you a totally free (public domain) well structured math library
that is very flexible, complete and performs well in resource constrained environments.  Fast RSA for example can
be performed with as little as 8KB of ram for data (again depending on build options).

\chapter{Getting Started with LibTomMath}
\section{Building Programs}
In order to use LibTomMath you must include ``tommath.h'' and link against the appropriate library file (typically
libtommath.a).  There is no library initialization required and the entire library is thread safe.

\section{Return Codes}
There are three possible return codes a function may return.

\index{MP\_OKAY}\index{MP\_YES}\index{MP\_NO}\index{MP\_VAL}\index{MP\_MEM}
\begin{figure}[h!]
\begin{center}
\begin{small}
\begin{tabular}{|l|l|}
\hline \textbf{Code} & \textbf{Meaning} \\
\hline MP\_OKAY & The function succeeded. \\
\hline MP\_VAL  & The function input was invalid. \\
\hline MP\_MEM  & Heap memory exhausted. \\
\hline &\\
\hline MP\_YES  & Response is yes. \\
\hline MP\_NO   & Response is no. \\
\hline
\end{tabular}
\end{small}
\end{center}
\caption{Return Codes}
\end{figure}

The last two codes listed are not actually ``return'ed'' by a function.  They are placed in an integer (the caller must
provide the address of an integer it can store to) which the caller can access.  To convert one of the three return codes
to a string use the following function.

\index{mp\_error\_to\_string}
\begin{alltt}
char *mp_error_to_string(int code);
\end{alltt}

This will return a pointer to a string which describes the given error code.  It will not work for the return codes
MP\_YES and MP\_NO.

\section{Data Types}
The basic ``multiple precision integer'' type is known as the ``mp\_int'' within LibTomMath.  This data type is used to
organize all of the data required to manipulate the integer it represents.  Within LibTomMath it has been prototyped
as the following.

\index{mp\_int}
\begin{alltt}
typedef struct  \{
    int used, alloc, sign;
    mp_digit *dp;
\} mp_int;
\end{alltt}

Where ``mp\_digit'' is a data type that represents individual digits of the integer.  By default, an mp\_digit is the
ISO C ``unsigned long'' data type and each digit is $28-$bits long.  The mp\_digit type can be configured to suit other
platforms by defining the appropriate macros.

All LTM functions that use the mp\_int type will expect a pointer to mp\_int structure.  You must allocate memory to
hold the structure itself by yourself (whether off stack or heap it does not matter).  The very first thing that must be
done to use an mp\_int is that it must be initialized.

\section{Function Organization}

The arithmetic functions of the library are all organized to have the same style prototype.  That is source operands
are passed on the left and the destination is on the right.  For instance,

\begin{alltt}
mp_add(&a, &b, &c);       /* c = a + b */
mp_mul(&a, &a, &c);       /* c = a * a */
mp_div(&a, &b, &c, &d);   /* c = [a/b], d = a mod b */
\end{alltt}

Another feature of the way the functions have been implemented is that source operands can be destination operands as well.
For instance,

\begin{alltt}
mp_add(&a, &b, &b);       /* b = a + b */
mp_div(&a, &b, &a, &c);   /* a = [a/b], c = a mod b */
\end{alltt}

This allows operands to be re-used which can make programming simpler.

\section{Initialization}
\subsection{Single Initialization}
A single mp\_int can be initialized with the ``mp\_init'' function.

\index{mp\_init}
\begin{alltt}
int mp_init (mp_int * a);
\end{alltt}

This function expects a pointer to an mp\_int structure and will initialize the members of the structure so the mp\_int
represents the default integer which is zero.  If the functions returns MP\_OKAY then the mp\_int is ready to be used
by the other LibTomMath functions.

\begin{small} \begin{alltt}
int main(void)
\{
   mp_int number;
   int result;

   if ((result = mp_init(&number)) != MP_OKAY) \{
      printf("Error initializing the number.  \%s",
             mp_error_to_string(result));
      return EXIT_FAILURE;
   \}

   /* use the number */

   return EXIT_SUCCESS;
\}
\end{alltt} \end{small}

\subsection{Single Free}
When you are finished with an mp\_int it is ideal to return the heap it used back to the system.  The following function
provides this functionality.

\index{mp\_clear}
\begin{alltt}
void mp_clear (mp_int * a);
\end{alltt}

The function expects a pointer to a previously initialized mp\_int structure and frees the heap it uses.  It sets the
pointer\footnote{The ``dp'' member.} within the mp\_int to \textbf{NULL} which is used to prevent double free situations.
Is is legal to call mp\_clear() twice on the same mp\_int in a row.

\begin{small} \begin{alltt}
int main(void)
\{
   mp_int number;
   int result;

   if ((result = mp_init(&number)) != MP_OKAY) \{
      printf("Error initializing the number.  \%s",
             mp_error_to_string(result));
      return EXIT_FAILURE;
   \}

   /* use the number */

   /* We're done with it. */
   mp_clear(&number);

   return EXIT_SUCCESS;
\}
\end{alltt} \end{small}

\subsection{Multiple Initializations}
Certain algorithms require more than one large integer.  In these instances it is ideal to initialize all of the mp\_int
variables in an ``all or nothing'' fashion.  That is, they are either all initialized successfully or they are all
not initialized.

The  mp\_init\_multi() function provides this functionality.

\index{mp\_init\_multi} \index{mp\_clear\_multi}
\begin{alltt}
int mp_init_multi(mp_int *mp, ...);
\end{alltt}

It accepts a \textbf{NULL} terminated list of pointers to mp\_int structures.  It will attempt to initialize them all
at once.  If the function returns MP\_OKAY then all of the mp\_int variables are ready to use, otherwise none of them
are available for use.  A complementary mp\_clear\_multi() function allows multiple mp\_int variables to be free'd
from the heap at the same time.

\begin{small} \begin{alltt}
int main(void)
\{
   mp_int num1, num2, num3;
   int result;

   if ((result = mp_init_multi(&num1,
                               &num2,
                               &num3, NULL)) != MP\_OKAY) \{
      printf("Error initializing the numbers.  \%s",
             mp_error_to_string(result));
      return EXIT_FAILURE;
   \}

   /* use the numbers */

   /* We're done with them. */
   mp_clear_multi(&num1, &num2, &num3, NULL);

   return EXIT_SUCCESS;
\}
\end{alltt} \end{small}

\subsection{Other Initializers}
To initialized and make a copy of an mp\_int the mp\_init\_copy() function has been provided.

\index{mp\_init\_copy}
\begin{alltt}
int mp_init_copy (mp_int * a, mp_int * b);
\end{alltt}

This function will initialize $a$ and make it a copy of $b$ if all goes well.

\begin{small} \begin{alltt}
int main(void)
\{
   mp_int num1, num2;
   int result;

   /* initialize and do work on num1 ... */

   /* We want a copy of num1 in num2 now */
   if ((result = mp_init_copy(&num2, &num1)) != MP_OKAY) \{
     printf("Error initializing the copy.  \%s",
             mp_error_to_string(result));
      return EXIT_FAILURE;
   \}

   /* now num2 is ready and contains a copy of num1 */

   /* We're done with them. */
   mp_clear_multi(&num1, &num2, NULL);

   return EXIT_SUCCESS;
\}
\end{alltt} \end{small}

Another less common initializer is mp\_init\_size() which allows the user to initialize an mp\_int with a given
default number of digits.  By default, all initializers allocate \textbf{MP\_PREC} digits.  This function lets
you override this behaviour.

\index{mp\_init\_size}
\begin{alltt}
int mp_init_size (mp_int * a, int size);
\end{alltt}

The $size$ parameter must be greater than zero.  If the function succeeds the mp\_int $a$ will be initialized
to have $size$ digits (which are all initially zero).

\begin{small} \begin{alltt}
int main(void)
\{
   mp_int number;
   int result;

   /* we need a 60-digit number */
   if ((result = mp_init_size(&number, 60)) != MP_OKAY) \{
      printf("Error initializing the number.  \%s",
             mp_error_to_string(result));
      return EXIT_FAILURE;
   \}

   /* use the number */

   return EXIT_SUCCESS;
\}
\end{alltt} \end{small}

\section{Maintenance Functions}
\subsection{Clear Leading Zeros}

This is used to ensure that leading zero digits are trimmed and the leading "used" digit will be non-zero.
It also fixes the sign if there are no more leading digits.

\index{mp\_clamp}
\begin{alltt}
void mp_clamp(mp_int *a);
\end{alltt}

\subsection{Zero Out}

This function will set the ``bigint'' to zeros without changing the amount of allocated memory.

\index{mp\_zero}
\begin{alltt}
void mp_zero(mp_int *a);
\end{alltt}


\subsection{Reducing Memory Usage}
When an mp\_int is in a state where it won't be changed again\footnote{A Diffie-Hellman modulus for instance.} excess
digits can be removed to return memory to the heap with the mp\_shrink() function.

\index{mp\_shrink}
\begin{alltt}
int mp_shrink (mp_int * a);
\end{alltt}

This will remove excess digits of the mp\_int $a$.  If the operation fails the mp\_int should be intact without the
excess digits being removed.  Note that you can use a shrunk mp\_int in further computations, however, such operations
will require heap operations which can be slow.  It is not ideal to shrink mp\_int variables that you will further
modify in the system (unless you are seriously low on memory).

\begin{small} \begin{alltt}
int main(void)
\{
   mp_int number;
   int result;

   if ((result = mp_init(&number)) != MP_OKAY) \{
      printf("Error initializing the number.  \%s",
             mp_error_to_string(result));
      return EXIT_FAILURE;
   \}

   /* use the number [e.g. pre-computation]  */

   /* We're done with it for now. */
   if ((result = mp_shrink(&number)) != MP_OKAY) \{
      printf("Error shrinking the number.  \%s",
             mp_error_to_string(result));
      return EXIT_FAILURE;
   \}

   /* use it .... */


   /* we're done with it. */
   mp_clear(&number);

   return EXIT_SUCCESS;
\}
\end{alltt} \end{small}

\subsection{Adding additional digits}

Within the mp\_int structure are two parameters which control the limitations of the array of digits that represent
the integer the mp\_int is meant to equal.   The \textit{used} parameter dictates how many digits are significant, that is,
contribute to the value of the mp\_int.  The \textit{alloc} parameter dictates how many digits are currently available in
the array.  If you need to perform an operation that requires more digits you will have to mp\_grow() the mp\_int to
your desired size.

\index{mp\_grow}
\begin{alltt}
int mp_grow (mp_int * a, int size);
\end{alltt}

This will grow the array of digits of $a$ to $size$.  If the \textit{alloc} parameter is already bigger than
$size$ the function will not do anything.

\begin{small} \begin{alltt}
int main(void)
\{
   mp_int number;
   int result;

   if ((result = mp_init(&number)) != MP_OKAY) \{
      printf("Error initializing the number.  \%s",
             mp_error_to_string(result));
      return EXIT_FAILURE;
   \}

   /* use the number */

   /* We need to add 20 digits to the number  */
   if ((result = mp_grow(&number, number.alloc + 20)) != MP_OKAY) \{
      printf("Error growing the number.  \%s",
             mp_error_to_string(result));
      return EXIT_FAILURE;
   \}


   /* use the number */

   /* we're done with it. */
   mp_clear(&number);

   return EXIT_SUCCESS;
\}
\end{alltt} \end{small}

\chapter{Basic Operations}
\section{Copying}

A so called ``deep copy'', where new memory is allocated and all contents of $a$ are copied verbatim into $b$ such that $b = a$ at the end.

\index{mp\_copy}
\begin{alltt}
int mp_copy (mp_int * a, mp_int *b);
\end{alltt}

You can also just swap $a$ and $b$. It does the normal pointer changing with a temporary pointer variable, just that you do not have to.

\index{mp\_exch}
\begin{alltt}
void mp_exch (mp_int * a, mp_int *b);
\end{alltt}

\section{Bit Counting}

To get the position of the lowest bit set (LSB, the Lowest Significant Bit; the number of bits which are zero before the first zero bit )

\index{mp\_cnt\_lsb}
\begin{alltt}
int mp_cnt_lsb(const mp_int *a);
\end{alltt}

To get the position of the highest bit set (MSB, the Most Significant Bit; the number of bits in teh ``bignum'')

\index{mp\_count\_bits}
\begin{alltt}
int mp_count_bits(const mp_int *a);
\end{alltt}


\section{Small Constants}
Setting mp\_ints to small constants is a relatively common operation.  To accommodate these instances there are two
small constant assignment functions.  The first function is used to set a single digit constant while the second sets
an ISO C style ``unsigned long'' constant.  The reason for both functions is efficiency.  Setting a single digit is quick but the
domain of a digit can change (it's always at least $0 \ldots 127$).

\subsection{Single Digit}

Setting a single digit can be accomplished with the following function.

\index{mp\_set}
\begin{alltt}
void mp_set (mp_int * a, mp_digit b);
\end{alltt}

This will zero the contents of $a$ and make it represent an integer equal to the value of $b$.  Note that this
function has a return type of \textbf{void}.  It cannot cause an error so it is safe to assume the function
succeeded.

\begin{small} \begin{alltt}
int main(void)
\{
   mp_int number;
   int result;

   if ((result = mp_init(&number)) != MP_OKAY) \{
      printf("Error initializing the number.  \%s",
             mp_error_to_string(result));
      return EXIT_FAILURE;
   \}

   /* set the number to 5 */
   mp_set(&number, 5);

   /* we're done with it. */
   mp_clear(&number);

   return EXIT_SUCCESS;
\}
\end{alltt} \end{small}

\subsection{Long Constants}

To set a constant that is the size of an ISO C ``unsigned long'' and larger than a single digit the following function
can be used.

\index{mp\_set\_int}
\begin{alltt}
int mp_set_int (mp_int * a, unsigned long b);
\end{alltt}

This will assign the value of the 32-bit variable $b$ to the mp\_int $a$.  Unlike mp\_set() this function will always
accept a 32-bit input regardless of the size of a single digit.  However, since the value may span several digits
this function can fail if it runs out of heap memory.

To get the ``unsigned long'' copy of an mp\_int the following function can be used.

\index{mp\_get\_int}
\begin{alltt}
unsigned long mp_get_int (mp_int * a);
\end{alltt}

This will return the 32 least significant bits of the mp\_int $a$.

\begin{small} \begin{alltt}
int main(void)
\{
   mp_int number;
   int result;

   if ((result = mp_init(&number)) != MP_OKAY) \{
      printf("Error initializing the number.  \%s",
             mp_error_to_string(result));
      return EXIT_FAILURE;
   \}

   /* set the number to 654321 (note this is bigger than 127) */
   if ((result = mp_set_int(&number, 654321)) != MP_OKAY) \{
      printf("Error setting the value of the number.  \%s",
             mp_error_to_string(result));
      return EXIT_FAILURE;
   \}

   printf("number == \%lu", mp_get_int(&number));

   /* we're done with it. */
   mp_clear(&number);

   return EXIT_SUCCESS;
\}
\end{alltt} \end{small}

This should output the following if the program succeeds.

\begin{alltt}
number == 654321
\end{alltt}

\subsection{Long Constants - platform dependent}

\index{mp\_set\_long}
\begin{alltt}
int mp_set_long (mp_int * a, unsigned long b);
\end{alltt}

This will assign the value of the platform-dependent sized variable $b$ to the mp\_int $a$.

To get the ``unsigned long'' copy of an mp\_int the following function can be used.

\index{mp\_get\_long}
\begin{alltt}
unsigned long mp_get_long (mp_int * a);
\end{alltt}

This will return the least significant bits of the mp\_int $a$ that fit into an ``unsigned long''.

\subsection{Long Long Constants}

\index{mp\_set\_long\_long}
\begin{alltt}
int mp_set_long_long (mp_int * a, unsigned long long b);
\end{alltt}

This will assign the value of the 64-bit variable $b$ to the mp\_int $a$.

To get the ``unsigned long long'' copy of an mp\_int the following function can be used.

\index{mp\_get\_long\_long}
\begin{alltt}
unsigned long long mp_get_long_long (mp_int * a);
\end{alltt}

This will return the 64 least significant bits of the mp\_int $a$.

\subsection{Initialize and Setting Constants}
To both initialize and set small constants the following two functions are available.
\index{mp\_init\_set} \index{mp\_init\_set\_int}
\begin{alltt}
int mp_init_set (mp_int * a, mp_digit b);
int mp_init_set_int (mp_int * a, unsigned long b);
\end{alltt}

Both functions work like the previous counterparts except they first mp\_init $a$ before setting the values.

\begin{alltt}
int main(void)
\{
   mp_int number1, number2;
   int    result;

   /* initialize and set a single digit */
   if ((result = mp_init_set(&number1, 100)) != MP_OKAY) \{
      printf("Error setting number1: \%s",
             mp_error_to_string(result));
      return EXIT_FAILURE;
   \}

   /* initialize and set a long */
   if ((result = mp_init_set_int(&number2, 1023)) != MP_OKAY) \{
      printf("Error setting number2: \%s",
             mp_error_to_string(result));
      return EXIT_FAILURE;
   \}

   /* display */
   printf("Number1, Number2 == \%lu, \%lu",
          mp_get_int(&number1), mp_get_int(&number2));

   /* clear */
   mp_clear_multi(&number1, &number2, NULL);

   return EXIT_SUCCESS;
\}
\end{alltt}

If this program succeeds it shall output.
\begin{alltt}
Number1, Number2 == 100, 1023
\end{alltt}

\section{Comparisons}

Comparisons in LibTomMath are always performed in a ``left to right'' fashion.  There are three possible return codes
for any comparison.

\index{MP\_GT} \index{MP\_EQ} \index{MP\_LT}
\begin{figure}[h]
\begin{center}
\begin{tabular}{|c|c|}
\hline \textbf{Result Code} & \textbf{Meaning} \\
\hline MP\_GT & $a > b$ \\
\hline MP\_EQ & $a = b$ \\
\hline MP\_LT & $a < b$ \\
\hline
\end{tabular}
\end{center}
\caption{Comparison Codes for $a, b$}
\label{fig:CMP}
\end{figure}

In figure \ref{fig:CMP} two integers $a$ and $b$ are being compared.  In this case $a$ is said to be ``to the left'' of
$b$.

\subsection{Unsigned comparison}

An unsigned comparison considers only the digits themselves and not the associated \textit{sign} flag of the
mp\_int structures.  This is analogous to an absolute comparison.  The function mp\_cmp\_mag() will compare two
mp\_int variables based on their digits only.

\index{mp\_cmp\_mag}
\begin{alltt}
int mp_cmp_mag(mp_int * a, mp_int * b);
\end{alltt}
This will compare $a$ to $b$ placing $a$ to the left of $b$.  This function cannot fail and will return one of the
three compare codes listed in figure \ref{fig:CMP}.

\begin{small} \begin{alltt}
int main(void)
\{
   mp_int number1, number2;
   int result;

   if ((result = mp_init_multi(&number1, &number2, NULL)) != MP_OKAY) \{
      printf("Error initializing the numbers.  \%s",
             mp_error_to_string(result));
      return EXIT_FAILURE;
   \}

   /* set the number1 to 5 */
   mp_set(&number1, 5);

   /* set the number2 to -6 */
   mp_set(&number2, 6);
   if ((result = mp_neg(&number2, &number2)) != MP_OKAY) \{
      printf("Error negating number2.  \%s",
             mp_error_to_string(result));
      return EXIT_FAILURE;
   \}

   switch(mp_cmp_mag(&number1, &number2)) \{
       case MP_GT:  printf("|number1| > |number2|"); break;
       case MP_EQ:  printf("|number1| = |number2|"); break;
       case MP_LT:  printf("|number1| < |number2|"); break;
   \}

   /* we're done with it. */
   mp_clear_multi(&number1, &number2, NULL);

   return EXIT_SUCCESS;
\}
\end{alltt} \end{small}

If this program\footnote{This function uses the mp\_neg() function which is discussed in section \ref{sec:NEG}.} completes
successfully it should print the following.

\begin{alltt}
|number1| < |number2|
\end{alltt}

This is because $\vert -6 \vert = 6$ and obviously $5 < 6$.

\subsection{Signed comparison}

To compare two mp\_int variables based on their signed value the mp\_cmp() function is provided.

\index{mp\_cmp}
\begin{alltt}
int mp_cmp(mp_int * a, mp_int * b);
\end{alltt}

This will compare $a$ to the left of $b$.  It will first compare the signs of the two mp\_int variables.  If they
differ it will return immediately based on their signs.  If the signs are equal then it will compare the digits
individually.  This function will return one of the compare conditions codes listed in figure \ref{fig:CMP}.

\begin{small} \begin{alltt}
int main(void)
\{
   mp_int number1, number2;
   int result;

   if ((result = mp_init_multi(&number1, &number2, NULL)) != MP_OKAY) \{
      printf("Error initializing the numbers.  \%s",
             mp_error_to_string(result));
      return EXIT_FAILURE;
   \}

   /* set the number1 to 5 */
   mp_set(&number1, 5);

   /* set the number2 to -6 */
   mp_set(&number2, 6);
   if ((result = mp_neg(&number2, &number2)) != MP_OKAY) \{
      printf("Error negating number2.  \%s",
             mp_error_to_string(result));
      return EXIT_FAILURE;
   \}

   switch(mp_cmp(&number1, &number2)) \{
       case MP_GT:  printf("number1 > number2"); break;
       case MP_EQ:  printf("number1 = number2"); break;
       case MP_LT:  printf("number1 < number2"); break;
   \}

   /* we're done with it. */
   mp_clear_multi(&number1, &number2, NULL);

   return EXIT_SUCCESS;
\}
\end{alltt} \end{small}

If this program\footnote{This function uses the mp\_neg() function which is discussed in section \ref{sec:NEG}.} completes
successfully it should print the following.

\begin{alltt}
number1 > number2
\end{alltt}

\subsection{Single Digit}

To compare a single digit against an mp\_int the following function has been provided.

\index{mp\_cmp\_d}
\begin{alltt}
int mp_cmp_d(mp_int * a, mp_digit b);
\end{alltt}

This will compare $a$ to the left of $b$ using a signed comparison.  Note that it will always treat $b$ as
positive.  This function is rather handy when you have to compare against small values such as $1$ (which often
comes up in cryptography).  The function cannot fail and will return one of the tree compare condition codes
listed in figure \ref{fig:CMP}.


\begin{small} \begin{alltt}
int main(void)
\{
   mp_int number;
   int result;

   if ((result = mp_init(&number)) != MP_OKAY) \{
      printf("Error initializing the number.  \%s",
             mp_error_to_string(result));
      return EXIT_FAILURE;
   \}

   /* set the number to 5 */
   mp_set(&number, 5);

   switch(mp_cmp_d(&number, 7)) \{
       case MP_GT:  printf("number > 7"); break;
       case MP_EQ:  printf("number = 7"); break;
       case MP_LT:  printf("number < 7"); break;
   \}

   /* we're done with it. */
   mp_clear(&number);

   return EXIT_SUCCESS;
\}
\end{alltt} \end{small}

If this program functions properly it will print out the following.

\begin{alltt}
number < 7
\end{alltt}

\section{Logical Operations}

Logical operations are operations that can be performed either with simple shifts or boolean operators such as
AND, XOR and OR directly.  These operations are very quick.

\subsection{Multiplication by two}

Multiplications and divisions by any power of two can be performed with quick logical shifts either left or
right depending on the operation.

When multiplying or dividing by two a special case routine can be used which are as follows.
\index{mp\_mul\_2} \index{mp\_div\_2}
\begin{alltt}
int mp_mul_2(mp_int * a, mp_int * b);
int mp_div_2(mp_int * a, mp_int * b);
\end{alltt}

The former will assign twice $a$ to $b$ while the latter will assign half $a$ to $b$.  These functions are fast
since the shift counts and masks are hardcoded into the routines.

\begin{small} \begin{alltt}
int main(void)
\{
   mp_int number;
   int result;

   if ((result = mp_init(&number)) != MP_OKAY) \{
      printf("Error initializing the number.  \%s",
             mp_error_to_string(result));
      return EXIT_FAILURE;
   \}

   /* set the number to 5 */
   mp_set(&number, 5);

   /* multiply by two */
   if ((result = mp\_mul\_2(&number, &number)) != MP_OKAY) \{
      printf("Error multiplying the number.  \%s",
             mp_error_to_string(result));
      return EXIT_FAILURE;
   \}
   switch(mp_cmp_d(&number, 7)) \{
       case MP_GT:  printf("2*number > 7"); break;
       case MP_EQ:  printf("2*number = 7"); break;
       case MP_LT:  printf("2*number < 7"); break;
   \}

   /* now divide by two */
   if ((result = mp\_div\_2(&number, &number)) != MP_OKAY) \{
      printf("Error dividing the number.  \%s",
             mp_error_to_string(result));
      return EXIT_FAILURE;
   \}
   switch(mp_cmp_d(&number, 7)) \{
       case MP_GT:  printf("2*number/2 > 7"); break;
       case MP_EQ:  printf("2*number/2 = 7"); break;
       case MP_LT:  printf("2*number/2 < 7"); break;
   \}

   /* we're done with it. */
   mp_clear(&number);

   return EXIT_SUCCESS;
\}
\end{alltt} \end{small}

If this program is successful it will print out the following text.

\begin{alltt}
2*number > 7
2*number/2 < 7
\end{alltt}

Since $10 > 7$ and $5 < 7$.

To multiply by a power of two the following function can be used.

\index{mp\_mul\_2d}
\begin{alltt}
int mp_mul_2d(mp_int * a, int b, mp_int * c);
\end{alltt}

This will multiply $a$ by $2^b$ and store the result in ``c''.  If the value of $b$ is less than or equal to
zero the function will copy $a$ to ``c'' without performing any further actions.  The multiplication itself
is implemented as a right-shift operation of $a$ by $b$ bits.

To divide by a power of two use the following.

\index{mp\_div\_2d}
\begin{alltt}
int mp_div_2d (mp_int * a, int b, mp_int * c, mp_int * d);
\end{alltt}
Which will divide $a$ by $2^b$, store the quotient in ``c'' and the remainder in ``d'.  If $b \le 0$ then the
function simply copies $a$ over to ``c'' and zeros $d$.  The variable $d$ may be passed as a \textbf{NULL}
value to signal that the remainder is not desired.  The division itself is implemented as a left-shift
operation of $a$ by $b$ bits.

\index{mp\_tc\_div\_2d}\label{arithrightshift}
\begin{alltt}
int mp_tc_div_2d (mp_int * a, int b, mp_int * c, mp_int * d);
\end{alltt}
The two-complement version of the function above. This can be used to implement arbitrary-precision two-complement integers together with the two-complement bit-wise operations at page \ref{tcbitwiseops}.


It is also not very uncommon to need just the power of two $2^b$;  for example the startvalue for the Newton method.

\index{mp\_2expt}
\begin{alltt}
int mp_2expt(mp_int *a, int b);
\end{alltt}
It is faster than doing it by shifting $1$ with \texttt{mp\_mul\_2d}.

\subsection{Polynomial Basis Operations}

Strictly speaking the organization of the integers within the mp\_int structures is what is known as a
``polynomial basis''.  This simply means a field element is stored by divisions of a radix.  For example, if
$f(x) = \sum_{i=0}^{k} y_ix^k$ for any vector $\vec y$ then the array of digits in $\vec y$ are said to be
the polynomial basis representation of $z$ if $f(\beta) = z$ for a given radix $\beta$.

To multiply by the polynomial $g(x) = x$ all you have todo is shift the digits of the basis left one place.  The
following function provides this operation.

\index{mp\_lshd}
\begin{alltt}
int mp_lshd (mp_int * a, int b);
\end{alltt}

This will multiply $a$ in place by $x^b$ which is equivalent to shifting the digits left $b$ places and inserting zeros
in the least significant digits.  Similarly to divide by a power of $x$ the following function is provided.

\index{mp\_rshd}
\begin{alltt}
void mp_rshd (mp_int * a, int b)
\end{alltt}
This will divide $a$ in place by $x^b$ and discard the remainder.  This function cannot fail as it performs the operations
in place and no new digits are required to complete it.

\subsection{AND, OR, XOR and COMPLEMENT Operations}

While AND, OR and XOR operations are not typical ``bignum functions'' they can be useful in several instances.  The
three functions are prototyped as follows.

\index{mp\_or} \index{mp\_and} \index{mp\_xor}
\begin{alltt}
int mp_or  (mp_int * a, mp_int * b, mp_int * c);
int mp_and (mp_int * a, mp_int * b, mp_int * c);
int mp_xor (mp_int * a, mp_int * b, mp_int * c);
\end{alltt}

Which compute $c = a \odot b$ where $\odot$ is one of OR, AND or XOR.

The following four functions allow implementing arbitrary-precision two-complement numbers.

\index{mp\_tc\_or} \index{mp\_tc\_and} \index{mp\_tc\_xor} \index{mp\_complement} \label{tcbitwiseops}
\begin{alltt}
int mp_tc_or  (mp_int * a, mp_int * b, mp_int * c);
int mp_tc_and (mp_int * a, mp_int * b, mp_int * c);
int mp_tc_xor (mp_int * a, mp_int * b, mp_int * c);
int mp_complement(const mp_int *a, mp_int *b);
\end{alltt}

They compute $c = a \odot b$ as above if both $a$ and $b$ are positive. Negative values are converted into their two-complement representations first. The function \texttt{mp\_complement} computes a two-complement $b = \sim a$.


\subsection{Bit Picking}
\index{mp\_get\_bit}
\begin{alltt}
int mp_get_bit(mp_int *a, int b)
\end{alltt}

Pick a bit: returns \texttt{MP\_YES} if the bit at position $b$ (0-index) is set, that is if it is 1 (one), \texttt{MP\_NO}
if the bit is 0 (zero) and \texttt{MP\_VAL} if $b < 0$.

\section{Addition and Subtraction}

To compute an addition or subtraction the following two functions can be used.

\index{mp\_add} \index{mp\_sub}
\begin{alltt}
int mp_add (mp_int * a, mp_int * b, mp_int * c);
int mp_sub (mp_int * a, mp_int * b, mp_int * c)
\end{alltt}

Which perform $c = a \odot b$ where $\odot$ is one of signed addition or subtraction.  The operations are fully sign
aware.

\section{Sign Manipulation}
\subsection{Negation}
\label{sec:NEG}
Simple integer negation can be performed with the following.

\index{mp\_neg}
\begin{alltt}
int mp_neg (mp_int * a, mp_int * b);
\end{alltt}

Which assigns $-a$ to $b$.

\subsection{Absolute}
Simple integer absolutes can be performed with the following.

\index{mp\_abs}
\begin{alltt}
int mp_abs (mp_int * a, mp_int * b);
\end{alltt}

Which assigns $\vert a \vert$ to $b$.

\section{Integer Division and Remainder}
To perform a complete and general integer division with remainder use the following function.

\index{mp\_div}
\begin{alltt}
int mp_div (mp_int * a, mp_int * b, mp_int * c, mp_int * d);
\end{alltt}

This divides $a$ by $b$ and stores the quotient in $c$ and $d$.  The signed quotient is computed such that
$bc + d = a$.  Note that either of $c$ or $d$ can be set to \textbf{NULL} if their value is not required.  If
$b$ is zero the function returns \textbf{MP\_VAL}.


\chapter{Multiplication and Squaring}
\section{Multiplication}
A full signed integer multiplication can be performed with the following.
\index{mp\_mul}
\begin{alltt}
int mp_mul (mp_int * a, mp_int * b, mp_int * c);
\end{alltt}
Which assigns the full signed product $ab$ to $c$.  This function actually breaks into one of four cases which are
specific multiplication routines optimized for given parameters.  First there are the Toom-Cook multiplications which
should only be used with very large inputs.  This is followed by the Karatsuba multiplications which are for moderate
sized inputs.  Then followed by the Comba and baseline multipliers.

Fortunately for the developer you don't really need to know this unless you really want to fine tune the system.  mp\_mul()
will determine on its own\footnote{Some tweaking may be required.} what routine to use automatically when it is called.

\begin{alltt}
int main(void)
\{
   mp_int number1, number2;
   int result;

   /* Initialize the numbers */
   if ((result = mp_init_multi(&number1,
                               &number2, NULL)) != MP_OKAY) \{
      printf("Error initializing the numbers.  \%s",
             mp_error_to_string(result));
      return EXIT_FAILURE;
   \}

   /* set the terms */
   if ((result = mp_set_int(&number, 257)) != MP_OKAY) \{
      printf("Error setting number1.  \%s",
             mp_error_to_string(result));
      return EXIT_FAILURE;
   \}

   if ((result = mp_set_int(&number2, 1023)) != MP_OKAY) \{
      printf("Error setting number2.  \%s",
             mp_error_to_string(result));
      return EXIT_FAILURE;
   \}

   /* multiply them */
   if ((result = mp_mul(&number1, &number2,
                        &number1)) != MP_OKAY) \{
      printf("Error multiplying terms.  \%s",
             mp_error_to_string(result));
      return EXIT_FAILURE;
   \}

   /* display */
   printf("number1 * number2 == \%lu", mp_get_int(&number1));

   /* free terms and return */
   mp_clear_multi(&number1, &number2, NULL);

   return EXIT_SUCCESS;
\}
\end{alltt}

If this program succeeds it shall output the following.

\begin{alltt}
number1 * number2 == 262911
\end{alltt}

\section{Squaring}
Since squaring can be performed faster than multiplication it is performed it's own function instead of just using
mp\_mul().

\index{mp\_sqr}
\begin{alltt}
int mp_sqr (mp_int * a, mp_int * b);
\end{alltt}

Will square $a$ and store it in $b$.  Like the case of multiplication there are four different squaring
algorithms all which can be called from mp\_sqr().  It is ideal to use mp\_sqr over mp\_mul when squaring terms because
of the speed difference.

\section{Tuning Polynomial Basis Routines}

Both of the Toom-Cook and Karatsuba multiplication algorithms are faster than the traditional $O(n^2)$ approach that
the Comba and baseline algorithms use.  At $O(n^{1.464973})$ and $O(n^{1.584962})$ running times respectively they require
considerably less work.  For example, a 10000-digit multiplication would take roughly 724,000 single precision
multiplications with Toom-Cook or 100,000,000 single precision multiplications with the standard Comba (a factor
of 138).

So why not always use Karatsuba or Toom-Cook?   The simple answer is that they have so much overhead that they're not
actually faster than Comba until you hit distinct  ``cutoff'' points.  For Karatsuba with the default configuration,
GCC 3.3.1 and an Athlon XP processor the cutoff point is roughly 110 digits (about 70 for the Intel P4).  That is, at
110 digits Karatsuba and Comba multiplications just about break even and for 110+ digits Karatsuba is faster.

Toom-Cook has incredible overhead and is probably only useful for very large inputs.  So far no known cutoff points
exist and for the most part I just set the cutoff points very high to make sure they're not called.

A demo program in the ``etc/'' directory of the project called ``tune.c'' can be used to find the cutoff points.  This
can be built with GCC as follows

\begin{alltt}
make XXX
\end{alltt}
Where ``XXX'' is one of the following entries from the table \ref{fig:tuning}.

\begin{figure}[h]
\begin{center}
\begin{small}
\begin{tabular}{|l|l|}
\hline \textbf{Value of XXX} & \textbf{Meaning} \\
\hline tune & Builds portable tuning application \\
\hline tune86 & Builds x86 (pentium and up) program for COFF \\
\hline tune86c & Builds x86 program for Cygwin \\
\hline tune86l & Builds x86 program for Linux (ELF format) \\
\hline
\end{tabular}
\end{small}
\end{center}
\caption{Build Names for Tuning Programs}
\label{fig:tuning}
\end{figure}

When the program is running it will output a series of measurements for different cutoff points.  It will first find
good Karatsuba squaring and multiplication points.  Then it proceeds to find Toom-Cook points.  Note that the Toom-Cook
tuning takes a very long time as the cutoff points are likely to be very high.

\chapter{Modular Reduction}

Modular reduction is process of taking the remainder of one quantity divided by another.  Expressed
as (\ref{eqn:mod}) the modular reduction is equivalent to the remainder of $b$ divided by $c$.

\begin{equation}
a \equiv b \mbox{ (mod }c\mbox{)}
\label{eqn:mod}
\end{equation}

Of particular interest to cryptography are reductions where $b$ is limited to the range $0 \le b < c^2$ since particularly
fast reduction algorithms can be written for the limited range.

Note that one of the four optimized reduction algorithms are automatically chosen in the modular exponentiation
algorithm mp\_exptmod when an appropriate modulus is detected.

\section{Straight Division}
In order to effect an arbitrary modular reduction the following algorithm is provided.

\index{mp\_mod}
\begin{alltt}
int mp_mod(mp_int *a, mp_int *b, mp_int *c);
\end{alltt}

This reduces $a$ modulo $b$ and stores the result in $c$.  The sign of $c$ shall agree with the sign
of $b$.  This algorithm accepts an input $a$ of any range and is not limited by $0 \le a < b^2$.

\section{Barrett Reduction}

Barrett reduction is a generic optimized reduction algorithm that requires pre--computation to achieve
a decent speedup over straight division.  First a $\mu$ value must be precomputed with the following function.

\index{mp\_reduce\_setup}
\begin{alltt}
int mp_reduce_setup(mp_int *a, mp_int *b);
\end{alltt}

Given a modulus in $b$ this produces the required $\mu$ value in $a$.  For any given modulus this only has to
be computed once.  Modular reduction can now be performed with the following.

\index{mp\_reduce}
\begin{alltt}
int mp_reduce(mp_int *a, mp_int *b, mp_int *c);
\end{alltt}

This will reduce $a$ in place modulo $b$ with the precomputed $\mu$ value in $c$.  $a$ must be in the range
$0 \le a < b^2$.

\begin{alltt}
int main(void)
\{
   mp_int   a, b, c, mu;
   int      result;

   /* initialize a,b to desired values, mp_init mu,
    * c and set c to 1...we want to compute a^3 mod b
    */

   /* get mu value */
   if ((result = mp_reduce_setup(&mu, b)) != MP_OKAY) \{
      printf("Error getting mu.  \%s",
             mp_error_to_string(result));
      return EXIT_FAILURE;
   \}

   /* square a to get c = a^2 */
   if ((result = mp_sqr(&a, &c)) != MP_OKAY) \{
      printf("Error squaring.  \%s",
             mp_error_to_string(result));
      return EXIT_FAILURE;
   \}

   /* now reduce `c' modulo b */
   if ((result = mp_reduce(&c, &b, &mu)) != MP_OKAY) \{
      printf("Error reducing.  \%s",
             mp_error_to_string(result));
      return EXIT_FAILURE;
   \}

   /* multiply a to get c = a^3 */
   if ((result = mp_mul(&a, &c, &c)) != MP_OKAY) \{
      printf("Error reducing.  \%s",
             mp_error_to_string(result));
      return EXIT_FAILURE;
   \}

   /* now reduce `c' modulo b  */
   if ((result = mp_reduce(&c, &b, &mu)) != MP_OKAY) \{
      printf("Error reducing.  \%s",
             mp_error_to_string(result));
      return EXIT_FAILURE;
   \}

   /* c now equals a^3 mod b */

   return EXIT_SUCCESS;
\}
\end{alltt}

This program will calculate $a^3 \mbox{ mod }b$ if all the functions succeed.

\section{Montgomery Reduction}

Montgomery is a specialized reduction algorithm for any odd moduli.  Like Barrett reduction a pre--computation
step is required.  This is accomplished with the following.

\index{mp\_montgomery\_setup}
\begin{alltt}
int mp_montgomery_setup(mp_int *a, mp_digit *mp);
\end{alltt}

For the given odd moduli $a$ the pre--computation value is placed in $mp$.  The reduction is computed with the
following.

\index{mp\_montgomery\_reduce}
\begin{alltt}
int mp_montgomery_reduce(mp_int *a, mp_int *m, mp_digit mp);
\end{alltt}
This reduces $a$ in place modulo $m$ with the pre--computed value $mp$.   $a$ must be in the range
$0 \le a < b^2$.

Montgomery reduction is faster than Barrett reduction for moduli smaller than the ``comba'' limit.  With the default
setup for instance, the limit is $127$ digits ($3556$--bits).   Note that this function is not limited to
$127$ digits just that it falls back to a baseline algorithm after that point.

An important observation is that this reduction does not return $a \mbox{ mod }m$ but $aR^{-1} \mbox{ mod }m$
where $R = \beta^n$, $n$ is the n number of digits in $m$ and $\beta$ is radix used (default is $2^{28}$).

To quickly calculate $R$ the following function was provided.

\index{mp\_montgomery\_calc\_normalization}
\begin{alltt}
int mp_montgomery_calc_normalization(mp_int *a, mp_int *b);
\end{alltt}
Which calculates $a = R$ for the odd moduli $b$ without using multiplication or division.

The normal modus operandi for Montgomery reductions is to normalize the integers before entering the system.  For
example, to calculate $a^3 \mbox { mod }b$ using Montgomery reduction the value of $a$ can be normalized by
multiplying it by $R$.  Consider the following code snippet.

\begin{alltt}
int main(void)
\{
   mp_int   a, b, c, R;
   mp_digit mp;
   int      result;

   /* initialize a,b to desired values,
    * mp_init R, c and set c to 1....
    */

   /* get normalization */
   if ((result = mp_montgomery_calc_normalization(&R, b)) != MP_OKAY) \{
      printf("Error getting norm.  \%s",
             mp_error_to_string(result));
      return EXIT_FAILURE;
   \}

   /* get mp value */
   if ((result = mp_montgomery_setup(&c, &mp)) != MP_OKAY) \{
      printf("Error setting up montgomery.  \%s",
             mp_error_to_string(result));
      return EXIT_FAILURE;
   \}

   /* normalize `a' so now a is equal to aR */
   if ((result = mp_mulmod(&a, &R, &b, &a)) != MP_OKAY) \{
      printf("Error computing aR.  \%s",
             mp_error_to_string(result));
      return EXIT_FAILURE;
   \}

   /* square a to get c = a^2R^2 */
   if ((result = mp_sqr(&a, &c)) != MP_OKAY) \{
      printf("Error squaring.  \%s",
             mp_error_to_string(result));
      return EXIT_FAILURE;
   \}

   /* now reduce `c' back down to c = a^2R^2 * R^-1 == a^2R */
   if ((result = mp_montgomery_reduce(&c, &b, mp)) != MP_OKAY) \{
      printf("Error reducing.  \%s",
             mp_error_to_string(result));
      return EXIT_FAILURE;
   \}

   /* multiply a to get c = a^3R^2 */
   if ((result = mp_mul(&a, &c, &c)) != MP_OKAY) \{
      printf("Error reducing.  \%s",
             mp_error_to_string(result));
      return EXIT_FAILURE;
   \}

   /* now reduce `c' back down to c = a^3R^2 * R^-1 == a^3R */
   if ((result = mp_montgomery_reduce(&c, &b, mp)) != MP_OKAY) \{
      printf("Error reducing.  \%s",
             mp_error_to_string(result));
      return EXIT_FAILURE;
   \}

   /* now reduce (again) `c' back down to c = a^3R * R^-1 == a^3 */
   if ((result = mp_montgomery_reduce(&c, &b, mp)) != MP_OKAY) \{
      printf("Error reducing.  \%s",
             mp_error_to_string(result));
      return EXIT_FAILURE;
   \}

   /* c now equals a^3 mod b */

   return EXIT_SUCCESS;
\}
\end{alltt}

This particular example does not look too efficient but it demonstrates the point of the algorithm.  By
normalizing the inputs the reduced results are always of the form $aR$ for some variable $a$.  This allows
a single final reduction to correct for the normalization and the fast reduction used within the algorithm.

For more details consider examining the file \textit{bn\_mp\_exptmod\_fast.c}.

\section{Restricted Diminished Radix}

``Diminished Radix'' reduction refers to reduction with respect to moduli that are amenable to simple
digit shifting and small multiplications.  In this case the ``restricted'' variant refers to moduli of the
form $\beta^k - p$ for some $k \ge 0$ and $0 < p < \beta$ where $\beta$ is the radix (default to $2^{28}$).

As in the case of Montgomery reduction there is a pre--computation phase required for a given modulus.

\index{mp\_dr\_setup}
\begin{alltt}
void mp_dr_setup(mp_int *a, mp_digit *d);
\end{alltt}

This computes the value required for the modulus $a$ and stores it in $d$.  This function cannot fail
and does not return any error codes.  After the pre--computation a reduction can be performed with the
following.

\index{mp\_dr\_reduce}
\begin{alltt}
int mp_dr_reduce(mp_int *a, mp_int *b, mp_digit mp);
\end{alltt}

This reduces $a$ in place modulo $b$ with the pre--computed value $mp$.  $b$ must be of a restricted
diminished radix form and $a$ must be in the range $0 \le a < b^2$.  Diminished radix reductions are
much faster than both Barrett and Montgomery reductions as they have a much lower asymptotic running time.

Since the moduli are restricted this algorithm is not particularly useful for something like Rabin, RSA or
BBS cryptographic purposes.  This reduction algorithm is useful for Diffie-Hellman and ECC where fixed
primes are acceptable.

Note that unlike Montgomery reduction there is no normalization process.  The result of this function is
equal to the correct residue.

\section{Unrestricted Diminished Radix}

Unrestricted reductions work much like the restricted counterparts except in this case the moduli is of the
form $2^k - p$ for $0 < p < \beta$.  In this sense the unrestricted reductions are more flexible as they
can be applied to a wider range of numbers.

\index{mp\_reduce\_2k\_setup}
\begin{alltt}
int mp_reduce_2k_setup(mp_int *a, mp_digit *d);
\end{alltt}

This will compute the required $d$ value for the given moduli $a$.

\index{mp\_reduce\_2k}
\begin{alltt}
int mp_reduce_2k(mp_int *a, mp_int *n, mp_digit d);
\end{alltt}

This will reduce $a$ in place modulo $n$ with the pre--computed value $d$.  From my experience this routine is
slower than mp\_dr\_reduce but faster for most moduli sizes than the Montgomery reduction.

\section{Combined Modular Reduction}

Some of the combinations of an arithmetic operations followed by a modular reduction can be done in a faster way. The ones implemented are:

Addition $d = (a + b) \mod c$
\index{mp\_addmod}
\begin{alltt}
int mp_addmod(const mp_int *a, const mp_int *b, const mp_int *c, mp_int *d);
\end{alltt}

Subtraction  $d = (a - b) \mod c$
\begin{alltt}
int mp_submod(const mp_int *a, const mp_int *b, const mp_int *c, mp_int *d);
\end{alltt}

Multiplication $d = (ab) \mod c$
\begin{alltt}
int mp_mulmod(const mp_int *a, const mp_int *b, const mp_int *c, mp_int *d);
\end{alltt}

Squaring  $d = (a^2) \mod c$
\begin{alltt}
int mp_sqrmod(const mp_int *a, const mp_int *b, const mp_int *c, mp_int *d);
\end{alltt}



\chapter{Exponentiation}
\section{Single Digit Exponentiation}
\index{mp\_expt\_d\_ex}
\begin{alltt}
int mp_expt_d_ex (mp_int * a, mp_digit b, mp_int * c, int fast)
\end{alltt}
This function computes $c = a^b$.

With parameter \textit{fast} set to $0$ the old version of the algorithm is used,
when \textit{fast} is $1$, a faster but not statically timed version of the algorithm is used.

The old version uses a simple binary left-to-right algorithm.
It is faster than repeated multiplications by $a$ for all values of $b$ greater than three.

The new version uses a binary right-to-left algorithm.

The difference between the old and the new version is that the old version always
executes $DIGIT\_BIT$ iterations. The new algorithm executes only $n$ iterations
where $n$ is equal to the position of the highest bit that is set in $b$.

\index{mp\_expt\_d}
\begin{alltt}
int mp_expt_d (mp_int * a, mp_digit b, mp_int * c)
\end{alltt}
mp\_expt\_d(a, b, c) is a wrapper function to mp\_expt\_d\_ex(a, b, c, 0).

\index{mp\_expt}
\begin{alltt}
int mp_expt (mp_int * a, const mp_int *b, mp_int * c)
\end{alltt}
Same as \texttt{mp\_expt\_d(a, b, c)} except that \texttt{b} is a \texttt{mp\_int}, Useful for small \texttt{MP\_xBIT}.

\section{Modular Exponentiation}
\index{mp\_exptmod}
\begin{alltt}
int mp_exptmod (mp_int * G, mp_int * X, mp_int * P, mp_int * Y)
\end{alltt}
This computes $Y \equiv G^X \mbox{ (mod }P\mbox{)}$ using a variable width sliding window algorithm.  This function
will automatically detect the fastest modular reduction technique to use during the operation.  For negative values of
$X$ the operation is performed as $Y \equiv (G^{-1} \mbox{ mod }P)^{\vert X \vert} \mbox{ (mod }P\mbox{)}$ provided that
$gcd(G, P) = 1$.

This function is actually a shell around the two internal exponentiation functions.  This routine will automatically
detect when Barrett, Montgomery, Restricted and Unrestricted Diminished Radix based exponentiation can be used.  Generally
moduli of the a ``restricted diminished radix'' form lead to the fastest modular exponentiations.  Followed by Montgomery
and the other two algorithms.

\section{Modulus a Power of Two}
\index{mp\_mod\_2d}
\begin{alltt}
int mp_mod_2d(const mp_int *a, int b, mp_int *c)
\end{alltt}
It calculates $c = a \mod 2^b$.

\section{Root Finding}
\index{mp\_n\_root}
\begin{alltt}
int mp_n_root (mp_int * a, mp_digit b, mp_int * c)
\end{alltt}
This computes $c = a^{1/b}$ such that $c^b \le a$ and $(c+1)^b > a$.  The implementation of this function is not
ideal for values of $b$ greater than three.  It will work but become very slow.  So unless you are working with very small
numbers (less than 1000 bits) I'd avoid $b > 3$ situations.  Will return a positive root only for even roots and return
a root with the sign of the input for odd roots.  For example, performing $4^{1/2}$ will return $2$ whereas $(-8)^{1/3}$
will return $-2$.

This algorithm uses the ``Newton Approximation'' method and will converge on the correct root fairly quickly.  Since
the algorithm requires raising $a$ to the power of $b$ it is not ideal to attempt to find roots for large
values of $b$.  If particularly large roots are required then a factor method could be used instead.  For example,
$a^{1/16}$ is equivalent to $\left (a^{1/4} \right)^{1/4}$ or simply
$\left ( \left ( \left ( a^{1/2} \right )^{1/2} \right )^{1/2} \right )^{1/2}$


The square root  $c = a^{1/2}$ (with the same conditions $c^2 \le a$ and $(c+1)^2 > a$) is implemented with a faster algorithm.

\index{mp\_sqrt}
\begin{alltt}
int mp_sqrt (mp_int * a, mp_digit b, mp_int * c)
\end{alltt}


\chapter{Logarithm}
\section{Integer Logarithm}
A logarithm function for positive integer input \texttt{a, base} computing  $\floor{\log_bx}$ such that $(\ilog_bx)^b \le x$.
\index{mp\_ilogb}
\begin{alltt}
int mp_ilogb(mp_int *a, mp_digit base, mp_int *c)
\end{alltt}
\subsection{Example}
\begin{alltt}
#include <stdlib.h>
#include <stdio.h>
#include <errno.h>
/* Must be defined before including tommath.h */
#define LTM_USE_EXTRA_FUNCTIONS
#include <tommath.h>

int main(int argc, char **argv)
{
   mp_int x, output;
   mp_digit base;
   int e;

   if (argc != 3) {
      fprintf(stderr,"Usage %s base x\textbackslash{}n", argv[0]);
      exit(EXIT_FAILURE);
   }
   if ((e = mp_init_multi(&x, &output, NULL)) != MP_OKAY) {
      fprintf(stderr,"mp_init failed: \textbackslash{}"%s\textbackslash{}"\textbackslash{}n",
                     mp_error_to_string(e));
              exit(EXIT_FAILURE);
   }
   errno = 0;
#ifdef MP_64BIT
   base = (mp_digit)strtoull(argv[1], NULL, 10);
#else
   base = (mp_digit)strtoul(argv[1], NULL, 10);
#endif
   if ((errno == ERANGE) || (base > (base & MP_MASK))) {
      fprintf(stderr,"strtoul(l) failed: input out of range\textbackslash{}n");
      exit(EXIT_FAILURE);
   }
   if ((e = mp_read_radix(&x, argv[2], 10)) != MP_OKAY) {
      fprintf(stderr,"mp_read_radix failed: \textbackslash{}"%s\textbackslash{}"\textbackslash{}n",
                      mp_error_to_string(e));
      exit(EXIT_FAILURE);
   }

   if ((e = mp_ilogb(&x, base, &output)) != MP_OKAY) {
      fprintf(stderr,"mp_ilogb failed: \textbackslash{}"%s\textbackslash{}"\textbackslash{}n",
                      mp_error_to_string(e));
      exit(EXIT_FAILURE);
   }

   if ((e = mp_fwrite(&output, 10, stdout)) != MP_OKAY) {
      fprintf(stderr,"mp_fwrite failed: \textbackslash{}"%s\textbackslash{}"\textbackslash{}n",
                      mp_error_to_string(e));
      exit(EXIT_FAILURE);
   }
   putchar('\textbackslash{}n');

   mp_clear_multi(&x, &output, NULL);
   exit(EXIT_SUCCESS);
}
\end{alltt}

\chapter{Large Prime Numbers}
\section{Trial Division}
\index{mp\_prime\_is\_divisible}
\begin{alltt}
int mp_prime_is_divisible (mp_int * a, int *result)
\end{alltt}
This will attempt to evenly divide $a$ by a list of primes\footnote{Default is the first 256 primes.} and store the
outcome in ``result''.  That is if $result = 0$ then $a$ is not divisible by the primes, otherwise it is.  Note that
if the function does not return \textbf{MP\_OKAY} the value in ``result'' should be considered undefined\footnote{Currently
the default is to set it to zero first.}.

\section{Fermat Test}
\index{mp\_prime\_fermat}
\begin{alltt}
int mp_prime_fermat (mp_int * a, mp_int * b, int *result)
\end{alltt}
Performs a Fermat primality test to the base $b$.  That is it computes $b^a \mbox{ mod }a$ and tests whether the value is
equal to $b$ or not.  If the values are equal then $a$ is probably prime and $result$ is set to one.  Otherwise $result$
is set to zero.

\section{Miller-Rabin Test}
\index{mp\_prime\_miller\_rabin}
\begin{alltt}
int mp_prime_miller_rabin (mp_int * a, mp_int * b, int *result)
\end{alltt}
Performs a Miller-Rabin test to the base $b$ of $a$.  This test is much stronger than the Fermat test and is very hard to
fool (besides with Carmichael numbers).  If $a$ passes the test (therefore is probably prime) $result$ is set to one.
Otherwise $result$ is set to zero.

Note that is suggested that you use the Miller-Rabin test instead of the Fermat test since all of the failures of
Miller-Rabin are a subset of the failures of the Fermat test.

\subsection{Required Number of Tests}
Generally to ensure a number is very likely to be prime you have to perform the Miller-Rabin with at least a half-dozen
or so unique bases.  However, it has been proven that the probability of failure goes down as the size of the input goes up.
This is why a simple function has been provided to help out.

\index{mp\_prime\_rabin\_miller\_trials}
\begin{alltt}
int mp_prime_rabin_miller_trials(int size)
\end{alltt}
This returns the number of trials required for a $2^{-96}$ (or lower) probability of failure for a given ``size'' expressed
in bits.  This comes in handy specially since larger numbers are slower to test.  For example, a 512-bit number would
require ten tests whereas a 1024-bit number would only require four tests.

You should always still perform a trial division before a Miller-Rabin test though.

A small table, broke in two for typographical reasons, with the number of rounds of Miller-Rabin tests is shown below.
The first column is the number of bits $b$ in the prime $p = 2^b$, the numbers in the first row represent the
probability that the number that all of the Miller-Rabin tests deemed a pseudoprime is actually a composite. There is a deterministic test for numbers smaller than $2^{80}$.

\begin{table}[h]
\begin{center}
\begin{tabular}{c c c c c c c}
\textbf{bits} & $\mathbf{2^{-80}}$ & $\mathbf{2^{-96}}$ & $\mathbf{2^{-112}}$ & $\mathbf{2^{-128}}$ & $\mathbf{2^{-160}}$ & $\mathbf{2^{-192}}$ \\
80    & 31 & 39 & 47 & 55 & 71 & 87  \\
96    & 29 & 37 & 45 & 53 & 69 & 85  \\
128   & 24 & 32 & 40 & 48 & 64 & 80  \\
160   & 19 & 27 & 35 & 43 & 59 & 75  \\
192   & 15 & 21 & 29 & 37 & 53 & 69  \\
256   & 10 & 15 & 20 & 27 & 43 & 59  \\
384   & 7  & 9  & 12 & 16 & 25 & 38  \\
512   & 5  & 7  & 9  & 12 & 18 & 26  \\
768   & 4  & 5  & 6  & 8  & 11 & 16  \\
1024  & 3  & 4  & 5  & 6  & 9  & 12  \\
1536  & 2  & 3  & 3  & 4  & 6  & 8   \\
2048  & 2  & 2  & 3  & 3  & 4  & 6   \\
3072  & 1  & 2  & 2  & 2  & 3  & 4   \\
4096  & 1  & 1  & 2  & 2  & 2  & 3   \\
6144  & 1  & 1  & 1  & 1  & 2  & 2   \\
8192  & 1  & 1  & 1  & 1  & 2  & 2   \\
12288 & 1  & 1  & 1  & 1  & 1  & 1   \\
16384 & 1  & 1  & 1  & 1  & 1  & 1   \\
24576 & 1  & 1  & 1  & 1  & 1  & 1   \\
32768 & 1  & 1  & 1  & 1  & 1  & 1
\end{tabular}
\caption{ Number of Miller-Rabin rounds. Part I } \label{table:millerrabinrunsp1}
\end{center}
\end{table}
\newpage
\begin{table}[h]
\begin{center}
\begin{tabular}{c c c c c c c c}
\textbf{bits} &$\mathbf{2^{-224}}$ & $\mathbf{2^{-256}}$ & $\mathbf{2^{-288}}$ & $\mathbf{2^{-320}}$ & $\mathbf{2^{-352}}$ & $\mathbf{2^{-384}}$ & $\mathbf{2^{-416}}$\\
80    & 103 & 119 & 135 & 151 & 167 & 183 & 199 \\
96    & 101 & 117 & 133 & 149 & 165 & 181 & 197 \\
128   & 96  & 112 & 128 & 144 & 160 & 176 & 192 \\
160   & 91  & 107 & 123 & 139 & 155 & 171 & 187 \\
192   & 85  & 101 & 117 & 133 & 149 & 165 & 181 \\
256   & 75  & 91  & 107 & 123 & 139 & 155 & 171 \\
384   & 54  & 70  & 86  & 102 & 118 & 134 & 150 \\
512   & 36  & 49  & 65  & 81  & 97  & 113 & 129 \\
768   & 22  & 29  & 37  & 47  & 58  & 70  & 86  \\
1024  & 16  & 21  & 26  & 33  & 40  & 48  & 58  \\
1536  & 10  & 13  & 17  & 21  & 25  & 30  & 35  \\
2048  & 8   & 10  & 13  & 15  & 18  & 22  & 26  \\
3072  & 5   & 7   & 8	& 10  & 12  & 14  & 17  \\
4096  & 4   & 5   & 6	& 8   & 9   & 11  & 12  \\
6144  & 3   & 4   & 4	& 5   & 6   & 7   & 8	\\
8192  & 2   & 3   & 3	& 4   & 5   & 6   & 6	\\
12288 & 2   & 2   & 2	& 3   & 3   & 4   & 4	\\
16384 & 1   & 2   & 2	& 2   & 3   & 3   & 3	\\
24576 & 1   & 1   & 2	& 2   & 2   & 2   & 2	\\
32768 & 1   & 1   & 1	& 1   & 2   & 2   & 2
\end{tabular}
\caption{ Number of Miller-Rabin rounds. Part II } \label{table:millerrabinrunsp2}
\end{center}
\end{table}

Determining the probability needed to pick the right column is a bit harder. Fips 186.4, for example has $2^{-80}$ for $512$ bit large numbers, $2^{-112}$ for $1024$ bits, and $2^{128}$ for $1536$ bits. It can be seen in table \ref{table:millerrabinrunsp1} that those combinations follow the diagonal from $(512,2^{-80})$ downwards and to the right to gain a lower probability of getting a composite declared a pseudoprime for the same amount of work or less.

If this version of the library has the strong Lucas-Selfridge and/or the Frobenius-Underwood test implemented only one or two rounds of the Miller-Rabin test with a random base is necessary for numbers larger than or equal to $1024$ bits.


\section{Strong Lucas-Selfridge Test}
\index{mp\_prime\_strong\_lucas\_selfridge}
\begin{alltt}
int mp_prime_strong_lucas_selfridge(const mp_int *a, int *result)
\end{alltt}
Performs a strong Lucas-Selfridge test. The strong Lucas-Selfridge test together with the Rabin-Miler test with bases $2$ and $3$ resemble the BPSW test. The single internal use is a compile-time option in \texttt{mp\_prime\_is\_prime} and can be excluded
from the Libtommath build if not needed.

\section{Frobenius (Underwood)  Test}
\index{mp\_prime\_frobenius\_underwood}
\begin{alltt}
int mp_prime_frobenius_underwood(const mp_int *N, int *result)
\end{alltt}
Performs the variant of the Frobenius test as described by Paul Underwood. The single internal use is in
\texttt{mp\_prime\_is\_prime} for \texttt{MP\_8BIT} only but can be included at build-time for all other sizes
if the preprocessor macro \texttt{LTM\_USE\_FROBENIUS\_TEST} is defined.

It returns \texttt{MP\_ITER} if the number of iterations is exhausted, assumes a composite as the input and sets \texttt{result} accordingly. This will reduce the set of available pseudoprimes by a very small amount: test with large datasets (more than $10^{10}$ numbers, both randomly chosen and sequences of odd numbers with a random start point) found only 31 (thirty-one) numbers with $a > 120$ and none at all with just an additional simple check for divisors $d < 2^8$.

\section{Primality Testing}
Testing if a number is a square can be done a bit faster than just by calculating the square root. It is used by the primality testing function described below.
\index{mp\_is\_square}
\begin{alltt}
int mp_is_square(const mp_int *arg, int *ret);
\end{alltt}


\index{mp\_prime\_is\_prime}
\begin{alltt}
int mp_prime_is_prime (mp_int * a, int t, int *result)
\end{alltt}
This will perform a trial division followed by two rounds of Miller-Rabin with bases 2 and 3 and a Lucas-Selfridge test. The Lucas-Selfridge test is replaced with a Frobenius-Underwood for \texttt{MP\_8BIT}. The Frobenius-Underwood test for all other sizes is available as a compile-time option with the preprocessor macro \texttt{LTM\_USE\_FROBENIUS\_TEST}. See file
\texttt{bn\_mp\_prime\_is\_prime.c} for the necessary details. It shall be noted that both functions are much slower than
the Miller-Rabin test and if speed is an essential issue, the macro \texttt{LTM\_USE\_FIPS\_ONLY} switches both functions, the Frobenius-Underwood test and the Lucas-Selfridge test off and their code will not even be compiled into the library.

If $t$ is set to a positive value $t$ additional rounds of the Miller-Rabin test with random bases will be performed to allow for Fips 186.4 (vid.~p.~126ff) compliance. The function \texttt{mp\_prime\_rabin\_miller\_trials} can be used to determine the number of rounds. It is vital that the function \texttt{mp\_rand()} has a cryptographically strong random number generator available.

One Miller-Rabin tests with a random base will be run automatically, so by setting $t$ to a positive value this function will run $t + 1$ Miller-Rabin tests with random bases.

If  $t$ is set to a negative value the test will run the deterministic Miller-Rabin test for the primes up to
$3317044064679887385961981$. That limit has to be checked by the caller. If $-t > 13$ than $-t - 13$ additional rounds of the
Miller-Rabin test will be performed but note that $-t$ is bounded by $1 \le -t < PRIME\_SIZE$ where $PRIME\_SIZE$ is the number
of primes in the prime number table (by default this is $256$) and the first 13 primes have already been used. It will return
\texttt{MP\_VAL} in case of$-t > PRIME\_SIZE$.

If $a$ passes all of the tests $result$ is set to one, otherwise it is set to zero.

\subsection{Deterministic Prime Test}
The test above is a probabilistic test and can fail by calling a composite a prime. It is extremely rare, so rare that it is of almost no practical relevance. For those users who get nervous about the word ``almost'' a deterministic primes test has been put in the packet ``extra''. Add it to the library by typing \texttt{make extra}.
\index{mp\_prime\_is\_prime\_deterministic}
\begin{alltt}
int mp_prime_is_prime_deterministic(const mp_int *z, int *result);
\end{alltt}
It has two disadvantages: the theory behind assumes the general Riemann hypothesis to be true and, much more significant, is very slow for large primes.

\begin{table}[h]
\begin{center}
\begin{tabular}{c c c}
 $\mathbf{2^n}$ & \textbf{Testing Time} &  \textbf{Generating Time}\\
    128 &   0m00.061s  &  0m00.176s \\
    256 &   0m00.490s  &  0m00.788s \\
    512 &   0m06.233s  &  0m05.257s \\
    768 &   0m34.974s  &  0m02.916s \\
   1024 &   2m12.393s  &  0m45.018s \\
   1536 &  11m39.058s  &  2m25.890s \\
   2048 &  45m13.408s  &  2m57.433s
\end{tabular}
\caption{Benchmarking \texttt{mp\_prime\_is\_prime\_deterministic}} \label{table:benchmarkprimetestdet}
\end{center}
\end{table}

The machine for that benchmark was an AMD A8-6600K and the prime generator was LibTomMaths own
 \texttt{mp\_prime\_random\_ex(\&z, 8, size, LTM\_PRIME\_SAFE, myrng, NULL)}.

Despite its large runtime it was the only way to include a deterministic primality test with a small memory footprint, no need for floating point functions and one that works with low \texttt{MP\_xBIT}, too.

\section{Next Prime}
\index{mp\_prime\_next\_prime}
\begin{alltt}
int mp_prime_next_prime(mp_int *a, int t, int bbs_style)
\end{alltt}
This finds the next prime after $a$ that passes mp\_prime\_is\_prime() with $t$ tests but see the documentation for
mp\_prime\_is\_prime for details regarding the use of the argument $t$.  Set $bbs\_style$ to one if you
want only the next prime congruent to $3 \mbox{ mod } 4$, otherwise set it to zero to find any next prime.

\section{Random Primes}
\index{mp\_prime\_random}
\begin{alltt}
int mp_prime_random(mp_int *a, int t, int size, int bbs,
                    ltm_prime_callback cb, void *dat)
\end{alltt}
This will find a prime greater than $256^{size}$ which can be ``bbs\_style'' or not depending on $bbs$ and must pass
$t$ rounds of tests but see the documentation for mp\_prime\_is\_prime for details regarding the use of the argument $t$.
The ``ltm\_prime\_callback'' is a typedef for

\begin{alltt}
typedef int ltm_prime_callback(unsigned char *dst, int len, void *dat);
\end{alltt}

Which is a function that must read $len$ bytes (and return the amount stored) into $dst$.  The $dat$ variable is simply
copied from the original input.  It can be used to pass RNG context data to the callback.  The function
mp\_prime\_random() is more suitable for generating primes which must be secret (as in the case of RSA) since there
is no skew on the least significant bits.

\textit{Note:}  As of v0.30 of the LibTomMath library this function has been deprecated.  It is still available
but users are encouraged to use the new mp\_prime\_random\_ex() function instead.

\subsection{Extended Generation}
\index{mp\_prime\_random\_ex}
\begin{alltt}
int mp_prime_random_ex(mp_int *a,    int t,
                       int     size, int flags,
                       ltm_prime_callback cb, void *dat);
\end{alltt}
This will generate a prime in $a$ using $t$ tests of the primality testing algorithms.  The variable $size$
specifies the bit length of the prime desired.  The variable $flags$ specifies one of several options available
(see fig. \ref{fig:primeopts}) which can be OR'ed together.  The callback parameters are used as in
mp\_prime\_random().

\begin{figure}[h]
\begin{center}
\begin{small}
\begin{tabular}{|r|l|}
\hline \textbf{Flag}         & \textbf{Meaning} \\
\hline LTM\_PRIME\_BBS       & Make the prime congruent to $3$ modulo $4$ \\
\hline LTM\_PRIME\_SAFE      & Make a prime $p$ such that $(p - 1)/2$ is also prime. \\
                             & This option implies LTM\_PRIME\_BBS as well. \\
\hline LTM\_PRIME\_2MSB\_OFF & Makes sure that the bit adjacent to the most significant bit \\
                             & Is forced to zero.  \\
\hline LTM\_PRIME\_2MSB\_ON  & Makes sure that the bit adjacent to the most significant bit \\
                             & Is forced to one. \\
\hline
\end{tabular}
\end{small}
\end{center}
\caption{Primality Generation Options}
\label{fig:primeopts}
\end{figure}

\chapter{Random Number Generation}
\section{PRNG}
\index{mp\_rand\_digit}
\begin{alltt}
int mp_rand_digit(mp_digit *r)
\end{alltt}
This function generates a random number in \texttt{r} of the size given in \texttt{r} (that is, the variable is used for in- and output) but not more than \texttt{MP\_MASK} bits.

\index{mp\_rand}
\begin{alltt}
int mp_rand(mp_int *a, int digits)
\end{alltt}
This function generates a random number of \texttt{digits} bits.

The random number generated with these two functions is cryptographically secure if the source of random numbers the operating systems offers is cryptographically secure. It will use \texttt{arc4random()} if the OS is a BSD flavor, Wincrypt on Windows, or \texttt{/dev/urandom} on all operating systems that have it.

\chapter{Small Prime Numbers}

Small prime numbers are those primes that fit in a word of LibTomMath, that is they are smaller than or equal to the size of the type behind \texttt{mp\_word}. Examples at the end of this chapter at page \ref{sec:spnexamples},
\section{Prime Sieve}
A prime sieve is implemented as a simple segmented Sieve of Eratosthenes. It is only moderately optimized for space and runtime but should be small enough and also fast enough for almost all use-cases; quite quick for sequential access but relatively slow for random access.

 The prime sieve and its functions are part of the ``extra'' package and can be compiled in with \texttt{make extra}. The macro \texttt{LTM\_USE\_EXTRA\_FUNCTIONS} has to be set before \texttt{tommath.h} is included. Printing the small primes fneeds \texttt{inttypes.h} which must be included before \texttt{tommath.h}.
\subsection{Initialization and Clearing}
Initializing. It cannot fail because it only sets some default values. Memory is allocated later according to needs.
\index{mp\_sieve\_init}
\begin{alltt}
void mp_sieve_init(mp_sieve *sieve);
\end{alltt}
The function \texttt{mp\_sieve\_init} is equivalent to
\begin{alltt}
mp_sieve sieve = {NULL, NULL, 0};
\end{alltt}

Free the memory used by the sieve with
\index{mp\_sieve\_clear}
\begin{alltt}
void mp_sieve_clear(mp_sieve *sieve);
\end{alltt}
\subsection{Primality Test of Small Numbers}
Individual small numbers can be tested for primality with
\index{mp\_is\_small\_prime}
\begin{alltt}
int mp_is_small_prime(LTM_SIEVE_UINT n, LTM_SIEVE_UINT *result,
                      mp_sieve *sieve);
\end{alltt}
The implementation of this function also does all of the heavy lifting, the building of the base sieve and the segment if one is necessary. The base sieve stays, so this function can be used to ``warm up'' the sieve and, if \texttt{n} is slightly larger than the upper limit of the base sieve, ``warm up'' the first segment, too. It will return \texttt{LTM\_SIEVE\_MAX\_REACHED} to flag the content of \texttt{result} as the last valid one.
\subsection{Find Adjacent Primes}
To find the prime bigger than a number \texttt{n} use
\index{mp\_next\_small\_prime}
\begin{alltt}
int mp_next_small_prime(LTM_SIEVE_UINT n, LTM_SIEVE_UINT *result,
                        mp_sieve *sieve);
\end{alltt}
and to find the one smaller than \texttt{n}
\begin{alltt}
int mp_prec_small_prime(LTM_SIEVE_UINT n, LTM_SIEVE_UINT *result,
                        mp_sieve *sieve);
\end{alltt}
\subsection{Prime Sequence}
The most common use of the small primes is in the form of a continuous sequence. To produce this sequence utilize
\index{mp\_small\_prime\_array}. The array \texttt{prime\_array} is allocated with \texttt{malloc} internally and needs to be free'd after use.
\begin{alltt}
int mp_small_prime_array(LTM_SIEVE_UINT start, LTM_SIEVE_UINT end,
                         mp_factors *factors,
                         mp_sieve *sieve);
\end{alltt}

\subsection{Useful Constants}
\begin{description}
\item[\texttt{LTM\_SIEVE\_BIGGEST\_PRIME}] \texttt{read-only} The biggest prime the sieve can offer. It is be $65\,521$ for \texttt{MP\_8BIT},
 $4\,294\,967\,291$ for \texttt{MP\_16BIT}, \texttt{MP\_32BIT} and \texttt{MP\_64BIT}; and
 $18\,446\,744\,073\,709\,551\,557$ for \texttt{MP\_64BIT} if the macro\\
 \texttt{LTM\_SIEVE\_USE\_LARGE\_SIEVE} is defined.

\item[\texttt{LTM\_SIEVE\_UINT}] \texttt{read-only}  The basic type for the numbers in the sieve. It is be \texttt{uint16\_t} for \texttt{MP\_8BIT}, \texttt{uint32\_t} for \texttt{MP\_16BIT}, \texttt{MP\_32BIT} and \texttt{MP\_64BIT}; and \texttt{uint64\_t} for \texttt{MP\_64BIT} if the macro \texttt{LTM\_SIEVE\_USE\_LARGE\_SIEVE} is defined.

\item[\texttt{LTM\_SIEVE\_UINT\_MAX}] \texttt{read-only} The maximum value of the type for the numbers in the sieve. It is \texttt{UINT16\_MAX} for \texttt{MP\_8BIT}, \texttt{UINT32\_MAX} for \texttt{MP\_16BIT}, \texttt{MP\_32BIT} and \texttt{MP\_64BIT}; and \texttt{UINT\_64MAX} for \texttt{MP\_64BIT} if the macro\\
\texttt{LTM\_SIEVE\_USE\_LARGE\_SIEVE} is defined.

\item[\texttt{LTM\_SIEVE\_UINT\_MAX\_SQRT}] \texttt{read-only} The square root of the maximum value of the type for the numbers in the sieve. It is \texttt{UINT8\_MAX} for \texttt{MP\_8BIT}, \texttt{UINT16\_MAX} for \texttt{MP\_16BIT}, \texttt{MP\_32BIT} and \texttt{MP\_64BIT}; and\texttt{UINT32\_MAX} for \texttt{MP\_64BIT} if the macro \texttt{LTM\_SIEVE\_USE\_LARGE\_SIEVE} is defined.

\item[\texttt{LTM\_SIEVE\_USE\_LARGE\_SIEVE}] \texttt{read-only} A flag to make a large sieve.  No advantage has been seen in using 64-bit integers if available except the ability to get a sieve up to $2^64$. But in this case the base sieve gets 0.25 Gibibytes large and the segments 0.5 Gibibytes (although you can change \texttt{LTM\_SIEVE\_RANGE\_A\_B} to get smaller segments) and need a long time to fill.

\item[\texttt{LTM\_SIEVE\_RANGE\_A\_B}] \texttt{read-write} The size of the sieve for the segment. It is set to \texttt{LTM\_SIEVE\_UINT\_MAX\_SQRT} per default but it can be changed. The default size is already small enough to fit into most CPU's L-2 caches but if \texttt{LTM\_SIEVE\_USE\_LARGE\_SIEVE} is defined the segment sieve grows quite large and setting \texttt{LTM\_SIEVE\_RANGE\_A\_B} to the size of the CPU's L-2 caches will show a significant advantage regarding the runtime, it more than doubles it. Because of that large penalty the default value is set to \texttt{0x400000uL} if both \texttt{MP\_64BIT} and \texttt{LTM\_SIEVE\_USE\_LARGE\_SIEVE} are defined. Needs to be set at the compile time of LibTomMath.

\item[\texttt{LTM\_SIEVE\_UINT\_NUM\_BITS}] \texttt{read-only} The number of bits in the type for the numbers in the sieve.\\
It is a shortcut for \verb!CHAR\_BIT * sizeof(LTM\_SIEVE\_UINT)!.

\item[\texttt{LTM\_SIEVE\_SIZE(bst)}] \texttt{function macro, read-only} Returns the entry \texttt{size} of \texttt{struct mp\_sieve}. It is a shortcut for \verb!sieve->size!.

\item[\texttt{LTM\_SIEVE\_PR\_UINT}] Choses the correct macro from \texttt{inttypes.h} to print a\\
 \texttt{LTM\_SIEVE\_UINT}. The header \texttt{inttypes.h} must be included before\\
 \texttt{tommath.h} for it to work.
\end{description}


\subsection{Examples}\label{sec:spnexamples}
\subsubsection{Initialization and Clearing}
Using a sieve follows the same procedure as using a bigint:
\begin{alltt}
/* Declaration */
mp_sieve sieve;

/*
   Initialization.
   Cannot fail, only sets a handful of default values.
   Memory allocation is done in the library itself
   according to needs.
 */
mp_sieve_init(&sieve);

/* use the sieve */

/* Clean up */
mp_sieve_clear(&sieve);
\end{alltt}
\subsubsection{Primality Test}
A small program to test the input of a small number for primality.
\begin{alltt}
#include <stdlib.h>
#include <stdio.h>
#include <errno.h>
/*inttypes.h must be included before tommath.h*/
#include <inttypes.h>
/* Must be defined before tommath.h is included */
#define LTM_USE_EXTRA_FUNCTIONS
#include "tommath.h"
int main(int argc, char **argv)
{
   LTM_SIEVE_UINT number;
   mp_sieve *base = NULL;
   mp_sieve *segment = NULL;
   LTM_SIEVE_UINT single_segment_a = 0;
   int e;

   /* variable holding the result of mp_is_small_prime */
   LTM_SIEVE_UINT result;

   if (argc != 2) {
      fprintf(stderr,"Usage %s number\textbackslash{}n", argv[0]);
      exit(EXIT_FAILURE);
   }

   number = (LTM_SIEVE_UINT) strtoul(argv[1],NULL, 10);
   if (errno == ERANGE) {
      fprintf(stderr,"strtoul(l) failed: input out of range\textbackslash{}n");
      goto LTM_ERR;
   }

   mp_sieve_init(&sieve);

   if ((e = mp_is_small_prime(number, &result, &sieve)) != MP_OKAY) {
      fprintf(stderr,"mp_is_small_number failed: \textbackslash{}"%s\textbackslash{}"\textbackslash{}n",
              mp_error_to_string(e));
      goto LTM_ERR;
   }

   printf("The number %" LTM_SIEVE_PR_UINT " is %s prime\textbackslash{}n",
           number,(result)?"":"not");


   mp_sieve_clear(&sieve);
   exit(EXIT_SUCCESS);
LTM_ERR:
   mp_sieve_clear(&sieve);
   exit(EXIT_FAILURE);
}
\end{alltt}
\subsubsection{Find Adjacent Primes}
To sum up all primes up to and including \texttt{LTM\_SIEVE\_BIGGEST\_PRIME} you might do something like:
\begin{alltt}
#include <stdlib.h>
#include <stdio.h>
#include <errno.h>
/* Must be defined before tommath.h is included */
#define LTM_USE_EXTRA_FUNCTIONS
#include <tommath.h>
int main(int argc, char **argv)
{
   LTM_SIEVE_UINT number;
   mp_sieve sieve;
   LTM_SIEVE_UINT k, ret;
   mp_int total, t;
   int e;

   if (argc != 2) {
      fprintf(stderr,"Usage %s integer\textbackslash{}n", argv[0]);
      exit(EXIT_FAILURE);
   }

   if ((e = mp_init_multi(&total, &t, NULL)) != MP_OKAY) {
      fprintf(stderr,"mp_init_multi(segment): \textbackslash{}"%s\textbackslash{}"\textbackslash{}n",
              mp_error_to_string(e));
      goto LTM_ERR_1;
   }
   errno = 0;
#if ( (defined MP_64BIT) && (defined LTM_SIEVE_USE_LARGE_SIEVE) )
   number = (LTM_SIEVE_UINT) strtoull(argv[1],NULL, 10);
#else
   number = (LTM_SIEVE_UINT) strtoul(argv[1],NULL, 10);
#endif
   if (errno == ERANGE) {
      fprintf(stderr,"strtoul(l) failed: input out of range\textbackslash{}n");
      return EXIT_FAILURE
   }

   mp_sieve_init(&sieve);

   for (k = 0, ret = 0; ret < number; k = ret) {
      if ((e = mp_next_small_prime(k + 1, &ret, &sieve)) != MP_OKAY) {
         if (e == LTM_SIEVE_MAX_REACHED) {
#ifdef MP_64BIT
            if ((e = mp_add_d(&total, (mp_digit) k, &total)) != MP_OKAY) {
               fprintf(stderr,"mp_add_d (1) failed: \textbackslash{}"%s\textbackslash{}"\textbackslash{}n",
                       mp_error_to_string(e));
               goto LTM_ERR;
            }
#else
            if ((e = mp_set_long(&t, k)) != MP_OKAY) {
               fprintf(stderr,"mp_set_long (1) failed: \textbackslash{}"%s\textbackslash{}"\textbackslash{}n",
                       mp_error_to_string(e));
               goto LTM_ERR;
            }
            if ((e = mp_add(&total, &t, &total)) != MP_OKAY) {
               fprintf(stderr,"mp_add (1) failed: \textbackslash{}"%s\textbackslash{}"\textbackslash{}n",
                       mp_error_to_string(e));
               goto LTM_ERR;
            }
#endif
            break;
         }
         fprintf(stderr,"mp_next_small_prime failed: \textbackslash{}"%s\textbackslash{}"\textbackslash{}n",
                 mp_error_to_string(e));
         goto LTM_ERR;
      }
      /* The check if the prime is below the given limit
       * cannot be done in the for-loop conditions because if
       * it could we wouldn't need the sieve in the first place.
       */
      if (ret <= number) {
#ifdef MP_64BIT
         if ((e = mp_add_d(&total, (mp_digit) k, &total)) != MP_OKAY) {
            fprintf(stderr,"mp_add_d failed: \textbackslash{}"%s\textbackslash{}"\textbackslash{}n",
                    mp_error_to_string(e));
            goto LTM_ERR;
         }
#else
         if ((e = mp_set_long(&t, k)) != MP_OKAY) {
            fprintf(stderr,"mp_set_long failed: \textbackslash{}"%s\textbackslash{}"\textbackslash{}n",
                    mp_error_to_string(e));
            goto LTM_ERR;
         }
         if ((e = mp_add(&total, &t, &total)) != MP_OKAY) {
            fprintf(stderr,"mp_add failed: \textbackslash{}"%s\textbackslash{}"\textbackslash{}n",
            mp_error_to_string(e));
            goto LTM_ERR;
         }
#endif
      }
   }
   printf("total: ");
   mp_fwrite(&total,10,stdout);
   putchar('\textbackslash{}n');

   mp_clear_multi(&total, &t, NULL);
   mp_sieve_clear(&sieve);
   exit(EXIT_SUCCESS);
LTM_ERR:
   mp_clear_multi(&total, &t, NULL);
   mp_sieve_clear(&sieve);
   exit(EXIT_FAILURE);
}
\end{alltt}
It took about a minute on the authors machine from 2015 to get the expected $425\,649\,736\,193\,687\,430$ for the sum of all primes up to $2^{32}$, about the same runtime as Pari/GP version 2.9.4 (with a GMP-5.1.3 kernel).

\subsubsection{Prime Sequence}
A short sequence of primes can be produced with:
\begin{alltt}
#include <stdlib.h>
#include <stdio.h>
#include <errno.h>
/* Must be defined before tommath.h is included */
#define LTM_USE_EXTRA_FUNCTIONS
#include <tommath.h>
int main(int argc, char **argv)
{
   LTM_SIEVE_UINT a, b;
   mp_factors factors;
   int e;

   if (argc != 3) {
      fprintf(stderr,"Usage %s start stop\textbackslash{}n", argv[0]);
      exit(EXIT_FAILURE);
   }

   errno = 0;
#if ( (defined MP_64BIT) && (defined LTM_SIEVE_USE_LARGE_SIEVE) )
   a = (LTM_SIEVE_UINT) strtoull(argv[1], NULL, 10);
   if (errno == ERANGE) {
      fprintf(stderr,"strtoull(start) failed: input out of range\textbackslash{}n");
      goto LTM_ERR;
   }
   errno = 0;
   b = (LTM_SIEVE_UINT) strtoull(argv[2], NULL, 10);
   if (errno == ERANGE) {
      fprintf(stderr,"strtoull(end) failed: input out of range\textbackslash{}n");
      goto LTM_ERR;
   }
#else
   a = (LTM_SIEVE_UINT) strtoul(argv[1], NULL, 10);
   if (errno == ERANGE) {
      fprintf(stderr,"strtoul(start) failed: input out of range\textbackslash{}n");
      goto LTM_ERR;
   }
   errno = 0;
   b = (LTM_SIEVE_UINT) strtoul(argv[2], NULL, 10);
   if (errno == ERANGE) {
      fprintf(stderr,"strtoul(end) failed: input out of range\textbackslash{}n");
      goto LTM_ERR;
   }
#endif

   if ((e = mp_small_prime_array(a, b, &factors)) != MP_OKAY) {
      fprintf(stderr,"mp_small_prime_array failed: \textbackslash{}"%s\textbackslash{}"\textbackslash{}n",
              mp_error_to_string(e));
      goto LTM_ERR;
   }
   
   mp_factors_print(&factors, 10, 0, stdout);

   mp_factors_clear(&factors);
   exit(EXIT_SUCCESS);
LTM_ERR:
   mp_factors_clear(&factors);
   exit(EXIT_FAILURE);
}
\end{alltt}
The array \texttt{prime\_array} will be of size $\pi(b) - \pi(a)$ times \verb!sizeof(LTM_SIEVE_UINT)! which can get quite large quite quickly\footnote{There are $203\,280\,221$ primes smaller than $2^{32}$.}. You might find the method involving the function \texttt{mp\_next\_small\_prime} more applicable for larger sequences.

%\subsubsection{Using the Useful Constants}

\section{Factorizing}
All of the functions described in this section are in the packet ``extra''. Add it to the library by typing \texttt{make extra}.

The decomposition of numbers into their prime-factors is covered by the function
\index{mp\_factor}
\begin{alltt}
int mp_factor(const mp_int *z, mp_factors *factors);
\end{alltt}
It will decompose the factors of the integer $z > 0$ into its prime factors\footnote{The methods used in this algorithm are reasonably fast but definitely not the fastest. A practical limit is at about 30-35 bit large factors, 40 bit with some patience.}, checks the result by multiplying the found factors and comparing them with the input, and sample them as numbers of the type \texttt{mp\_int} in the list \texttt{factors}. The structure of this list is described by
\index{mp\_factors}
\begin{alltt}
typedef struct {
   int length, alloc;
   mp_int *factors;
} mp_factors;
\end{alltt}
There are a handful of functions to help with the management of that list.

\begin{description}
\item
\index{mp\_factors\_init}
\verb!int mp_factors_init(mp_factors *f);!\\
Initialize the factor list by allocating a certain amount of memory and setting \texttt{length = 0} and \texttt{alloc} to the amount of memory pre-allocated. The exact amount is defined at compile time by the macro \texttt{LTM\_TRIAL\_GROWTH} in \texttt{tommath.h}.
\item
\index{mp\_factors\_clear}
\verb!void mp_factors_clear(mp_factors *f);!\\
This function free's all memory used.
\item
\index{mp\_factors\_zero}
\verb!int mp_factors_zero(mp_factors *f);!\\
Remove the elements of the factor list and allocate (fresh) memory of default size in that order.
\item
\index{mp\_factors\_add}
\verb!int mp_factors_add(const mp_int *a, mp_factors *f);!\\
Add a factor of type \texttt{mp\_int} to the list.
\item
\index{mp\_factors\_sort}
\verb!int mp_factors_sort(mp_factors *f);!\\
The factors in the list are not necessarily in increasing order. This functions changes that. It does it with the insert-sort algorithm, a good choice for the task it has been written for (the list is most likely already ordered) but not for many other tasks involving large lists in random order.
\item
\index{mp\_factors\_print}
\texttt{int mp\_factors\_print(mp\_factors *f, int base, char delimiter,\\
\hphantom{int mp\_factors\_print(} FILE *stream);}\\
Prints the element of the list in base \texttt{base} to \texttt{stream} with the delimiter \texttt{delimiter}. The default delimiter is a comma (ASCII \texttt{0x2c}).
\item
\index{mp\_factors\_product}
\verb!int mp_factors_product(mp_factors *factors, mp_int *p);!\\
Multiplies all elements of the list. It does not recognize sparse lists, every zero in the list gets multiplied, too. It does multiply the list with a binary-splitting algorithm which assumes a highly or better fully sorted list to work optimally.
\end{description}

The function \texttt{mp\_factor} uses two different factorization algorithms. The first one does just trial division with the small primes generated by a sieve and is
\index{mp\_trial}
\begin{alltt}
int mp_trial(const mp_int *a, int limit, 
             mp_factors *factors, mp_int *r);
\end{alltt}
It tries all small primes up to the limit \texttt{limit}, puts all factors it finds in the list \texttt{factors} and the remainder in \texttt{r},

The other function is the Pollard-Rho algorithm.
\index{mp\_pollard\_rho}
\begin{alltt}
int mp_pollard_rho(const mp_int *n, mp_int *factor);
\end{alltt}
It is used to compute all factors left over by the function \texttt{mp\_factor}. Both functions \texttt{mp\_trial} and \texttt{mp\_pollard\_rho} are not meant to be used as a standalone function, please consult the source of the respective functions for the necessary information.


The function \texttt{mp\_factors\_product} does already most of the work so it was not much left to do to implement a function to compute a primorial.
\index{mp\_primorial}
\begin{alltt}
int mp_primorial(const LTM_SIEVE_UINT n, mp_int *p);
\end{alltt}

\subsection{Test for (perfect) Powers}
\index{mp\_ispower}
\begin{alltt}
int mp_ispower(const mp_int *z, int *result, mp_int *rootout,
                 mp_int *exponent);
\end{alltt}
Tests if $z$ is a power, that is $z = a^b$ with $b$ prime but $a$ might be composite. Computes the actual roots to do so and in case of success sets \texttt{result} to \texttt{MP\_YES}, puts $a$ in \texttt{rootout} and the exponent $b$ in \texttt{exponent} or $0$ (zero) in both and \texttt{result} to \texttt{MP\_NO} if case of a failure to find one.

\index{mp\_isperfpower}
\begin{alltt}
int mp_isperfpower(const mp_int *z, int *result, mp_int *rootout,
                   mp_int *exponent)
\end{alltt}
Same as \texttt{mp\_ispower} but searches for perfect or prime powers, that is $z = a^b$ such that $a, b$ are prime. Uses \texttt{mp\_prime\_is\_prime} which is a probabilistic prime tester and only roots below $2^{64}$ are save.
\chapter{Input and Output}
\section{ASCII Conversions}
\subsection{To ASCII}
\index{mp\_toradix}
\begin{alltt}
int mp_toradix (mp_int * a, char *str, int radix);
\end{alltt}
This still store $a$ in ``str'' as a base-``radix'' string of ASCII chars.  This function appends a NUL character
to terminate the string.  Valid values of ``radix'' line in the range $[2, 64]$.  To determine the size (exact) required
by the conversion before storing any data use the following function.

\index{mp\_toradix\_n}
\begin{alltt}
int mp_toradix_n (mp_int * a, char *str, int radix, int maxlen);
\end{alltt}

Like \texttt{mp\_toradix} but stores up to maxlen-1 chars and always a NULL byte.

\index{mp\_radix\_size}
\begin{alltt}
int mp_radix_size (mp_int * a, int radix, int *size)
\end{alltt}
This stores in ``size'' the number of characters (including space for the NUL terminator) required.  Upon error this
function returns an error code and ``size'' will be zero.

If \texttt{LTM\_NO\_FILE} is not defined a function to write to a file is also available.
\index{mp\_fwrite}
\begin{alltt}
int mp_fwrite(const mp_int *a, int radix, FILE *stream);
\end{alltt}


\subsection{From ASCII}
\index{mp\_read\_radix}
\begin{alltt}
int mp_read_radix (mp_int * a, char *str, int radix);
\end{alltt}
This will read the base-``radix'' NUL terminated string from ``str'' into $a$.  It will stop reading when it reads a
character it does not recognize (which happens to include th NUL char... imagine that...).  A single leading $-$ sign
can be used to denote a negative number.

If \texttt{LTM\_NO\_FILE} is not defined a function to read from a file is also available.
\index{mp\_fread}
\begin{alltt}
int mp_fread(mp_int *a, int radix, FILE *stream);
\end{alltt}


\section{Binary Conversions}

Converting an mp\_int to and from binary is another keen idea.

\index{mp\_unsigned\_bin\_size}
\begin{alltt}
int mp_unsigned_bin_size(mp_int *a);
\end{alltt}

This will return the number of bytes (octets) required to store the unsigned copy of the integer $a$.

\index{mp\_to\_unsigned\_bin}
\begin{alltt}
int mp_to_unsigned_bin(mp_int *a, unsigned char *b);
\end{alltt}
This will store $a$ into the buffer $b$ in big--endian format.  Fortunately this is exactly what DER (or is it ASN?)
requires.  It does not store the sign of the integer.

\index{mp\_to\_unsigned\_bin\_n}
\begin{alltt}
int mp_to_unsigned_bin_n(const mp_int *a, unsigned char *b, unsigned long *outlen)
\end{alltt}
Like \texttt{mp\_to\_unsigned\_bin} but checks if the value at \texttt{*outlen} is larger than or equal to the output of \texttt{mp\_unsigned\_bin\_size(a)} and sets \texttt{*outlen} to the output of \texttt{mp\_unsigned\_bin\_size(a)} or returns \texttt{MP\_VAL} if the test failed.


\index{mp\_read\_unsigned\_bin}
\begin{alltt}
int mp_read_unsigned_bin(mp_int *a, unsigned char *b, int c);
\end{alltt}
This will read in an unsigned big--endian array of bytes (octets) from $b$ of length $c$ into $a$.  The resulting
integer $a$ will always be positive.

For those who acknowledge the existence of negative numbers (heretic!) there are ``signed'' versions of the
previous functions.
\index{mp\_signed\_bin\_size} \index{mp\_to\_signed\_bin} \index{mp\_read\_signed\_bin}
\begin{alltt}
int mp_signed_bin_size(mp_int *a);
int mp_read_signed_bin(mp_int *a, unsigned char *b, int c);
int mp_to_signed_bin(mp_int *a, unsigned char *b);
\end{alltt}
They operate essentially the same as the unsigned copies except they prefix the data with zero or non--zero
byte depending on the sign.  If the sign is zpos (e.g. not negative) the prefix is zero, otherwise the prefix
is non--zero.

The two functions \texttt{mp\_import} and \texttt{mp\_export} implement the corresponding GMP functions as described at \url{http://gmplib.org/manual/Integer-Import-and-Export.html}.
\index{mp\_import} \index{mp\_export}
\begin{alltt}
int mp_import(mp_int *rop, size_t count, int order, size_t size, int endian, size_t nails, const void *op);
int mp_export(void *rop, size_t *countp, int order, size_t size, int endian, size_t nails, const mp_int *op);
\end{alltt}

\chapter{Algebraic Functions}
\section{Extended Euclidean Algorithm}
\index{mp\_exteuclid}
\begin{alltt}
int mp_exteuclid(mp_int *a, mp_int *b,
                 mp_int *U1, mp_int *U2, mp_int *U3);
\end{alltt}

This finds the triple U1/U2/U3 using the Extended Euclidean algorithm such that the following equation holds.

\begin{equation}
a \cdot U1 + b \cdot U2 = U3
\end{equation}

Any of the U1/U2/U3 parameters can be set to \textbf{NULL} if they are not desired.

\section{Greatest Common Divisor}
\index{mp\_gcd}
\begin{alltt}
int mp_gcd (mp_int * a, mp_int * b, mp_int * c)
\end{alltt}
This will compute the greatest common divisor of $a$ and $b$ and store it in $c$.

\section{Least Common Multiple}
\index{mp\_lcm}
\begin{alltt}
int mp_lcm (mp_int * a, mp_int * b, mp_int * c)
\end{alltt}
This will compute the least common multiple of $a$ and $b$ and store it in $c$.

\section{Jacobi Symbol}
\index{mp\_jacobi}
\begin{alltt}
int mp_jacobi (mp_int * a, mp_int * p, int *c)
\end{alltt}
This will compute the Jacobi symbol for $a$ with respect to $p$.  If $p$ is prime this essentially computes the Legendre
symbol.  The result is stored in $c$ and can take on one of three values $\lbrace -1, 0, 1 \rbrace$.  If $p$ is prime
then the result will be $-1$ when $a$ is not a quadratic residue modulo $p$.  The result will be $0$ if $a$ divides $p$
and the result will be $1$ if $a$ is a quadratic residue modulo $p$.

\section{Kronecker Symbol}
\index{mp\_kronecker}
\begin{alltt}
int mp_kronecker (mp_int * a, mp_int * p, int *c)
\end{alltt}
Extension of the Jacoby symbol to all $\lbrace a, p \rbrace \in \mathbb{Z}$ .


\section{Modular square root}
\index{mp\_sqrtmod\_prime}
\begin{alltt}
int mp_sqrtmod_prime(mp_int *n, mp_int *p, mp_int *r)
\end{alltt}

This will solve the modular equation $r^2 = n \mod p$ where $p$ is a prime number greater than 2 (odd prime).
The result is returned in the third argument $r$, the function returns \textbf{MP\_OKAY} on success,
other return values indicate failure.

The implementation is split for two different cases:

1. if $p \mod 4 == 3$ we apply \href{http://cacr.uwaterloo.ca/hac/}{Handbook of Applied Cryptography algorithm 3.36} and compute $r$ directly as
$r = n^{(p+1)/4} \mod p$

2. otherwise we use \href{https://en.wikipedia.org/wiki/Tonelli-Shanks_algorithm}{Tonelli-Shanks algorithm}

The function does not check the primality of parameter $p$ thus it is up to the caller to assure that this parameter
is a prime number. When $p$ is a composite the function behaviour is undefined, it may even return a false-positive
\textbf{MP\_OKAY}.

\section{Modular Inverse}
\index{mp\_invmod}
\begin{alltt}
int mp_invmod (mp_int * a, mp_int * b, mp_int * c)
\end{alltt}
Computes the multiplicative inverse of $a$ modulo $b$ and stores the result in $c$ such that $ac \equiv 1 \mbox{ (mod }b\mbox{)}$.

\section{Single Digit Functions}

For those using small numbers (\textit{snicker snicker}) there are several ``helper'' functions

\index{mp\_add\_d} \index{mp\_sub\_d} \index{mp\_mul\_d} \index{mp\_div\_d} \index{mp\_mod\_d}
\begin{alltt}
int mp_add_d(mp_int *a, mp_digit b, mp_int *c);
int mp_sub_d(mp_int *a, mp_digit b, mp_int *c);
int mp_mul_d(mp_int *a, mp_digit b, mp_int *c);
int mp_div_d(mp_int *a, mp_digit b, mp_int *c, mp_digit *d);
int mp_mod_d(mp_int *a, mp_digit b, mp_digit *c);
\end{alltt}

These work like the full mp\_int capable variants except the second parameter $b$ is a mp\_digit.  These
functions fairly handy if you have to work with relatively small numbers since you will not have to allocate
an entire mp\_int to store a number like $1$ or $2$.


The division by three can be made faster by replacing the division with a multiplication by the multiplicative inverse of three.

\index{mp\_div\_3}
\begin{alltt}
int mp_div_3(const mp_int *a, mp_int *c, mp_digit *d);
\end{alltt}

\chapter{Little Helpers}
It is never wrong to have some useful little shortcuts at hand.
\section{Function Macros}
To make this overview simpler the macros are given as function prototypes. The return of logic macros is \texttt{MP\_NO} or \texttt{MP\_YES} respectively.

\index{mp\_iseven}
\begin{alltt}
int mp_iseven(mp_int *a)
\end{alltt}
Checks if $a = 0 mod 2$

\index{mp\_isodd}
\begin{alltt}
int mp_isodd(mp_int *a)
\end{alltt}
Checks if $a = 1 mod 2$

\index{mp\_isneg}
\begin{alltt}
int mp_isneg(mp_int *a)
\end{alltt}
Checks if $a < 0$


\index{mp\_iszero}
\begin{alltt}
int mp_iszero(mp_int *a)
\end{alltt}
Checks if $a = 0$. It does not check if the amount of memory allocated for $a$ is also minimal.


Other macros which are either shortcuts to normal functions or just other names for them do have their place in a programmer's life, too!

\subsection{Renamings}
\index{mp\_mag\_size}
\begin{alltt}
#define mp_mag_size(mp) mp_unsigned_bin_size(mp)
\end{alltt}


\index{mp\_raw\_size}
\begin{alltt}
#define mp_raw_size(mp) mp_signed_bin_size(mp)
\end{alltt}


\index{mp\_read\_mag}
\begin{alltt}
#define mp_read_mag(mp, str, len) mp_read_unsigned_bin((mp), (str), (len))
\end{alltt}


\index{mp\_read\_raw}
\begin{alltt}
 #define mp_read_raw(mp, str, len) mp_read_signed_bin((mp), (str), (len))
\end{alltt}


\index{mp\_tomag}
\begin{alltt}
#define mp_tomag(mp, str) mp_to_unsigned_bin((mp), (str))
\end{alltt}


\index{mp\_toraw}
\begin{alltt}
#define mp_toraw(mp, str)         mp_to_signed_bin((mp), (str))
\end{alltt}



\subsection{Shortcuts}

\index{mp\_tobinary}
\begin{alltt}
#define mp_tobinary(M, S) mp_toradix((M), (S), 2)
\end{alltt}


\index{mp\_tooctal}
\begin{alltt}
#define mp_tooctal(M, S) mp_toradix((M), (S), 8)
\end{alltt}


\index{mp\_todecimal}
\begin{alltt}
#define mp_todecimal(M, S) mp_toradix((M), (S), 10)
\end{alltt}


\index{mp\_tohex}
\begin{alltt}
#define mp_tohex(M, S)     mp_toradix((M), (S), 16)
\end{alltt}


\documentclass[synpaper]{book}
\usepackage{hyperref}
\usepackage{makeidx}
\usepackage{amssymb}
\usepackage{color}
\usepackage{alltt}
\usepackage{graphicx}
\usepackage{layout}
\def\union{\cup}
\def\intersect{\cap}
\def\getsrandom{\stackrel{\rm R}{\gets}}
\def\cross{\times}
\def\cat{\hspace{0.5em} \| \hspace{0.5em}}
\def\catn{$\|$}
\def\divides{\hspace{0.3em} | \hspace{0.3em}}
\def\nequiv{\not\equiv}
\def\approx{\raisebox{0.2ex}{\mbox{\small $\sim$}}}
\def\lcm{{\rm lcm}}
\def\gcd{{\rm gcd}}
\def\log{{\rm log}}
\def\ilog{{\rm ilog}}
\def\ord{{\rm ord}}
\def\abs{{\mathit abs}}
\def\rep{{\mathit rep}}
\def\mod{{\mathit\ mod\ }}
\renewcommand{\pmod}[1]{\ ({\rm mod\ }{#1})}
\newcommand{\floor}[1]{\left\lfloor{#1}\right\rfloor}
\newcommand{\ceil}[1]{\left\lceil{#1}\right\rceil}
\def\Or{{\rm\ or\ }}
\def\And{{\rm\ and\ }}
\def\iff{\hspace{1em}\Longleftrightarrow\hspace{1em}}
\def\implies{\Rightarrow}
\def\undefined{{\rm ``undefined"}}
\def\Proof{\vspace{1ex}\noindent {\bf Proof:}\hspace{1em}}
\let\oldphi\phi
\def\phi{\varphi}
\def\Pr{{\rm Pr}}
\newcommand{\str}[1]{{\mathbf{#1}}}
\def\F{{\mathbb F}}
\def\N{{\mathbb N}}
\def\Z{{\mathbb Z}}
\def\R{{\mathbb R}}
\def\C{{\mathbb C}}
\def\Q{{\mathbb Q}}
\definecolor{DGray}{gray}{0.5}
\newcommand{\emailaddr}[1]{\mbox{$<${#1}$>$}}
\def\twiddle{\raisebox{0.3ex}{\mbox{\tiny $\sim$}}}
\def\gap{\vspace{0.5ex}}
\makeindex
\begin{document}
\frontmatter
\pagestyle{empty}
\title{LibTomMath User Manual \\ v1.1.0}
\author{LibTom Projects \\ www.libtom.net}
\maketitle
This text, the library and the accompanying textbook are all hereby placed in the public domain.  This book has been
formatted for B5 [176x250] paper using the \LaTeX{} {\em book} macro package.

\vspace{10cm}

\begin{flushright}Open Source.  Open Academia.  Open Minds.

\mbox{ }
LibTom Projects

\& originally

Tom St Denis,

Ontario, Canada
\end{flushright}

\tableofcontents
\listoffigures
\mainmatter
\pagestyle{headings}
\chapter{Introduction}
\section{What is LibTomMath?}
LibTomMath is a library of source code which provides a series of efficient and carefully written functions for manipulating
large integer numbers.  It was written in portable ISO C source code so that it will build on any platform with a conforming
C compiler.

In a nutshell the library was written from scratch with verbose comments to help instruct computer science students how
to implement ``bignum'' math.  However, the resulting code has proven to be very useful.  It has been used by numerous
universities, commercial and open source software developers.  It has been used on a variety of platforms ranging from
Linux and Windows based x86 to ARM based Gameboys and PPC based MacOS machines.

\section{License}
As of the v0.25 the library source code has been placed in the public domain with every new release.  As of the v0.28
release the textbook ``Implementing Multiple Precision Arithmetic'' has been placed in the public domain with every new
release as well.  This textbook is meant to compliment the project by providing a more solid walkthrough of the development
algorithms used in the library.

Since both\footnote{Note that the MPI files under mtest/ are copyrighted by Michael Fromberger.  They are not required to use LibTomMath.} are in the
public domain everyone is entitled to do with them as they see fit.

\section{Building LibTomMath}

LibTomMath is meant to be very ``GCC friendly'' as it comes with a makefile well suited for GCC.  However, the library will
also build in MSVC, Borland C out of the box.  For any other ISO C compiler a makefile will have to be made by the end
developer. Please consider to commit such a makefile to the LibTomMath developers, currently residing at
\url{http://github.com/libtom/libtommath}, if successfully done so.

Intel's C-compiler (ICC) is sufficiently compatible with GCC, at least the newer versions, to replace GCC for building the static and the shared library. Editing the makefiles is not needed, just set the shell variable \texttt{CC} as shown below.
\begin{alltt}
CC=/home/czurnieden/intel/bin/icc make
\end{alltt}

ICC does not know all options available for GCC and LibTomMath uses two diagnostics \texttt{-Wbad-function-cast} and \texttt{-Wcast-align} that are not supported by ICC resulting in the warnings:
\begin{alltt}
icc: command line warning #10148: option '-Wbad-function-cast' not supported
icc: command line warning #10148: option '-Wcast-align' not supported
\end{alltt}
It is possible to mute this ICC warning with the compiler flag \texttt{-diag-disable=10006}\footnote{It is not recommended to suppress warnings without a very good reason but there is no harm in doing so in this very special case.}.

\subsection{Static Libraries}
To build as a static library for GCC issue the following
\begin{alltt}
make
\end{alltt}

command.  This will build the library and archive the object files in ``libtommath.a''.  Now you link against
that and include ``tommath.h'' within your programs.  Alternatively to build with MSVC issue the following
\begin{alltt}
nmake -f makefile.msvc
\end{alltt}

This will build the library and archive the object files in ``tommath.lib''.  This has been tested with MSVC
version 6.00 with service pack 5.

\subsection{Shared Libraries}
\subsubsection{GNU based Operating Systems}
To build as a shared library for GCC issue the following
\begin{alltt}
make -f makefile.shared
\end{alltt}
This requires the ``libtool'' package (common on most Linux/BSD systems).  It will build LibTomMath as both shared
and static then install (by default) into /usr/lib as well as install the header files in /usr/include.  The shared
library (resource) will be called ``libtommath.la'' while the static library called ``libtommath.a''.  Generally
you use libtool to link your application against the shared object.
\subsubsection{Microsoft Windows based Operating Systems}
There is limited support for making a ``DLL'' in windows via the ``makefile.cygwin\_dll'' makefile.  It requires
Cygwin to work with since it requires the auto-export/import functionality.  The resulting DLL and import library
``libtommath.dll.a'' can be used to link LibTomMath dynamically to any Windows program using Cygwin.
\subsubsection{OpenBSD}
OpenBSD replaced some of their GNU-tools, especially \texttt{libtool} with their own, slightly different versions. To ease the workload of LibTomMath's developer team, only a static library can be build with the included \texttt{makefile.unix}.

The wrong \texttt{make} will result in errors like:
\begin{alltt}
*** Parse error in /home/user/GITHUB/libtommath: Need an operator in 'LIBNAME' )
*** Parse error: Need an operator in 'endif' (makefile.shared:8)
*** Parse error: Need an operator in 'CROSS_COMPILE' (makefile_include.mk:16)
*** Parse error: Need an operator in 'endif' (makefile_include.mk:18)
*** Parse error: Missing dependency operator (makefile_include.mk:22)
*** Parse error: Missing dependency operator (makefile_include.mk:23)
...
\end{alltt}
The wrong \texttt{libtool} will build it all fine but when it comes to the final linking fails with
\begin{alltt}
...
cc -I./ -Wall -Wsign-compare -Wextra -Wshadow -Wsystem-headers -Wdeclaration-afo...
cc -I./ -Wall -Wsign-compare -Wextra -Wshadow -Wsystem-headers -Wdeclaration-afo...
cc -I./ -Wall -Wsign-compare -Wextra -Wshadow -Wsystem-headers -Wdeclaration-afo...
libtool --mode=link --tag=CC cc  bn_error.lo bn_fast_mp_invmod.lo bn_fast_mp_mo 
libtool: link: cc bn_error.lo bn_fast_mp_invmod.lo bn_fast_mp_montgomery_reduce0
bn_error.lo: file not recognized: File format not recognized
cc: error: linker command failed with exit code 1 (use -v to see invocation)
Error while executing cc bn_error.lo bn_fast_mp_invmod.lo bn_fast_mp_montgomery0
gmake: *** [makefile.shared:64: libtommath.la] Error 1
\end{alltt}

To build a shared library with OpenBSD\footnote{Tested with OpenBSD version 6.4} the GNU versions of \texttt{make} and \texttt{libtool} are needed.
\begin{alltt}
$ sudo pkg_add gmake libtool
\end{alltt}
At this time two versions of \texttt{libtool} are installed and both are named \texttt{libtool}, unfortunately but GNU \texttt{libtool} has been placed in \texttt{/usr/local/bin/} and the native version in \texttt{/usr/bin/}. The path might be different in other versions of OpenBSD but both programs differ in the output of \texttt{libtool --version}
\begin{alltt}
$ /usr/local/bin/libtool --version                              
libtool (GNU libtool) 2.4.2
Written by Gordon Matzigkeit <gord@gnu.ai.mit.edu>, 1996

Copyright (C) 2011 Free Software Foundation, Inc.
This is free software; see the source for copying conditions.  There is NO
warranty; not even for MERCHANTABILITY or FITNESS FOR A PARTICULAR PURPOSE.
$ libtool --version
libtool (not (GNU libtool)) 1.5.26
\end{alltt}

The shared library should build now with
\begin{alltt}
LIBTOOL="/usr/local/bin/libtool" gmake -f makefile.shared
\end{alltt}
You might need to run a \texttt{gmake -f makefile.shared clean} first.

\subsubsection{NetBSD}
NetBSD is not as strict as OpenBSD but still needs \texttt{gmake} to build the shared library. \texttt{libtool} may also not exist in a fresh install.
\begin{alltt}
pkg_add gmake libtool
\end{alltt}
Please check with \texttt{libtool --version} that installed libtool is indeed a GNU libtool.
Build the shared library by typing:
\begin{alltt}
gmake -f makefile.shared
\end{alltt}

\subsection{Testing}
To build the library and the test harness type

\begin{alltt}
make test
\end{alltt}

This will build the library, ``test'' and ``mtest/mtest''.  The ``test'' program will accept test vectors and verify the
results.  ``mtest/mtest'' will generate test vectors using the MPI library by Michael Fromberger\footnote{A copy of MPI
is included in the package}.  Simply pipe mtest into test using

\begin{alltt}
mtest/mtest | test
\end{alltt}

If you do not have a ``/dev/urandom'' style RNG source you will have to write your own PRNG and simply pipe that into
mtest.  For example, if your PRNG program is called ``myprng'' simply invoke

\begin{alltt}
myprng | mtest/mtest | test
\end{alltt}

This will output a row of numbers that are increasing.  Each column is a different test (such as addition, multiplication, etc)
that is being performed.  The numbers represent how many times the test was invoked.  If an error is detected the program
will exit with a dump of the relevant numbers it was working with.

\section{Build Configuration}
LibTomMath can configured at build time in three phases we shall call ``depends'', ``tweaks'' and ``trims''.
Each phase changes how the library is built and they are applied one after another respectively.

To make the system more powerful you can tweak the build process.  Classes are defined in the file
``tommath\_superclass.h''.  By default, the symbol ``LTM\_ALL'' shall be defined which simply
instructs the system to build all of the functions.  This is how LibTomMath used to be packaged.  This will give you
access to every function LibTomMath offers.

However, there are cases where such a build is not optional.  For instance, you want to perform RSA operations.  You
don't need the vast majority of the library to perform these operations.  Aside from LTM\_ALL there is
another pre--defined class ``SC\_RSA\_1'' which works in conjunction with the RSA from LibTomCrypt.  Additional
classes can be defined base on the need of the user.

\subsection{Build Depends}
In the file tommath\_class.h you will see a large list of C ``defines'' followed by a series of ``ifdefs''
which further define symbols.  All of the symbols (technically they're macros $\ldots$) represent a given C source
file.  For instance, BN\_MP\_ADD\_C represents the file ``bn\_mp\_add.c''.  When a define has been enabled the
function in the respective file will be compiled and linked into the library.  Accordingly when the define
is absent the file will not be compiled and not contribute any size to the library.

You will also note that the header tommath\_class.h is actually recursively included (it includes itself twice).
This is to help resolve as many dependencies as possible.  In the last pass the symbol LTM\_LAST will be defined.
This is useful for ``trims''.

\subsection{Build Tweaks}
A tweak is an algorithm ``alternative''.  For example, to provide trade-offs (usually between size and space).
They can be enabled at any pass of the configuration phase.

\begin{small}
\begin{center}
\begin{tabular}{|l|l|}
\hline \textbf{Define} & \textbf{Purpose} \\
\hline BN\_MP\_DIV\_SMALL & Enables a slower, smaller and equally \\
                          & functional mp\_div() function \\
\hline
\end{tabular}
\end{center}
\end{small}

\subsection{Build Trims}
A trim is a manner of removing functionality from a function that is not required.  For instance, to perform
RSA cryptography you only require exponentiation with odd moduli so even moduli support can be safely removed.
Build trims are meant to be defined on the last pass of the configuration which means they are to be defined
only if LTM\_LAST has been defined.

\subsubsection{Moduli Related}
\begin{small}
\begin{center}
\begin{tabular}{|l|l|}
\hline \textbf{Restriction} & \textbf{Undefine} \\
\hline Exponentiation with odd moduli only & BN\_S\_MP\_EXPTMOD\_C \\
                                           & BN\_MP\_REDUCE\_C \\
                                           & BN\_MP\_REDUCE\_SETUP\_C \\
                                           & BN\_S\_MP\_MUL\_HIGH\_DIGS\_C \\
                                           & BN\_FAST\_S\_MP\_MUL\_HIGH\_DIGS\_C \\
\hline Exponentiation with random odd moduli & (The above plus the following) \\
                                           & BN\_MP\_REDUCE\_2K\_C \\
                                           & BN\_MP\_REDUCE\_2K\_SETUP\_C \\
                                           & BN\_MP\_REDUCE\_IS\_2K\_C \\
                                           & BN\_MP\_DR\_IS\_MODULUS\_C \\
                                           & BN\_MP\_DR\_REDUCE\_C \\
                                           & BN\_MP\_DR\_SETUP\_C \\
\hline Modular inverse odd moduli only     & BN\_MP\_INVMOD\_SLOW\_C \\
\hline Modular inverse (both, smaller/slower) & BN\_FAST\_MP\_INVMOD\_C \\
\hline
\end{tabular}
\end{center}
\end{small}

\subsubsection{Operand Size Related}
\begin{small}
\begin{center}
\begin{tabular}{|l|l|}
\hline \textbf{Restriction} & \textbf{Undefine} \\
\hline Moduli $\le 2560$ bits              & BN\_MP\_MONTGOMERY\_REDUCE\_C \\
                                           & BN\_S\_MP\_MUL\_DIGS\_C \\
                                           & BN\_S\_MP\_MUL\_HIGH\_DIGS\_C \\
                                           & BN\_S\_MP\_SQR\_C \\
\hline Polynomial Schmolynomial            & BN\_MP\_KARATSUBA\_MUL\_C \\
                                           & BN\_MP\_KARATSUBA\_SQR\_C \\
                                           & BN\_MP\_TOOM\_MUL\_C \\
                                           & BN\_MP\_TOOM\_SQR\_C \\

\hline
\end{tabular}
\end{center}
\end{small}


\section{Purpose of LibTomMath}
Unlike  GNU MP (GMP) Library, LIP, OpenSSL or various other commercial kits (Miracl), LibTomMath was not written with
bleeding edge performance in mind.  First and foremost LibTomMath was written to be entirely open.  Not only is the
source code public domain (unlike various other GPL/etc licensed code), not only is the code freely downloadable but the
source code is also accessible for computer science students attempting to learn ``BigNum'' or multiple precision
arithmetic techniques.

LibTomMath was written to be an instructive collection of source code.  This is why there are many comments, only one
function per source file and often I use a ``middle-road'' approach where I don't cut corners for an extra 2\% speed
increase.

Source code alone cannot really teach how the algorithms work which is why I also wrote a textbook that accompanies
the library (beat that!).

So you may be thinking ``should I use LibTomMath?'' and the answer is a definite maybe.  Let me tabulate what I think
are the pros and cons of LibTomMath by comparing it to the math routines from GnuPG\footnote{GnuPG v1.2.3 versus LibTomMath v0.28}.

\newpage\begin{figure}[h]
\begin{small}
\begin{center}
\begin{tabular}{|l|c|c|l|}
\hline \textbf{Criteria} & \textbf{Pro} & \textbf{Con} & \textbf{Notes} \\
\hline Few lines of code per file & X & & GnuPG $ = 300.9$, LibTomMath  $ = 71.97$ \\
\hline Commented function prototypes & X && GnuPG function names are cryptic. \\
\hline Speed && X & LibTomMath is slower.  \\
\hline Totally free & X & & GPL has unfavourable restrictions.\\
\hline Large function base & X & & GnuPG is barebones. \\
\hline Five modular reduction algorithms & X & & Faster modular exponentiation for a variety of moduli. \\
\hline Portable & X & & GnuPG requires configuration to build. \\
\hline
\end{tabular}
\end{center}
\end{small}
\caption{LibTomMath Valuation}
\end{figure}

It may seem odd to compare LibTomMath to GnuPG since the math in GnuPG is only a small portion of the entire application.
However, LibTomMath was written with cryptography in mind.  It provides essentially all of the functions a cryptosystem
would require when working with large integers.

So it may feel tempting to just rip the math code out of GnuPG (or GnuMP where it was taken from originally) in your
own application but I think there are reasons not to.  While LibTomMath is slower than libraries such as GnuMP it is
not normally significantly slower.  On x86 machines the difference is normally a factor of two when performing modular
exponentiations.  It depends largely on the processor, compiler and the moduli being used.

Essentially the only time you would not use LibTomMath is when blazing speed is the primary concern.  However,
on the other side of the coin LibTomMath offers you a totally free (public domain) well structured math library
that is very flexible, complete and performs well in resource constrained environments.  Fast RSA for example can
be performed with as little as 8KB of ram for data (again depending on build options).

\chapter{Getting Started with LibTomMath}
\section{Building Programs}
In order to use LibTomMath you must include ``tommath.h'' and link against the appropriate library file (typically
libtommath.a).  There is no library initialization required and the entire library is thread safe.

\section{Return Codes}
There are three possible return codes a function may return.

\index{MP\_OKAY}\index{MP\_YES}\index{MP\_NO}\index{MP\_VAL}\index{MP\_MEM}
\begin{figure}[h!]
\begin{center}
\begin{small}
\begin{tabular}{|l|l|}
\hline \textbf{Code} & \textbf{Meaning} \\
\hline MP\_OKAY & The function succeeded. \\
\hline MP\_VAL  & The function input was invalid. \\
\hline MP\_MEM  & Heap memory exhausted. \\
\hline &\\
\hline MP\_YES  & Response is yes. \\
\hline MP\_NO   & Response is no. \\
\hline
\end{tabular}
\end{small}
\end{center}
\caption{Return Codes}
\end{figure}

The last two codes listed are not actually ``return'ed'' by a function.  They are placed in an integer (the caller must
provide the address of an integer it can store to) which the caller can access.  To convert one of the three return codes
to a string use the following function.

\index{mp\_error\_to\_string}
\begin{alltt}
char *mp_error_to_string(int code);
\end{alltt}

This will return a pointer to a string which describes the given error code.  It will not work for the return codes
MP\_YES and MP\_NO.

\section{Data Types}
The basic ``multiple precision integer'' type is known as the ``mp\_int'' within LibTomMath.  This data type is used to
organize all of the data required to manipulate the integer it represents.  Within LibTomMath it has been prototyped
as the following.

\index{mp\_int}
\begin{alltt}
typedef struct  \{
    int used, alloc, sign;
    mp_digit *dp;
\} mp_int;
\end{alltt}

Where ``mp\_digit'' is a data type that represents individual digits of the integer.  By default, an mp\_digit is the
ISO C ``unsigned long'' data type and each digit is $28-$bits long.  The mp\_digit type can be configured to suit other
platforms by defining the appropriate macros.

All LTM functions that use the mp\_int type will expect a pointer to mp\_int structure.  You must allocate memory to
hold the structure itself by yourself (whether off stack or heap it does not matter).  The very first thing that must be
done to use an mp\_int is that it must be initialized.

\section{Function Organization}

The arithmetic functions of the library are all organized to have the same style prototype.  That is source operands
are passed on the left and the destination is on the right.  For instance,

\begin{alltt}
mp_add(&a, &b, &c);       /* c = a + b */
mp_mul(&a, &a, &c);       /* c = a * a */
mp_div(&a, &b, &c, &d);   /* c = [a/b], d = a mod b */
\end{alltt}

Another feature of the way the functions have been implemented is that source operands can be destination operands as well.
For instance,

\begin{alltt}
mp_add(&a, &b, &b);       /* b = a + b */
mp_div(&a, &b, &a, &c);   /* a = [a/b], c = a mod b */
\end{alltt}

This allows operands to be re-used which can make programming simpler.

\section{Initialization}
\subsection{Single Initialization}
A single mp\_int can be initialized with the ``mp\_init'' function.

\index{mp\_init}
\begin{alltt}
int mp_init (mp_int * a);
\end{alltt}

This function expects a pointer to an mp\_int structure and will initialize the members of the structure so the mp\_int
represents the default integer which is zero.  If the functions returns MP\_OKAY then the mp\_int is ready to be used
by the other LibTomMath functions.

\begin{small} \begin{alltt}
int main(void)
\{
   mp_int number;
   int result;

   if ((result = mp_init(&number)) != MP_OKAY) \{
      printf("Error initializing the number.  \%s",
             mp_error_to_string(result));
      return EXIT_FAILURE;
   \}

   /* use the number */

   return EXIT_SUCCESS;
\}
\end{alltt} \end{small}

\subsection{Single Free}
When you are finished with an mp\_int it is ideal to return the heap it used back to the system.  The following function
provides this functionality.

\index{mp\_clear}
\begin{alltt}
void mp_clear (mp_int * a);
\end{alltt}

The function expects a pointer to a previously initialized mp\_int structure and frees the heap it uses.  It sets the
pointer\footnote{The ``dp'' member.} within the mp\_int to \textbf{NULL} which is used to prevent double free situations.
Is is legal to call mp\_clear() twice on the same mp\_int in a row.

\begin{small} \begin{alltt}
int main(void)
\{
   mp_int number;
   int result;

   if ((result = mp_init(&number)) != MP_OKAY) \{
      printf("Error initializing the number.  \%s",
             mp_error_to_string(result));
      return EXIT_FAILURE;
   \}

   /* use the number */

   /* We're done with it. */
   mp_clear(&number);

   return EXIT_SUCCESS;
\}
\end{alltt} \end{small}

\subsection{Multiple Initializations}
Certain algorithms require more than one large integer.  In these instances it is ideal to initialize all of the mp\_int
variables in an ``all or nothing'' fashion.  That is, they are either all initialized successfully or they are all
not initialized.

The  mp\_init\_multi() function provides this functionality.

\index{mp\_init\_multi} \index{mp\_clear\_multi}
\begin{alltt}
int mp_init_multi(mp_int *mp, ...);
\end{alltt}

It accepts a \textbf{NULL} terminated list of pointers to mp\_int structures.  It will attempt to initialize them all
at once.  If the function returns MP\_OKAY then all of the mp\_int variables are ready to use, otherwise none of them
are available for use.  A complementary mp\_clear\_multi() function allows multiple mp\_int variables to be free'd
from the heap at the same time.

\begin{small} \begin{alltt}
int main(void)
\{
   mp_int num1, num2, num3;
   int result;

   if ((result = mp_init_multi(&num1,
                               &num2,
                               &num3, NULL)) != MP\_OKAY) \{
      printf("Error initializing the numbers.  \%s",
             mp_error_to_string(result));
      return EXIT_FAILURE;
   \}

   /* use the numbers */

   /* We're done with them. */
   mp_clear_multi(&num1, &num2, &num3, NULL);

   return EXIT_SUCCESS;
\}
\end{alltt} \end{small}

\subsection{Other Initializers}
To initialized and make a copy of an mp\_int the mp\_init\_copy() function has been provided.

\index{mp\_init\_copy}
\begin{alltt}
int mp_init_copy (mp_int * a, mp_int * b);
\end{alltt}

This function will initialize $a$ and make it a copy of $b$ if all goes well.

\begin{small} \begin{alltt}
int main(void)
\{
   mp_int num1, num2;
   int result;

   /* initialize and do work on num1 ... */

   /* We want a copy of num1 in num2 now */
   if ((result = mp_init_copy(&num2, &num1)) != MP_OKAY) \{
     printf("Error initializing the copy.  \%s",
             mp_error_to_string(result));
      return EXIT_FAILURE;
   \}

   /* now num2 is ready and contains a copy of num1 */

   /* We're done with them. */
   mp_clear_multi(&num1, &num2, NULL);

   return EXIT_SUCCESS;
\}
\end{alltt} \end{small}

Another less common initializer is mp\_init\_size() which allows the user to initialize an mp\_int with a given
default number of digits.  By default, all initializers allocate \textbf{MP\_PREC} digits.  This function lets
you override this behaviour.

\index{mp\_init\_size}
\begin{alltt}
int mp_init_size (mp_int * a, int size);
\end{alltt}

The $size$ parameter must be greater than zero.  If the function succeeds the mp\_int $a$ will be initialized
to have $size$ digits (which are all initially zero).

\begin{small} \begin{alltt}
int main(void)
\{
   mp_int number;
   int result;

   /* we need a 60-digit number */
   if ((result = mp_init_size(&number, 60)) != MP_OKAY) \{
      printf("Error initializing the number.  \%s",
             mp_error_to_string(result));
      return EXIT_FAILURE;
   \}

   /* use the number */

   return EXIT_SUCCESS;
\}
\end{alltt} \end{small}

\section{Maintenance Functions}
\subsection{Clear Leading Zeros}

This is used to ensure that leading zero digits are trimmed and the leading "used" digit will be non-zero.
It also fixes the sign if there are no more leading digits.

\index{mp\_clamp}
\begin{alltt}
void mp_clamp(mp_int *a);
\end{alltt}

\subsection{Zero Out}

This function will set the ``bigint'' to zeros without changing the amount of allocated memory.

\index{mp\_zero}
\begin{alltt}
void mp_zero(mp_int *a);
\end{alltt}


\subsection{Reducing Memory Usage}
When an mp\_int is in a state where it won't be changed again\footnote{A Diffie-Hellman modulus for instance.} excess
digits can be removed to return memory to the heap with the mp\_shrink() function.

\index{mp\_shrink}
\begin{alltt}
int mp_shrink (mp_int * a);
\end{alltt}

This will remove excess digits of the mp\_int $a$.  If the operation fails the mp\_int should be intact without the
excess digits being removed.  Note that you can use a shrunk mp\_int in further computations, however, such operations
will require heap operations which can be slow.  It is not ideal to shrink mp\_int variables that you will further
modify in the system (unless you are seriously low on memory).

\begin{small} \begin{alltt}
int main(void)
\{
   mp_int number;
   int result;

   if ((result = mp_init(&number)) != MP_OKAY) \{
      printf("Error initializing the number.  \%s",
             mp_error_to_string(result));
      return EXIT_FAILURE;
   \}

   /* use the number [e.g. pre-computation]  */

   /* We're done with it for now. */
   if ((result = mp_shrink(&number)) != MP_OKAY) \{
      printf("Error shrinking the number.  \%s",
             mp_error_to_string(result));
      return EXIT_FAILURE;
   \}

   /* use it .... */


   /* we're done with it. */
   mp_clear(&number);

   return EXIT_SUCCESS;
\}
\end{alltt} \end{small}

\subsection{Adding additional digits}

Within the mp\_int structure are two parameters which control the limitations of the array of digits that represent
the integer the mp\_int is meant to equal.   The \textit{used} parameter dictates how many digits are significant, that is,
contribute to the value of the mp\_int.  The \textit{alloc} parameter dictates how many digits are currently available in
the array.  If you need to perform an operation that requires more digits you will have to mp\_grow() the mp\_int to
your desired size.

\index{mp\_grow}
\begin{alltt}
int mp_grow (mp_int * a, int size);
\end{alltt}

This will grow the array of digits of $a$ to $size$.  If the \textit{alloc} parameter is already bigger than
$size$ the function will not do anything.

\begin{small} \begin{alltt}
int main(void)
\{
   mp_int number;
   int result;

   if ((result = mp_init(&number)) != MP_OKAY) \{
      printf("Error initializing the number.  \%s",
             mp_error_to_string(result));
      return EXIT_FAILURE;
   \}

   /* use the number */

   /* We need to add 20 digits to the number  */
   if ((result = mp_grow(&number, number.alloc + 20)) != MP_OKAY) \{
      printf("Error growing the number.  \%s",
             mp_error_to_string(result));
      return EXIT_FAILURE;
   \}


   /* use the number */

   /* we're done with it. */
   mp_clear(&number);

   return EXIT_SUCCESS;
\}
\end{alltt} \end{small}

\chapter{Basic Operations}
\section{Copying}

A so called ``deep copy'', where new memory is allocated and all contents of $a$ are copied verbatim into $b$ such that $b = a$ at the end.

\index{mp\_copy}
\begin{alltt}
int mp_copy (mp_int * a, mp_int *b);
\end{alltt}

You can also just swap $a$ and $b$. It does the normal pointer changing with a temporary pointer variable, just that you do not have to.

\index{mp\_exch}
\begin{alltt}
void mp_exch (mp_int * a, mp_int *b);
\end{alltt}

\section{Bit Counting}

To get the position of the lowest bit set (LSB, the Lowest Significant Bit; the number of bits which are zero before the first zero bit )

\index{mp\_cnt\_lsb}
\begin{alltt}
int mp_cnt_lsb(const mp_int *a);
\end{alltt}

To get the position of the highest bit set (MSB, the Most Significant Bit; the number of bits in teh ``bignum'')

\index{mp\_count\_bits}
\begin{alltt}
int mp_count_bits(const mp_int *a);
\end{alltt}


\section{Small Constants}
Setting mp\_ints to small constants is a relatively common operation.  To accommodate these instances there are two
small constant assignment functions.  The first function is used to set a single digit constant while the second sets
an ISO C style ``unsigned long'' constant.  The reason for both functions is efficiency.  Setting a single digit is quick but the
domain of a digit can change (it's always at least $0 \ldots 127$).

\subsection{Single Digit}

Setting a single digit can be accomplished with the following function.

\index{mp\_set}
\begin{alltt}
void mp_set (mp_int * a, mp_digit b);
\end{alltt}

This will zero the contents of $a$ and make it represent an integer equal to the value of $b$.  Note that this
function has a return type of \textbf{void}.  It cannot cause an error so it is safe to assume the function
succeeded.

\begin{small} \begin{alltt}
int main(void)
\{
   mp_int number;
   int result;

   if ((result = mp_init(&number)) != MP_OKAY) \{
      printf("Error initializing the number.  \%s",
             mp_error_to_string(result));
      return EXIT_FAILURE;
   \}

   /* set the number to 5 */
   mp_set(&number, 5);

   /* we're done with it. */
   mp_clear(&number);

   return EXIT_SUCCESS;
\}
\end{alltt} \end{small}

\subsection{Long Constants}

To set a constant that is the size of an ISO C ``unsigned long'' and larger than a single digit the following function
can be used.

\index{mp\_set\_int}
\begin{alltt}
int mp_set_int (mp_int * a, unsigned long b);
\end{alltt}

This will assign the value of the 32-bit variable $b$ to the mp\_int $a$.  Unlike mp\_set() this function will always
accept a 32-bit input regardless of the size of a single digit.  However, since the value may span several digits
this function can fail if it runs out of heap memory.

To get the ``unsigned long'' copy of an mp\_int the following function can be used.

\index{mp\_get\_int}
\begin{alltt}
unsigned long mp_get_int (mp_int * a);
\end{alltt}

This will return the 32 least significant bits of the mp\_int $a$.

\begin{small} \begin{alltt}
int main(void)
\{
   mp_int number;
   int result;

   if ((result = mp_init(&number)) != MP_OKAY) \{
      printf("Error initializing the number.  \%s",
             mp_error_to_string(result));
      return EXIT_FAILURE;
   \}

   /* set the number to 654321 (note this is bigger than 127) */
   if ((result = mp_set_int(&number, 654321)) != MP_OKAY) \{
      printf("Error setting the value of the number.  \%s",
             mp_error_to_string(result));
      return EXIT_FAILURE;
   \}

   printf("number == \%lu", mp_get_int(&number));

   /* we're done with it. */
   mp_clear(&number);

   return EXIT_SUCCESS;
\}
\end{alltt} \end{small}

This should output the following if the program succeeds.

\begin{alltt}
number == 654321
\end{alltt}

\subsection{Long Constants - platform dependent}

\index{mp\_set\_long}
\begin{alltt}
int mp_set_long (mp_int * a, unsigned long b);
\end{alltt}

This will assign the value of the platform-dependent sized variable $b$ to the mp\_int $a$.

To get the ``unsigned long'' copy of an mp\_int the following function can be used.

\index{mp\_get\_long}
\begin{alltt}
unsigned long mp_get_long (mp_int * a);
\end{alltt}

This will return the least significant bits of the mp\_int $a$ that fit into an ``unsigned long''.

\subsection{Long Long Constants}

\index{mp\_set\_long\_long}
\begin{alltt}
int mp_set_long_long (mp_int * a, unsigned long long b);
\end{alltt}

This will assign the value of the 64-bit variable $b$ to the mp\_int $a$.

To get the ``unsigned long long'' copy of an mp\_int the following function can be used.

\index{mp\_get\_long\_long}
\begin{alltt}
unsigned long long mp_get_long_long (mp_int * a);
\end{alltt}

This will return the 64 least significant bits of the mp\_int $a$.

\subsection{Initialize and Setting Constants}
To both initialize and set small constants the following two functions are available.
\index{mp\_init\_set} \index{mp\_init\_set\_int}
\begin{alltt}
int mp_init_set (mp_int * a, mp_digit b);
int mp_init_set_int (mp_int * a, unsigned long b);
\end{alltt}

Both functions work like the previous counterparts except they first mp\_init $a$ before setting the values.

\begin{alltt}
int main(void)
\{
   mp_int number1, number2;
   int    result;

   /* initialize and set a single digit */
   if ((result = mp_init_set(&number1, 100)) != MP_OKAY) \{
      printf("Error setting number1: \%s",
             mp_error_to_string(result));
      return EXIT_FAILURE;
   \}

   /* initialize and set a long */
   if ((result = mp_init_set_int(&number2, 1023)) != MP_OKAY) \{
      printf("Error setting number2: \%s",
             mp_error_to_string(result));
      return EXIT_FAILURE;
   \}

   /* display */
   printf("Number1, Number2 == \%lu, \%lu",
          mp_get_int(&number1), mp_get_int(&number2));

   /* clear */
   mp_clear_multi(&number1, &number2, NULL);

   return EXIT_SUCCESS;
\}
\end{alltt}

If this program succeeds it shall output.
\begin{alltt}
Number1, Number2 == 100, 1023
\end{alltt}

\section{Comparisons}

Comparisons in LibTomMath are always performed in a ``left to right'' fashion.  There are three possible return codes
for any comparison.

\index{MP\_GT} \index{MP\_EQ} \index{MP\_LT}
\begin{figure}[h]
\begin{center}
\begin{tabular}{|c|c|}
\hline \textbf{Result Code} & \textbf{Meaning} \\
\hline MP\_GT & $a > b$ \\
\hline MP\_EQ & $a = b$ \\
\hline MP\_LT & $a < b$ \\
\hline
\end{tabular}
\end{center}
\caption{Comparison Codes for $a, b$}
\label{fig:CMP}
\end{figure}

In figure \ref{fig:CMP} two integers $a$ and $b$ are being compared.  In this case $a$ is said to be ``to the left'' of
$b$.

\subsection{Unsigned comparison}

An unsigned comparison considers only the digits themselves and not the associated \textit{sign} flag of the
mp\_int structures.  This is analogous to an absolute comparison.  The function mp\_cmp\_mag() will compare two
mp\_int variables based on their digits only.

\index{mp\_cmp\_mag}
\begin{alltt}
int mp_cmp_mag(mp_int * a, mp_int * b);
\end{alltt}
This will compare $a$ to $b$ placing $a$ to the left of $b$.  This function cannot fail and will return one of the
three compare codes listed in figure \ref{fig:CMP}.

\begin{small} \begin{alltt}
int main(void)
\{
   mp_int number1, number2;
   int result;

   if ((result = mp_init_multi(&number1, &number2, NULL)) != MP_OKAY) \{
      printf("Error initializing the numbers.  \%s",
             mp_error_to_string(result));
      return EXIT_FAILURE;
   \}

   /* set the number1 to 5 */
   mp_set(&number1, 5);

   /* set the number2 to -6 */
   mp_set(&number2, 6);
   if ((result = mp_neg(&number2, &number2)) != MP_OKAY) \{
      printf("Error negating number2.  \%s",
             mp_error_to_string(result));
      return EXIT_FAILURE;
   \}

   switch(mp_cmp_mag(&number1, &number2)) \{
       case MP_GT:  printf("|number1| > |number2|"); break;
       case MP_EQ:  printf("|number1| = |number2|"); break;
       case MP_LT:  printf("|number1| < |number2|"); break;
   \}

   /* we're done with it. */
   mp_clear_multi(&number1, &number2, NULL);

   return EXIT_SUCCESS;
\}
\end{alltt} \end{small}

If this program\footnote{This function uses the mp\_neg() function which is discussed in section \ref{sec:NEG}.} completes
successfully it should print the following.

\begin{alltt}
|number1| < |number2|
\end{alltt}

This is because $\vert -6 \vert = 6$ and obviously $5 < 6$.

\subsection{Signed comparison}

To compare two mp\_int variables based on their signed value the mp\_cmp() function is provided.

\index{mp\_cmp}
\begin{alltt}
int mp_cmp(mp_int * a, mp_int * b);
\end{alltt}

This will compare $a$ to the left of $b$.  It will first compare the signs of the two mp\_int variables.  If they
differ it will return immediately based on their signs.  If the signs are equal then it will compare the digits
individually.  This function will return one of the compare conditions codes listed in figure \ref{fig:CMP}.

\begin{small} \begin{alltt}
int main(void)
\{
   mp_int number1, number2;
   int result;

   if ((result = mp_init_multi(&number1, &number2, NULL)) != MP_OKAY) \{
      printf("Error initializing the numbers.  \%s",
             mp_error_to_string(result));
      return EXIT_FAILURE;
   \}

   /* set the number1 to 5 */
   mp_set(&number1, 5);

   /* set the number2 to -6 */
   mp_set(&number2, 6);
   if ((result = mp_neg(&number2, &number2)) != MP_OKAY) \{
      printf("Error negating number2.  \%s",
             mp_error_to_string(result));
      return EXIT_FAILURE;
   \}

   switch(mp_cmp(&number1, &number2)) \{
       case MP_GT:  printf("number1 > number2"); break;
       case MP_EQ:  printf("number1 = number2"); break;
       case MP_LT:  printf("number1 < number2"); break;
   \}

   /* we're done with it. */
   mp_clear_multi(&number1, &number2, NULL);

   return EXIT_SUCCESS;
\}
\end{alltt} \end{small}

If this program\footnote{This function uses the mp\_neg() function which is discussed in section \ref{sec:NEG}.} completes
successfully it should print the following.

\begin{alltt}
number1 > number2
\end{alltt}

\subsection{Single Digit}

To compare a single digit against an mp\_int the following function has been provided.

\index{mp\_cmp\_d}
\begin{alltt}
int mp_cmp_d(mp_int * a, mp_digit b);
\end{alltt}

This will compare $a$ to the left of $b$ using a signed comparison.  Note that it will always treat $b$ as
positive.  This function is rather handy when you have to compare against small values such as $1$ (which often
comes up in cryptography).  The function cannot fail and will return one of the tree compare condition codes
listed in figure \ref{fig:CMP}.


\begin{small} \begin{alltt}
int main(void)
\{
   mp_int number;
   int result;

   if ((result = mp_init(&number)) != MP_OKAY) \{
      printf("Error initializing the number.  \%s",
             mp_error_to_string(result));
      return EXIT_FAILURE;
   \}

   /* set the number to 5 */
   mp_set(&number, 5);

   switch(mp_cmp_d(&number, 7)) \{
       case MP_GT:  printf("number > 7"); break;
       case MP_EQ:  printf("number = 7"); break;
       case MP_LT:  printf("number < 7"); break;
   \}

   /* we're done with it. */
   mp_clear(&number);

   return EXIT_SUCCESS;
\}
\end{alltt} \end{small}

If this program functions properly it will print out the following.

\begin{alltt}
number < 7
\end{alltt}

\section{Logical Operations}

Logical operations are operations that can be performed either with simple shifts or boolean operators such as
AND, XOR and OR directly.  These operations are very quick.

\subsection{Multiplication by two}

Multiplications and divisions by any power of two can be performed with quick logical shifts either left or
right depending on the operation.

When multiplying or dividing by two a special case routine can be used which are as follows.
\index{mp\_mul\_2} \index{mp\_div\_2}
\begin{alltt}
int mp_mul_2(mp_int * a, mp_int * b);
int mp_div_2(mp_int * a, mp_int * b);
\end{alltt}

The former will assign twice $a$ to $b$ while the latter will assign half $a$ to $b$.  These functions are fast
since the shift counts and masks are hardcoded into the routines.

\begin{small} \begin{alltt}
int main(void)
\{
   mp_int number;
   int result;

   if ((result = mp_init(&number)) != MP_OKAY) \{
      printf("Error initializing the number.  \%s",
             mp_error_to_string(result));
      return EXIT_FAILURE;
   \}

   /* set the number to 5 */
   mp_set(&number, 5);

   /* multiply by two */
   if ((result = mp\_mul\_2(&number, &number)) != MP_OKAY) \{
      printf("Error multiplying the number.  \%s",
             mp_error_to_string(result));
      return EXIT_FAILURE;
   \}
   switch(mp_cmp_d(&number, 7)) \{
       case MP_GT:  printf("2*number > 7"); break;
       case MP_EQ:  printf("2*number = 7"); break;
       case MP_LT:  printf("2*number < 7"); break;
   \}

   /* now divide by two */
   if ((result = mp\_div\_2(&number, &number)) != MP_OKAY) \{
      printf("Error dividing the number.  \%s",
             mp_error_to_string(result));
      return EXIT_FAILURE;
   \}
   switch(mp_cmp_d(&number, 7)) \{
       case MP_GT:  printf("2*number/2 > 7"); break;
       case MP_EQ:  printf("2*number/2 = 7"); break;
       case MP_LT:  printf("2*number/2 < 7"); break;
   \}

   /* we're done with it. */
   mp_clear(&number);

   return EXIT_SUCCESS;
\}
\end{alltt} \end{small}

If this program is successful it will print out the following text.

\begin{alltt}
2*number > 7
2*number/2 < 7
\end{alltt}

Since $10 > 7$ and $5 < 7$.

To multiply by a power of two the following function can be used.

\index{mp\_mul\_2d}
\begin{alltt}
int mp_mul_2d(mp_int * a, int b, mp_int * c);
\end{alltt}

This will multiply $a$ by $2^b$ and store the result in ``c''.  If the value of $b$ is less than or equal to
zero the function will copy $a$ to ``c'' without performing any further actions.  The multiplication itself
is implemented as a right-shift operation of $a$ by $b$ bits.

To divide by a power of two use the following.

\index{mp\_div\_2d}
\begin{alltt}
int mp_div_2d (mp_int * a, int b, mp_int * c, mp_int * d);
\end{alltt}
Which will divide $a$ by $2^b$, store the quotient in ``c'' and the remainder in ``d'.  If $b \le 0$ then the
function simply copies $a$ over to ``c'' and zeros $d$.  The variable $d$ may be passed as a \textbf{NULL}
value to signal that the remainder is not desired.  The division itself is implemented as a left-shift
operation of $a$ by $b$ bits.

\index{mp\_tc\_div\_2d}\label{arithrightshift}
\begin{alltt}
int mp_tc_div_2d (mp_int * a, int b, mp_int * c, mp_int * d);
\end{alltt}
The two-complement version of the function above. This can be used to implement arbitrary-precision two-complement integers together with the two-complement bit-wise operations at page \ref{tcbitwiseops}.


It is also not very uncommon to need just the power of two $2^b$;  for example the startvalue for the Newton method.

\index{mp\_2expt}
\begin{alltt}
int mp_2expt(mp_int *a, int b);
\end{alltt}
It is faster than doing it by shifting $1$ with \texttt{mp\_mul\_2d}.

\subsection{Polynomial Basis Operations}

Strictly speaking the organization of the integers within the mp\_int structures is what is known as a
``polynomial basis''.  This simply means a field element is stored by divisions of a radix.  For example, if
$f(x) = \sum_{i=0}^{k} y_ix^k$ for any vector $\vec y$ then the array of digits in $\vec y$ are said to be
the polynomial basis representation of $z$ if $f(\beta) = z$ for a given radix $\beta$.

To multiply by the polynomial $g(x) = x$ all you have todo is shift the digits of the basis left one place.  The
following function provides this operation.

\index{mp\_lshd}
\begin{alltt}
int mp_lshd (mp_int * a, int b);
\end{alltt}

This will multiply $a$ in place by $x^b$ which is equivalent to shifting the digits left $b$ places and inserting zeros
in the least significant digits.  Similarly to divide by a power of $x$ the following function is provided.

\index{mp\_rshd}
\begin{alltt}
void mp_rshd (mp_int * a, int b)
\end{alltt}
This will divide $a$ in place by $x^b$ and discard the remainder.  This function cannot fail as it performs the operations
in place and no new digits are required to complete it.

\subsection{AND, OR, XOR and COMPLEMENT Operations}

While AND, OR and XOR operations are not typical ``bignum functions'' they can be useful in several instances.  The
three functions are prototyped as follows.

\index{mp\_or} \index{mp\_and} \index{mp\_xor}
\begin{alltt}
int mp_or  (mp_int * a, mp_int * b, mp_int * c);
int mp_and (mp_int * a, mp_int * b, mp_int * c);
int mp_xor (mp_int * a, mp_int * b, mp_int * c);
\end{alltt}

Which compute $c = a \odot b$ where $\odot$ is one of OR, AND or XOR.

The following four functions allow implementing arbitrary-precision two-complement numbers.

\index{mp\_tc\_or} \index{mp\_tc\_and} \index{mp\_tc\_xor} \index{mp\_complement} \label{tcbitwiseops}
\begin{alltt}
int mp_tc_or  (mp_int * a, mp_int * b, mp_int * c);
int mp_tc_and (mp_int * a, mp_int * b, mp_int * c);
int mp_tc_xor (mp_int * a, mp_int * b, mp_int * c);
int mp_complement(const mp_int *a, mp_int *b);
\end{alltt}

They compute $c = a \odot b$ as above if both $a$ and $b$ are positive. Negative values are converted into their two-complement representations first. The function \texttt{mp\_complement} computes a two-complement $b = \sim a$.


\subsection{Bit Picking}
\index{mp\_get\_bit}
\begin{alltt}
int mp_get_bit(mp_int *a, int b)
\end{alltt}

Pick a bit: returns \texttt{MP\_YES} if the bit at position $b$ (0-index) is set, that is if it is 1 (one), \texttt{MP\_NO}
if the bit is 0 (zero) and \texttt{MP\_VAL} if $b < 0$.

\section{Addition and Subtraction}

To compute an addition or subtraction the following two functions can be used.

\index{mp\_add} \index{mp\_sub}
\begin{alltt}
int mp_add (mp_int * a, mp_int * b, mp_int * c);
int mp_sub (mp_int * a, mp_int * b, mp_int * c)
\end{alltt}

Which perform $c = a \odot b$ where $\odot$ is one of signed addition or subtraction.  The operations are fully sign
aware.

\section{Sign Manipulation}
\subsection{Negation}
\label{sec:NEG}
Simple integer negation can be performed with the following.

\index{mp\_neg}
\begin{alltt}
int mp_neg (mp_int * a, mp_int * b);
\end{alltt}

Which assigns $-a$ to $b$.

\subsection{Absolute}
Simple integer absolutes can be performed with the following.

\index{mp\_abs}
\begin{alltt}
int mp_abs (mp_int * a, mp_int * b);
\end{alltt}

Which assigns $\vert a \vert$ to $b$.

\section{Integer Division and Remainder}
To perform a complete and general integer division with remainder use the following function.

\index{mp\_div}
\begin{alltt}
int mp_div (mp_int * a, mp_int * b, mp_int * c, mp_int * d);
\end{alltt}

This divides $a$ by $b$ and stores the quotient in $c$ and $d$.  The signed quotient is computed such that
$bc + d = a$.  Note that either of $c$ or $d$ can be set to \textbf{NULL} if their value is not required.  If
$b$ is zero the function returns \textbf{MP\_VAL}.


\chapter{Multiplication and Squaring}
\section{Multiplication}
A full signed integer multiplication can be performed with the following.
\index{mp\_mul}
\begin{alltt}
int mp_mul (mp_int * a, mp_int * b, mp_int * c);
\end{alltt}
Which assigns the full signed product $ab$ to $c$.  This function actually breaks into one of four cases which are
specific multiplication routines optimized for given parameters.  First there are the Toom-Cook multiplications which
should only be used with very large inputs.  This is followed by the Karatsuba multiplications which are for moderate
sized inputs.  Then followed by the Comba and baseline multipliers.

Fortunately for the developer you don't really need to know this unless you really want to fine tune the system.  mp\_mul()
will determine on its own\footnote{Some tweaking may be required.} what routine to use automatically when it is called.

\begin{alltt}
int main(void)
\{
   mp_int number1, number2;
   int result;

   /* Initialize the numbers */
   if ((result = mp_init_multi(&number1,
                               &number2, NULL)) != MP_OKAY) \{
      printf("Error initializing the numbers.  \%s",
             mp_error_to_string(result));
      return EXIT_FAILURE;
   \}

   /* set the terms */
   if ((result = mp_set_int(&number, 257)) != MP_OKAY) \{
      printf("Error setting number1.  \%s",
             mp_error_to_string(result));
      return EXIT_FAILURE;
   \}

   if ((result = mp_set_int(&number2, 1023)) != MP_OKAY) \{
      printf("Error setting number2.  \%s",
             mp_error_to_string(result));
      return EXIT_FAILURE;
   \}

   /* multiply them */
   if ((result = mp_mul(&number1, &number2,
                        &number1)) != MP_OKAY) \{
      printf("Error multiplying terms.  \%s",
             mp_error_to_string(result));
      return EXIT_FAILURE;
   \}

   /* display */
   printf("number1 * number2 == \%lu", mp_get_int(&number1));

   /* free terms and return */
   mp_clear_multi(&number1, &number2, NULL);

   return EXIT_SUCCESS;
\}
\end{alltt}

If this program succeeds it shall output the following.

\begin{alltt}
number1 * number2 == 262911
\end{alltt}

\section{Squaring}
Since squaring can be performed faster than multiplication it is performed it's own function instead of just using
mp\_mul().

\index{mp\_sqr}
\begin{alltt}
int mp_sqr (mp_int * a, mp_int * b);
\end{alltt}

Will square $a$ and store it in $b$.  Like the case of multiplication there are four different squaring
algorithms all which can be called from mp\_sqr().  It is ideal to use mp\_sqr over mp\_mul when squaring terms because
of the speed difference.

\section{Tuning Polynomial Basis Routines}

Both of the Toom-Cook and Karatsuba multiplication algorithms are faster than the traditional $O(n^2)$ approach that
the Comba and baseline algorithms use.  At $O(n^{1.464973})$ and $O(n^{1.584962})$ running times respectively they require
considerably less work.  For example, a 10000-digit multiplication would take roughly 724,000 single precision
multiplications with Toom-Cook or 100,000,000 single precision multiplications with the standard Comba (a factor
of 138).

So why not always use Karatsuba or Toom-Cook?   The simple answer is that they have so much overhead that they're not
actually faster than Comba until you hit distinct  ``cutoff'' points.  For Karatsuba with the default configuration,
GCC 3.3.1 and an Athlon XP processor the cutoff point is roughly 110 digits (about 70 for the Intel P4).  That is, at
110 digits Karatsuba and Comba multiplications just about break even and for 110+ digits Karatsuba is faster.

Toom-Cook has incredible overhead and is probably only useful for very large inputs.  So far no known cutoff points
exist and for the most part I just set the cutoff points very high to make sure they're not called.

A demo program in the ``etc/'' directory of the project called ``tune.c'' can be used to find the cutoff points.  This
can be built with GCC as follows

\begin{alltt}
make XXX
\end{alltt}
Where ``XXX'' is one of the following entries from the table \ref{fig:tuning}.

\begin{figure}[h]
\begin{center}
\begin{small}
\begin{tabular}{|l|l|}
\hline \textbf{Value of XXX} & \textbf{Meaning} \\
\hline tune & Builds portable tuning application \\
\hline tune86 & Builds x86 (pentium and up) program for COFF \\
\hline tune86c & Builds x86 program for Cygwin \\
\hline tune86l & Builds x86 program for Linux (ELF format) \\
\hline
\end{tabular}
\end{small}
\end{center}
\caption{Build Names for Tuning Programs}
\label{fig:tuning}
\end{figure}

When the program is running it will output a series of measurements for different cutoff points.  It will first find
good Karatsuba squaring and multiplication points.  Then it proceeds to find Toom-Cook points.  Note that the Toom-Cook
tuning takes a very long time as the cutoff points are likely to be very high.

\chapter{Modular Reduction}

Modular reduction is process of taking the remainder of one quantity divided by another.  Expressed
as (\ref{eqn:mod}) the modular reduction is equivalent to the remainder of $b$ divided by $c$.

\begin{equation}
a \equiv b \mbox{ (mod }c\mbox{)}
\label{eqn:mod}
\end{equation}

Of particular interest to cryptography are reductions where $b$ is limited to the range $0 \le b < c^2$ since particularly
fast reduction algorithms can be written for the limited range.

Note that one of the four optimized reduction algorithms are automatically chosen in the modular exponentiation
algorithm mp\_exptmod when an appropriate modulus is detected.

\section{Straight Division}
In order to effect an arbitrary modular reduction the following algorithm is provided.

\index{mp\_mod}
\begin{alltt}
int mp_mod(mp_int *a, mp_int *b, mp_int *c);
\end{alltt}

This reduces $a$ modulo $b$ and stores the result in $c$.  The sign of $c$ shall agree with the sign
of $b$.  This algorithm accepts an input $a$ of any range and is not limited by $0 \le a < b^2$.

\section{Barrett Reduction}

Barrett reduction is a generic optimized reduction algorithm that requires pre--computation to achieve
a decent speedup over straight division.  First a $\mu$ value must be precomputed with the following function.

\index{mp\_reduce\_setup}
\begin{alltt}
int mp_reduce_setup(mp_int *a, mp_int *b);
\end{alltt}

Given a modulus in $b$ this produces the required $\mu$ value in $a$.  For any given modulus this only has to
be computed once.  Modular reduction can now be performed with the following.

\index{mp\_reduce}
\begin{alltt}
int mp_reduce(mp_int *a, mp_int *b, mp_int *c);
\end{alltt}

This will reduce $a$ in place modulo $b$ with the precomputed $\mu$ value in $c$.  $a$ must be in the range
$0 \le a < b^2$.

\begin{alltt}
int main(void)
\{
   mp_int   a, b, c, mu;
   int      result;

   /* initialize a,b to desired values, mp_init mu,
    * c and set c to 1...we want to compute a^3 mod b
    */

   /* get mu value */
   if ((result = mp_reduce_setup(&mu, b)) != MP_OKAY) \{
      printf("Error getting mu.  \%s",
             mp_error_to_string(result));
      return EXIT_FAILURE;
   \}

   /* square a to get c = a^2 */
   if ((result = mp_sqr(&a, &c)) != MP_OKAY) \{
      printf("Error squaring.  \%s",
             mp_error_to_string(result));
      return EXIT_FAILURE;
   \}

   /* now reduce `c' modulo b */
   if ((result = mp_reduce(&c, &b, &mu)) != MP_OKAY) \{
      printf("Error reducing.  \%s",
             mp_error_to_string(result));
      return EXIT_FAILURE;
   \}

   /* multiply a to get c = a^3 */
   if ((result = mp_mul(&a, &c, &c)) != MP_OKAY) \{
      printf("Error reducing.  \%s",
             mp_error_to_string(result));
      return EXIT_FAILURE;
   \}

   /* now reduce `c' modulo b  */
   if ((result = mp_reduce(&c, &b, &mu)) != MP_OKAY) \{
      printf("Error reducing.  \%s",
             mp_error_to_string(result));
      return EXIT_FAILURE;
   \}

   /* c now equals a^3 mod b */

   return EXIT_SUCCESS;
\}
\end{alltt}

This program will calculate $a^3 \mbox{ mod }b$ if all the functions succeed.

\section{Montgomery Reduction}

Montgomery is a specialized reduction algorithm for any odd moduli.  Like Barrett reduction a pre--computation
step is required.  This is accomplished with the following.

\index{mp\_montgomery\_setup}
\begin{alltt}
int mp_montgomery_setup(mp_int *a, mp_digit *mp);
\end{alltt}

For the given odd moduli $a$ the pre--computation value is placed in $mp$.  The reduction is computed with the
following.

\index{mp\_montgomery\_reduce}
\begin{alltt}
int mp_montgomery_reduce(mp_int *a, mp_int *m, mp_digit mp);
\end{alltt}
This reduces $a$ in place modulo $m$ with the pre--computed value $mp$.   $a$ must be in the range
$0 \le a < b^2$.

Montgomery reduction is faster than Barrett reduction for moduli smaller than the ``comba'' limit.  With the default
setup for instance, the limit is $127$ digits ($3556$--bits).   Note that this function is not limited to
$127$ digits just that it falls back to a baseline algorithm after that point.

An important observation is that this reduction does not return $a \mbox{ mod }m$ but $aR^{-1} \mbox{ mod }m$
where $R = \beta^n$, $n$ is the n number of digits in $m$ and $\beta$ is radix used (default is $2^{28}$).

To quickly calculate $R$ the following function was provided.

\index{mp\_montgomery\_calc\_normalization}
\begin{alltt}
int mp_montgomery_calc_normalization(mp_int *a, mp_int *b);
\end{alltt}
Which calculates $a = R$ for the odd moduli $b$ without using multiplication or division.

The normal modus operandi for Montgomery reductions is to normalize the integers before entering the system.  For
example, to calculate $a^3 \mbox { mod }b$ using Montgomery reduction the value of $a$ can be normalized by
multiplying it by $R$.  Consider the following code snippet.

\begin{alltt}
int main(void)
\{
   mp_int   a, b, c, R;
   mp_digit mp;
   int      result;

   /* initialize a,b to desired values,
    * mp_init R, c and set c to 1....
    */

   /* get normalization */
   if ((result = mp_montgomery_calc_normalization(&R, b)) != MP_OKAY) \{
      printf("Error getting norm.  \%s",
             mp_error_to_string(result));
      return EXIT_FAILURE;
   \}

   /* get mp value */
   if ((result = mp_montgomery_setup(&c, &mp)) != MP_OKAY) \{
      printf("Error setting up montgomery.  \%s",
             mp_error_to_string(result));
      return EXIT_FAILURE;
   \}

   /* normalize `a' so now a is equal to aR */
   if ((result = mp_mulmod(&a, &R, &b, &a)) != MP_OKAY) \{
      printf("Error computing aR.  \%s",
             mp_error_to_string(result));
      return EXIT_FAILURE;
   \}

   /* square a to get c = a^2R^2 */
   if ((result = mp_sqr(&a, &c)) != MP_OKAY) \{
      printf("Error squaring.  \%s",
             mp_error_to_string(result));
      return EXIT_FAILURE;
   \}

   /* now reduce `c' back down to c = a^2R^2 * R^-1 == a^2R */
   if ((result = mp_montgomery_reduce(&c, &b, mp)) != MP_OKAY) \{
      printf("Error reducing.  \%s",
             mp_error_to_string(result));
      return EXIT_FAILURE;
   \}

   /* multiply a to get c = a^3R^2 */
   if ((result = mp_mul(&a, &c, &c)) != MP_OKAY) \{
      printf("Error reducing.  \%s",
             mp_error_to_string(result));
      return EXIT_FAILURE;
   \}

   /* now reduce `c' back down to c = a^3R^2 * R^-1 == a^3R */
   if ((result = mp_montgomery_reduce(&c, &b, mp)) != MP_OKAY) \{
      printf("Error reducing.  \%s",
             mp_error_to_string(result));
      return EXIT_FAILURE;
   \}

   /* now reduce (again) `c' back down to c = a^3R * R^-1 == a^3 */
   if ((result = mp_montgomery_reduce(&c, &b, mp)) != MP_OKAY) \{
      printf("Error reducing.  \%s",
             mp_error_to_string(result));
      return EXIT_FAILURE;
   \}

   /* c now equals a^3 mod b */

   return EXIT_SUCCESS;
\}
\end{alltt}

This particular example does not look too efficient but it demonstrates the point of the algorithm.  By
normalizing the inputs the reduced results are always of the form $aR$ for some variable $a$.  This allows
a single final reduction to correct for the normalization and the fast reduction used within the algorithm.

For more details consider examining the file \textit{bn\_mp\_exptmod\_fast.c}.

\section{Restricted Diminished Radix}

``Diminished Radix'' reduction refers to reduction with respect to moduli that are amenable to simple
digit shifting and small multiplications.  In this case the ``restricted'' variant refers to moduli of the
form $\beta^k - p$ for some $k \ge 0$ and $0 < p < \beta$ where $\beta$ is the radix (default to $2^{28}$).

As in the case of Montgomery reduction there is a pre--computation phase required for a given modulus.

\index{mp\_dr\_setup}
\begin{alltt}
void mp_dr_setup(mp_int *a, mp_digit *d);
\end{alltt}

This computes the value required for the modulus $a$ and stores it in $d$.  This function cannot fail
and does not return any error codes.  After the pre--computation a reduction can be performed with the
following.

\index{mp\_dr\_reduce}
\begin{alltt}
int mp_dr_reduce(mp_int *a, mp_int *b, mp_digit mp);
\end{alltt}

This reduces $a$ in place modulo $b$ with the pre--computed value $mp$.  $b$ must be of a restricted
diminished radix form and $a$ must be in the range $0 \le a < b^2$.  Diminished radix reductions are
much faster than both Barrett and Montgomery reductions as they have a much lower asymptotic running time.

Since the moduli are restricted this algorithm is not particularly useful for something like Rabin, RSA or
BBS cryptographic purposes.  This reduction algorithm is useful for Diffie-Hellman and ECC where fixed
primes are acceptable.

Note that unlike Montgomery reduction there is no normalization process.  The result of this function is
equal to the correct residue.

\section{Unrestricted Diminished Radix}

Unrestricted reductions work much like the restricted counterparts except in this case the moduli is of the
form $2^k - p$ for $0 < p < \beta$.  In this sense the unrestricted reductions are more flexible as they
can be applied to a wider range of numbers.

\index{mp\_reduce\_2k\_setup}
\begin{alltt}
int mp_reduce_2k_setup(mp_int *a, mp_digit *d);
\end{alltt}

This will compute the required $d$ value for the given moduli $a$.

\index{mp\_reduce\_2k}
\begin{alltt}
int mp_reduce_2k(mp_int *a, mp_int *n, mp_digit d);
\end{alltt}

This will reduce $a$ in place modulo $n$ with the pre--computed value $d$.  From my experience this routine is
slower than mp\_dr\_reduce but faster for most moduli sizes than the Montgomery reduction.

\section{Combined Modular Reduction}

Some of the combinations of an arithmetic operations followed by a modular reduction can be done in a faster way. The ones implemented are:

Addition $d = (a + b) \mod c$
\index{mp\_addmod}
\begin{alltt}
int mp_addmod(const mp_int *a, const mp_int *b, const mp_int *c, mp_int *d);
\end{alltt}

Subtraction  $d = (a - b) \mod c$
\begin{alltt}
int mp_submod(const mp_int *a, const mp_int *b, const mp_int *c, mp_int *d);
\end{alltt}

Multiplication $d = (ab) \mod c$
\begin{alltt}
int mp_mulmod(const mp_int *a, const mp_int *b, const mp_int *c, mp_int *d);
\end{alltt}

Squaring  $d = (a^2) \mod c$
\begin{alltt}
int mp_sqrmod(const mp_int *a, const mp_int *b, const mp_int *c, mp_int *d);
\end{alltt}



\chapter{Exponentiation}
\section{Single Digit Exponentiation}
\index{mp\_expt\_d\_ex}
\begin{alltt}
int mp_expt_d_ex (mp_int * a, mp_digit b, mp_int * c, int fast)
\end{alltt}
This function computes $c = a^b$.

With parameter \textit{fast} set to $0$ the old version of the algorithm is used,
when \textit{fast} is $1$, a faster but not statically timed version of the algorithm is used.

The old version uses a simple binary left-to-right algorithm.
It is faster than repeated multiplications by $a$ for all values of $b$ greater than three.

The new version uses a binary right-to-left algorithm.

The difference between the old and the new version is that the old version always
executes $DIGIT\_BIT$ iterations. The new algorithm executes only $n$ iterations
where $n$ is equal to the position of the highest bit that is set in $b$.

\index{mp\_expt\_d}
\begin{alltt}
int mp_expt_d (mp_int * a, mp_digit b, mp_int * c)
\end{alltt}
mp\_expt\_d(a, b, c) is a wrapper function to mp\_expt\_d\_ex(a, b, c, 0).

\index{mp\_expt}
\begin{alltt}
int mp_expt (mp_int * a, const mp_int *b, mp_int * c)
\end{alltt}
Same as \texttt{mp\_expt\_d(a, b, c)} except that \texttt{b} is a \texttt{mp\_int}, Useful for small \texttt{MP\_xBIT}.

\section{Modular Exponentiation}
\index{mp\_exptmod}
\begin{alltt}
int mp_exptmod (mp_int * G, mp_int * X, mp_int * P, mp_int * Y)
\end{alltt}
This computes $Y \equiv G^X \mbox{ (mod }P\mbox{)}$ using a variable width sliding window algorithm.  This function
will automatically detect the fastest modular reduction technique to use during the operation.  For negative values of
$X$ the operation is performed as $Y \equiv (G^{-1} \mbox{ mod }P)^{\vert X \vert} \mbox{ (mod }P\mbox{)}$ provided that
$gcd(G, P) = 1$.

This function is actually a shell around the two internal exponentiation functions.  This routine will automatically
detect when Barrett, Montgomery, Restricted and Unrestricted Diminished Radix based exponentiation can be used.  Generally
moduli of the a ``restricted diminished radix'' form lead to the fastest modular exponentiations.  Followed by Montgomery
and the other two algorithms.

\section{Modulus a Power of Two}
\index{mp\_mod\_2d}
\begin{alltt}
int mp_mod_2d(const mp_int *a, int b, mp_int *c)
\end{alltt}
It calculates $c = a \mod 2^b$.

\section{Root Finding}
\index{mp\_n\_root}
\begin{alltt}
int mp_n_root (mp_int * a, mp_digit b, mp_int * c)
\end{alltt}
This computes $c = a^{1/b}$ such that $c^b \le a$ and $(c+1)^b > a$.  The implementation of this function is not
ideal for values of $b$ greater than three.  It will work but become very slow.  So unless you are working with very small
numbers (less than 1000 bits) I'd avoid $b > 3$ situations.  Will return a positive root only for even roots and return
a root with the sign of the input for odd roots.  For example, performing $4^{1/2}$ will return $2$ whereas $(-8)^{1/3}$
will return $-2$.

This algorithm uses the ``Newton Approximation'' method and will converge on the correct root fairly quickly.  Since
the algorithm requires raising $a$ to the power of $b$ it is not ideal to attempt to find roots for large
values of $b$.  If particularly large roots are required then a factor method could be used instead.  For example,
$a^{1/16}$ is equivalent to $\left (a^{1/4} \right)^{1/4}$ or simply
$\left ( \left ( \left ( a^{1/2} \right )^{1/2} \right )^{1/2} \right )^{1/2}$


The square root  $c = a^{1/2}$ (with the same conditions $c^2 \le a$ and $(c+1)^2 > a$) is implemented with a faster algorithm.

\index{mp\_sqrt}
\begin{alltt}
int mp_sqrt (mp_int * a, mp_digit b, mp_int * c)
\end{alltt}


\chapter{Logarithm}
\section{Integer Logarithm}
A logarithm function for positive integer input \texttt{a, base} computing  $\floor{\log_bx}$ such that $(\ilog_bx)^b \le x$.
\index{mp\_ilogb}
\begin{alltt}
int mp_ilogb(mp_int *a, mp_digit base, mp_int *c)
\end{alltt}
\subsection{Example}
\begin{alltt}
#include <stdlib.h>
#include <stdio.h>
#include <errno.h>
/* Must be defined before including tommath.h */
#define LTM_USE_EXTRA_FUNCTIONS
#include <tommath.h>

int main(int argc, char **argv)
{
   mp_int x, output;
   mp_digit base;
   int e;

   if (argc != 3) {
      fprintf(stderr,"Usage %s base x\textbackslash{}n", argv[0]);
      exit(EXIT_FAILURE);
   }
   if ((e = mp_init_multi(&x, &output, NULL)) != MP_OKAY) {
      fprintf(stderr,"mp_init failed: \textbackslash{}"%s\textbackslash{}"\textbackslash{}n",
                     mp_error_to_string(e));
              exit(EXIT_FAILURE);
   }
   errno = 0;
#ifdef MP_64BIT
   base = (mp_digit)strtoull(argv[1], NULL, 10);
#else
   base = (mp_digit)strtoul(argv[1], NULL, 10);
#endif
   if ((errno == ERANGE) || (base > (base & MP_MASK))) {
      fprintf(stderr,"strtoul(l) failed: input out of range\textbackslash{}n");
      exit(EXIT_FAILURE);
   }
   if ((e = mp_read_radix(&x, argv[2], 10)) != MP_OKAY) {
      fprintf(stderr,"mp_read_radix failed: \textbackslash{}"%s\textbackslash{}"\textbackslash{}n",
                      mp_error_to_string(e));
      exit(EXIT_FAILURE);
   }

   if ((e = mp_ilogb(&x, base, &output)) != MP_OKAY) {
      fprintf(stderr,"mp_ilogb failed: \textbackslash{}"%s\textbackslash{}"\textbackslash{}n",
                      mp_error_to_string(e));
      exit(EXIT_FAILURE);
   }

   if ((e = mp_fwrite(&output, 10, stdout)) != MP_OKAY) {
      fprintf(stderr,"mp_fwrite failed: \textbackslash{}"%s\textbackslash{}"\textbackslash{}n",
                      mp_error_to_string(e));
      exit(EXIT_FAILURE);
   }
   putchar('\textbackslash{}n');

   mp_clear_multi(&x, &output, NULL);
   exit(EXIT_SUCCESS);
}
\end{alltt}

\chapter{Large Prime Numbers}
\section{Trial Division}
\index{mp\_prime\_is\_divisible}
\begin{alltt}
int mp_prime_is_divisible (mp_int * a, int *result)
\end{alltt}
This will attempt to evenly divide $a$ by a list of primes\footnote{Default is the first 256 primes.} and store the
outcome in ``result''.  That is if $result = 0$ then $a$ is not divisible by the primes, otherwise it is.  Note that
if the function does not return \textbf{MP\_OKAY} the value in ``result'' should be considered undefined\footnote{Currently
the default is to set it to zero first.}.

\section{Fermat Test}
\index{mp\_prime\_fermat}
\begin{alltt}
int mp_prime_fermat (mp_int * a, mp_int * b, int *result)
\end{alltt}
Performs a Fermat primality test to the base $b$.  That is it computes $b^a \mbox{ mod }a$ and tests whether the value is
equal to $b$ or not.  If the values are equal then $a$ is probably prime and $result$ is set to one.  Otherwise $result$
is set to zero.

\section{Miller-Rabin Test}
\index{mp\_prime\_miller\_rabin}
\begin{alltt}
int mp_prime_miller_rabin (mp_int * a, mp_int * b, int *result)
\end{alltt}
Performs a Miller-Rabin test to the base $b$ of $a$.  This test is much stronger than the Fermat test and is very hard to
fool (besides with Carmichael numbers).  If $a$ passes the test (therefore is probably prime) $result$ is set to one.
Otherwise $result$ is set to zero.

Note that is suggested that you use the Miller-Rabin test instead of the Fermat test since all of the failures of
Miller-Rabin are a subset of the failures of the Fermat test.

\subsection{Required Number of Tests}
Generally to ensure a number is very likely to be prime you have to perform the Miller-Rabin with at least a half-dozen
or so unique bases.  However, it has been proven that the probability of failure goes down as the size of the input goes up.
This is why a simple function has been provided to help out.

\index{mp\_prime\_rabin\_miller\_trials}
\begin{alltt}
int mp_prime_rabin_miller_trials(int size)
\end{alltt}
This returns the number of trials required for a $2^{-96}$ (or lower) probability of failure for a given ``size'' expressed
in bits.  This comes in handy specially since larger numbers are slower to test.  For example, a 512-bit number would
require ten tests whereas a 1024-bit number would only require four tests.

You should always still perform a trial division before a Miller-Rabin test though.

A small table, broke in two for typographical reasons, with the number of rounds of Miller-Rabin tests is shown below.
The first column is the number of bits $b$ in the prime $p = 2^b$, the numbers in the first row represent the
probability that the number that all of the Miller-Rabin tests deemed a pseudoprime is actually a composite. There is a deterministic test for numbers smaller than $2^{80}$.

\begin{table}[h]
\begin{center}
\begin{tabular}{c c c c c c c}
\textbf{bits} & $\mathbf{2^{-80}}$ & $\mathbf{2^{-96}}$ & $\mathbf{2^{-112}}$ & $\mathbf{2^{-128}}$ & $\mathbf{2^{-160}}$ & $\mathbf{2^{-192}}$ \\
80    & 31 & 39 & 47 & 55 & 71 & 87  \\
96    & 29 & 37 & 45 & 53 & 69 & 85  \\
128   & 24 & 32 & 40 & 48 & 64 & 80  \\
160   & 19 & 27 & 35 & 43 & 59 & 75  \\
192   & 15 & 21 & 29 & 37 & 53 & 69  \\
256   & 10 & 15 & 20 & 27 & 43 & 59  \\
384   & 7  & 9  & 12 & 16 & 25 & 38  \\
512   & 5  & 7  & 9  & 12 & 18 & 26  \\
768   & 4  & 5  & 6  & 8  & 11 & 16  \\
1024  & 3  & 4  & 5  & 6  & 9  & 12  \\
1536  & 2  & 3  & 3  & 4  & 6  & 8   \\
2048  & 2  & 2  & 3  & 3  & 4  & 6   \\
3072  & 1  & 2  & 2  & 2  & 3  & 4   \\
4096  & 1  & 1  & 2  & 2  & 2  & 3   \\
6144  & 1  & 1  & 1  & 1  & 2  & 2   \\
8192  & 1  & 1  & 1  & 1  & 2  & 2   \\
12288 & 1  & 1  & 1  & 1  & 1  & 1   \\
16384 & 1  & 1  & 1  & 1  & 1  & 1   \\
24576 & 1  & 1  & 1  & 1  & 1  & 1   \\
32768 & 1  & 1  & 1  & 1  & 1  & 1
\end{tabular}
\caption{ Number of Miller-Rabin rounds. Part I } \label{table:millerrabinrunsp1}
\end{center}
\end{table}
\newpage
\begin{table}[h]
\begin{center}
\begin{tabular}{c c c c c c c c}
\textbf{bits} &$\mathbf{2^{-224}}$ & $\mathbf{2^{-256}}$ & $\mathbf{2^{-288}}$ & $\mathbf{2^{-320}}$ & $\mathbf{2^{-352}}$ & $\mathbf{2^{-384}}$ & $\mathbf{2^{-416}}$\\
80    & 103 & 119 & 135 & 151 & 167 & 183 & 199 \\
96    & 101 & 117 & 133 & 149 & 165 & 181 & 197 \\
128   & 96  & 112 & 128 & 144 & 160 & 176 & 192 \\
160   & 91  & 107 & 123 & 139 & 155 & 171 & 187 \\
192   & 85  & 101 & 117 & 133 & 149 & 165 & 181 \\
256   & 75  & 91  & 107 & 123 & 139 & 155 & 171 \\
384   & 54  & 70  & 86  & 102 & 118 & 134 & 150 \\
512   & 36  & 49  & 65  & 81  & 97  & 113 & 129 \\
768   & 22  & 29  & 37  & 47  & 58  & 70  & 86  \\
1024  & 16  & 21  & 26  & 33  & 40  & 48  & 58  \\
1536  & 10  & 13  & 17  & 21  & 25  & 30  & 35  \\
2048  & 8   & 10  & 13  & 15  & 18  & 22  & 26  \\
3072  & 5   & 7   & 8	& 10  & 12  & 14  & 17  \\
4096  & 4   & 5   & 6	& 8   & 9   & 11  & 12  \\
6144  & 3   & 4   & 4	& 5   & 6   & 7   & 8	\\
8192  & 2   & 3   & 3	& 4   & 5   & 6   & 6	\\
12288 & 2   & 2   & 2	& 3   & 3   & 4   & 4	\\
16384 & 1   & 2   & 2	& 2   & 3   & 3   & 3	\\
24576 & 1   & 1   & 2	& 2   & 2   & 2   & 2	\\
32768 & 1   & 1   & 1	& 1   & 2   & 2   & 2
\end{tabular}
\caption{ Number of Miller-Rabin rounds. Part II } \label{table:millerrabinrunsp2}
\end{center}
\end{table}

Determining the probability needed to pick the right column is a bit harder. Fips 186.4, for example has $2^{-80}$ for $512$ bit large numbers, $2^{-112}$ for $1024$ bits, and $2^{128}$ for $1536$ bits. It can be seen in table \ref{table:millerrabinrunsp1} that those combinations follow the diagonal from $(512,2^{-80})$ downwards and to the right to gain a lower probability of getting a composite declared a pseudoprime for the same amount of work or less.

If this version of the library has the strong Lucas-Selfridge and/or the Frobenius-Underwood test implemented only one or two rounds of the Miller-Rabin test with a random base is necessary for numbers larger than or equal to $1024$ bits.


\section{Strong Lucas-Selfridge Test}
\index{mp\_prime\_strong\_lucas\_selfridge}
\begin{alltt}
int mp_prime_strong_lucas_selfridge(const mp_int *a, int *result)
\end{alltt}
Performs a strong Lucas-Selfridge test. The strong Lucas-Selfridge test together with the Rabin-Miler test with bases $2$ and $3$ resemble the BPSW test. The single internal use is a compile-time option in \texttt{mp\_prime\_is\_prime} and can be excluded
from the Libtommath build if not needed.

\section{Frobenius (Underwood)  Test}
\index{mp\_prime\_frobenius\_underwood}
\begin{alltt}
int mp_prime_frobenius_underwood(const mp_int *N, int *result)
\end{alltt}
Performs the variant of the Frobenius test as described by Paul Underwood. The single internal use is in
\texttt{mp\_prime\_is\_prime} for \texttt{MP\_8BIT} only but can be included at build-time for all other sizes
if the preprocessor macro \texttt{LTM\_USE\_FROBENIUS\_TEST} is defined.

It returns \texttt{MP\_ITER} if the number of iterations is exhausted, assumes a composite as the input and sets \texttt{result} accordingly. This will reduce the set of available pseudoprimes by a very small amount: test with large datasets (more than $10^{10}$ numbers, both randomly chosen and sequences of odd numbers with a random start point) found only 31 (thirty-one) numbers with $a > 120$ and none at all with just an additional simple check for divisors $d < 2^8$.

\section{Primality Testing}
Testing if a number is a square can be done a bit faster than just by calculating the square root. It is used by the primality testing function described below.
\index{mp\_is\_square}
\begin{alltt}
int mp_is_square(const mp_int *arg, int *ret);
\end{alltt}


\index{mp\_prime\_is\_prime}
\begin{alltt}
int mp_prime_is_prime (mp_int * a, int t, int *result)
\end{alltt}
This will perform a trial division followed by two rounds of Miller-Rabin with bases 2 and 3 and a Lucas-Selfridge test. The Lucas-Selfridge test is replaced with a Frobenius-Underwood for \texttt{MP\_8BIT}. The Frobenius-Underwood test for all other sizes is available as a compile-time option with the preprocessor macro \texttt{LTM\_USE\_FROBENIUS\_TEST}. See file
\texttt{bn\_mp\_prime\_is\_prime.c} for the necessary details. It shall be noted that both functions are much slower than
the Miller-Rabin test and if speed is an essential issue, the macro \texttt{LTM\_USE\_FIPS\_ONLY} switches both functions, the Frobenius-Underwood test and the Lucas-Selfridge test off and their code will not even be compiled into the library.

If $t$ is set to a positive value $t$ additional rounds of the Miller-Rabin test with random bases will be performed to allow for Fips 186.4 (vid.~p.~126ff) compliance. The function \texttt{mp\_prime\_rabin\_miller\_trials} can be used to determine the number of rounds. It is vital that the function \texttt{mp\_rand()} has a cryptographically strong random number generator available.

One Miller-Rabin tests with a random base will be run automatically, so by setting $t$ to a positive value this function will run $t + 1$ Miller-Rabin tests with random bases.

If  $t$ is set to a negative value the test will run the deterministic Miller-Rabin test for the primes up to
$3317044064679887385961981$. That limit has to be checked by the caller. If $-t > 13$ than $-t - 13$ additional rounds of the
Miller-Rabin test will be performed but note that $-t$ is bounded by $1 \le -t < PRIME\_SIZE$ where $PRIME\_SIZE$ is the number
of primes in the prime number table (by default this is $256$) and the first 13 primes have already been used. It will return
\texttt{MP\_VAL} in case of$-t > PRIME\_SIZE$.

If $a$ passes all of the tests $result$ is set to one, otherwise it is set to zero.

\subsection{Deterministic Prime Test}
The test above is a probabilistic test and can fail by calling a composite a prime. It is extremely rare, so rare that it is of almost no practical relevance. For those users who get nervous about the word ``almost'' a deterministic primes test has been put in the packet ``extra''. Add it to the library by typing \texttt{make extra}.
\index{mp\_prime\_is\_prime\_deterministic}
\begin{alltt}
int mp_prime_is_prime_deterministic(const mp_int *z, int *result);
\end{alltt}
It has two disadvantages: the theory behind assumes the general Riemann hypothesis to be true and, much more significant, is very slow for large primes.

\begin{table}[h]
\begin{center}
\begin{tabular}{c c c}
 $\mathbf{2^n}$ & \textbf{Testing Time} &  \textbf{Generating Time}\\
    128 &   0m00.061s  &  0m00.176s \\
    256 &   0m00.490s  &  0m00.788s \\
    512 &   0m06.233s  &  0m05.257s \\
    768 &   0m34.974s  &  0m02.916s \\
   1024 &   2m12.393s  &  0m45.018s \\
   1536 &  11m39.058s  &  2m25.890s \\
   2048 &  45m13.408s  &  2m57.433s
\end{tabular}
\caption{Benchmarking \texttt{mp\_prime\_is\_prime\_deterministic}} \label{table:benchmarkprimetestdet}
\end{center}
\end{table}

The machine for that benchmark was an AMD A8-6600K and the prime generator was LibTomMaths own
 \texttt{mp\_prime\_random\_ex(\&z, 8, size, LTM\_PRIME\_SAFE, myrng, NULL)}.

Despite its large runtime it was the only way to include a deterministic primality test with a small memory footprint, no need for floating point functions and one that works with low \texttt{MP\_xBIT}, too.

\section{Next Prime}
\index{mp\_prime\_next\_prime}
\begin{alltt}
int mp_prime_next_prime(mp_int *a, int t, int bbs_style)
\end{alltt}
This finds the next prime after $a$ that passes mp\_prime\_is\_prime() with $t$ tests but see the documentation for
mp\_prime\_is\_prime for details regarding the use of the argument $t$.  Set $bbs\_style$ to one if you
want only the next prime congruent to $3 \mbox{ mod } 4$, otherwise set it to zero to find any next prime.

\section{Random Primes}
\index{mp\_prime\_random}
\begin{alltt}
int mp_prime_random(mp_int *a, int t, int size, int bbs,
                    ltm_prime_callback cb, void *dat)
\end{alltt}
This will find a prime greater than $256^{size}$ which can be ``bbs\_style'' or not depending on $bbs$ and must pass
$t$ rounds of tests but see the documentation for mp\_prime\_is\_prime for details regarding the use of the argument $t$.
The ``ltm\_prime\_callback'' is a typedef for

\begin{alltt}
typedef int ltm_prime_callback(unsigned char *dst, int len, void *dat);
\end{alltt}

Which is a function that must read $len$ bytes (and return the amount stored) into $dst$.  The $dat$ variable is simply
copied from the original input.  It can be used to pass RNG context data to the callback.  The function
mp\_prime\_random() is more suitable for generating primes which must be secret (as in the case of RSA) since there
is no skew on the least significant bits.

\textit{Note:}  As of v0.30 of the LibTomMath library this function has been deprecated.  It is still available
but users are encouraged to use the new mp\_prime\_random\_ex() function instead.

\subsection{Extended Generation}
\index{mp\_prime\_random\_ex}
\begin{alltt}
int mp_prime_random_ex(mp_int *a,    int t,
                       int     size, int flags,
                       ltm_prime_callback cb, void *dat);
\end{alltt}
This will generate a prime in $a$ using $t$ tests of the primality testing algorithms.  The variable $size$
specifies the bit length of the prime desired.  The variable $flags$ specifies one of several options available
(see fig. \ref{fig:primeopts}) which can be OR'ed together.  The callback parameters are used as in
mp\_prime\_random().

\begin{figure}[h]
\begin{center}
\begin{small}
\begin{tabular}{|r|l|}
\hline \textbf{Flag}         & \textbf{Meaning} \\
\hline LTM\_PRIME\_BBS       & Make the prime congruent to $3$ modulo $4$ \\
\hline LTM\_PRIME\_SAFE      & Make a prime $p$ such that $(p - 1)/2$ is also prime. \\
                             & This option implies LTM\_PRIME\_BBS as well. \\
\hline LTM\_PRIME\_2MSB\_OFF & Makes sure that the bit adjacent to the most significant bit \\
                             & Is forced to zero.  \\
\hline LTM\_PRIME\_2MSB\_ON  & Makes sure that the bit adjacent to the most significant bit \\
                             & Is forced to one. \\
\hline
\end{tabular}
\end{small}
\end{center}
\caption{Primality Generation Options}
\label{fig:primeopts}
\end{figure}

\chapter{Random Number Generation}
\section{PRNG}
\index{mp\_rand\_digit}
\begin{alltt}
int mp_rand_digit(mp_digit *r)
\end{alltt}
This function generates a random number in \texttt{r} of the size given in \texttt{r} (that is, the variable is used for in- and output) but not more than \texttt{MP\_MASK} bits.

\index{mp\_rand}
\begin{alltt}
int mp_rand(mp_int *a, int digits)
\end{alltt}
This function generates a random number of \texttt{digits} bits.

The random number generated with these two functions is cryptographically secure if the source of random numbers the operating systems offers is cryptographically secure. It will use \texttt{arc4random()} if the OS is a BSD flavor, Wincrypt on Windows, or \texttt{/dev/urandom} on all operating systems that have it.

\chapter{Small Prime Numbers}

Small prime numbers are those primes that fit in a word of LibTomMath, that is they are smaller than or equal to the size of the type behind \texttt{mp\_word}. Examples at the end of this chapter at page \ref{sec:spnexamples},
\section{Prime Sieve}
A prime sieve is implemented as a simple segmented Sieve of Eratosthenes. It is only moderately optimized for space and runtime but should be small enough and also fast enough for almost all use-cases; quite quick for sequential access but relatively slow for random access.

 The prime sieve and its functions are part of the ``extra'' package and can be compiled in with \texttt{make extra}. The macro \texttt{LTM\_USE\_EXTRA\_FUNCTIONS} has to be set before \texttt{tommath.h} is included. Printing the small primes fneeds \texttt{inttypes.h} which must be included before \texttt{tommath.h}.
\subsection{Initialization and Clearing}
Initializing. It cannot fail because it only sets some default values. Memory is allocated later according to needs.
\index{mp\_sieve\_init}
\begin{alltt}
void mp_sieve_init(mp_sieve *sieve);
\end{alltt}
The function \texttt{mp\_sieve\_init} is equivalent to
\begin{alltt}
mp_sieve sieve = {NULL, NULL, 0};
\end{alltt}

Free the memory used by the sieve with
\index{mp\_sieve\_clear}
\begin{alltt}
void mp_sieve_clear(mp_sieve *sieve);
\end{alltt}
\subsection{Primality Test of Small Numbers}
Individual small numbers can be tested for primality with
\index{mp\_is\_small\_prime}
\begin{alltt}
int mp_is_small_prime(LTM_SIEVE_UINT n, LTM_SIEVE_UINT *result,
                      mp_sieve *sieve);
\end{alltt}
The implementation of this function also does all of the heavy lifting, the building of the base sieve and the segment if one is necessary. The base sieve stays, so this function can be used to ``warm up'' the sieve and, if \texttt{n} is slightly larger than the upper limit of the base sieve, ``warm up'' the first segment, too. It will return \texttt{LTM\_SIEVE\_MAX\_REACHED} to flag the content of \texttt{result} as the last valid one.
\subsection{Find Adjacent Primes}
To find the prime bigger than a number \texttt{n} use
\index{mp\_next\_small\_prime}
\begin{alltt}
int mp_next_small_prime(LTM_SIEVE_UINT n, LTM_SIEVE_UINT *result,
                        mp_sieve *sieve);
\end{alltt}
and to find the one smaller than \texttt{n}
\begin{alltt}
int mp_prec_small_prime(LTM_SIEVE_UINT n, LTM_SIEVE_UINT *result,
                        mp_sieve *sieve);
\end{alltt}
\subsection{Prime Sequence}
The most common use of the small primes is in the form of a continuous sequence. To produce this sequence utilize
\index{mp\_small\_prime\_array}. The array \texttt{prime\_array} is allocated with \texttt{malloc} internally and needs to be free'd after use.
\begin{alltt}
int mp_small_prime_array(LTM_SIEVE_UINT start, LTM_SIEVE_UINT end,
                         mp_factors *factors,
                         mp_sieve *sieve);
\end{alltt}

\subsection{Useful Constants}
\begin{description}
\item[\texttt{LTM\_SIEVE\_BIGGEST\_PRIME}] \texttt{read-only} The biggest prime the sieve can offer. It is be $65\,521$ for \texttt{MP\_8BIT},
 $4\,294\,967\,291$ for \texttt{MP\_16BIT}, \texttt{MP\_32BIT} and \texttt{MP\_64BIT}; and
 $18\,446\,744\,073\,709\,551\,557$ for \texttt{MP\_64BIT} if the macro\\
 \texttt{LTM\_SIEVE\_USE\_LARGE\_SIEVE} is defined.

\item[\texttt{LTM\_SIEVE\_UINT}] \texttt{read-only}  The basic type for the numbers in the sieve. It is be \texttt{uint16\_t} for \texttt{MP\_8BIT}, \texttt{uint32\_t} for \texttt{MP\_16BIT}, \texttt{MP\_32BIT} and \texttt{MP\_64BIT}; and \texttt{uint64\_t} for \texttt{MP\_64BIT} if the macro \texttt{LTM\_SIEVE\_USE\_LARGE\_SIEVE} is defined.

\item[\texttt{LTM\_SIEVE\_UINT\_MAX}] \texttt{read-only} The maximum value of the type for the numbers in the sieve. It is \texttt{UINT16\_MAX} for \texttt{MP\_8BIT}, \texttt{UINT32\_MAX} for \texttt{MP\_16BIT}, \texttt{MP\_32BIT} and \texttt{MP\_64BIT}; and \texttt{UINT\_64MAX} for \texttt{MP\_64BIT} if the macro\\
\texttt{LTM\_SIEVE\_USE\_LARGE\_SIEVE} is defined.

\item[\texttt{LTM\_SIEVE\_UINT\_MAX\_SQRT}] \texttt{read-only} The square root of the maximum value of the type for the numbers in the sieve. It is \texttt{UINT8\_MAX} for \texttt{MP\_8BIT}, \texttt{UINT16\_MAX} for \texttt{MP\_16BIT}, \texttt{MP\_32BIT} and \texttt{MP\_64BIT}; and\texttt{UINT32\_MAX} for \texttt{MP\_64BIT} if the macro \texttt{LTM\_SIEVE\_USE\_LARGE\_SIEVE} is defined.

\item[\texttt{LTM\_SIEVE\_USE\_LARGE\_SIEVE}] \texttt{read-only} A flag to make a large sieve.  No advantage has been seen in using 64-bit integers if available except the ability to get a sieve up to $2^64$. But in this case the base sieve gets 0.25 Gibibytes large and the segments 0.5 Gibibytes (although you can change \texttt{LTM\_SIEVE\_RANGE\_A\_B} to get smaller segments) and need a long time to fill.

\item[\texttt{LTM\_SIEVE\_RANGE\_A\_B}] \texttt{read-write} The size of the sieve for the segment. It is set to \texttt{LTM\_SIEVE\_UINT\_MAX\_SQRT} per default but it can be changed. The default size is already small enough to fit into most CPU's L-2 caches but if \texttt{LTM\_SIEVE\_USE\_LARGE\_SIEVE} is defined the segment sieve grows quite large and setting \texttt{LTM\_SIEVE\_RANGE\_A\_B} to the size of the CPU's L-2 caches will show a significant advantage regarding the runtime, it more than doubles it. Because of that large penalty the default value is set to \texttt{0x400000uL} if both \texttt{MP\_64BIT} and \texttt{LTM\_SIEVE\_USE\_LARGE\_SIEVE} are defined. Needs to be set at the compile time of LibTomMath.

\item[\texttt{LTM\_SIEVE\_UINT\_NUM\_BITS}] \texttt{read-only} The number of bits in the type for the numbers in the sieve.\\
It is a shortcut for \verb!CHAR\_BIT * sizeof(LTM\_SIEVE\_UINT)!.

\item[\texttt{LTM\_SIEVE\_SIZE(bst)}] \texttt{function macro, read-only} Returns the entry \texttt{size} of \texttt{struct mp\_sieve}. It is a shortcut for \verb!sieve->size!.

\item[\texttt{LTM\_SIEVE\_PR\_UINT}] Choses the correct macro from \texttt{inttypes.h} to print a\\
 \texttt{LTM\_SIEVE\_UINT}. The header \texttt{inttypes.h} must be included before\\
 \texttt{tommath.h} for it to work.
\end{description}


\subsection{Examples}\label{sec:spnexamples}
\subsubsection{Initialization and Clearing}
Using a sieve follows the same procedure as using a bigint:
\begin{alltt}
/* Declaration */
mp_sieve sieve;

/*
   Initialization.
   Cannot fail, only sets a handful of default values.
   Memory allocation is done in the library itself
   according to needs.
 */
mp_sieve_init(&sieve);

/* use the sieve */

/* Clean up */
mp_sieve_clear(&sieve);
\end{alltt}
\subsubsection{Primality Test}
A small program to test the input of a small number for primality.
\begin{alltt}
#include <stdlib.h>
#include <stdio.h>
#include <errno.h>
/*inttypes.h must be included before tommath.h*/
#include <inttypes.h>
/* Must be defined before tommath.h is included */
#define LTM_USE_EXTRA_FUNCTIONS
#include "tommath.h"
int main(int argc, char **argv)
{
   LTM_SIEVE_UINT number;
   mp_sieve *base = NULL;
   mp_sieve *segment = NULL;
   LTM_SIEVE_UINT single_segment_a = 0;
   int e;

   /* variable holding the result of mp_is_small_prime */
   LTM_SIEVE_UINT result;

   if (argc != 2) {
      fprintf(stderr,"Usage %s number\textbackslash{}n", argv[0]);
      exit(EXIT_FAILURE);
   }

   number = (LTM_SIEVE_UINT) strtoul(argv[1],NULL, 10);
   if (errno == ERANGE) {
      fprintf(stderr,"strtoul(l) failed: input out of range\textbackslash{}n");
      goto LTM_ERR;
   }

   mp_sieve_init(&sieve);

   if ((e = mp_is_small_prime(number, &result, &sieve)) != MP_OKAY) {
      fprintf(stderr,"mp_is_small_number failed: \textbackslash{}"%s\textbackslash{}"\textbackslash{}n",
              mp_error_to_string(e));
      goto LTM_ERR;
   }

   printf("The number %" LTM_SIEVE_PR_UINT " is %s prime\textbackslash{}n",
           number,(result)?"":"not");


   mp_sieve_clear(&sieve);
   exit(EXIT_SUCCESS);
LTM_ERR:
   mp_sieve_clear(&sieve);
   exit(EXIT_FAILURE);
}
\end{alltt}
\subsubsection{Find Adjacent Primes}
To sum up all primes up to and including \texttt{LTM\_SIEVE\_BIGGEST\_PRIME} you might do something like:
\begin{alltt}
#include <stdlib.h>
#include <stdio.h>
#include <errno.h>
/* Must be defined before tommath.h is included */
#define LTM_USE_EXTRA_FUNCTIONS
#include <tommath.h>
int main(int argc, char **argv)
{
   LTM_SIEVE_UINT number;
   mp_sieve sieve;
   LTM_SIEVE_UINT k, ret;
   mp_int total, t;
   int e;

   if (argc != 2) {
      fprintf(stderr,"Usage %s integer\textbackslash{}n", argv[0]);
      exit(EXIT_FAILURE);
   }

   if ((e = mp_init_multi(&total, &t, NULL)) != MP_OKAY) {
      fprintf(stderr,"mp_init_multi(segment): \textbackslash{}"%s\textbackslash{}"\textbackslash{}n",
              mp_error_to_string(e));
      goto LTM_ERR_1;
   }
   errno = 0;
#if ( (defined MP_64BIT) && (defined LTM_SIEVE_USE_LARGE_SIEVE) )
   number = (LTM_SIEVE_UINT) strtoull(argv[1],NULL, 10);
#else
   number = (LTM_SIEVE_UINT) strtoul(argv[1],NULL, 10);
#endif
   if (errno == ERANGE) {
      fprintf(stderr,"strtoul(l) failed: input out of range\textbackslash{}n");
      return EXIT_FAILURE
   }

   mp_sieve_init(&sieve);

   for (k = 0, ret = 0; ret < number; k = ret) {
      if ((e = mp_next_small_prime(k + 1, &ret, &sieve)) != MP_OKAY) {
         if (e == LTM_SIEVE_MAX_REACHED) {
#ifdef MP_64BIT
            if ((e = mp_add_d(&total, (mp_digit) k, &total)) != MP_OKAY) {
               fprintf(stderr,"mp_add_d (1) failed: \textbackslash{}"%s\textbackslash{}"\textbackslash{}n",
                       mp_error_to_string(e));
               goto LTM_ERR;
            }
#else
            if ((e = mp_set_long(&t, k)) != MP_OKAY) {
               fprintf(stderr,"mp_set_long (1) failed: \textbackslash{}"%s\textbackslash{}"\textbackslash{}n",
                       mp_error_to_string(e));
               goto LTM_ERR;
            }
            if ((e = mp_add(&total, &t, &total)) != MP_OKAY) {
               fprintf(stderr,"mp_add (1) failed: \textbackslash{}"%s\textbackslash{}"\textbackslash{}n",
                       mp_error_to_string(e));
               goto LTM_ERR;
            }
#endif
            break;
         }
         fprintf(stderr,"mp_next_small_prime failed: \textbackslash{}"%s\textbackslash{}"\textbackslash{}n",
                 mp_error_to_string(e));
         goto LTM_ERR;
      }
      /* The check if the prime is below the given limit
       * cannot be done in the for-loop conditions because if
       * it could we wouldn't need the sieve in the first place.
       */
      if (ret <= number) {
#ifdef MP_64BIT
         if ((e = mp_add_d(&total, (mp_digit) k, &total)) != MP_OKAY) {
            fprintf(stderr,"mp_add_d failed: \textbackslash{}"%s\textbackslash{}"\textbackslash{}n",
                    mp_error_to_string(e));
            goto LTM_ERR;
         }
#else
         if ((e = mp_set_long(&t, k)) != MP_OKAY) {
            fprintf(stderr,"mp_set_long failed: \textbackslash{}"%s\textbackslash{}"\textbackslash{}n",
                    mp_error_to_string(e));
            goto LTM_ERR;
         }
         if ((e = mp_add(&total, &t, &total)) != MP_OKAY) {
            fprintf(stderr,"mp_add failed: \textbackslash{}"%s\textbackslash{}"\textbackslash{}n",
            mp_error_to_string(e));
            goto LTM_ERR;
         }
#endif
      }
   }
   printf("total: ");
   mp_fwrite(&total,10,stdout);
   putchar('\textbackslash{}n');

   mp_clear_multi(&total, &t, NULL);
   mp_sieve_clear(&sieve);
   exit(EXIT_SUCCESS);
LTM_ERR:
   mp_clear_multi(&total, &t, NULL);
   mp_sieve_clear(&sieve);
   exit(EXIT_FAILURE);
}
\end{alltt}
It took about a minute on the authors machine from 2015 to get the expected $425\,649\,736\,193\,687\,430$ for the sum of all primes up to $2^{32}$, about the same runtime as Pari/GP version 2.9.4 (with a GMP-5.1.3 kernel).

\subsubsection{Prime Sequence}
A short sequence of primes can be produced with:
\begin{alltt}
#include <stdlib.h>
#include <stdio.h>
#include <errno.h>
/* Must be defined before tommath.h is included */
#define LTM_USE_EXTRA_FUNCTIONS
#include <tommath.h>
int main(int argc, char **argv)
{
   LTM_SIEVE_UINT a, b;
   mp_factors factors;
   int e;

   if (argc != 3) {
      fprintf(stderr,"Usage %s start stop\textbackslash{}n", argv[0]);
      exit(EXIT_FAILURE);
   }

   errno = 0;
#if ( (defined MP_64BIT) && (defined LTM_SIEVE_USE_LARGE_SIEVE) )
   a = (LTM_SIEVE_UINT) strtoull(argv[1], NULL, 10);
   if (errno == ERANGE) {
      fprintf(stderr,"strtoull(start) failed: input out of range\textbackslash{}n");
      goto LTM_ERR;
   }
   errno = 0;
   b = (LTM_SIEVE_UINT) strtoull(argv[2], NULL, 10);
   if (errno == ERANGE) {
      fprintf(stderr,"strtoull(end) failed: input out of range\textbackslash{}n");
      goto LTM_ERR;
   }
#else
   a = (LTM_SIEVE_UINT) strtoul(argv[1], NULL, 10);
   if (errno == ERANGE) {
      fprintf(stderr,"strtoul(start) failed: input out of range\textbackslash{}n");
      goto LTM_ERR;
   }
   errno = 0;
   b = (LTM_SIEVE_UINT) strtoul(argv[2], NULL, 10);
   if (errno == ERANGE) {
      fprintf(stderr,"strtoul(end) failed: input out of range\textbackslash{}n");
      goto LTM_ERR;
   }
#endif

   if ((e = mp_small_prime_array(a, b, &factors)) != MP_OKAY) {
      fprintf(stderr,"mp_small_prime_array failed: \textbackslash{}"%s\textbackslash{}"\textbackslash{}n",
              mp_error_to_string(e));
      goto LTM_ERR;
   }
   
   mp_factors_print(&factors, 10, 0, stdout);

   mp_factors_clear(&factors);
   exit(EXIT_SUCCESS);
LTM_ERR:
   mp_factors_clear(&factors);
   exit(EXIT_FAILURE);
}
\end{alltt}
The array \texttt{prime\_array} will be of size $\pi(b) - \pi(a)$ times \verb!sizeof(LTM_SIEVE_UINT)! which can get quite large quite quickly\footnote{There are $203\,280\,221$ primes smaller than $2^{32}$.}. You might find the method involving the function \texttt{mp\_next\_small\_prime} more applicable for larger sequences.

%\subsubsection{Using the Useful Constants}

\section{Factorizing}
All of the functions described in this section are in the packet ``extra''. Add it to the library by typing \texttt{make extra}.

The decomposition of numbers into their prime-factors is covered by the function
\index{mp\_factor}
\begin{alltt}
int mp_factor(const mp_int *z, mp_factors *factors);
\end{alltt}
It will decompose the factors of the integer $z > 0$ into its prime factors\footnote{The methods used in this algorithm are reasonably fast but definitely not the fastest. A practical limit is at about 30-35 bit large factors, 40 bit with some patience.}, checks the result by multiplying the found factors and comparing them with the input, and sample them as numbers of the type \texttt{mp\_int} in the list \texttt{factors}. The structure of this list is described by
\index{mp\_factors}
\begin{alltt}
typedef struct {
   int length, alloc;
   mp_int *factors;
} mp_factors;
\end{alltt}
There are a handful of functions to help with the management of that list.

\begin{description}
\item
\index{mp\_factors\_init}
\verb!int mp_factors_init(mp_factors *f);!\\
Initialize the factor list by allocating a certain amount of memory and setting \texttt{length = 0} and \texttt{alloc} to the amount of memory pre-allocated. The exact amount is defined at compile time by the macro \texttt{LTM\_TRIAL\_GROWTH} in \texttt{tommath.h}.
\item
\index{mp\_factors\_clear}
\verb!void mp_factors_clear(mp_factors *f);!\\
This function free's all memory used.
\item
\index{mp\_factors\_zero}
\verb!int mp_factors_zero(mp_factors *f);!\\
Remove the elements of the factor list and allocate (fresh) memory of default size in that order.
\item
\index{mp\_factors\_add}
\verb!int mp_factors_add(const mp_int *a, mp_factors *f);!\\
Add a factor of type \texttt{mp\_int} to the list.
\item
\index{mp\_factors\_sort}
\verb!int mp_factors_sort(mp_factors *f);!\\
The factors in the list are not necessarily in increasing order. This functions changes that. It does it with the insert-sort algorithm, a good choice for the task it has been written for (the list is most likely already ordered) but not for many other tasks involving large lists in random order.
\item
\index{mp\_factors\_print}
\texttt{int mp\_factors\_print(mp\_factors *f, int base, char delimiter,\\
\hphantom{int mp\_factors\_print(} FILE *stream);}\\
Prints the element of the list in base \texttt{base} to \texttt{stream} with the delimiter \texttt{delimiter}. The default delimiter is a comma (ASCII \texttt{0x2c}).
\item
\index{mp\_factors\_product}
\verb!int mp_factors_product(mp_factors *factors, mp_int *p);!\\
Multiplies all elements of the list. It does not recognize sparse lists, every zero in the list gets multiplied, too. It does multiply the list with a binary-splitting algorithm which assumes a highly or better fully sorted list to work optimally.
\end{description}

The function \texttt{mp\_factor} uses two different factorization algorithms. The first one does just trial division with the small primes generated by a sieve and is
\index{mp\_trial}
\begin{alltt}
int mp_trial(const mp_int *a, int limit, 
             mp_factors *factors, mp_int *r);
\end{alltt}
It tries all small primes up to the limit \texttt{limit}, puts all factors it finds in the list \texttt{factors} and the remainder in \texttt{r},

The other function is the Pollard-Rho algorithm.
\index{mp\_pollard\_rho}
\begin{alltt}
int mp_pollard_rho(const mp_int *n, mp_int *factor);
\end{alltt}
It is used to compute all factors left over by the function \texttt{mp\_factor}. Both functions \texttt{mp\_trial} and \texttt{mp\_pollard\_rho} are not meant to be used as a standalone function, please consult the source of the respective functions for the necessary information.


The function \texttt{mp\_factors\_product} does already most of the work so it was not much left to do to implement a function to compute a primorial.
\index{mp\_primorial}
\begin{alltt}
int mp_primorial(const LTM_SIEVE_UINT n, mp_int *p);
\end{alltt}

\subsection{Test for (perfect) Powers}
\index{mp\_ispower}
\begin{alltt}
int mp_ispower(const mp_int *z, int *result, mp_int *rootout,
                 mp_int *exponent);
\end{alltt}
Tests if $z$ is a power, that is $z = a^b$ with $b$ prime but $a$ might be composite. Computes the actual roots to do so and in case of success sets \texttt{result} to \texttt{MP\_YES}, puts $a$ in \texttt{rootout} and the exponent $b$ in \texttt{exponent} or $0$ (zero) in both and \texttt{result} to \texttt{MP\_NO} if case of a failure to find one.

\index{mp\_isperfpower}
\begin{alltt}
int mp_isperfpower(const mp_int *z, int *result, mp_int *rootout,
                   mp_int *exponent)
\end{alltt}
Same as \texttt{mp\_ispower} but searches for perfect or prime powers, that is $z = a^b$ such that $a, b$ are prime. Uses \texttt{mp\_prime\_is\_prime} which is a probabilistic prime tester and only roots below $2^{64}$ are save.
\chapter{Input and Output}
\section{ASCII Conversions}
\subsection{To ASCII}
\index{mp\_toradix}
\begin{alltt}
int mp_toradix (mp_int * a, char *str, int radix);
\end{alltt}
This still store $a$ in ``str'' as a base-``radix'' string of ASCII chars.  This function appends a NUL character
to terminate the string.  Valid values of ``radix'' line in the range $[2, 64]$.  To determine the size (exact) required
by the conversion before storing any data use the following function.

\index{mp\_toradix\_n}
\begin{alltt}
int mp_toradix_n (mp_int * a, char *str, int radix, int maxlen);
\end{alltt}

Like \texttt{mp\_toradix} but stores up to maxlen-1 chars and always a NULL byte.

\index{mp\_radix\_size}
\begin{alltt}
int mp_radix_size (mp_int * a, int radix, int *size)
\end{alltt}
This stores in ``size'' the number of characters (including space for the NUL terminator) required.  Upon error this
function returns an error code and ``size'' will be zero.

If \texttt{LTM\_NO\_FILE} is not defined a function to write to a file is also available.
\index{mp\_fwrite}
\begin{alltt}
int mp_fwrite(const mp_int *a, int radix, FILE *stream);
\end{alltt}


\subsection{From ASCII}
\index{mp\_read\_radix}
\begin{alltt}
int mp_read_radix (mp_int * a, char *str, int radix);
\end{alltt}
This will read the base-``radix'' NUL terminated string from ``str'' into $a$.  It will stop reading when it reads a
character it does not recognize (which happens to include th NUL char... imagine that...).  A single leading $-$ sign
can be used to denote a negative number.

If \texttt{LTM\_NO\_FILE} is not defined a function to read from a file is also available.
\index{mp\_fread}
\begin{alltt}
int mp_fread(mp_int *a, int radix, FILE *stream);
\end{alltt}


\section{Binary Conversions}

Converting an mp\_int to and from binary is another keen idea.

\index{mp\_unsigned\_bin\_size}
\begin{alltt}
int mp_unsigned_bin_size(mp_int *a);
\end{alltt}

This will return the number of bytes (octets) required to store the unsigned copy of the integer $a$.

\index{mp\_to\_unsigned\_bin}
\begin{alltt}
int mp_to_unsigned_bin(mp_int *a, unsigned char *b);
\end{alltt}
This will store $a$ into the buffer $b$ in big--endian format.  Fortunately this is exactly what DER (or is it ASN?)
requires.  It does not store the sign of the integer.

\index{mp\_to\_unsigned\_bin\_n}
\begin{alltt}
int mp_to_unsigned_bin_n(const mp_int *a, unsigned char *b, unsigned long *outlen)
\end{alltt}
Like \texttt{mp\_to\_unsigned\_bin} but checks if the value at \texttt{*outlen} is larger than or equal to the output of \texttt{mp\_unsigned\_bin\_size(a)} and sets \texttt{*outlen} to the output of \texttt{mp\_unsigned\_bin\_size(a)} or returns \texttt{MP\_VAL} if the test failed.


\index{mp\_read\_unsigned\_bin}
\begin{alltt}
int mp_read_unsigned_bin(mp_int *a, unsigned char *b, int c);
\end{alltt}
This will read in an unsigned big--endian array of bytes (octets) from $b$ of length $c$ into $a$.  The resulting
integer $a$ will always be positive.

For those who acknowledge the existence of negative numbers (heretic!) there are ``signed'' versions of the
previous functions.
\index{mp\_signed\_bin\_size} \index{mp\_to\_signed\_bin} \index{mp\_read\_signed\_bin}
\begin{alltt}
int mp_signed_bin_size(mp_int *a);
int mp_read_signed_bin(mp_int *a, unsigned char *b, int c);
int mp_to_signed_bin(mp_int *a, unsigned char *b);
\end{alltt}
They operate essentially the same as the unsigned copies except they prefix the data with zero or non--zero
byte depending on the sign.  If the sign is zpos (e.g. not negative) the prefix is zero, otherwise the prefix
is non--zero.

The two functions \texttt{mp\_import} and \texttt{mp\_export} implement the corresponding GMP functions as described at \url{http://gmplib.org/manual/Integer-Import-and-Export.html}.
\index{mp\_import} \index{mp\_export}
\begin{alltt}
int mp_import(mp_int *rop, size_t count, int order, size_t size, int endian, size_t nails, const void *op);
int mp_export(void *rop, size_t *countp, int order, size_t size, int endian, size_t nails, const mp_int *op);
\end{alltt}

\chapter{Algebraic Functions}
\section{Extended Euclidean Algorithm}
\index{mp\_exteuclid}
\begin{alltt}
int mp_exteuclid(mp_int *a, mp_int *b,
                 mp_int *U1, mp_int *U2, mp_int *U3);
\end{alltt}

This finds the triple U1/U2/U3 using the Extended Euclidean algorithm such that the following equation holds.

\begin{equation}
a \cdot U1 + b \cdot U2 = U3
\end{equation}

Any of the U1/U2/U3 parameters can be set to \textbf{NULL} if they are not desired.

\section{Greatest Common Divisor}
\index{mp\_gcd}
\begin{alltt}
int mp_gcd (mp_int * a, mp_int * b, mp_int * c)
\end{alltt}
This will compute the greatest common divisor of $a$ and $b$ and store it in $c$.

\section{Least Common Multiple}
\index{mp\_lcm}
\begin{alltt}
int mp_lcm (mp_int * a, mp_int * b, mp_int * c)
\end{alltt}
This will compute the least common multiple of $a$ and $b$ and store it in $c$.

\section{Jacobi Symbol}
\index{mp\_jacobi}
\begin{alltt}
int mp_jacobi (mp_int * a, mp_int * p, int *c)
\end{alltt}
This will compute the Jacobi symbol for $a$ with respect to $p$.  If $p$ is prime this essentially computes the Legendre
symbol.  The result is stored in $c$ and can take on one of three values $\lbrace -1, 0, 1 \rbrace$.  If $p$ is prime
then the result will be $-1$ when $a$ is not a quadratic residue modulo $p$.  The result will be $0$ if $a$ divides $p$
and the result will be $1$ if $a$ is a quadratic residue modulo $p$.

\section{Kronecker Symbol}
\index{mp\_kronecker}
\begin{alltt}
int mp_kronecker (mp_int * a, mp_int * p, int *c)
\end{alltt}
Extension of the Jacoby symbol to all $\lbrace a, p \rbrace \in \mathbb{Z}$ .


\section{Modular square root}
\index{mp\_sqrtmod\_prime}
\begin{alltt}
int mp_sqrtmod_prime(mp_int *n, mp_int *p, mp_int *r)
\end{alltt}

This will solve the modular equation $r^2 = n \mod p$ where $p$ is a prime number greater than 2 (odd prime).
The result is returned in the third argument $r$, the function returns \textbf{MP\_OKAY} on success,
other return values indicate failure.

The implementation is split for two different cases:

1. if $p \mod 4 == 3$ we apply \href{http://cacr.uwaterloo.ca/hac/}{Handbook of Applied Cryptography algorithm 3.36} and compute $r$ directly as
$r = n^{(p+1)/4} \mod p$

2. otherwise we use \href{https://en.wikipedia.org/wiki/Tonelli-Shanks_algorithm}{Tonelli-Shanks algorithm}

The function does not check the primality of parameter $p$ thus it is up to the caller to assure that this parameter
is a prime number. When $p$ is a composite the function behaviour is undefined, it may even return a false-positive
\textbf{MP\_OKAY}.

\section{Modular Inverse}
\index{mp\_invmod}
\begin{alltt}
int mp_invmod (mp_int * a, mp_int * b, mp_int * c)
\end{alltt}
Computes the multiplicative inverse of $a$ modulo $b$ and stores the result in $c$ such that $ac \equiv 1 \mbox{ (mod }b\mbox{)}$.

\section{Single Digit Functions}

For those using small numbers (\textit{snicker snicker}) there are several ``helper'' functions

\index{mp\_add\_d} \index{mp\_sub\_d} \index{mp\_mul\_d} \index{mp\_div\_d} \index{mp\_mod\_d}
\begin{alltt}
int mp_add_d(mp_int *a, mp_digit b, mp_int *c);
int mp_sub_d(mp_int *a, mp_digit b, mp_int *c);
int mp_mul_d(mp_int *a, mp_digit b, mp_int *c);
int mp_div_d(mp_int *a, mp_digit b, mp_int *c, mp_digit *d);
int mp_mod_d(mp_int *a, mp_digit b, mp_digit *c);
\end{alltt}

These work like the full mp\_int capable variants except the second parameter $b$ is a mp\_digit.  These
functions fairly handy if you have to work with relatively small numbers since you will not have to allocate
an entire mp\_int to store a number like $1$ or $2$.


The division by three can be made faster by replacing the division with a multiplication by the multiplicative inverse of three.

\index{mp\_div\_3}
\begin{alltt}
int mp_div_3(const mp_int *a, mp_int *c, mp_digit *d);
\end{alltt}

\chapter{Little Helpers}
It is never wrong to have some useful little shortcuts at hand.
\section{Function Macros}
To make this overview simpler the macros are given as function prototypes. The return of logic macros is \texttt{MP\_NO} or \texttt{MP\_YES} respectively.

\index{mp\_iseven}
\begin{alltt}
int mp_iseven(mp_int *a)
\end{alltt}
Checks if $a = 0 mod 2$

\index{mp\_isodd}
\begin{alltt}
int mp_isodd(mp_int *a)
\end{alltt}
Checks if $a = 1 mod 2$

\index{mp\_isneg}
\begin{alltt}
int mp_isneg(mp_int *a)
\end{alltt}
Checks if $a < 0$


\index{mp\_iszero}
\begin{alltt}
int mp_iszero(mp_int *a)
\end{alltt}
Checks if $a = 0$. It does not check if the amount of memory allocated for $a$ is also minimal.


Other macros which are either shortcuts to normal functions or just other names for them do have their place in a programmer's life, too!

\subsection{Renamings}
\index{mp\_mag\_size}
\begin{alltt}
#define mp_mag_size(mp) mp_unsigned_bin_size(mp)
\end{alltt}


\index{mp\_raw\_size}
\begin{alltt}
#define mp_raw_size(mp) mp_signed_bin_size(mp)
\end{alltt}


\index{mp\_read\_mag}
\begin{alltt}
#define mp_read_mag(mp, str, len) mp_read_unsigned_bin((mp), (str), (len))
\end{alltt}


\index{mp\_read\_raw}
\begin{alltt}
 #define mp_read_raw(mp, str, len) mp_read_signed_bin((mp), (str), (len))
\end{alltt}


\index{mp\_tomag}
\begin{alltt}
#define mp_tomag(mp, str) mp_to_unsigned_bin((mp), (str))
\end{alltt}


\index{mp\_toraw}
\begin{alltt}
#define mp_toraw(mp, str)         mp_to_signed_bin((mp), (str))
\end{alltt}



\subsection{Shortcuts}

\index{mp\_tobinary}
\begin{alltt}
#define mp_tobinary(M, S) mp_toradix((M), (S), 2)
\end{alltt}


\index{mp\_tooctal}
\begin{alltt}
#define mp_tooctal(M, S) mp_toradix((M), (S), 8)
\end{alltt}


\index{mp\_todecimal}
\begin{alltt}
#define mp_todecimal(M, S) mp_toradix((M), (S), 10)
\end{alltt}


\index{mp\_tohex}
\begin{alltt}
#define mp_tohex(M, S)     mp_toradix((M), (S), 16)
\end{alltt}


\input{bn.ind}

\end{document}


\end{document}


\end{document}


\end{document}
