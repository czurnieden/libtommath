\documentclass[a4paper,10pt]{article}

% This bunch of macros grew to that size in a couple of years,
% some parts are even decades old, so be aware of the risks of diving in!

\usepackage{palatino}
%\usepackage{lmodern}

% Narrow the margins a bit to fit in the code
\usepackage[margin=1.2in]{geometry}
\usepackage{amsmath, amsthm,amsfonts,amssymb}
\usepackage{graphicx}
\usepackage{fixltx2e}
\usepackage{listings}
\usepackage{algorithm}

\usepackage{hyperref}
\usepackage{longtable}
\usepackage{siunitx}
\usepackage{graphicx}

\usepackage{float}
% pseudocode
\usepackage[end]{algpseudocode}
\newcommand*\Let[2]{\State #1 $\gets$ #2}
\algrenewcommand\alglinenumber[1]{{\sf\scriptsize#1}}
\algrenewcommand\algorithmicrequire{\textbf{Input:}}
\algrenewcommand\algorithmicensure{\textbf{Output:}}

\makeatletter
\let\OldStatex\Statex
\renewcommand{\Statex}[1][3]{%
  \setlength\@tempdima{\algorithmicindent}%
  \OldStatex\hskip\dimexpr#1\@tempdima\relax}
\makeatother

\algnewcommand\algorithmicblackcomment[1]{\hfill\(\blacktriangleright\) #1}
\algnewcommand{\CommentMulti}[1]{ \(\blacktriangleright\) #1}
\makeatletter
\algnewcommand{\CommentInlineMulti}[1]{\Statex[\theALG@nested] \(\triangleright\) #1}

% "algorithmic" without "algorithm" lacks all of the horizontal lines
% The top lines are in the  "captionof" redefinition below and the trailing
% line is here
\algrenewcommand\ALG@endalgorithmic{\Statex[-1]\hrulefill}

\usepackage{xpatch}
% too many fractions in the algorithms
% TODO: do it individually per case
\xpatchcmd{\algorithmic}{\itemsep\z@}{\itemsep=.25ex}{}{}
\makeatother


% shamelessly stolen from http://tex.stackexchange.com/questions/53357/switch-cases-in-algorithmic
% New definitions
\algnewcommand\algorithmicswitch{\textbf{switch}}
\algnewcommand\algorithmiccase{\textbf{case}}
\algnewcommand\algorithmicdefault{\textbf{default}}

\newcommand{\longsub}[1]{\text{\textit{\scriptsize{#1}}}}
\newcommand{\RETURN}{\State \textbf{return} }
\newcommand{\Break}{\State \textbf{break} }
\newcommand{\Continue}{\State \textbf{break} }

% New "environments"
\algdef{SE}[SWITCH]{Switch}{EndSwitch}[1]{\algorithmicswitch\ #1\ \algorithmicdo}{\algorithmicend\ \algorithmicswitch}%
\algdef{SE}[CASE]{Case}{EndCase}[1]{\algorithmiccase\ #1}{\algorithmicend\ \algorithmiccase}%
\algdef{SE}[DEFAULT]{Default}{EndDefault}[1]{\algorithmicdefault\ #1}{\algorithmicend\ \algorithmicdefault}%
\algtext*{EndSwitch}%
\algtext*{EndCase}%

\algdef{SE}[DOWHILE]{Do}{doWhile}{\algorithmicdo}[1]{\algorithmicwhile\ #1}


\makeatletter
\def\hrulefillthick{\leavevmode\leaders\hrule height .85pt\hfill\kern\z@}
\makeatother
% shamelessly stolen from http://tex.stackexchange.com/questions/33866/algorithm-tag-and-page-break
\usepackage[font=small,labelfont=bf,width=.8\linewidth]{caption}
\DeclareCaptionFormat{algor}{%
  \hrulefillthick\par\offinterlineskip\vskip2pt%
     \textbf{#1#2}#3\offinterlineskip\hrulefill}
\DeclareCaptionStyle{algori}{singlelinecheck=off,format=algor,labelsep=space}
\captionsetup[algorithm]{style=algori}
%Still necessary?
\MakeRobust{\Call}

% http://tex.stackexchange.com/questions/77996/how-to-show-a-hint-when-lstlisting-is-breaking-page
\usepackage[framemethod=tikz]{mdframed}
% define the frame style for the listing:
\mdfdefinestyle{note}
  {
    hidealllines = true,
    skipabove    = .5\baselineskip,
    skipbelow    = .5\baselineskip,
    singleextra  = {},
    firstextra   = {
      \node[below right,overlay,align=center,font=\continuingfont]
        at (O) {\continuingtext};
    },
    secondextra  = {
      \node[above right,overlay,align=center,font=\continuingfont]
        at (O |- P) {\continuedtext};
    },
    middleextra  = {
      \node[below right,overlay,align=center,font=\continuingfont]
        at (O) {\continuingtext};
      \node[above right,overlay,align=center,font=\continuingfont]
        at (O |- P) {\continuedtext};
    }
  }
\newcommand*\continuingfont{\footnotesize\itshape}
\newcommand*\continuingtext{\hspace{2em}Listing continues on next page}
\newcommand*\continuedtext{\hspace{2em}Continuing from last page}
% the trick with mdframed writes over the footnote-line,
% this lowers the footnote
\setlength{\skip\footins}{5ex}

\ifpdf
\pdfcompresslevel=9
\pdfinfo{
   /Title      (LibTomMath's Algorithms: Lars Helmström's Barret\_todecimal)
   /Author     (Editor: Christoph Zurnieden)
   /Keywords   (arbitrary-precision algorithm libtommath libtom)
}
\fi


\providecommand{\abs}[1]{\left\lvert#1\right\rvert}
\providecommand{\floor}[1]{\left\lfloor#1\right\rfloor}
\providecommand{\ceil}[1]{\left\lceil#1\right\rceil}

\DeclareMathOperator{\AND}{\wedge}
\DeclareMathOperator{\OR}{\vee}
\DeclareMathOperator{\sgn}{sgn}

% don't forget to add an empty statement {} after that command to get a space
% I did not want to add another package to this already very heavy prelude
% but you can include the package "xspace" and change the macro to
% \newcommand{\nthroot}{$n^{\text{\tiny th}}$-root\xspace}
% some say that xspace produces more problems than it solves, vid.:
% https://tex.stackexchange.com/questions/86565/drawbacks-of-xspace
% where you'll find out that its own creator does not recommend it.
\newcommand{\nthroot}{$n^{\text{\tiny th}}$-root}

\newcommand{\RaiseNum}[2]{\raisebox{#1}{$\scriptstyle #2$}}

% wonder if I'll ever need it
% C&P from the amsthm documentation
\theoremstyle{plain} % is said to be the default
\newtheorem{thm}{Theorem}[section]
\newtheorem{lem}[thm]{Lemma}
\newtheorem{prop}[thm]{Proposition}

\theoremstyle{definition}
\newtheorem{defn}{Definition}[section]
\newtheorem{conj}{Conjecture}[section]
\newtheorem{exmp}{Example}[section]

\theoremstyle{remark}
\newtheorem*{rem}{Remark}
\newtheorem*{note}{Note}
\newtheorem{case}{Case}

% to get the notes a counter (lacks internationalization)
%\newcounter{notice}
%\newenvironment{notice}{
%\stepcounter{notice}
%\par\bigskip\noindent{\bfseries Note \arabic{notice}}
%\par\medskip}{\par\medskip}

% shamelessly stolen from https://tex.stackexchange.com/a/94466
% still lacks internationalization
\global\mdfdefinestyle{notice}{%
linecolor=gray,linewidth=1pt,%
leftmargin=1cm,rightmargin=1cm,
}
\newenvironment{fnotice}[1]{%
\mdfsetup{%
frametitle={\colorbox{white}{\,#1\,}},
%frametitle={\tikz\node[fill=white,rectangle,inner sep=0pt,outer sep=0pt]{#1};},
frametitleaboveskip=-\ht\strutbox,
%frametitleaboveskip=-0.5\ht\strutbox,
frametitlealignment=\raggedright
}%
\begin{mdframed}[style=notice]
}{\end{mdframed}}

\definecolor{lightgray}{rgb}{.9,.9,.9}
\definecolor{darkgray}{rgb}{.4,.4,.4}
\definecolor{purple}{rgb}{0.65, 0.12, 0.82}

\lstdefinelanguage{parigp}{
  alsoletter={\\},
  keywords={ for, forcomposite, fordiv, fordivfactored, forell, forfactored, forpart,%
             forperm, forprime, forsquarefree, forstep, forsubgroup, forsubset,%
             forvec, if, iferr, next, return, until, while, local, alias, allocatemem,%
             apply, call, default, extern, externstr, fold, getabstime, getenv,%
             getheap, getrand, getstack, gettime, getwalltime, global, inline, input,%
             localbitprec, localprec, print, print1, printf, printp, printsep,%
             printsep1, printtex, quit, read, readstr, readvec, self, setrand,%
             system, type, uninline, version, write, write1, writebin, writetex},
  keywordstyle=\bfseries,
  morekeywords={[2]{log, floor, ceil,gcd, lcm,sqrt, sqrtn,sqrtint,%
              sin,cos,tan,asin,acos,atan,sinh,cosh,tanh, asinh,acosh,atanh,%
              gamma,lngamma,psi,dilog,%
              % own functions
              flog2, newton, halley, bisection}},
  keywordstyle={[2]{\color{darkgray}\bfseries}},
  sensitive=false,
  comment=[l]{\\\\},
  commentstyle=\ttfamily,
  stringstyle=\ttfamily,
  morestring=[b]',
  morestring=[b]"
}

\lstdefinestyle{code}{
   extendedchars=true,
   basicstyle=\footnotesize\ttfamily,
   showstringspaces=false,
   showspaces=false,
   tabsize=2,
   breaklines=true,
   showtabs=false,
   xleftmargin=.2\textwidth, xrightmargin=.2\textwidth,
   frame=lines
%   texcl= true
}

\lstdefinestyle{widercode}{
   extendedchars=true,
   basicstyle=\footnotesize\ttfamily,
   showstringspaces=false,
   showspaces=false,
   tabsize=2,
   breaklines=true,
   showtabs=false,
   xleftmargin=.05\textwidth, xrightmargin=.05\textwidth,
   frame=lines
}

\lstnewenvironment{pblisting}[1]
  {%
    \lstset{style=widercode,#1}%
    \mdframed[style=note]%
  }
  {%
    \endmdframed
  }
% to get subsubsection numbered, too
\setcounter{secnumdepth}{3}
\setcounter{tocdepth}{3}

% Boxes with the gray background
% shamelessly stolen from https://tex.stackexchange.com/a/7545
\usepackage{calc}
\newlength{\DepthReference}
\settodepth{\DepthReference}{g}
\newlength{\HeightReference}
\settoheight{\HeightReference}{T}
\newlength{\Width}
\newcommand{\newcolorbox}[2][gray!30] {%
  \settowidth{\Width}{#2}%
  \setlength{\fboxsep}{.9pt}%
  \fcolorbox{black}{#1}{%
    \raisebox{-\DepthReference} {%
      \parbox[b][\HeightReference+\DepthReference][c]{\Width}{\centering#2}%
    }%
  }%
}
\setlength{\fboxsep}{1pt}
%no, a framed ox with grey background is just too much here
%\newcommand*{\ttchar}[1]{\newcolorbox{\texttt{\strut#1}}}
\newcommand*{\ttchar}[1]{\texttt{\textbf{\strut#1}}}
\newcommand*{\smalltt}[1]{\small\texttt{#1}}

\makeatletter
% Idea shamlessly stolen from Stephan Lehmke in
% https://tex.stackexchange.com/a/53716
\newcommand\circled
{%
  \mathpalette\@circled
}
\newcommand\@circled[2]
{%
  \mathbin%
  {%
    \ooalign{\hidewidth$#1#2$\hidewidth\crcr$#1\bigcirc$}%
  }%
}
\newcommand{\odiv}{\circled{\div}}
\makeatother

\sisetup{output-exponent-marker=\ensuremath{\mathrm{e}}}

\begin{document}
\title{LibTomMath's Algorithms: Lars Helmstr\"om's Barret\_todecimal}
\author{Current editor: Christoph Zurnieden\\
        \small{\texttt{$<$czurnieden@gmx.de$>$}}}
\date{Last change: \today}
\maketitle

%\tableofcontents
% Courier because Computer Modern has no boldface
\renewcommand{\ttdefault}{pcr}

\begin{abstract}
This artikel describes the algorithm behind the function \smalltt{Barret\_todecimal} in Lars Helmstr\"om's post to the TCL list\footnote{Available online at: \url{https://sourceforge.net/p/tcl/mailman/message/31972670/}}.
\end{abstract}

\section{Pseudocode}

\subsection{Setup}
% TODO: no minipage enclosure to reduce width, it is put on the next page if it is in one.
\begin{center}
  \captionof{algorithm}{Barret\_todecimal\label{alg:barrettodecimal}}
  \begin{algorithmic}[1]
    \Require{$a > 0$, big-integer, $b$, a buffer of suitable size}
    \Ensure{$b$ filled with the value of $a$ converted to a decimal number}
    \Function{Barret\_todecimal}{$a$}
      \CommentInlineMulti{Setup and zero out three buffers that are able to hold 20 bigints each}
      \Let{$n^\ell$}{$\{20,0\}$}
      \Let{$s^\ell$}{$\{20,0\}$}
      \Let{$m^\ell$}{$\{20,0\}$}

      \CommentInlineMulti{Set initial values: radix, shift and reciprocal}
      \Let{$n$}{$1000$}
      \Let{$n^\ell_0$}{$n$}
      \Let{$s$}{$20$}
      \Let{$s^\ell_0$}{$s$}

      \Let{$M$}{$8389$}\Comment{$\ceil{ \left(8 \cdot 2^{20}\right)/1000}$}
      \Let{$m^\ell_0$}{$1049$}\Comment{$\ceil{M/8}$}

      \Let{$\rho$}{$1$}\Comment{Index into the buffers $n^\ell, s^\ell, m^\ell$}
      \While{$p \le h$}
        \Let{$n$}{$n^2$}
        \If{$n > a$}
           \Break
        \EndIf
        \CommentInlineMulti{Compute $M = \ceil{\left(2^{2s}\right)/n^2}$}
        \Let{$M_2$}{$2M^2$}
        \Let{$M_4$}{$\floor{\frac{nM^4}{2^{s + 6}}}$}
        \Let{$M$}{$\floor{\frac{\left(M_2 - M_4\right)}{2^6}} + 1$}
        \Let{$m^\ell_\rho$}{$M$}
        \Let{$\rho$}{$\rho + 1$}
        \Let{$M$}{$8M$}
      \EndWhile
      \If{$\rho \ge 20$}
         \RETURN error
      \EndIf
      \CommentInlineMulti{Use the pre-computed reciprocals recursively}
      \State \Call{Barrett\_todecimalRecursion}{$a, n^\ell, s^\ell, m^\ell, \rho - 1, 1, b$}
    \EndFunction
  \end{algorithmic}
\end{center}

The first five rounds and the last one result in the following values
\begin{table}[h]
\begin{center}
\begin{tabular}{c l r l }
\textbf{Index}&$\mathbf{n}$&\textbf{Shift}&$\mathbf{M}$\\
1  & $n^1$    &  20 & $1049$\\
2  & $n^2$    &  40 & $1099512$\\
3  & $n^4$    &  80 & $1208925819615$\\
4  & $n^8$    & 160 & $1461501637330902918203685$\\
5  & $n^{16}$ & 320 & $2135987035920910082395021706169552114602704522357$\\
\multicolumn{4}{c}{\textellipsis}\\
20 & $n^{\left({2^{20}}\right)}$ & 20971520 & $\approx 3.7554941942\,e3167328$
\end{tabular}
\captionof{table}{Results of first five rounds and the last one}\label{tab:firstfiveloops}
\end{center}
\end{table}

Are $3\,167\,329$ decimal digits sufficient?
The largest number possible in LTM is\\
 \-\hspace{1cm}\smalltt{(1<<(sizeof(int) * CHAR\_BIT) - 1) * MP\_DIGIT\_MAX}\\
 bits large. That is for 32 bit integers and \smalltt{MP\_64BIT}: $2^{31} \cdot 60 = 128\,849\,018\,880$ ($38\,787\,419\,594$ decimal digits) which overflows several functions in LTM, so the actually possible largest value is only\\
 \-\hspace{1cm}\smalltt{(1<<(sizeof(int) * CHAR\_BIT - 1)) / MP\_DIGIT\_MAX}\\
 bits large: $\floor{2^{31}/60} = 35\,791\,394$ under the same conditions as above. In decimal digits: $\floor{35\,791\,394\log_{10}(2)} = 10\,774\,283$ which is a bit more than 3 times larger than $3\,167\,329$, so the answer is: no, it is not sufficient.

The next entries in the list are be
\begin{center}
\begin{longtable}{c c l l l}
\multicolumn{1}{c}{\textbf{Index}}&
\multicolumn{1}{c}{$\mathbf{n}$}&
\multicolumn{1}{c}{\textbf{Shift}}&
\multicolumn{1}{c}{$\mathbf{M}$}&
\multicolumn{1}{l}{\textbf{Comment}}\\
\endfirsthead
%

\multicolumn{1}{c}{\textbf{Index}}&
\multicolumn{1}{c}{$\mathbf{n}$}&
\multicolumn{1}{c}{\textbf{Shift}}&
\multicolumn{1}{c}{$\mathbf{M}$}&
\multicolumn{1}{l}{\textbf{Comment}}\\
\endhead
% To get a bit more "headroom"
\rule{0pt}{1ex}\\
\multicolumn{5}{c}{{\footnotesize{\textit{Continued on next page}}}} \\
\endfoot
\endlastfoot
21 & $n^{\left(2^{21}\right)}$ & 41943040      & $1.4103736642\,e6334657$     & \\
22 & $n^{\left(2^{22}\right)}$ & 83886080      & $1.9891538728\,e12669314$  
                                                                      & $> 10^{10\,774\,283}$ \\
23 & $n^{\left(2^{23}\right)}$ & 167772160     & $3.9567331296\,e25338628$    &  \\
24 & $n^{\left(2^{24}\right)}$ & 335544320     & $1.5655737058\,e50677257$    &  \\
25 & $n^{\left(2^{25}\right)}$ & 671088640     & $2.4510210284\,e101354514$   &  \\
26 & $n^{\left(2^{26}\right)}$ & 1342177280    & $6.0075040819\,e202709028$   &  \\
27 & $n^{\left(2^{27}\right)}$ & 2684354560    & $3.6090105294\,e405418057$   &  \\
28 & $n^{\left(2^{28}\right)}$ & 5368709120    & $1.3024957001\,e810836115$   &  \\
29 & $n^{\left(2^{29}\right)}$ & 10737418240   & $1.6964950488\,e1621672230$  &  \\
30 & $n^{\left(2^{30}\right)}$ & 21474836480   & $2.8780954507\,e3243344460$  &  \\
31 & $n^{\left(2^{31}\right)}$ & 42949672960   & $8.2834334231\,e6486688920$  &  \\
32 & $n^{\left(2^{32}\right)}$ & 85899345920   & $6.8615269276\,e12973377841$ &  \\
33 & $n^{\left(2^{33}\right)}$ & 171798691840  & $4.7080551778\,e25946755683$ &  \\
34 & $n^{\left(2^{34}\right)}$ & 343597383680  & $2.2165783558\,e51893511367$ 
                                                                  & $> 10^{38\,787\,419\,594}$ \\
\end{longtable}
\captionof{table}{Results of rounds 21-34)}\label{tab:rounds21to34}
\end{center}

To save some stack memory compute the upper limit of the number $n$ of reciprocals needed calculate
\begin{equation}
n = \floor{\log_2 \left( \floor{\frac{b_a}{2\floor{\log_2 10}}} + 2 \right)  }
\end{equation}
with $b_a = \floor{\log_2 a} + 1$ the number of bits in $a$, assuming $1\,000$ as the start-value and allocate the memory for the lists on the heap instead.

\subsection{Recursion}

\begin{center}
\begin{minipage}{.9\linewidth}
  \captionof{algorithm}{Barret\_todecimalRecursion\label{alg:barrettodecimalrecursion}}
  \begin{algorithmic}[1]
    \Require{$a > 0$, big-integer, $n^\ell, s^\ell, m^\ell, \rho, L, b$, from Barret\_todecimal}
    \Ensure{$b$ filled with the value of $a$ converted to a decimal number}
    \Function{Barret\_todecimalRecusive}{$a,n^\ell, s^\ell, m^\ell, \rho, L, b$}
    \If{$\rho < 0$}
      \Let{$b$}{$b | a$}\Comment{Convert $a$ to a decimal string and add to $b$. See text for details}
      \RETURN
    \Else
      \Let{$q$}{$\floor{\left(a\cdot m^\ell_\rho\right) / 2^{s^\ell_\rho}}$}\Comment{Quotient}
      \Let{$r$}{$ a - n^\ell_\rho \cdot q$}
      \If{$r < 0$}
        \Let{$q$}{$q - 1$}
        \Let{$r$}{$r + n^\ell_\rho$}
      \EndIf
      \Let{$\rho$}{$\rho + 1$}
      \If{$\left(L = 1\right) \AND \left(q = 0\right) $}
        \State \Call{Barret\_todecimalRecusive}{$r,n^\ell, s^\ell, m^\ell, \rho, 1, b$}
      \Else
        \State \Call{Barret\_todecimalRecusive}{$q,n^\ell, s^\ell, m^\ell, \rho, L, b$}
        \State \Call{Barret\_todecimalRecusive}{$r,n^\ell, s^\ell, m^\ell, \rho, 0, b$}
      \EndIf
    \EndIf
    \EndFunction
  \end{algorithmic}
 \end{minipage}
\end{center}

If the chunk is small enough ($\rho < 0$ for 0-based lists) we can build that part of the output.
The smallest chunk has 4 decimal digits in our case ($n^1 = 1000$) so we need up to three leading zeros if the parameter ``left'' is true or just build the decimal number without any further ado.
This is also a place where a bit of optimization can take place because four decimal digits is a rather small chunk and has most probably been chosen to be fit for \smalltt{MP\_8BIT} ($\log_{10}(2^{16}) \approx 4.8$).
If we want a bigger chunk we can stop at a higher index and use \smalltt{mp\_to\_radix} to convert\footnote{Or strip the relevant part out of \smalltt{mp\_to\_radix} which is just the inner loop. All the checks are not needed because they have already been done}. Handling leading zeros is a bit more complicated in that case but not much.

\section{Computing the Reciprocals}

The reciprocals are the values of the sequence
$\ceil{\left(2^s/n\right)^{\left(2^k\right)}}  =  \ceil{\left(2^{2^ks}\right)/n^{\left(2^k\right)}}$
(with $k = \{0,1,2,3,\dots\}$, $s = 20$, and $n = 1\,000$). We cannot simply square because $2^s \nmid n$ and hence $\ceil{\left(2^s/n\right)}^{\left(2^k\right)} \not= \ceil{\left(2^s/n\right)^{\left(2^k\right)}}$ but we can use one round of Newton-Raphson to overcome this inconvenient obstacle.

\begin{equation}
x_{n+1} = x_n\left(2 - Dx_n\right) = 2x_n - Dx_n^2
\end{equation}

By substituting $x_n = \ceil{2^s/n}^2$ and\footnote{Here $\odiv$ stands for ``use truncating division (rounding to next integer in direction of zero) later in the algebraic transformations because truncating division is all we have in big--integer land''} $D = \left(n \odiv 2^s\right)^2$ we get

\begin{equation}\label{eq:newtonraphson01}
x_{n+1} = 2\ceil{2^s/n}^2 - \floor{\frac{\ceil{2^s/n}^4n^2}{2^{2s}}}
\end{equation}


The error $\left(1- Dx_n\right)^2$ is $1$, assuming the same substitutions as above, but that is misleading. The error without the truncating division is about \num{2e-34} and a simple addition of $1$ will not do. The following short and highly unoptimized Pari/\kern-1.2pt gp script will wet this dry theory with some numbers:

\lstset{language=parigp}
\begin{pblisting}{caption={Print error},label=lst:parigperror}
NR_error(extra, limit) = {
   local(n, s, M, t);
   for(m=0,extra,
      n = 1000;
      s= 20;
      for(k=0,limit,
         M = (2*ceil(2^(s+m)/n)^2 - ( ceil(2^(s+m)/n)^4*n^2)\2^(2*(s+m)))\2^(2*m) + 1;
         t = ceil(2^(2*s)/n^2);
         t = M-t;
         printf(t);
         if( k != limit,
            printf(", ")
         );
         n = n^2;
         s = 2*s;
      );
       print("");
   );
}
\end{pblisting}

Running \texttt{NR\_error(3, 22)} lasts a long time and needs a lot of memory but eventually prints:

\begin{table}[h]
\begin{center}\resizebox{\textwidth}{!}{%
\begin{tabular}{c | c  c  c  c  c  c  c  c  c  c  c  c  c  c  c  c  c  c  c  c  c c c}
\multicolumn{1}{c}{\texttt{\textbf{extra}}}&\multicolumn{18}{c}{\textbf{Error}}\\
$0$ & $0$& $1$& $1$& $1$& $0$& $0$& $1$& $0$& $-$$2$& $1$& $0$&
   $1$& $-$$2$& $1$& $-$$1$& $0$& $-$$3$& $0$& $0$& $0$ & $0$& $1$& $0$ \\
$1$ & $0$& $0$& $0$& $0$& $0$& $1$& $0$& $-$$1$& $0$& $0$& $1$&
   $0$& $0$& $0$& $0$& $-$$1$& $-$$1$& $-$$1$& $0$& $0$& $0$& $0$& $0$ \\
$2$ &  $0$& $0$& $0$& $0$& $0$& $0$& $0$&  $0$& $0$& $0$& $0$&
   $0$& $0$& $0$& $0$&  $0$&  $0$&  $0$& $0$& $0$& $0$& $0$& $0$ \\
$3$ &$0$& $0$& $0$& $0$& $0$& $0$& $0$&  $0$& $0$& $0$& $0$& $0$&
   $0$& $0$& $0$&  $0$&  $0$&  $0$& $0$& $0$& $0$& $0$& $0$   
\end{tabular}}
\captionof{table}{Results of \texttt{NR\_error(3, 22)}}\label{tab:nr_error_3_22}
\end{center}
\end{table}

\subsection{Expanding to other bases}

The largest problem for other bases is generating a good fitting $s$. With base $10$ it is sufficient to use a simple $s = 2s$ and come close to what is actually needed.
\lstset{language=parigp}
\begin{pblisting}{caption={Find start values for $n$ for table \ref{tab:startvaluesn}},label=lst:startvaluesn}
max_size_n(limit_s) = {
   local(t, n, s = 40, ro8);
   for(k= 2, 63,
      for(m = 1, limit_s + 1,
         n = k^m; t = ceil(log(n)/log(2));
         if( t > limit_s,
            t = m-1; n = k^t;ro8 = ceil(ceil(2^(s+3)/n)/8);
            print(k ", " t ", " n ", " ro8);
            break(1);
         );
      );
   );
}

\end{pblisting}
\begin{center}
\begin{longtable}{r r r r | r r r r}
\multicolumn{1}{r}{$\mathbf{b}$}&
\multicolumn{1}{r}{$\mathbf{k}$}&
\multicolumn{1}{r}{$\mathbf{n = b^k}$}&
\multicolumn{1}{r}{$\mathbf{M = \ceil{\ceil{2^{s+3}/n}/8}}$}&
\multicolumn{1}{r}{$\mathbf{b}$}&
\multicolumn{1}{r}{$\mathbf{k}$}&
\multicolumn{1}{r}{$\mathbf{n = b^k}$}&
\multicolumn{1}{r}{$\mathbf{M = \ceil{\ceil{2^{s+3}/n}/8}}$}\\
\endfirsthead
%
\multicolumn{1}{r}{$\mathbf{b}$}&
\multicolumn{1}{r}{$\mathbf{k}$}&
\multicolumn{1}{r}{$\mathbf{n = b^k}$}&
\multicolumn{1}{r}{$\mathbf{M = \ceil{\ceil{2^{s+3}/n}/8}}$}&
\multicolumn{1}{r}{$\mathbf{b}$}&
\multicolumn{1}{r}{$\mathbf{k}$}&
\multicolumn{1}{r}{$\mathbf{n = b^k}$}&
\multicolumn{1}{r}{$\mathbf{M = \ceil{\ceil{2^{s+3}/n}/8}}$}\\
\endhead
% To get a bit more "headroom"
\rule{0pt}{1ex}\\
\multicolumn{6}{c}{{\footnotesize{\textit{Continued on next page}}}} \\
\endfoot
\endlastfoot
2  & 20 & 1048576 & 1048576  &  33 & 3  & 35937   & 30595532 \\
3  & 12 & 531441  & 2068926  &  34 & 3  & 39304   & 27974548 \\
4  & 10 & 1048576 & 1048576  &  35 & 3  & 42875   & 25644587 \\
5  & 8  & 390625  & 2814750  &  36 & 3  & 46656   & 23566351 \\
6  & 7  & 279936  & 3927726  &  37 & 3  & 50653   & 21706743 \\
7  & 7  & 823543  & 1335100  &  38 & 3  & 54872   & 20037754 \\
8  & 6  & 262144  & 4194304  &  39 & 3  & 59319   & 18535573 \\
9  & 6  & 531441  & 2068926  &  40 & 3  & 64000   & 17179870 \\
10 & 6  & 1000000 & 1099512  &  41 & 3  & 68921   & 15953217 \\
11 & 5  & 161051  & 6827103  &  42 & 3  & 74088   & 14840617 \\
12 & 5  & 248832  & 4418691  &  43 & 3  & 79507   & 13829118 \\
13 & 5  & 371293  & 2961305  &  44 & 3  & 85184   & 12907490 \\
14 & 5  & 537824  & 2044371  &  45 & 3  & 91125   & 12065972 \\
15 & 5  & 759375  & 1447917  &  46 & 3  & 97336   & 11296043 \\
16 & 5  & 1048576 & 1048576  &  47 & 3  & 103823  & 10590251 \\
17 & 4  & 83521   & 13164494 &  48 & 3  & 110592  & 9942054  \\
18 & 4  & 104976  & 10473934 &  49 & 3  & 117649  & 9345695  \\
19 & 4  & 130321  & 8436949  &  50 & 3  & 125000  & 8796094  \\
20 & 4  & 160000  & 6871948  &  51 & 3  & 132651  & 8288755  \\
21 & 4  & 194481  & 5653569  &  52 & 3  & 140608  & 7819695  \\
22 & 4  & 234256  & 4693633  &  53 & 3  & 148877  & 7385370  \\
23 & 4  & 279841  & 3929059  &  54 & 3  & 157464  & 6982623  \\
24 & 4  & 331776  & 3314018  &  55 & 3  & 166375  & 6608635  \\
25 & 4  & 390625  & 2814750  &  56 & 3  & 175616  & 6260886  \\
26 & 4  & 456976  & 2406060  &  57 & 3  & 185193  & 5937113  \\
27 & 4  & 531441  & 2068926  &  58 & 3  & 195112  & 5635285  \\
28 & 4  & 614656  & 1788825  &  59 & 3  & 205379  & 5353574  \\
29 & 4  & 707281  & 1554562  &  60 & 3  & 216000  & 5090332  \\
30 & 4  & 810000  & 1357422  &  61 & 3  & 226981  & 4844070  \\
31 & 4  & 923521  & 1190565  &  62 & 3  & 238328  & 4613439  \\
32 & 4  & 1048576 & 1048576  &  63 & 3  & 250047  & 4397220

\end{longtable}
\captionof{table}{Start--values for $n$ for assorted bases $b$ with $s = 40$}\label{tab:startvaluesn}
\end{center}

If we start with $s = 20$ for all bases the start--values for $n$ in table \ref{tab:startvaluesn} the exponents $k$ would need to be halved and be way too small for all values above base $10$. There are no need for changes in algorithm \ref{alg:barrettodecimal} (\smalltt{Barrett\_todecimal}, just plug in the values from the table above for $n$ and $M$ and set $s = 80$ instead.

Nothing needs to be changed in algorithm \ref{alg:barrettodecimalrecursion} (\smalltt{Barret\_todecimalRecursion}) but the number of digits produced in the leafs which are represented by the exponent $k$. With e.g.: base $10$ the algorithm produces $6$ digits now instead of $3$.


The little script used to generate the values for the table \ref{tab:baserelations} is:

\lstset{language=parigp}
\begin{pblisting}{caption={Find sizes of bases 3, 5, and 10},label=lst:parigpsizebases3510}
sizes_of_bases(limit) = {
   local(s, n3, n5, n10, t3, t5, t10);
   s = 10;
   n3 = 729;
   n5 = 625;
   n10 = 1000;
   for(k=0, limit,
      t3 = ceil(log(n3)/log(2));
      t5 = ceil(log(n5)/log(2));
      t10 = ceil(log(n10)/log(2));
      print(k ": " s ", " t3 ", " t5 ", " t10);
      s = 2*s;
      n3 = n3^2;
      n5 = n5^2;
      n10 = n10^2;
  );
}
\end{pblisting}


\begin{center}
\begin{longtable}{r r r r}
\multicolumn{1}{r}{$\mathbf{s/2}$}&
\multicolumn{1}{r}{$\mathbf{3^6}$}&
\multicolumn{1}{r}{$\mathbf{5^4}$}&
\multicolumn{1}{r}{$\mathbf{10^3}$} \\
\endfirsthead
%
\multicolumn{1}{r}{$\mathbf{s}$}&
\multicolumn{1}{r}{$\mathbf{3^6}$}&
\multicolumn{1}{r}{$\mathbf{5^4}$}&
\multicolumn{1}{r}{$\mathbf{10^3}$} \\
\endhead
% To get a bit more "headroom"
\rule{0pt}{1ex}\\
\multicolumn{4}{c}{{\footnotesize{\textit{Continued on next page}}}} \\
\endfoot
\endlastfoot
10       & 10       & 10       & 10       \\
20       & 20       & 19       & 20       \\
40       & 39       & 38       & 40       \\
80       & 77       & 75       & 80       \\
160      & 153      & 149      & 160      \\
320      & 305      & 298      & 319      \\
640      & 609      & 595      & 638      \\
1280     & 1218     & 1189     & 1276     \\
2560     & 2435     & 2378     & 2552     \\
5120     & 4870     & 4756     & 5103     \\
10240    & 9739     & 9511     & 10205    \\
20480    & 19477    & 19022    & 20410    \\
40960    & 38953    & 38043    & 40820    \\
81920    & 77905    & 76085    & 81640    \\
163840   & 155809   & 152170   & 163280   \\
327680   & 311617   & 304340   & 326559   \\
655360   & 623233   & 608680   & 653118   \\
1310720  & 1246466  & 1217360  & 1306236  \\
2621440  & 2492931  & 2434719  & 2612471  \\
5242880  & 4985861  & 4869437  & 5224942  \\
10485760 & 9971722  & 9738873  & 10449883 \\
20971520 & 19943444 & 19477745 & 20899765 \\
41943040 & 39886888 & 38955490 & 41799529
\end{longtable}
\captionof{table}{Relation of $s = 2s$ to bases $3$, $5$ and $10$}\label{tab:baserelations}
\end{center}

The difference between the behaviour of base $10$ and $s$ is relatively similar but it gets quite large with the other bases. It is a bit complicated to generate the values programmatically but as all of the numbers up to $k=28$ stay below $2^{32}$ a small table would be enough.

\subsection{Some optimizations}\label{subsec:optimizations}

\subsubsection{Factoring the bases}
If the input of the function \smalltt{Barret\_todecimal} is restricted to base $10$, base $5$ can be used for optimization by observing that $10 = 2\cdot5$ which can safe about a third ($\log(2)/\log(5) \approx 0.4307$) of the magnitude in multiplication.

Factoring all of the other even bases (table \ref{tab:factoreven}) gives a hint of another way to reduce the amount of necessary computation: the bases that are powers of two need shifts only. This can be done, without a lot of effort and memory, in \smalltt{mp\_to\_radix}, too.

Using all of the factors is less attractive because every other additional factor needs its own large cache and a method to use it. Most of the odd bases are prime already, none of the rest has more than two factors and all the work and memory for the three bases $30$, $42$, and $60$ that would need such a special treatment is just not economical.
% TODO: Nuh, still way too much white. Make three columns?
\begin{table}[H]
\begin{center}
\begin{tabular}{l c c c l c c c}
\textbf{base}&\multicolumn{3}{c}{\textbf{factors}}&\textbf{base}&\multicolumn{3}{c}{\textbf{factors}} \\
2  & $ 2^1 $&        &         & 34 & $ 2^1 $&$ 17^1 $& 	       \\
4  & $ 2^2 $&        &         & 36 & $ 2^2 $&$ 3^2 $ & 	       \\
6  & $ 2^1 $&$ 3^1 $ &         & 38 & $ 2^1 $&$ 19^1 $& 	       \\
8  & $ 2^3 $&        &         & 40 & $ 2^3 $&$ 5^1 $ & 	       \\
10 & $ 2^1 $&$ 5^1 $ &         & 42 & $ 2^1 $&$ 3^1 $ &$  7^1 $        \\
12 & $ 2^2 $&$ 3^1 $ &         & 44 & $ 2^2 $&$ 11^1 $& 	       \\
14 & $ 2^1 $&$ 7^1 $ &         & 46 & $ 2^1 $&$ 23^1 $& 	       \\
16 & $ 2^4 $&        &         & 48 & $ 2^4 $&$ 3^1 $ & 	       \\
18 & $ 2^1 $&$ 3^2 $ &         & 50 & $ 2^1 $&$ 5^2 $ & 	       \\
20 & $ 2^2 $&$ 5^1 $ &         & 52 & $ 2^2 $&$ 13^1 $& 	       \\
22 & $ 2^1 $&$ 11^1 $&         & 54 & $ 2^1 $&$ 3^3 $ & 	       \\
24 & $ 2^3 $&$ 3^1 $ &         & 56 & $ 2^3 $&$ 7^1 $ & 	       \\
26 & $ 2^1 $&$ 13^1 $&         & 58 & $ 2^1 $&$ 29^1 $& 	       \\
28 & $ 2^2 $&$ 7^1 $ &         & 60 & $ 2^2 $&$ 3^1 $ &$  5^1 $        \\
30 & $ 2^1 $&$ 3^1 $ &$  5^1 $ & 62 & $ 2^1 $&$ 31^1 $& 	       \\
32 & $ 2^5 $&&                 & 64 & $ 2^6 $&         &
\end{tabular}
\captionof{table}{Factors of the even bases}\label{tab:factoreven}
\end{center}
\end{table}


\subsubsection{Larger Leafs}
The algorithm as it is now goes to the very end of the recursion until the actual number conversion is done. The amount of converted digits is very small down there, just the number of digits in $n^1-1$ which is a mere $\ceil{\log_{10}(999)} = 3$ digits in case of base $10$.
To get more digits we need to stop at an earlier stage $k > 0$ of the recursion to do the conversion instead of $k = 0$. The maximum number of digits needed for the base $\beta$is
\begin{equation}
\ceil{\log_\beta\left(n^{\left(2^k\right)}-1\right)}\quad\text{with}\quad n = \beta^\kappa
\end{equation}
Here $\beta$ is the base and the value of $\kappa$ can be taken from table \ref{tab:startvaluesn} but keep in mind that the shift--value is $s = 40$ there instead of the original $s = 20$.

The algorithm itself needs more information than the algorithm \smalltt{Barret\_todecimalRecursion} in \ref{alg:barrettodecimalrecursion} offers. We also need the base $\beta$ and the exponent of the start--value $\kappa$.

\begin{center}
\begin{minipage}{.9\linewidth}
  \captionof{algorithm}{to\_string\label{alg:tostring}}
  \begin{algorithmic}[1]
    \Require{$a$, big--integer, $b$, buffer, $\beta$, $\kappa$, $k$, $L$, bool (is left?)}
    \Ensure{$b$ filled with the value of $a$ converted to base $\beta$}
    \Function{to\_string}{$a$, $b$,$\beta$, $\kappa$, $k$, $L$}
    \Let{$K$}{$\kappa + 2^k - 1$}\Comment{Cut--off at $k=5$ on the authors machine, around 600 bits.}
    \Let{$S$}{A temporary buffer of size $K$}
    \Let{$S$}{Fill the whole buffer with the ASCII digit zero, \smalltt{0x30}}
    \Let{$d$}{$0$}
    \While{$a \not= 0$}
       \Let{$q$}{$\floor{\frac{a}{\beta}}$} \Comment{Big--integer division}
       \Let{$r$}{$a - \left(q \cdot \beta\right)$}
       \Let{$c$}{$T_r$}\Comment{$T$ a table mapping $r$ to a character}
       \Let{$S_{K-d}$}{$c$}\Comment{Fill buffer $S$ from the end}
       \Let{$d$}{$d + 1$}
       \Let{$a$}{$q$}
    \EndWhile
    \EndFunction
  \end{algorithmic}
\end{minipage}
\end{center}




% \section{Convert {\texttt{Barret\_todecimal}} to an iterative algorithm}

% T.\kern-.2pt b\kern-.5pt.d.


\section{Convert a decimal string to a big integer}
The \smalltt{mp\_from\_decimal} is a much less complicated method. For simplicity only the recursive algorithm is given here.

\begin{center}
\begin{minipage}{.9\linewidth}
  \captionof{algorithm}{from\_decimal\label{alg:fromdecimal}}
  \begin{algorithmic}[1]
    \Require{$a$, big--integer, $b_{10}$, decimal string}
    \Ensure{$a$ filled with the value of $b_{10}$ converted from a decimal number}
    \Function{from\_decimal}{$a$}
    \Let{$\ell$}{$\#b_{10}$} \Comment{Number of bytes (characters) in the string}
    \If{The first character in $b_{10}$ is a minus--sign}
       \Let{$\ell$}{$\ell - 1$}
       \If{$\ell = 0$} \Comment{No digits beside the minus--sign}
          \RETURN {$\quad$error}
       \EndIf
       \Let{$a$}{$-a$}
       \State{Strip the byte for the character representing the sign from $b_{10}$}
    \EndIf
    \If{$\ell < C$}\Comment{$C$ is a (tunable) cut--off value. $C = 100$ seems reasonable.}
      \Let{$a$}{$b_{10} \to b_\beta$}\Comment{Convert the decimal number in $b$ to the bigint $a$.}
    \Else
      \State \Call{from\_decimal\_recursion}{$a$, $b_{10}$, $0$, $\ell$}
    \EndIf
    \EndFunction
  \end{algorithmic}
\end{minipage}
\end{center}

The recursion itself is as straight forward as a recursion can be.

\begin{center}
\begin{minipage}{.9\linewidth}
  \captionof{algorithm}{from\_decimal\_recursion\label{alg:fromdecimalrecursion}}
  \begin{algorithmic}[1]
    \Require{$a$, big--integer, $b_{10}$ decimal string, $s$ start, $e$ end, given by from\_decimal}
    \Ensure{$a$ filled with the value of $b_{10}$ converted from a decimal number}
    \Function{from\_decimal\_recursion}{$a$,$b_{10}$, $s$, $e$}
    \Let{$\ell$}{$e - s$ }
    \If{$\ell < C$}\Comment{$C$ is the cut--off value from algorithm\ref{alg:fromdecimal}}
      \Let{$a$}{$b_{10} \to b_\beta$}\Comment{Convert the decimal number in $b_{10}$ from bytes $s$ to $e$ %
                                              to the bigint $a$.}
      \RETURN
    \Else
      \Let{$m$}{$\floor{\frac{\ell}{2}}$}
      \State \Call{from\_decimal\_recursion}{$A$, $b_{10}$, $s$, $s + m + 1$}
      \State \Call{from\_decimal\_recursion}{$B$, $b_{10}$, $s + m + 1$, $e$}
      \Let{$T$}{$10^{\ell - m - 1}$}\label{alg:fromdecimalrecursion:line:T}
      \Let{$a$}{$A \cdot T + B$}
    \EndIf
    \EndFunction
  \end{algorithmic}
\end{minipage}
\end{center}

The same optimization described in section \ref{subsec:optimizations} is applicable here, too. Change the line \ref{alg:fromdecimalrecursion:line:T} in algorithm \ref{alg:fromdecimalrecursion} from $T = 10^{\ell - m - 1}$ to $T = 5^{\ell - m - 1} \cdot 2^{\ell - m - 1}$.

\subsection{Extending to other bases}

The algorithms \ref{alg:fromdecimal} \smalltt{from\_decimal} and \ref{alg:fromdecimalrecursion} \smalltt{from\_decimal\_recursion} can be extended to all other bases with minimal effort.


\begin{center}
\begin{minipage}{.9\linewidth}
  \captionof{algorithm}{from\_string\label{alg:fromstring}}
  \begin{algorithmic}[1]
    \Require{$a$, big--integer, $b_r$, string, $r$, the base}
    \Ensure{$a$ filled with the value of $b_r$ converted from a number in base $r$}
    \Function{from\_string}{$a$}
    \If{$r < 2 \and r > 64$}
       \RETURN {$\quad$error}
    \EndIf
    \Let{$\ell$}{$\#b_r$} \Comment{Number of bytes (characters) in the string}
    \If{The first character in $b_r$ is a minus--sign}
       \Let{$\ell$}{$\ell - 1$}
       \If{$\ell = 0$} \Comment{No digits following the minus--sign}
          \RETURN {$\quad$error}
       \EndIf
       \Let{$a$}{$-a$}
       \State{Strip the byte for the character representing the sign from $b_r$}
    \EndIf
    \If{$\ell < C$}\Comment{$C$ is a (tunable) cut--off value.}
      \Let{$a$}{$b_r \to b_\beta$}\Comment{Convert the number in $b_r$ to the bigint $a$.}
    \Else
      \State \Call{from\_number\_recursion}{$a$, $b_r$, $0$, $\ell$, $r$}
    \EndIf
    \EndFunction
  \end{algorithmic}
\end{minipage}
\end{center}

The changes needed for the recursion are also minimal.

\begin{center}
\begin{minipage}{.9\linewidth}
  \captionof{algorithm}{from\_string\_recursion\label{alg:fromstringrecursion}}
  \begin{algorithmic}[1]
    \Require{$a$, big--integer, $b_r$ decimal string, $s$ start, $e$ end, $r$, the base, given by from\_string}
    \Ensure{$a$ filled with the value of $b_r$ converted from a number of base $r$}
    \Function{from\_string\_recursion}{$a$,$b_r$, $s$, $e$, $r$}
    \Let{$\ell$}{$e - s$ }
    \If{$\ell < C$}\Comment{$C$ is the cut--off value from algorithm\ref{alg:fromstring}}
      \Let{$a$}{$b_r \to b_\beta$}\Comment{Convert the number in $b_r$ from bytes $s$ to $e$ %
                                              to the bigint $a$.}
      \RETURN
    \Else
      \Let{$m$}{$\floor{\frac{\ell}{2}}$}
      \State \Call{from\_decimal\_recursion}{$A$, $b_r$, $s$, $s + m + 1$, $r$}
      \State \Call{from\_decimal\_recursion}{$B$, $b_r$, $s + m + 1$, $e$, $r$}
      \Let{$T$}{$r^{\ell - m - 1}$}
      \Let{$a$}{$A \cdot T + B$}
    \EndIf
    \EndFunction
  \end{algorithmic}
\end{minipage}
\end{center}

\subsection{Partial string conversion}
Both recursive algorithms to convert a string to a big--integer need a method for the last mile, the actual conversion. This can be done with an already existing iterative algorithm. It would need a copy of the part of the string it is supposed to convert which would double the amount of memory needed for the input--string.
But with almost all checks and balances already done, all we need is the inner loop of the actual conversion.

\begin{center}
\begin{minipage}{.9\linewidth}
  \captionof{algorithm}{from\_string\_small\label{alg:fromstringsmall}}
  \begin{algorithmic}[1]
    \Require{$a$, big--integer, $b_r$ decimal string, $s$ start, $e$ end, $r$, the base}
    \Ensure{$a$ filled with the value of $b_r$ from byte $s$ to byte $e$ converted from a number of base $r$}
    \Function{from\_string\_small}{$a$,$b_r$, $s$, $e$, $r$}
    \For{character (byte) $c$ in $b_r$ from byte $s$ to byte $e$}
       \If{$c < 0x28$}\Comment{Rough check if we have a usable ASCII character}
          \RETURN {$quad$ error}
       \EndIf
       \Let{$p$}{$c - 0x28$}\Comment{Make the position $p$ $0$--based to be able to use a table}
       \State \CommentMulti{ Here $t$ is a small table for a fast a character--value map}
       \Let{$v$}{$t_p$}
       \If{$v \ge base$} \Comment{Wrong character?}
          \RETURN {$\quad$ error}
       \EndIf
       \Let{$a$}{$a \cdot r + v$}
    \EndFor
    \EndFunction
  \end{algorithmic}
\end{minipage}
\end{center}





\end{document}
