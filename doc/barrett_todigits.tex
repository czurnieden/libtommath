\documentclass[a4paper,10pt]{article}

% This bunch of macros grew to that size in a couple of years,
% some parts are even decades old, so be aware of the risks of diving in!

\usepackage{palatino}
%\usepackage{lmodern}

% Narrow the margins a bit to fit in the code
\usepackage[margin=1.2in]{geometry}
\usepackage{amsmath, amsthm,amsfonts,amssymb}
\usepackage{graphicx}
\usepackage{fixltx2e}
\usepackage{listings}
\usepackage{algorithm}

\usepackage{hyperref}
\usepackage{longtable}
\usepackage{siunitx}
\usepackage{graphicx}

% pseudocode
\usepackage[end]{algpseudocode}
\newcommand*\Let[2]{\State #1 $\gets$ #2}
\algrenewcommand\alglinenumber[1]{{\sf\scriptsize#1}}
\algrenewcommand\algorithmicrequire{\textbf{Input:}}
\algrenewcommand\algorithmicensure{\textbf{Output:}}

\makeatletter
\let\OldStatex\Statex
\renewcommand{\Statex}[1][3]{%
  \setlength\@tempdima{\algorithmicindent}%
  \OldStatex\hskip\dimexpr#1\@tempdima\relax}
\makeatother

\algnewcommand\algorithmicblackcomment[1]{\hfill\(\blacktriangleright\) #1}
\algnewcommand{\CommentMulti}[1]{ \(\blacktriangleright\) #1}
\makeatletter
\algnewcommand{\CommentInlineMulti}[1]{\Statex[\theALG@nested] \(\triangleright\) #1}

% "algorithmic" without "algorithm" lacks all of the horizontal lines
% The top lines are in the  "captionof" redefinition below and the trailing
% line is here
\algrenewcommand\ALG@endalgorithmic{\Statex[-1]\hrulefill}

\usepackage{xpatch}
% too many fractions in the algorithms
% TODO: do it individually per case
\xpatchcmd{\algorithmic}{\itemsep\z@}{\itemsep=.25ex}{}{}
\makeatother


% shamelessly stolen from http://tex.stackexchange.com/questions/53357/switch-cases-in-algorithmic
% New definitions
\algnewcommand\algorithmicswitch{\textbf{switch}}
\algnewcommand\algorithmiccase{\textbf{case}}
\algnewcommand\algorithmicdefault{\textbf{default}}

\newcommand{\longsub}[1]{\text{\textit{\scriptsize{#1}}}}
\newcommand{\RETURN}{\State \textbf{return} }
\newcommand{\Break}{\State \textbf{break} }
\newcommand{\Continue}{\State \textbf{break} }

% New "environments"
\algdef{SE}[SWITCH]{Switch}{EndSwitch}[1]{\algorithmicswitch\ #1\ \algorithmicdo}{\algorithmicend\ \algorithmicswitch}%
\algdef{SE}[CASE]{Case}{EndCase}[1]{\algorithmiccase\ #1}{\algorithmicend\ \algorithmiccase}%
\algdef{SE}[DEFAULT]{Default}{EndDefault}[1]{\algorithmicdefault\ #1}{\algorithmicend\ \algorithmicdefault}%
\algtext*{EndSwitch}%
\algtext*{EndCase}%

\algdef{SE}[DOWHILE]{Do}{doWhile}{\algorithmicdo}[1]{\algorithmicwhile\ #1}


\makeatletter
\def\hrulefillthick{\leavevmode\leaders\hrule height .85pt\hfill\kern\z@}
\makeatother
% shamelessly stolen from http://tex.stackexchange.com/questions/33866/algorithm-tag-and-page-break
\usepackage[font=small,labelfont=bf,width=.8\linewidth]{caption}
\DeclareCaptionFormat{algor}{%
  \hrulefillthick\par\offinterlineskip\vskip2pt%
     \textbf{#1#2}#3\offinterlineskip\hrulefill}
\DeclareCaptionStyle{algori}{singlelinecheck=off,format=algor,labelsep=space}
\captionsetup[algorithm]{style=algori}
%Still necessary?
\MakeRobust{\Call}

% http://tex.stackexchange.com/questions/77996/how-to-show-a-hint-when-lstlisting-is-breaking-page
\usepackage[framemethod=tikz]{mdframed}
% define the frame style for the listing:
\mdfdefinestyle{note}
  {
    hidealllines = true,
    skipabove    = .5\baselineskip,
    skipbelow    = .5\baselineskip,
    singleextra  = {},
    firstextra   = {
      \node[below right,overlay,align=center,font=\continuingfont]
        at (O) {\continuingtext};
    },
    secondextra  = {
      \node[above right,overlay,align=center,font=\continuingfont]
        at (O |- P) {\continuedtext};
    },
    middleextra  = {
      \node[below right,overlay,align=center,font=\continuingfont]
        at (O) {\continuingtext};
      \node[above right,overlay,align=center,font=\continuingfont]
        at (O |- P) {\continuedtext};
    }
  }
\newcommand*\continuingfont{\footnotesize\itshape}
\newcommand*\continuingtext{\hspace{2em}Listing continues on next page}
\newcommand*\continuedtext{\hspace{2em}Continuing from last page}
% the trick with mdframed writes over the footnote-line,
% this lowers the footnote
\setlength{\skip\footins}{5ex}

\ifpdf
\pdfcompresslevel=9
\pdfinfo{
   /Title      (LibTomMath's Algorithms: Lars Helmström's Barret\_todecimal)
   /Author     (Editor: Christoph Zurnieden)
   /Keywords   (arbitrary-precision algorithm libtommath libtom)
}
\fi


\providecommand{\abs}[1]{\left\lvert#1\right\rvert}
\providecommand{\floor}[1]{\left\lfloor#1\right\rfloor}
\providecommand{\ceil}[1]{\left\lceil#1\right\rceil}

\DeclareMathOperator{\AND}{\wedge}
\DeclareMathOperator{\OR}{\vee}
\DeclareMathOperator{\sgn}{sgn}

% don't forget to add an empty statement {} after that command to get a space
% I did not want to add another package to this already very heavy prelude
% but you can include the package "xspace" and change the macro to
% \newcommand{\nthroot}{$n^{\text{\tiny th}}$-root\xspace}
% some say that xspace produces more problems than it solves, vid.:
% https://tex.stackexchange.com/questions/86565/drawbacks-of-xspace
% where you'll find out that its own creator does not recommend it.
\newcommand{\nthroot}{$n^{\text{\tiny th}}$-root}

\newcommand{\RaiseNum}[2]{\raisebox{#1}{$\scriptstyle #2$}}

% wonder if I'll ever need it
% C&P from the amsthm documentation
\theoremstyle{plain} % is said to be the default
\newtheorem{thm}{Theorem}[section]
\newtheorem{lem}[thm]{Lemma}
\newtheorem{prop}[thm]{Proposition}

\theoremstyle{definition}
\newtheorem{defn}{Definition}[section]
\newtheorem{conj}{Conjecture}[section]
\newtheorem{exmp}{Example}[section]

\theoremstyle{remark}
\newtheorem*{rem}{Remark}
\newtheorem*{note}{Note}
\newtheorem{case}{Case}

% to get the notes a counter (lacks internationalization)
%\newcounter{notice}
%\newenvironment{notice}{
%\stepcounter{notice}
%\par\bigskip\noindent{\bfseries Note \arabic{notice}}
%\par\medskip}{\par\medskip}

% shamelessly stolen from https://tex.stackexchange.com/a/94466
% still lacks internationalization
\global\mdfdefinestyle{notice}{%
linecolor=gray,linewidth=1pt,%
leftmargin=1cm,rightmargin=1cm,
}
\newenvironment{fnotice}[1]{%
\mdfsetup{%
frametitle={\colorbox{white}{\,#1\,}},
%frametitle={\tikz\node[fill=white,rectangle,inner sep=0pt,outer sep=0pt]{#1};},
frametitleaboveskip=-\ht\strutbox,
%frametitleaboveskip=-0.5\ht\strutbox,
frametitlealignment=\raggedright
}%
\begin{mdframed}[style=notice]
}{\end{mdframed}}

\definecolor{lightgray}{rgb}{.9,.9,.9}
\definecolor{darkgray}{rgb}{.4,.4,.4}
\definecolor{purple}{rgb}{0.65, 0.12, 0.82}

\lstdefinelanguage{parigp}{
  alsoletter={\\},
  keywords={ for, forcomposite, fordiv, fordivfactored, forell, forfactored, forpart, forperm,%
             forprime, forsquarefree, forstep, forsubgroup, forsubset, forvec,%
             if, iferr, next, return, until, while, %
             local,%
             alias, allocatemem, apply, call, default,%
             extern, externstr, fold, getabstime, getenv, getheap, getrand, getstack, gettime,%
             getwalltime, global, inline, input%
             localbitprec, localprec,%
             print, print1, printf, printp, printsep, printsep1, printtex,%
             quit, read, readstr, readvec, self, setrand, system, type, uninline,%
             version, write, write1, writebin, writetex},
  keywordstyle=\bfseries,
  morekeywords={[2]{log, floor, ceil,gcd, lcm,sqrt, sqrtn,sqrtint,%
              sin,cos,tan,asin,acos,atan,sinh,cosh,tanh, asinh,acosh,atanh,%
              gamma,lngamma,psi,dilog,%
              % own functions
              flog2, newton, halley, bisection}},
  keywordstyle={[2]{\color{darkgray}\bfseries}},
  sensitive=false,
  comment=[l]{\\\\},
  commentstyle=\ttfamily,
  stringstyle=\ttfamily,
  morestring=[b]',
  morestring=[b]"
}

\lstdefinestyle{code}{
   extendedchars=true,
   basicstyle=\footnotesize\ttfamily,
   showstringspaces=false,
   showspaces=false,
   tabsize=2,
   breaklines=true,
   showtabs=false,
   xleftmargin=.2\textwidth, xrightmargin=.2\textwidth,
   frame=lines
%   texcl= true
}

\lstdefinestyle{widercode}{
   extendedchars=true,
   basicstyle=\footnotesize\ttfamily,
   showstringspaces=false,
   showspaces=false,
   tabsize=2,
   breaklines=true,
   showtabs=false,
   xleftmargin=.05\textwidth, xrightmargin=.05\textwidth,
   frame=lines
}

\lstnewenvironment{pblisting}[1]
  {%
    \lstset{style=widercode,#1}%
    \mdframed[style=note]%
  }
  {%
    \endmdframed
  }
% to get subsubsection numbered, too
\setcounter{secnumdepth}{3}
\setcounter{tocdepth}{3}

% Boxes with the gray background
% shamelessly stolen from https://tex.stackexchange.com/a/7545
\usepackage{calc}
\newlength{\DepthReference}
\settodepth{\DepthReference}{g}
\newlength{\HeightReference}
\settoheight{\HeightReference}{T}
\newlength{\Width}
\newcommand{\newcolorbox}[2][gray!30] {%
  \settowidth{\Width}{#2}%
  \setlength{\fboxsep}{.9pt}%
  \fcolorbox{black}{#1}{%
    \raisebox{-\DepthReference} {%
      \parbox[b][\HeightReference+\DepthReference][c]{\Width}{\centering#2}%
    }%
  }%
}
\setlength{\fboxsep}{1pt}
%no, a framed ox with grey background is just too much here
%\newcommand*{\ttchar}[1]{\newcolorbox{\texttt{\strut#1}}}
\newcommand*{\ttchar}[1]{\texttt{\textbf{\strut#1}}}
\newcommand*{\smalltt}[1]{\small\texttt{#1}}

\makeatletter
% Idea shamlessly stolen from Stephan Lehmke in
% https://tex.stackexchange.com/a/53716
\newcommand\circled
{%
  \mathpalette\@circled
}
\newcommand\@circled[2]
{%
  \mathbin%
  {%
    \ooalign{\hidewidth$#1#2$\hidewidth\crcr$#1\bigcirc$}%
  }%
}
\newcommand{\odiv}{\circled{\div}}
\makeatother

\sisetup{output-exponent-marker=\ensuremath{\mathrm{e}}}

\begin{document}
\title{LibTomMath's Algorithms: Lars Helmstr\"om's Barret\_todecimal}
\author{Current editor: Christoph Zurnieden\\
        \small{\texttt{$<$czurnieden@gmx.de$>$}}}
\date{Last change: \today}
\maketitle

%\tableofcontents
% Courier because Computer Modern has no boldface
\renewcommand{\ttdefault}{pcr}

\begin{abstract}
This artikel describes the algorithm behind the function \smalltt{Barret\_todecimal} in Lars Helmstr\"om's post to the TCL list\footnote{Available online at: \url{https://sourceforge.net/p/tcl/mailman/message/31972670/}}.
\end{abstract}

\section{Pseudocode}

\subsection{Setup}
\begin{center}
  \captionof{algorithm}{Barret\_todecimal\label{alg:barrettodecimal}}
  \begin{algorithmic}[1]
    \Require{$a > 0$, big-integer, $b$, a buffer of suitable size}
    \Ensure{$b$ filled with the value of $a$ converted to a decimal number}
    \Function{Barret\_todecimal}{$a$}
      \CommentInlineMulti{Setup and zero out three buffers that are able to hold 20 bigints each}
      \Let{$n^\ell$}{$\{20,0\}$}
      \Let{$s^\ell$}{$\{20,0\}$}
      \Let{$m^\ell$}{$\{20,0\}$}

      \CommentInlineMulti{Set initial values: radix, shift and reciprocal}
      \Let{$n$}{$1000$}
      \Let{$n^\ell_0$}{$n$}
      \Let{$s$}{$20$}
      \Let{$s^\ell_0$}{$s$}

      \Let{$M$}{$8389$}\Comment{$\ceil{ \left(8 \cdot 2^{20}\right)/1000}$}
      \Let{$m^\ell_0$}{$1049$}\Comment{$\ceil{M/8}$}

      \Let{$\rho$}{$1$}\Comment{Index into the buffers $n^\ell, s^\ell, m^\ell$}
      \While{$p \le h$}
        \Let{$n$}{$n^2$}
        \If{$n > a$}
           \Break
        \EndIf
        \CommentInlineMulti{Compute $M = \ceil{\left(2^{2s}\right)/n^2}$}
        \Let{$M_2$}{$2M^2$}
        \Let{$M_4$}{$\floor{\frac{nM^4}{2^{s + 6}}}$}
        \Let{$M$}{$\floor{\frac{\left(M_2 - M_4\right)}{2^6}} + 1$}
        \Let{$m^\ell_\rho$}{$M$}
        \Let{$\rho$}{$\rho + 1$}
        \Let{$M$}{$8M$}
      \EndWhile
      \If{$\rho \ge 20$}
         \RETURN error
      \EndIf
      \CommentInlineMulti{Use the pre-computed reciprocals recursively}
      \State \Call{Barrett\_todecimalRecursion}{$a, n^\ell, s^\ell, m^\ell, \rho - 1, 1, b$}
    \EndFunction
  \end{algorithmic}
\end{center}

The first five rounds and the last one result in the following values
\begin{table}[h]
\begin{center}
\begin{tabular}{c l r l }
\textbf{Index}&\textbf{$\mathbf{n}$}&\textbf{Shift}&\textbf{$\mathbf{M}$}\\
1  & $n^1$  &  20 & $1049$\\
2  & $n^2$  &  40 & $1099512$\\
3  & $n^4$  &  80 & $1208925819615$\\
4  & $n^8$  & 160 & $1461501637330902918203685$\\
5  & $n^{16}$ & 320 & $2135987035920910082395021706169552114602704522357$\\
\multicolumn{4}{c}{\textellipsis}\\
20 & $n^{\left({2^{20}}\right)}$ & 20971520 & $\approx 3.7554941942\,e3167328$
\end{tabular}
\captionof{table}{Results of first five rounds and the last one}\label{tab:firstfiveloops}
\end{center}
\end{table}

Are $3\,167\,329$ decimal digits sufficient?
The largest number possible in LTM is\\
 \-\hspace{1cm}\smalltt{(1<<(sizeof(int) * CHAR\_BIT) - 1) * MP\_DIGIT\_MAX}\\
 bits large. That is for 32 bit integers and \smalltt{MP\_64BIT}: $2^{31} \cdot 60 = 128\,849\,018\,880$ ($38\,787\,419\,594$ decimal digits) which overflows several functions in LTM, so the actually possible largest value is only\\
 \-\hspace{1cm}\smalltt{(1<<(sizeof(int) * CHAR\_BIT - 1)) / MP\_DIGIT\_MAX}\\
 bits large: $\floor{2^{31}/60} = 35\,791\,394$ under the same conditions as above. In decimal digits: $\floor{35\,791\,394\log_{10}(2)} = 10\,774\,283$ which is a bit more than 3 times larger than $3\,167\,329$, so the answer is: no, it is not sufficient.

The next entries in the list are be
\begin{center}
\begin{longtable}{c c l l l}
\multicolumn{1}{c}{\textbf{Index}}&
\multicolumn{1}{c}{$\mathbf{n}$}&
\multicolumn{1}{c}{\textbf{Shift}}&
\multicolumn{1}{c}{$\mathbf{M}$}&
\multicolumn{1}{l}{\textbf{Comment}}\\
\endfirsthead
%

\multicolumn{1}{c}{\textbf{Index}}&
\multicolumn{1}{c}{$\mathbf{n}$}&
\multicolumn{1}{c}{\textbf{Shift}}&
\multicolumn{1}{c}{$\mathbf{M}$}&
\multicolumn{1}{l}{\textbf{Comment}}\\
\endhead
% To get a bit more "headroom"
\rule{0pt}{1ex}\\
\multicolumn{5}{c}{{\footnotesize{\textit{Continued on next page}}}} \\
\endfoot
\endlastfoot
21 & $n^{\left(2^{21}\right)}$ & 41943040      & $1.4103736642\,e6334657$     & \\
22 & $n^{\left(2^{22}\right)}$ & 83886080      & $1.9891538728\,e12669314$  
                                                                      & $> 10^{10\,774\,283}$ \\
23 & $n^{\left(2^{23}\right)}$ & 167772160     & $3.9567331296\,e25338628$    &  \\
24 & $n^{\left(2^{24}\right)}$ & 335544320     & $1.5655737058\,e50677257$    &  \\
25 & $n^{\left(2^{25}\right)}$ & 671088640     & $2.4510210284\,e101354514$   &  \\
26 & $n^{\left(2^{26}\right)}$ & 1342177280    & $6.0075040819\,e202709028$   &  \\
27 & $n^{\left(2^{27}\right)}$ & 2684354560    & $3.6090105294\,e405418057$   &  \\
28 & $n^{\left(2^{28}\right)}$ & 5368709120    & $1.3024957001\,e810836115$   &  \\
29 & $n^{\left(2^{29}\right)}$ & 10737418240   & $1.6964950488\,e1621672230$  &  \\
30 & $n^{\left(2^{30}\right)}$ & 21474836480   & $2.8780954507\,e3243344460$  &  \\
31 & $n^{\left(2^{31}\right)}$ & 42949672960   & $8.2834334231\,e6486688920$  &  \\
32 & $n^{\left(2^{32}\right)}$ & 85899345920   & $6.8615269276\,e12973377841$ &  \\
33 & $n^{\left(2^{33}\right)}$ & 171798691840  & $4.7080551778\,e25946755683$ &  \\
34 & $n^{\left(2^{34}\right)}$ & 343597383680  & $2.2165783558\,e51893511367$ 
                                                                      & $> 10^{38\,787\,419\,594}$ \\
\end{longtable}
\captionof{table}{Results of rounds 21-34)}\label{tab:rounds21to34}
\end{center}

To save some stack memory compute the upper limit of the number $n$ of reciprocals needed calculate
\begin{equation}
n = \floor{\log_2 \left( \floor{\frac{b_a}{2\floor{\log_2 10}}} + 2 \right)  }
\end{equation}
with $b_a = \floor{\log_2 a} + 1$ the number of bits in $a$, assuming $1\,000$ as the start-value and allocate the memory for the lists on the heap instead.

\subsection{Recursion}

\begin{center}
  \captionof{algorithm}{Barret\_todecimalRecursion\label{alg:barrettodecimalrecursion}}
  \begin{algorithmic}[1]
    \Require{$a > 0$, big-integer, $n^\ell, s^\ell, m^\ell, \rho, L, b$, from Barret\_todecimal}
    \Ensure{$b$ filled with the value of $a$ converted to a decimal number}
    \Function{Barret\_todecimalRecusive}{$a,n^\ell, s^\ell, m^\ell, \rho, L, b$}
    \If{$\rho < 0$}
      \Let{$b$}{$b | a$}\Comment{Convert $a$ to a decimal string and add to $b$. See text for details}
      \RETURN
    \Else
      \Let{$q$}{$\floor{\left(a\cdot m^\ell_\rho\right) / 2^{s^\ell_\rho}}$}\Comment{Quotient}
      \Let{$r$}{$ a - n^\ell_\rho \cdot q$}
      \If{$r < 0$}
        \Let{$q$}{$q - 1$}
        \Let{$r$}{$r + n^\ell_\rho$}
      \EndIf
      \Let{$\rho$}{$\rho + 1$}
      \If{$\left(L = 1\right) \AND \left(q = 0\right) $}
        \State \Call{Barret\_todecimalRecusive}{$r,n^\ell, s^\ell, m^\ell, \rho, 1, b$}
      \Else
        \State \Call{Barret\_todecimalRecusive}{$q,n^\ell, s^\ell, m^\ell, \rho, L, b$}
        \State \Call{Barret\_todecimalRecusive}{$r,n^\ell, s^\ell, m^\ell, \rho, 0, b$}
      \EndIf
    \EndIf
    \EndFunction
  \end{algorithmic}
\end{center}

If the chunk is small enough ($\rho < 0$ for 0-based lists) we can build that part of the output.
The smallest chunk has 4 decimal digits in our case ($n^1 = 1000$) so we need up to three leading zeros if the parameter ``left'' is true or just build the decimal number without any further ado.
This is also a place where a bit of optimization can take place because four decimal digits is a rather small chunk and has most probably been chosen to be fit for \smalltt{MP\_8BIT} ($\log_{10}(2^{16}) \approx 4.8$).
If we want a bigger chunk we can stop at a higher index and use \smalltt{mp\_to\_radix} to convert\footnote{Or strip the relevant part out of \smalltt{mp\_to\_radix} which is just the inner loop. All the checks are not needed because they have already been done}. Handling leading zeros is a bit more complicated in that case but not much.

\section{Computing the Reciprocals}

The reciprocals are the values of the sequence
$\ceil{\left(2^s/n\right)^{\left(2^k\right)}}  =  \ceil{\left(2^{2^ks}\right)/n^{\left(2^k\right)}}$
(with $k = \{0,1,2,3,\dots\}$, $s = 20$, and $n = 1\,000$). We cannot simply square because $2^s \nmid n$ and hence $\ceil{\left(2^s/n\right)}^{\left(2^k\right)} \not= \ceil{\left(2^s/n\right)^{\left(2^k\right)}}$ but we can use one round of Newton-Raphson to overcome this inconvenient obstacle.

\begin{equation}
x_{n+1} = x_n\left(2 - Dx_n\right) = 2x_n - Dx_n^2
\end{equation}

By substituting $x_n = \ceil{2^s/n}^2$ and\footnote{Here $\odiv$ stands for ``use truncating division (rounding to next integer in direction of zero) later in the algebraic transformations because truncating division is all we have in big--integer land''} $D = \left(n \odiv 2^s\right)^2$ we get

\begin{equation}\label{eq:newtonraphson01}
x_{n+1} = 2\ceil{2^s/n}^2 - \floor{\frac{\ceil{2^s/n}^4n^2}{2^{2s}}}
\end{equation}

\begin{equation}
x_{n+1} = 2\left\lceil 2^s/n\right\rceil^2 - \left\lfloor\frac{\left\lceil 2^s/n\right\rceil^4 n^2}{2^{2s}}\right\rfloor
\end{equation}

The error $\left(1- Dx_n\right)^2$ is $1$, assuming the same substitutions as above, but that is misleading. The error without the truncating division is about \num{2e-34} and a simple addition of $1$ will not do. The following short and highly unoptimized Pari/\kern-1.2pt gp script will wet this dry theory with some numbers:

\lstset{language=parigp}
\begin{pblisting}{caption={Print error},label=lst:parigperror}
NR_error(extra, limit) = {
   local(n, s, M, t);
   for(m=0,extra,
      n = 1000;
      s= 20;
      for(k=0,limit,
         M = (2*ceil(2^(s+m)/n)^2 - ( ceil(2^(s+m)/n)^4*n^2)\2^(2*(s+m)))\2^(2*m) + 1;
         t = ceil(2^(2*s)/n^2);
         t = M-t;
         printf(t);
         if( k != limit,
            printf(", ")
         );
         n = n^2;
         s = 2*s;
      );
       print("");
   );
}
\end{pblisting}

Running \texttt{NR\_error(3, 22)} lasts a long time and needs a lot of memory but eventually prints:

\begin{table}[h]
\begin{center}\resizebox{\textwidth}{!}{%
\begin{tabular}{c | c  c  c  c  c  c  c  c  c  c  c  c  c  c  c  c  c  c  c  c  c c c}
\multicolumn{1}{c}{\texttt{\textbf{extra}}}&\multicolumn{18}{c}{\textbf{Error}}\\
$0$ & $0$& $1$& $1$& $1$& $0$& $0$& $1$& $0$& $-$$2$& $1$& $0$& $1$& $-$$2$& $1$& $-$$1$& $0$& $-$$3$& $0$& $0$& $0$ & $0$& $1$& $0$ \\
$1$ & $0$& $0$& $0$& $0$& $0$& $1$& $0$& $-$$1$& $0$& $0$& $1$& $0$& $0$& $0$& $0$& $-$$1$& $-$$1$& $-$$1$& $0$& $0$& $0$& $0$& $0$ \\
$2$ &  $0$& $0$& $0$& $0$& $0$& $0$& $0$&  $0$& $0$& $0$& $0$& $0$& $0$& $0$& $0$&  $0$&  $0$&  $0$& $0$& $0$& $0$& $0$& $0$ \\
$3$ &$0$& $0$& $0$& $0$& $0$& $0$& $0$&  $0$& $0$& $0$& $0$& $0$& $0$& $0$& $0$&  $0$&  $0$&  $0$& $0$& $0$& $0$& $0$& $0$   
\end{tabular}}
\captionof{table}{Results of \texttt{NR\_error(3, 22)}}\label{tab:nr_error_3_22}
\end{center}
\end{table}

\subsection{Expanding to other bases}

The largest problem for other bases is generating a good fitting $s$. With base $10$ it is sufficient to use a simple $s = 2s$ and come close to what is actually needed. The upper limit of the of $n$ in the table \ref{tab:startvaluesn} is set to $\log_2(n) \le 20$.

The little script used to generate the values for the table \ref{tab:startvaluesn} is:

\lstset{language=parigp}
\begin{pblisting}{caption={Find start values for $n$},label=lst:parigpsizebases3510}
max_size_n(limit_s) = {
   local(t, n);
   for(k= 2, 63,
      for(m = 1, limit_s + 1,
         n = k^m;
         t = ceil(log(n)/log(2));
         if( t > limit_s,
            t = m-1;
            n = k^t;
            print(k ", " t ", " n);
            break(1);
         );
      );
   );
}

\end{pblisting}
\begin{center}
\begin{longtable}{r r r}
\multicolumn{1}{r}{$\mathbf{b}$}&
\multicolumn{1}{r}{$\mathbf{k}$}&
\multicolumn{1}{r}{$\mathbf{n = b^k}$} \\
\endfirsthead
%
\multicolumn{1}{r}{$\mathbf{b}$}&
\multicolumn{1}{r}{$\mathbf{k}$}&
\multicolumn{1}{r}{$\mathbf{n = b^k}$} \\
\endhead
% To get a bit more "headroom"
\rule{0pt}{1ex}\\
\multicolumn{3}{c}{{\footnotesize{\textit{Continued on next page}}}} \\
\endfoot
\endlastfoot
% TODO: make use of the whole textwidth
2 & 20 & 1048576\\
3 & 12 & 531441\\
4 & 10 & 1048576\\
5 & 8 & 390625\\
6 & 7 & 279936\\
7 & 7 & 823543\\
8 & 6 & 262144\\
9 & 6 & 531441\\
10 & 6 & 1000000\\
11 & 5 & 161051\\
12 & 5 & 248832\\
13 & 5 & 371293\\
14 & 5 & 537824\\
15 & 5 & 759375\\
16 & 5 & 1048576\\
17 & 4 & 83521\\
18 & 4 & 104976\\
19 & 4 & 130321\\
20 & 4 & 160000\\
21 & 4 & 194481\\
22 & 4 & 234256\\
23 & 4 & 279841\\
24 & 4 & 331776\\
25 & 4 & 390625\\
26 & 4 & 456976\\
27 & 4 & 531441\\
28 & 4 & 614656\\
29 & 4 & 707281\\
30 & 4 & 810000\\
31 & 4 & 923521\\
32 & 4 & 1048576\\
33 & 3 & 35937\\
34 & 3 & 39304\\
35 & 3 & 42875\\
36 & 3 & 46656\\
37 & 3 & 50653\\
38 & 3 & 54872\\
39 & 3 & 59319\\
40 & 3 & 64000\\
41 & 3 & 68921\\
42 & 3 & 74088\\
43 & 3 & 79507\\
44 & 3 & 85184\\
45 & 3 & 91125\\
46 & 3 & 97336\\
47 & 3 & 103823\\
48 & 3 & 110592\\
49 & 3 & 117649\\
50 & 3 & 125000\\
51 & 3 & 132651\\
52 & 3 & 140608\\
53 & 3 & 148877\\
54 & 3 & 157464\\
55 & 3 & 166375\\
56 & 3 & 175616\\
57 & 3 & 185193\\
58 & 3 & 195112\\
59 & 3 & 205379\\
60 & 3 & 216000\\
61 & 3 & 226981\\
62 & 3 & 238328\\
63 & 3 & 250047\\
\end{longtable}
\captionof{table}{Startvalues for $n$ for assorted bases $b$}\label{tab:startvaluesn}
\end{center}

The little script used to generate the values for the table \ref{tab:baserelations} is:

\lstset{language=parigp}
\begin{pblisting}{caption={Find sizes of bases 3, 5, and 10},label=lst:parigpsizebases3510}
sizes_of_bases(limit) = {
   local(s, n3, n5, n10, t3, t5, t10);
   s = 10;
   n3 = 729;
   n5 = 625;
   n10 = 1000;
   for(k=0, limit,
      t3 = ceil(log(n3)/log(2));
      t5 = ceil(log(n5)/log(2));
      t10 = ceil(log(n10)/log(2));
      print(k ": " s ", " t3 ", " t5 ", " t10);
      s = 2*s;
      n3 = n3^2;
      n5 = n5^2;
      n10 = n10^2;
  );
}
\end{pblisting}


\begin{center}
\begin{longtable}{r r r r}
\multicolumn{1}{r}{$\mathbf{s/2}$}&
\multicolumn{1}{r}{$\mathbf{3^6}$}&
\multicolumn{1}{r}{$\mathbf{5^4}$}&
\multicolumn{1}{r}{$\mathbf{10^3}$} \\
\endfirsthead
%
\multicolumn{1}{r}{$\mathbf{s}$}&
\multicolumn{1}{r}{$\mathbf{3^6}$}&
\multicolumn{1}{r}{$\mathbf{5^4}$}&
\multicolumn{1}{r}{$\mathbf{10^3}$} \\
\endhead
% To get a bit more "headroom"
\rule{0pt}{1ex}\\
\multicolumn{4}{c}{{\footnotesize{\textit{Continued on next page}}}} \\
\endfoot
\endlastfoot
10 & 10 & 10 & 10 \\
20 & 20 & 19 & 20 \\
40 & 39 & 38 & 40 \\
80 & 77 & 75 & 80 \\
160 & 153 & 149 & 160 \\
320 & 305 & 298 & 319 \\
640 & 609 & 595 & 638 \\
1280 & 1218 & 1189 & 1276 \\
2560 & 2435 & 2378 & 2552 \\
5120 & 4870 & 4756 & 5103 \\
10240 & 9739 & 9511 & 10205 \\
20480 & 19477 & 19022 & 20410 \\
40960 & 38953 & 38043 & 40820 \\
81920 & 77905 & 76085 & 81640 \\
163840 & 155809 & 152170 & 163280 \\
327680 & 311617 & 304340 & 326559 \\
655360 & 623233 & 608680 & 653118 \\
1310720 & 1246466 & 1217360 & 1306236 \\
2621440 & 2492931 & 2434719 & 2612471 \\
5242880 & 4985861 & 4869437 & 5224942 \\
10485760 & 9971722 & 9738873 & 10449883 \\
20971520 & 19943444 & 19477745 & 20899765 \\
41943040 & 39886888 & 38955490 & 41799529
\end{longtable}
\captionof{table}{Relation of $s = 2s$ to bases $3$, $5$ and $10$}\label{tab:baserelations}
\end{center}

The difference between the behaviour of base $10$ and $s$ is relatively similar but it gets quite large with the other bases. It is a bit complicated to generate the values programmatically but as all of the numbers up to $k=28$ stay below $2^{32}$ a small table would be enough.

If the input of the function \smalltt{Barret\_todecimal} is restricted to base $10$, base $5$ can be used for optimization by observing that $10 = 2\cdot5$ which can safe about a third ($\log(2)/\log(5) \approx 0.4307$) of the magnitude in multiplication.

\section{Convert \smalltt{Barret\_todecimal} to an iterative algorithm}

T.\kern-.2pt b\kern-.5pt.d.


\section{\smalltt{from_decimal}
T.\kern-.2pt b\kern-.5pt.d.




\end{document}
